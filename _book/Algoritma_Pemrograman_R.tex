% Options for packages loaded elsewhere
\PassOptionsToPackage{unicode}{hyperref}
\PassOptionsToPackage{hyphens}{url}
%
\documentclass[
]{book}
\usepackage{amsmath,amssymb}
\usepackage{iftex}
\ifPDFTeX
  \usepackage[T1]{fontenc}
  \usepackage[utf8]{inputenc}
  \usepackage{textcomp} % provide euro and other symbols
\else % if luatex or xetex
  \usepackage{unicode-math} % this also loads fontspec
  \defaultfontfeatures{Scale=MatchLowercase}
  \defaultfontfeatures[\rmfamily]{Ligatures=TeX,Scale=1}
\fi
\usepackage{lmodern}
\ifPDFTeX\else
  % xetex/luatex font selection
\fi
% Use upquote if available, for straight quotes in verbatim environments
\IfFileExists{upquote.sty}{\usepackage{upquote}}{}
\IfFileExists{microtype.sty}{% use microtype if available
  \usepackage[]{microtype}
  \UseMicrotypeSet[protrusion]{basicmath} % disable protrusion for tt fonts
}{}
\makeatletter
\@ifundefined{KOMAClassName}{% if non-KOMA class
  \IfFileExists{parskip.sty}{%
    \usepackage{parskip}
  }{% else
    \setlength{\parindent}{0pt}
    \setlength{\parskip}{6pt plus 2pt minus 1pt}}
}{% if KOMA class
  \KOMAoptions{parskip=half}}
\makeatother
\usepackage{xcolor}
\usepackage{color}
\usepackage{fancyvrb}
\newcommand{\VerbBar}{|}
\newcommand{\VERB}{\Verb[commandchars=\\\{\}]}
\DefineVerbatimEnvironment{Highlighting}{Verbatim}{commandchars=\\\{\}}
% Add ',fontsize=\small' for more characters per line
\usepackage{framed}
\definecolor{shadecolor}{RGB}{248,248,248}
\newenvironment{Shaded}{\begin{snugshade}}{\end{snugshade}}
\newcommand{\AlertTok}[1]{\textcolor[rgb]{0.94,0.16,0.16}{#1}}
\newcommand{\AnnotationTok}[1]{\textcolor[rgb]{0.56,0.35,0.01}{\textbf{\textit{#1}}}}
\newcommand{\AttributeTok}[1]{\textcolor[rgb]{0.13,0.29,0.53}{#1}}
\newcommand{\BaseNTok}[1]{\textcolor[rgb]{0.00,0.00,0.81}{#1}}
\newcommand{\BuiltInTok}[1]{#1}
\newcommand{\CharTok}[1]{\textcolor[rgb]{0.31,0.60,0.02}{#1}}
\newcommand{\CommentTok}[1]{\textcolor[rgb]{0.56,0.35,0.01}{\textit{#1}}}
\newcommand{\CommentVarTok}[1]{\textcolor[rgb]{0.56,0.35,0.01}{\textbf{\textit{#1}}}}
\newcommand{\ConstantTok}[1]{\textcolor[rgb]{0.56,0.35,0.01}{#1}}
\newcommand{\ControlFlowTok}[1]{\textcolor[rgb]{0.13,0.29,0.53}{\textbf{#1}}}
\newcommand{\DataTypeTok}[1]{\textcolor[rgb]{0.13,0.29,0.53}{#1}}
\newcommand{\DecValTok}[1]{\textcolor[rgb]{0.00,0.00,0.81}{#1}}
\newcommand{\DocumentationTok}[1]{\textcolor[rgb]{0.56,0.35,0.01}{\textbf{\textit{#1}}}}
\newcommand{\ErrorTok}[1]{\textcolor[rgb]{0.64,0.00,0.00}{\textbf{#1}}}
\newcommand{\ExtensionTok}[1]{#1}
\newcommand{\FloatTok}[1]{\textcolor[rgb]{0.00,0.00,0.81}{#1}}
\newcommand{\FunctionTok}[1]{\textcolor[rgb]{0.13,0.29,0.53}{\textbf{#1}}}
\newcommand{\ImportTok}[1]{#1}
\newcommand{\InformationTok}[1]{\textcolor[rgb]{0.56,0.35,0.01}{\textbf{\textit{#1}}}}
\newcommand{\KeywordTok}[1]{\textcolor[rgb]{0.13,0.29,0.53}{\textbf{#1}}}
\newcommand{\NormalTok}[1]{#1}
\newcommand{\OperatorTok}[1]{\textcolor[rgb]{0.81,0.36,0.00}{\textbf{#1}}}
\newcommand{\OtherTok}[1]{\textcolor[rgb]{0.56,0.35,0.01}{#1}}
\newcommand{\PreprocessorTok}[1]{\textcolor[rgb]{0.56,0.35,0.01}{\textit{#1}}}
\newcommand{\RegionMarkerTok}[1]{#1}
\newcommand{\SpecialCharTok}[1]{\textcolor[rgb]{0.81,0.36,0.00}{\textbf{#1}}}
\newcommand{\SpecialStringTok}[1]{\textcolor[rgb]{0.31,0.60,0.02}{#1}}
\newcommand{\StringTok}[1]{\textcolor[rgb]{0.31,0.60,0.02}{#1}}
\newcommand{\VariableTok}[1]{\textcolor[rgb]{0.00,0.00,0.00}{#1}}
\newcommand{\VerbatimStringTok}[1]{\textcolor[rgb]{0.31,0.60,0.02}{#1}}
\newcommand{\WarningTok}[1]{\textcolor[rgb]{0.56,0.35,0.01}{\textbf{\textit{#1}}}}
\usepackage{longtable,booktabs,array}
\usepackage{calc} % for calculating minipage widths
% Correct order of tables after \paragraph or \subparagraph
\usepackage{etoolbox}
\makeatletter
\patchcmd\longtable{\par}{\if@noskipsec\mbox{}\fi\par}{}{}
\makeatother
% Allow footnotes in longtable head/foot
\IfFileExists{footnotehyper.sty}{\usepackage{footnotehyper}}{\usepackage{footnote}}
\makesavenoteenv{longtable}
\usepackage{graphicx}
\makeatletter
\def\maxwidth{\ifdim\Gin@nat@width>\linewidth\linewidth\else\Gin@nat@width\fi}
\def\maxheight{\ifdim\Gin@nat@height>\textheight\textheight\else\Gin@nat@height\fi}
\makeatother
% Scale images if necessary, so that they will not overflow the page
% margins by default, and it is still possible to overwrite the defaults
% using explicit options in \includegraphics[width, height, ...]{}
\setkeys{Gin}{width=\maxwidth,height=\maxheight,keepaspectratio}
% Set default figure placement to htbp
\makeatletter
\def\fps@figure{htbp}
\makeatother
\setlength{\emergencystretch}{3em} % prevent overfull lines
\providecommand{\tightlist}{%
  \setlength{\itemsep}{0pt}\setlength{\parskip}{0pt}}
\setcounter{secnumdepth}{5}
\usepackage{booktabs}
\usepackage{lscape}

\ifLuaTeX
  \usepackage{selnolig}  % disable illegal ligatures
\fi
\usepackage[]{natbib}
\bibliographystyle{apalike}
\IfFileExists{bookmark.sty}{\usepackage{bookmark}}{\usepackage{hyperref}}
\IfFileExists{xurl.sty}{\usepackage{xurl}}{} % add URL line breaks if available
\urlstyle{same}
\hypersetup{
  pdftitle={Algoritma dan Pemrograman R},
  pdfauthor={Bakti Siregar, M.Sc},
  hidelinks,
  pdfcreator={LaTeX via pandoc}}

\title{Algoritma dan Pemrograman R}
\author{Bakti Siregar, M.Sc}
\date{06 April 2023}

\usepackage{amsthm}
\newtheorem{theorem}{Teorema}[chapter]
\newtheorem{lemma}{Lemma}[chapter]
\newtheorem{corollary}{Corollary}[chapter]
\newtheorem{proposition}{Proposition}[chapter]
\newtheorem{conjecture}{Conjecture}[chapter]
\theoremstyle{definition}
\newtheorem{definition}{Definition}[chapter]
\theoremstyle{definition}
\newtheorem{example}{Contoh}[chapter]
\theoremstyle{definition}
\newtheorem{exercise}{Exercise}[chapter]
\theoremstyle{definition}
\newtheorem{hypothesis}{Hypothesis}[chapter]
\theoremstyle{remark}
\newtheorem*{remark}{Remark}
\newtheorem*{solution}{Solution}
\begin{document}
\maketitle

{
\setcounter{tocdepth}{1}
\tableofcontents
}
\hypertarget{kata-pengantar}{%
\chapter*{Kata Pengantar}\label{kata-pengantar}}
\addcontentsline{toc}{chapter}{Kata Pengantar}

Bahasa pemrograman R telah menjadi alat yang kuat bagi para ilmuwan data, analis statistik, dan praktisi analisis numerik di seluruh dunia. Dengan kemampuan yang luar biasa dalam manipulasi data, visualisasi, dan analisis statistik, R memungkinkan para profesional untuk menggali wawasan berharga dari kumpulan data yang kompleks.

Dalam buku ini, penulis menyediakan materi dasar-dasar bahasa pemrograman R hingga tingkat yang lebih mendalam. Penulis juga menjelaskan beberapa konsep-konsep penting, sintaksis dasar, struktur data, serta memberikan contoh nyata tentang bagaimana R dapat digunakan dalam berbagai konteks. Modul ini dirancang untuk membantu pembaca yang baru mengenal pemrograman maupun yang telah memiliki pengalaman sebelumnya dalam bahasa lain.

\hypertarget{ringkasan-pembaelajaran}{%
\section*{Ringkasan Pembaelajaran}\label{ringkasan-pembaelajaran}}
\addcontentsline{toc}{section}{Ringkasan Pembaelajaran}

Adapun isi pembelajaran dalam modul ini adalah sebagai berikut:

\begin{itemize}
\tightlist
\item
  \textbf{Minggu 1-2: Pengenalan Pemrograman R}

  \begin{itemize}
  \tightlist
  \item
    Pengenalan konsep dasar pemrograman.
  \item
    Pengenalan lingkungan dan pengaturan awal bahasa R.
  \item
    Menulis program sederhana dalam R.
  \end{itemize}
\item
  \textbf{Minggu 3-4: Variabel dan Tipe Data}

  \begin{itemize}
  \tightlist
  \item
    Mengenal tipe data dasar dalam R: numerik, karakter, logika.
  \item
    Deklarasi dan penggunaan variabel.
  \item
    Konversi antar tipe data.
  \end{itemize}
\item
  \textbf{Minggu 5-6: Struktur Kontrol}

  \begin{itemize}
  \tightlist
  \item
    Penggunaan pernyataan kondisional (if, else).
  \item
    Penggunaan perulangan (for, while) untuk mengatur alur program.
  \item
    Studi kasus penggunaan struktur kontrol dalam pemrograman.
  \end{itemize}
\item
  \textbf{Minggu 7-8: Fungsi}

  \begin{itemize}
  \tightlist
  \item
    Pengenalan konsep fungsi dalam pemrograman.
  \item
    Membuat dan memanggil fungsi dalam R.
  \item
    Penggunaan parameter dalam fungsi.
  \end{itemize}
\item
  \textbf{Minggu 9-10: Struktur Data}

  \begin{itemize}
  \tightlist
  \item
    Pengenalan array dan matriks dalam R.
  \item
    Penggunaan vektor dan faktor.
  \item
    Pengenalan konsep dataframe untuk manipulasi data tabular.
  \end{itemize}
\item
  \textbf{Minggu 11-12: Algoritma Dasar}

  \begin{itemize}
  \tightlist
  \item
    Pengenalan konsep algoritma dan kompleksitas.
  \item
    Pemahaman tentang pencarian dan pengurutan.
  \item
    Implementasi algoritma pencarian dan pengurutan sederhana dalam R.
  \end{itemize}
\item
  \textbf{Minggu 13-14: Pengenalan Analisis Data}

  \begin{itemize}
  \tightlist
  \item
    Pengenalan pustaka dasar untuk analisis data di R.
  \item
    Pemahaman tentang statistik dasar dan visualisasi data.
  \item
    Penerapan analisis sederhana pada dataset kecil.
  \end{itemize}
\item
  \textbf{Minggu 15-16: Proyek Akhir}
  Siswa diminta untuk membuat proyek kecil menggunakan R yang menggabungkan konsep-konsep yang telah dipelajari. Proyek dapat berupa analisis data, pemecahan masalah, atau aplikasi sederhana.
\end{itemize}

\hypertarget{tim-penyusun}{%
\section*{Tim Penyusun}\label{tim-penyusun}}
\addcontentsline{toc}{section}{Tim Penyusun}

Berikut ini adalah nama dan biografi singkat para penulis:

\begin{itemize}
\tightlist
\item
  \textbf{Bakti Siregar, M.Sc} adalah Ketua Program Studi di Jurusan Statistika Universitas Matana. Lulusan Magister Matematika Terapan dari National Sun Yat Sen University, Taiwan. Beliau juga merupakan dosen dan konsultan Data Scientist di perusahaan-perusahaan ternama seperti JNE, Samora Group, Pertamina, dan lainnya. Beliau memiliki antusiasme khusus dalam mengajar Big Data Analytics, Machine Learning, Optimisasi, dan Analisis Time Series di bidang keuangan dan investasi. Keahliannya juga terlihat dalam penggunaan bahasa pemrograman Statistik seperti R Studio dan Python. Beliau mengaplikasikan sistem basis data MySQL/NoSQL dalam pembelajaran manajemen data, serta mahir dalam menggunakan tools Big Data seperti Spark dan Hadoop. Beberapa project beliau dapat dilihat di link berikut: \href{https://rpubs.com/dsciencelabs}{Rpubs}, \href{https://github.com/dsciencelabs}{Github}, \href{https://dsciencelabs.github.io/web/index.html}{Website}, dan \href{https://www.kaggle.com/baktisiregar/code}{Kaggle}.
\end{itemize}

\hypertarget{ucapan-terima-kasih}{%
\section*{Ucapan Terima Kasih}\label{ucapan-terima-kasih}}
\addcontentsline{toc}{section}{Ucapan Terima Kasih}

Kami berharap modul ini akan menjadi panduan yang bermanfaat bagi Anda dalam menguasai bahasa pemrograman R. Semoga dengan memahami konsep-konsep yang disajikan dalam modul ini, Anda akan dapat mengaplikasikan R dalam proyek-proyek analisis data dan statistik yang sebenarnya.

Terima kasih kepada semua yang telah berkontribusi dalam pembuatan modul ini, serta kepada Anda, pembaca, yang telah memilih modul ini sebagai sumber pengetahuan Anda. Kami berharap Anda menikmati perjalanan Anda dalam memahami bahasa pemrograman R.

\hypertarget{masukan-saran}{%
\section*{Masukan \& Saran}\label{masukan-saran}}
\addcontentsline{toc}{section}{Masukan \& Saran}

Semua masukan dan tanggapan Anda sangat berarti bagi kami untuk memperbaiki modul ini kedepannya. Bagi para pembaca/pengguna yang ingin menyampaikan masukan dan tanggapan, dipersilahkan melalui kontak dibawak ini!

\textbf{Email:} \href{mailto:dsciencelabs@outlook.com}{\nolinkurl{dsciencelabs@outlook.com}}

\hypertarget{pengantar-metode-numerik}{%
\chapter{Pengantar Metode Numerik}\label{pengantar-metode-numerik}}

Metode Numerik diterapkan ketika solusi analitis tidak mungkin atau sulit ditemukan, terutama dalam situasi di mana terdapat unsur ketidakpastian atau kompleksitas yang tinggi. Metode ini digunakan untuk memecahkan berbagai masalah, seperti:

\begin{itemize}
\tightlist
\item
  \textbf{Pencarian Akar Persamaan:} mencari akar (solusi) dari persamaan non-linear, yang mungkin sulit atau bahkan tidak mungkin diselesaikan secara analitis.
\item
  \textbf{Interpolasi dan Pendekatan Fungsi:} membangun fungsi pendekatan yang sesuai untuk set data, yang berguna dalam analisis dan prediksi.
\item
  \textbf{Integrasi Numerik:} menghitung integral dari fungsi yang kompleks atau tidak memiliki bentuk integral tertutup.
\item
  \textbf{Pemecahan Sistem Persamaan Linear:} menemukan solusi sistem persamaan linear, yang mungkin memiliki banyak variabel dan persamaan.
\item
  \textbf{Optimisasi Numerik:} mencari nilai maksimum atau minimum dari fungsi yang kompleks dengan cara mengiterasi melalui berbagai kemungkinan solusi.
\item
  \textbf{Pemodelan Numerik:} digunakan dalam simulasi numerik untuk memodelkan fenomena ilmiah atau teknik, seperti dinamika fluida, mekanika struktur, dan banyak lagi.
\end{itemize}

Pemahaman tentang Metode Numerik sangat penting dalam berbagai bidang, termasuk ilmu komputer, teknik, ilmu pengetahuan alam, ekonomi, dan masih banyak lagi. Keterampilan dalam mengaplikasikan metode numerik membantu dalam merancang algoritma yang efisien, mengatasi masalah yang kompleks, dan membuat prediksi yang lebih baik dalam dunia nyata.

\hypertarget{metode-vs-analitik-numerik}{%
\section{Metode vs Analitik Numerik}\label{metode-vs-analitik-numerik}}

Perbedaan antara metode numerik dan metode analitik menurut definisinya dapat dijelaskan sebagai berikut:

\begin{itemize}
\tightlist
\item
  \textbf{Metode Analitik} adalah pendekatan matematis untuk memecahkan masalah secara eksak atau analitis. Dalam metode ini, solusi ditemukan melalui manipulasi aljabar dan penerapan aturan matematika. Solusi analitis menghasilkan bentuk persamaan atau fungsi yang menggambarkan solusi secara tepat. Contoh dari metode analitik termasuk menghitung turunan, integral tertutup, dan menyelesaikan persamaan aljabar dengan cara mengisolasi variabel.
\item
  \textbf{Metode Analitik} adalah pendekatan matematis untuk memecahkan masalah secara eksak atau analitis. Dalam metode ini, solusi ditemukan melalui manipulasi aljabar dan penerapan aturan matematika. Solusi analitis menghasilkan bentuk persamaan atau fungsi yang menggambarkan solusi secara tepat. Contoh dari metode analitik termasuk menghitung turunan, integral tertutup, dan menyelesaikan persamaan aljabar dengan cara mengisolasi variabel.
\end{itemize}

Adapun perberdaan metode numerik dan metode analitik berdasarkan kelebihan dan kekurangannya dapat diuraikan sebagai berikut:

\begin{itemize}
\tightlist
\item
  Solusi metode numerik selalu berbentuk angka, sedangkan solusi metode analitik dapat berbentuk fungsi matematik yang selanjutnya dapat dievaluasi untuk menghasilkan nilai dalam bentuk angka.
\item
  Solusi dari metode numerik berupa hampiran, sedangkan metode analitik berupa solusi sejati. Kondisi ini berakibat pada nilai error metode analitik adalah 0, sedangkan metode numerik \(\neq 0\).
\item
  Metode analitik cocok untuk permasalahan dengan model terbatas dan sederhana, sedangkan metode numerik cocok dengan semua jenis permasalahan.
\end{itemize}

\hypertarget{tahapan-metode-numerik}{%
\section{Tahapan Metode Numerik}\label{tahapan-metode-numerik}}

Tahapan dalam penerapan Metode Numerik melibatkan serangkaian langkah yang disusun secara sistematis untuk memecahkan masalah matematika secara numerik. Berikut adalah tahapan umum yang terlibat dalam Metode Numerik:

\begin{itemize}
\tightlist
\item
  \textbf{Perumusan Masalah:} Tentukan masalah matematika yang ingin Anda selesaikan secara numerik, misalnya, mencari akar persamaan, menghitung integral, atau menyelesaikan sistem persamaan linear.
\item
  \textbf{Pemilihan Metode:} Pilih metode numerik yang sesuai untuk masalah yang Anda hadapi. Misalnya, jika Anda ingin mencari akar persamaan, Anda dapat menggunakan metode seperti metode bisection, metode Newton-Raphson, atau metode iteratif lainnya.
\item
  \textbf{Pembuatan Algoritma:} Buat algoritma berdasarkan metode yang dipilih. Algoritma ini harus berisi langkah-langkah yang jelas dan terdefinisi untuk mencari solusi numerik.
\item
  \textbf{Pemrograman:} Implementasikan algoritma dalam bahasa pemrograman, seperti R, Python, MATLAB, atau bahasa lainnya. Pemrograman melibatkan mengubah langkah-langkah algoritma menjadi kode yang dapat dieksekusi oleh komputer.
\item
  \textbf{Inisialisasi:} Inisialisasi nilai awal yang diperlukan untuk memulai iterasi algoritma. Nilai inisialisasi ini dapat mempengaruhi konvergensi dan akurasi solusi.
\item
  \textbf{Iterasi:} Jalankan algoritma secara berulang (iteratif) sesuai dengan langkah-langkah yang telah ditentukan. Setiap iterasi mengarahkan solusi lebih dekat ke solusi akhir yang diinginkan.
\item
  \textbf{Konvergensi dan Kriteria Berhenti:} Tentukan kriteria berhenti iterasi. Misalnya, berhenti saat perbedaan antara solusi saat ini dan solusi sebelumnya cukup kecil, atau ketika jumlah iterasi mencapai batas tertentu.
\item
  \textbf{Analisis Hasil:} Analisis hasil dari algoritma numerik. Evaluasi apakah solusi yang ditemukan sudah cukup akurat sesuai dengan kriteria yang ditentukan.
\item
  \textbf{Presentasi dan Interpretasi:} Presentasikan solusi numerik secara grafis atau dalam bentuk yang mudah dimengerti. Jelaskan hasilnya dalam konteks masalah yang dipecahkan.
\item
  \textbf{Validasi:} Validasi hasil dengan cara membandingkannya dengan solusi analitis (jika tersedia) atau dengan hasil lain yang diperoleh melalui metode numerik alternatif.

  \begin{itemize}
  \tightlist
  \item
    \textbf{Optimisasi dan Peningkatan:} Jika diperlukan, pertimbangkan untuk mengoptimasi algoritma atau mengganti metode numerik untuk meningkatkan akurasi atau efisiensi solusi.
  \item
    \textbf{Penggunaan Kembali:} Terapkan kembali metode numerik dan algoritma yang telah diuji dan diimplementasikan untuk masalah serupa di masa depan.
  \end{itemize}
\end{itemize}

Tahapan-tahapan ini membantu memastikan bahwa penerapan Metode Numerik dilakukan secara sistematis dan menghasilkan solusi yang akurat dan dapat diandalkan untuk berbagai masalah matematika yang kompleks.

\hypertarget{acuracy}{%
\section{Akurasi dan Presisi}\label{acuracy}}

Perhatikan Gambar \ref{fig:akurasi} berikut:

\begin{figure}

{\centering \includegraphics[width=0.75\linewidth]{./images/akurasi} 

}

\caption{4 ilustrasi terkait akurasi dan presisi}\label{fig:akurasi}
\end{figure}

Pada Gambar \ref{fig:akurasi} terdapat 4 buah kondisi ketika kita menembakkan beberapa perluru pada sebuah sasaran. Tujuan kita disini adalah untuk menembak bagian tengah sasaran tersebut.

Pada Gambar (a) dan (c) pada Gambar \ref{fig:akurasi} merupakan gambar yang menunjukkan seseorang telah berhasil mengenai bagian tengah sasaran tersebut dapat kita katakan pula tembakan pada kedua gambar tersebut akurat. Akurat dalam hal ini dapat diartikan suatu kondisi dimana kedekatan lubang peluru dengan pusat sasaran. Secara umum akurasi diartikan sebagai tingkat kedekatan pengukuran kuantitas terhadap nilai sebenarnya.

Terdapat dua buah cara untuk mengukur akurasi. Metode pengukuran akurasi antara lain: error absolut dan error relatif. Error absolut merupakan nilai absolut dari selisih antara nilai sebenarnya \(x\) dengan nilai observasi \(x'\). Error absolut dapat dituliskan menggunakan Persamaan \eqref{eq:errorabsolut}.

\begin{equation}
   \epsilon_A=\left|x-x'\right|
  \label{eq:errorabsolut}
\end{equation}

Pengukuran lain yang sering digunakan untuk mengukur akurasi adalah error relatif. Berbeda dengan error absolut, error relatif membagi selisih antara nilai sebenarnya \(x\) dan nilai observasi \(x'\) dengan nilai sebenarnya. Hasil yang diperoleh merupakan nilai tanpa satuan. Persamaan error relatif disajikan pada Persamaan \eqref{eq:errorrelatif}.

\begin{equation}
   \epsilon_R=\left|\frac{x-x'}{x}\right|
  \label{eq:errorrelatif}
\end{equation}

Dalam suatu pengukuran, hal lain yang perlu diperhatikan selain akurasi adalah presisi. Presisi adalah sejauh mana pengulangan pengukuran dalam kondisi yang tidak berubah mendapat hasil yang sama. Berdasarkan Gambar \ref{fig:akurasi}, Gambar (a) dan (b) menunjukkan kepresisian yang tinggi. Hal ini terlihat dari jarak antara lubang peluru yang saling berdekatan dan mengelompok.

Berdasarkan Gambar \ref{fig:akurasi} dapat kita simpulkan bahwa dalam suatu sistem pengukuran akan terdapat 4 buah kondisi. Pengukuran akurat dan presisi (Gambar (a)), tidak akurat namun presisi (Gambar (b)), akurat namun tidak presisi (Gambar (c)), dan tidak akurat serta tidak presisi (Gambar (d)). Dalam analisa numerik error atau kesalahan menjadi hal yang perlu diperhatikan.

\hypertarget{error-numerik}{%
\section{Error Numerik}\label{error-numerik}}

Ketika kita menggunakan Metode Numerik untuk memecahkan masalah matematika, kita akan selalu dihadapkan pada konsep error numerik. Error numerik mengacu pada perbedaan antara solusi yang dihitung secara numerik dengan solusi yang sebenarnya atau solusi eksak dari masalah matematika tersebut. Karena kita menggunakan pendekatan angka-angka (numerik) daripada solusi eksak (analitis), error numerik merupakan hal yang tak terhindarkan. Terdapat beberapa jenis error numerik yang biasanya muncul dalam penerapan Metode Numerik, diperlihatkan pada subtopik berikut.

\hypertarget{error-mutlak-absolute-error}{%
\subsection{\texorpdfstring{Error Mutlak \emph{(Absolute Error)}}{Error Mutlak (Absolute Error)}}\label{error-mutlak-absolute-error}}

Ini adalah perbedaan absolut antara solusi yang dihitung secara numerik dan solusi eksak. Dinyatakan dalam satuan yang sama dengan solusi itu sendiri (misalnya, dalam angka atau satuan fisik tertentu). Misalkan Anda memiliki solusi eksak dari suatu masalah dan hasil yang dihitung secara numerik dalam bahasa pemrograman R. Anda dapat menghitung absolute error \eqref{eq:errorabsolut} sebagai perbedaan absolut antara solusi eksak dan hasil numerik.

\begin{Shaded}
\begin{Highlighting}[]
\CommentTok{\# Contoh solusi eksak}
\NormalTok{solusi\_eksak }\OtherTok{\textless{}{-}} \FloatTok{10.5}

\CommentTok{\# Contoh hasil numerik}
\NormalTok{hasil\_numerik }\OtherTok{\textless{}{-}} \FloatTok{10.2}

\CommentTok{\# Menghitung absolute error}
\NormalTok{absolute\_error }\OtherTok{\textless{}{-}} \FunctionTok{abs}\NormalTok{(solusi\_eksak }\SpecialCharTok{{-}}\NormalTok{ hasil\_numerik)}

\CommentTok{\# Menampilkan hasil}
\FunctionTok{print}\NormalTok{(}\FunctionTok{paste}\NormalTok{(}\StringTok{"Absolute Error:"}\NormalTok{, absolute\_error))}
\end{Highlighting}
\end{Shaded}

\begin{verbatim}
## [1] "Absolute Error: 0.300000000000001"
\end{verbatim}

\hypertarget{error-relatif-relative-error}{%
\subsection{\texorpdfstring{Error Relatif \emph{(Relative Error)}}{Error Relatif (Relative Error)}}\label{error-relatif-relative-error}}

Ini adalah perbandingan antara absolute error dengan solusi eksak. Dinyatakan sebagai persentase atau dalam bentuk pecahan, sehingga memberikan gambaran seberapa besar error dibandingkan dengan ukuran solusi. Untuk menghitung absolute error, dapat dilakukan seperti pada contoh sebelumnya. Kemudian, kita menghitung relative error \eqref{eq:errorrelatif} dengan membagi absolute error dengan nilai absolut dari solusi eksak, dan kemudian mengalikannya dengan 100 untuk mendapatkan persentase.

\begin{Shaded}
\begin{Highlighting}[]
\CommentTok{\# Contoh solusi eksak}
\NormalTok{solusi\_eksak }\OtherTok{\textless{}{-}} \FloatTok{10.5}

\CommentTok{\# Contoh hasil numerik}
\NormalTok{hasil\_numerik }\OtherTok{\textless{}{-}} \FloatTok{10.2}

\CommentTok{\# Menghitung absolute error}
\NormalTok{absolute\_error }\OtherTok{\textless{}{-}} \FunctionTok{abs}\NormalTok{(solusi\_eksak }\SpecialCharTok{{-}}\NormalTok{ hasil\_numerik)}

\CommentTok{\# Menghitung relative error \& konversi ke persentase}
\NormalTok{relative\_error }\OtherTok{\textless{}{-}}\NormalTok{ absolute\_error }\SpecialCharTok{/} \FunctionTok{abs}\NormalTok{(solusi\_eksak) }\SpecialCharTok{*} \DecValTok{100}  

\CommentTok{\# Menampilkan hasil}
\FunctionTok{print}\NormalTok{(}\FunctionTok{paste}\NormalTok{(}\StringTok{"Relative Error:"}\NormalTok{, relative\_error, }\StringTok{"\%"}\NormalTok{))}
\end{Highlighting}
\end{Shaded}

\begin{verbatim}
## [1] "Relative Error: 2.85714285714286 %"
\end{verbatim}

\hypertarget{truncation-error-error-pemotongan}{%
\subsection{Truncation Error (Error Pemotongan)}\label{truncation-error-error-pemotongan}}

Truncation error muncul ketika kita menggunakan pendekatan yang lebih sederhana atau mengabaikan suku-suku yang lebih kecil dalam perhitungan. Ini terutama terjadi dalam pendekatan deret tak hingga atau pendekatan iteratif yang menghentikan iterasi setelah jumlah langkah tertentu. Misalkan kita ingin menghitung nilai eksponensial menggunakan deret Taylor yang dipotong setelah beberapa suku:

\begin{Shaded}
\begin{Highlighting}[]
\CommentTok{\# Fungsi untuk menghitung nilai eksponensial menggunakan deret Taylor}
\NormalTok{exp\_approx }\OtherTok{\textless{}{-}} \ControlFlowTok{function}\NormalTok{(x, n) \{}
\NormalTok{  approx\_value }\OtherTok{\textless{}{-}} \DecValTok{1}  \CommentTok{\# Suku pertama dari deret}
  \ControlFlowTok{for}\NormalTok{ (i }\ControlFlowTok{in} \DecValTok{1}\SpecialCharTok{:}\NormalTok{n) \{}
\NormalTok{    approx\_value }\OtherTok{\textless{}{-}}\NormalTok{ approx\_value }\SpecialCharTok{+}\NormalTok{ (x}\SpecialCharTok{\^{}}\NormalTok{i) }\SpecialCharTok{/} \FunctionTok{factorial}\NormalTok{(i)}
\NormalTok{  \}}
  \FunctionTok{return}\NormalTok{(approx\_value)}
\NormalTok{\}}

\CommentTok{\# Nilai eksak dari e\^{}2}
\NormalTok{exp\_exact }\OtherTok{\textless{}{-}} \FunctionTok{exp}\NormalTok{(}\DecValTok{2}\NormalTok{)}

\CommentTok{\# Hitung nilai perkiraan eksponensial dengan deret Taylor (misal, n = 5)}
\NormalTok{n\_terms }\OtherTok{\textless{}{-}} \DecValTok{5}
\NormalTok{exp\_approximation }\OtherTok{\textless{}{-}} \FunctionTok{exp\_approx}\NormalTok{(}\DecValTok{2}\NormalTok{, n\_terms)}

\CommentTok{\# Hitung truncation error}
\NormalTok{truncation\_error }\OtherTok{\textless{}{-}} \FunctionTok{abs}\NormalTok{(exp\_exact }\SpecialCharTok{{-}}\NormalTok{ exp\_approximation)}

\CommentTok{\# Menampilkan hasil}
\FunctionTok{print}\NormalTok{(}\FunctionTok{paste}\NormalTok{(}\StringTok{"Truncation Error:"}\NormalTok{, truncation\_error))}
\end{Highlighting}
\end{Shaded}

\begin{verbatim}
## [1] "Truncation Error: 0.122389432263984"
\end{verbatim}

\hypertarget{error-pembulatan-round-off-error}{%
\subsection{\texorpdfstring{Error Pembulatan \emph{(Round-off Error)}}{Error Pembulatan (Round-off Error)}}\label{error-pembulatan-round-off-error}}

Ini adalah kesalahan yang muncul karena keterbatasan representasi angka di komputer. Karena komputer hanya dapat merepresentasikan angka dalam bentuk biner (format floating-point), beberapa angka tidak dapat direpresentasikan secara tepat. Akibatnya, terdapat ketidakakuratan dalam perhitungan. Berikut adalah contoh sederhana menggunakan R:

\begin{Shaded}
\begin{Highlighting}[]
\CommentTok{\# Contoh perhitungan yang menghasilkan round{-}off error}
\NormalTok{result }\OtherTok{=} \FloatTok{1.0} \SpecialCharTok{/} \FloatTok{3.0} \SpecialCharTok{+} \FloatTok{2.0} \SpecialCharTok{/} \FloatTok{3.0}

\CommentTok{\# Tampilkan hasil}
\FunctionTok{print}\NormalTok{(result)}
\end{Highlighting}
\end{Shaded}

\begin{verbatim}
## [1] 1
\end{verbatim}

\hypertarget{error-stabilitas-stability-error}{%
\subsection{\texorpdfstring{Error Stabilitas \emph{(Stability Error)}}{Error Stabilitas (Stability Error)}}\label{error-stabilitas-stability-error}}

Ketika sebuah metode numerik tidak stabil, hasil yang dihitung secara numerik bisa sangat sensitif terhadap perubahan kecil dalam input atau iterasi. Ini dapat menyebabkan perubahan besar dalam hasil yang diharapkan. Misalkan kita ingin mengilustrasikan error stabilitas dengan menggunakan metode numerik iteratif yang mengalami divergensi. Berikut adalah contoh sederhana:

\begin{Shaded}
\begin{Highlighting}[]
\CommentTok{\# Contoh metode iteratif yang mengalami divergensi}
\NormalTok{unstable\_method }\OtherTok{\textless{}{-}} \ControlFlowTok{function}\NormalTok{(x, n) \{}
  \ControlFlowTok{for}\NormalTok{ (i }\ControlFlowTok{in} \DecValTok{1}\SpecialCharTok{:}\NormalTok{n) \{}
\NormalTok{    x }\OtherTok{\textless{}{-}} \DecValTok{2} \SpecialCharTok{*}\NormalTok{ x}
\NormalTok{  \}}
  \FunctionTok{return}\NormalTok{(x)}
\NormalTok{\}}

\CommentTok{\# Nilai awal}
\NormalTok{initial\_value }\OtherTok{\textless{}{-}} \FloatTok{0.1}

\CommentTok{\# Hitung nilai setelah 10 iterasi}
\NormalTok{n\_iterations }\OtherTok{\textless{}{-}} \DecValTok{10}
\NormalTok{final\_value }\OtherTok{\textless{}{-}} \FunctionTok{unstable\_method}\NormalTok{(initial\_value, n\_iterations)}

\CommentTok{\# Tampilkan hasil}
\FunctionTok{print}\NormalTok{(}\FunctionTok{paste}\NormalTok{(}\StringTok{"Final Value after"}\NormalTok{, n\_iterations, }\StringTok{"iterations:"}\NormalTok{, final\_value))}
\end{Highlighting}
\end{Shaded}

\begin{verbatim}
## [1] "Final Value after 10 iterations: 102.4"
\end{verbatim}

Untuk mengatasi error numerik, penting untuk memahami jenis error yang mungkin muncul dan mengambil langkah-langkah untuk meminimalkan dampaknya. Ini dapat melibatkan penggunaan metode yang lebih akurat, mengatur parameter iterasi dengan bijaksana, dan menggunakan teknik numerik yang tepat untuk masalah yang dihadapi. Selain itu, memahami tingkat akurasi yang diperlukan dalam konteks aplikasi juga penting untuk memutuskan apakah hasil numerik sudah cukup akurat atau tidak.

\hypertarget{studi-kasus-error-numerik}{%
\section{Studi Kasus Error Numerik}\label{studi-kasus-error-numerik}}

\hypertarget{ilustrasi-jenis-error-numerik}{%
\subsection{Ilustrasi Jenis error numerik}\label{ilustrasi-jenis-error-numerik}}

Berikut adalah contoh kasus sederhana yang mengilustrasikan beberapa jenis error numerik dalam bahasa pemrograman R:

\begin{Shaded}
\begin{Highlighting}[]
\CommentTok{\# Contoh kasus: Menghitung nilai akar kuadrat dari 2}
\NormalTok{nilai\_eksak }\OtherTok{\textless{}{-}} \FunctionTok{sqrt}\NormalTok{(}\DecValTok{2}\NormalTok{)  }\CommentTok{\# Nilai akar kuadrat dari 2 sebenarnya}

\CommentTok{\# Pendekatan numerik menggunakan deret Taylor (truncation error)}
\NormalTok{deret\_taylor }\OtherTok{\textless{}{-}} \ControlFlowTok{function}\NormalTok{(x, n) \{}
\NormalTok{  approx\_value }\OtherTok{\textless{}{-}} \DecValTok{1}
  \ControlFlowTok{for}\NormalTok{ (i }\ControlFlowTok{in} \DecValTok{1}\SpecialCharTok{:}\NormalTok{n) \{}
\NormalTok{    approx\_value }\OtherTok{\textless{}{-}}\NormalTok{ approx\_value }\SpecialCharTok{+}\NormalTok{ ((}\SpecialCharTok{{-}}\DecValTok{1}\NormalTok{)}\SpecialCharTok{\^{}}\NormalTok{i }\SpecialCharTok{*}\NormalTok{ (x }\SpecialCharTok{{-}} \DecValTok{1}\NormalTok{)}\SpecialCharTok{\^{}}\NormalTok{i) }\SpecialCharTok{/}\NormalTok{ (i }\SpecialCharTok{*} \FunctionTok{factorial}\NormalTok{(i))}
\NormalTok{  \}}
  \FunctionTok{return}\NormalTok{(approx\_value)}
\NormalTok{\}}

\CommentTok{\# Hitung nilai perkiraan akar kuadrat dari 2 dengan deret Taylor (misal, n = 5)}
\NormalTok{n\_terms }\OtherTok{\textless{}{-}} \DecValTok{5}
\NormalTok{nilai\_approx\_taylor }\OtherTok{\textless{}{-}} \FunctionTok{deret\_taylor}\NormalTok{(}\DecValTok{2}\NormalTok{, n\_terms)}

\CommentTok{\# Hitung absolute error}
\NormalTok{absolute\_error\_taylor }\OtherTok{\textless{}{-}} \FunctionTok{abs}\NormalTok{(nilai\_eksak }\SpecialCharTok{{-}}\NormalTok{ nilai\_approx\_taylor)}

\CommentTok{\# Hitung relative error}
\NormalTok{relative\_error\_taylor }\OtherTok{\textless{}{-}}\NormalTok{ absolute\_error\_taylor }\SpecialCharTok{/} \FunctionTok{abs}\NormalTok{(nilai\_eksak) }\SpecialCharTok{*} \DecValTok{100}

\CommentTok{\# Tampilkan hasil}
\FunctionTok{print}\NormalTok{(}\FunctionTok{paste}\NormalTok{(}\StringTok{"Nilai Akar Kuadrat dari 2 (Eksak):"}\NormalTok{, nilai\_eksak))}
\end{Highlighting}
\end{Shaded}

\begin{verbatim}
## [1] "Nilai Akar Kuadrat dari 2 (Eksak): 1.4142135623731"
\end{verbatim}

\begin{Shaded}
\begin{Highlighting}[]
\FunctionTok{print}\NormalTok{(}\FunctionTok{paste}\NormalTok{(}\StringTok{"Nilai Pendekatan Akar Kuadrat (Deret Taylor):"}\NormalTok{, nilai\_approx\_taylor))}
\end{Highlighting}
\end{Shaded}

\begin{verbatim}
## [1] "Nilai Pendekatan Akar Kuadrat (Deret Taylor): 0.203194444444444"
\end{verbatim}

\begin{Shaded}
\begin{Highlighting}[]
\FunctionTok{print}\NormalTok{(}\FunctionTok{paste}\NormalTok{(}\StringTok{"Absolute Error (Deret Taylor):"}\NormalTok{, absolute\_error\_taylor))}
\end{Highlighting}
\end{Shaded}

\begin{verbatim}
## [1] "Absolute Error (Deret Taylor): 1.21101911792865"
\end{verbatim}

\begin{Shaded}
\begin{Highlighting}[]
\FunctionTok{print}\NormalTok{(}\FunctionTok{paste}\NormalTok{(}\StringTok{"Relative Error (Deret Taylor):"}\NormalTok{, relative\_error\_taylor, }\StringTok{"\%"}\NormalTok{))}
\end{Highlighting}
\end{Shaded}

\begin{verbatim}
## [1] "Relative Error (Deret Taylor): 85.63198304339 %"
\end{verbatim}

\hypertarget{perhitungan-integral}{%
\subsection{Perhitungan Integral}\label{perhitungan-integral}}

Andaikan kita ingin melihat bagaimana error numerik dapat mempengaruhi perhitungan integral numerik dengan menggunakan metode kuadratur numerik. Kita akan mencoba menghitung integral dari fungsi \(f(x)=x^2\) di rentang \([0, 2]\) menggunakan metode kuadratur numerik (misalnya, metode trapesium) dan membandingkannya dengan nilai eksak integral.

\begin{Shaded}
\begin{Highlighting}[]
\CommentTok{\# Fungsi untuk menghitung integral numerik dengan metode trapesium}
\NormalTok{integral\_trapezoidal }\OtherTok{\textless{}{-}} \ControlFlowTok{function}\NormalTok{(f, a, b, n) \{}
\NormalTok{  h }\OtherTok{\textless{}{-}}\NormalTok{ (b }\SpecialCharTok{{-}}\NormalTok{ a) }\SpecialCharTok{/}\NormalTok{ n}
\NormalTok{  sum }\OtherTok{\textless{}{-}} \FloatTok{0.5} \SpecialCharTok{*}\NormalTok{ (}\FunctionTok{f}\NormalTok{(a) }\SpecialCharTok{+} \FunctionTok{f}\NormalTok{(b))}
  \ControlFlowTok{for}\NormalTok{ (i }\ControlFlowTok{in} \DecValTok{1}\SpecialCharTok{:}\NormalTok{(n }\SpecialCharTok{{-}} \DecValTok{1}\NormalTok{)) \{}
\NormalTok{    sum }\OtherTok{\textless{}{-}}\NormalTok{ sum }\SpecialCharTok{+} \FunctionTok{f}\NormalTok{(a }\SpecialCharTok{+}\NormalTok{ i }\SpecialCharTok{*}\NormalTok{ h)}
\NormalTok{  \}}
\NormalTok{  integral\_approx }\OtherTok{\textless{}{-}}\NormalTok{ h }\SpecialCharTok{*}\NormalTok{ sum}
  \FunctionTok{return}\NormalTok{(integral\_approx)}
\NormalTok{\}}

\CommentTok{\# Fungsi asli f(x) = x\^{}3}
\NormalTok{f }\OtherTok{\textless{}{-}} \ControlFlowTok{function}\NormalTok{(x) \{}
  \FunctionTok{return}\NormalTok{(x}\SpecialCharTok{\^{}}\DecValTok{3}\NormalTok{)}
\NormalTok{\}}

\CommentTok{\# Rentang integral [a, b]}
\NormalTok{a }\OtherTok{\textless{}{-}} \DecValTok{0}
\NormalTok{b }\OtherTok{\textless{}{-}} \DecValTok{2}

\CommentTok{\# Nilai eksak integral dari f(x) = x\^{}3}
\NormalTok{integral\_exact }\OtherTok{\textless{}{-}}\NormalTok{ (b}\SpecialCharTok{\^{}}\DecValTok{4}\NormalTok{) }\SpecialCharTok{/} \DecValTok{4}

\CommentTok{\# Hitung nilai perkiraan integral dengan metode trapesium (misal, n = 4)}
\NormalTok{n\_intervals }\OtherTok{\textless{}{-}} \DecValTok{4}
\NormalTok{integral\_approx }\OtherTok{\textless{}{-}} \FunctionTok{integral\_trapezoidal}\NormalTok{(f, a, b, n\_intervals)}

\CommentTok{\# Hitung absolute error}
\NormalTok{absolute\_error }\OtherTok{\textless{}{-}} \FunctionTok{abs}\NormalTok{(integral\_exact }\SpecialCharTok{{-}}\NormalTok{ integral\_approx)}

\CommentTok{\# Hitung relative error}
\NormalTok{relative\_error }\OtherTok{\textless{}{-}}\NormalTok{ absolute\_error }\SpecialCharTok{/} \FunctionTok{abs}\NormalTok{(integral\_exact) }\SpecialCharTok{*} \DecValTok{100}

\CommentTok{\# Tampilkan hasil}
\FunctionTok{print}\NormalTok{(}\FunctionTok{paste}\NormalTok{(}\StringTok{"Nilai Eksak Integral:"}\NormalTok{, integral\_exact))}
\end{Highlighting}
\end{Shaded}

\begin{verbatim}
## [1] "Nilai Eksak Integral: 4"
\end{verbatim}

\begin{Shaded}
\begin{Highlighting}[]
\FunctionTok{print}\NormalTok{(}\FunctionTok{paste}\NormalTok{(}\StringTok{"Nilai Perkiraan Integral (Metode Trapesium):"}\NormalTok{, integral\_approx))}
\end{Highlighting}
\end{Shaded}

\begin{verbatim}
## [1] "Nilai Perkiraan Integral (Metode Trapesium): 4.25"
\end{verbatim}

\begin{Shaded}
\begin{Highlighting}[]
\FunctionTok{print}\NormalTok{(}\FunctionTok{paste}\NormalTok{(}\StringTok{"Absolute Error (Metode Trapesium):"}\NormalTok{, absolute\_error))}
\end{Highlighting}
\end{Shaded}

\begin{verbatim}
## [1] "Absolute Error (Metode Trapesium): 0.25"
\end{verbatim}

\begin{Shaded}
\begin{Highlighting}[]
\FunctionTok{print}\NormalTok{(}\FunctionTok{paste}\NormalTok{(}\StringTok{"Relative Error (Metode Trapesium):"}\NormalTok{, relative\_error, }\StringTok{"\%"}\NormalTok{))}
\end{Highlighting}
\end{Shaded}

\begin{verbatim}
## [1] "Relative Error (Metode Trapesium): 6.25 %"
\end{verbatim}

Dalam contoh ini, kita mencoba menghitung integral dari fungsi \(f(x)=x^2\) dengan rentang di rentang \([0, 2]\) menggunakan metode trapesium. Kita membandingkan hasil perkiraan integral dengan nilai eksak integral \({b^4 \over 4}\). Semakin banyak interval (nilai \(n\)) yang digunakan, semakin mendekati hasil perkiraan dengan nilai eksak, tetapi juga menghasilkan error numerik. Anda dapat mengubah nilai \(n\) untuk melihat bagaimana error numerik berubah dengan peningkatan jumlah interval.

Mari lihat contoh kedua, gunakan fungsi sederhana \(y = x^2\) di rentang \([0, 1]\)

\begin{Shaded}
\begin{Highlighting}[]
\CommentTok{\# Fungsi yang ingin diintegrasikan: y = x\^{}2}
\NormalTok{f }\OtherTok{\textless{}{-}} \ControlFlowTok{function}\NormalTok{(x) \{}
  \FunctionTok{return}\NormalTok{(x}\SpecialCharTok{\^{}}\DecValTok{2}\NormalTok{)}
\NormalTok{\}}

\CommentTok{\# Batas integrasi}
\NormalTok{a }\OtherTok{\textless{}{-}} \DecValTok{0}  \CommentTok{\# Batas bawah}
\NormalTok{b }\OtherTok{\textless{}{-}} \DecValTok{1}  \CommentTok{\# Batas atas}

\CommentTok{\# Jumlah partisi (subinterval)}
\NormalTok{n\_partitions }\OtherTok{\textless{}{-}} \DecValTok{1000}

\CommentTok{\# Lebar setiap subinterval}
\NormalTok{dx }\OtherTok{\textless{}{-}}\NormalTok{ (b }\SpecialCharTok{{-}}\NormalTok{ a) }\SpecialCharTok{/}\NormalTok{ n\_partitions}

\CommentTok{\# Hitung integral numerik dengan metode kuadratur numerik (Metode trapesium)}
\NormalTok{numerical\_integral }\OtherTok{\textless{}{-}} \DecValTok{0}
\ControlFlowTok{for}\NormalTok{ (i }\ControlFlowTok{in} \DecValTok{1}\SpecialCharTok{:}\NormalTok{n\_partitions) \{}
\NormalTok{  x\_left }\OtherTok{\textless{}{-}}\NormalTok{ a }\SpecialCharTok{+}\NormalTok{ (i }\SpecialCharTok{{-}} \DecValTok{1}\NormalTok{) }\SpecialCharTok{*}\NormalTok{ dx}
\NormalTok{  x\_right }\OtherTok{\textless{}{-}}\NormalTok{ a }\SpecialCharTok{+}\NormalTok{ i }\SpecialCharTok{*}\NormalTok{ dx}
\NormalTok{  numerical\_integral }\OtherTok{\textless{}{-}}\NormalTok{ numerical\_integral }\SpecialCharTok{+}\NormalTok{ (}\FunctionTok{f}\NormalTok{(x\_left) }\SpecialCharTok{+} \FunctionTok{f}\NormalTok{(x\_right)) }\SpecialCharTok{*}\NormalTok{ dx }\SpecialCharTok{/} \DecValTok{2}
\NormalTok{\}}

\CommentTok{\# Integral sebenarnya (analitis)}
\NormalTok{actual\_integral }\OtherTok{\textless{}{-}}\NormalTok{ b}\SpecialCharTok{\^{}}\DecValTok{3} \SpecialCharTok{/} \DecValTok{3} \SpecialCharTok{{-}}\NormalTok{ a}\SpecialCharTok{\^{}}\DecValTok{3} \SpecialCharTok{/} \DecValTok{3}

\CommentTok{\# Hitung absolute error}
\NormalTok{absolute\_error }\OtherTok{\textless{}{-}} \FunctionTok{abs}\NormalTok{(actual\_integral }\SpecialCharTok{{-}}\NormalTok{ numerical\_integral)}

\CommentTok{\# Hitung relative error}
\NormalTok{relative\_error }\OtherTok{\textless{}{-}}\NormalTok{ absolute\_error }\SpecialCharTok{/} \FunctionTok{abs}\NormalTok{(actual\_integral) }\SpecialCharTok{*} \DecValTok{100}

\CommentTok{\# Tampilkan hasil}
\FunctionTok{print}\NormalTok{(}\FunctionTok{paste}\NormalTok{(}\StringTok{"Integral Numerik (Metode Trapesium):"}\NormalTok{, numerical\_integral))}
\end{Highlighting}
\end{Shaded}

\begin{verbatim}
## [1] "Integral Numerik (Metode Trapesium): 0.3333335"
\end{verbatim}

\begin{Shaded}
\begin{Highlighting}[]
\FunctionTok{print}\NormalTok{(}\FunctionTok{paste}\NormalTok{(}\StringTok{"Integral Analitis:"}\NormalTok{, actual\_integral))}
\end{Highlighting}
\end{Shaded}

\begin{verbatim}
## [1] "Integral Analitis: 0.333333333333333"
\end{verbatim}

\begin{Shaded}
\begin{Highlighting}[]
\FunctionTok{print}\NormalTok{(}\FunctionTok{paste}\NormalTok{(}\StringTok{"Absolute Error:"}\NormalTok{, absolute\_error))}
\end{Highlighting}
\end{Shaded}

\begin{verbatim}
## [1] "Absolute Error: 1.6666666652343e-07"
\end{verbatim}

\begin{Shaded}
\begin{Highlighting}[]
\FunctionTok{print}\NormalTok{(}\FunctionTok{paste}\NormalTok{(}\StringTok{"Relative Error:"}\NormalTok{, relative\_error, }\StringTok{"\%"}\NormalTok{))}
\end{Highlighting}
\end{Shaded}

\begin{verbatim}
## [1] "Relative Error: 4.99999999570289e-05 %"
\end{verbatim}

\hypertarget{terapan-error-numerik}{%
\section{Terapan Error Numerik}\label{terapan-error-numerik}}

Pada bagian ini diperlihatkan contoh penerapan metode error numerik dalam kehidupan sehari-hari.

\hypertarget{studi-kasus-keuangan}{%
\subsection{Studi Kasus Keuangan}\label{studi-kasus-keuangan}}

Salah satu contoh studi kasus error numerik dalam keuangan adalah ketika melakukan perhitungan pada instrumen keuangan yang memiliki tingkat bunga yang sangat rendah atau sangat tinggi. Perhitungan yang melibatkan tingkat bunga rendah atau tinggi dapat menyebabkan error numerik karena keterbatasan representasi angka di komputer.

Misalkan kita ingin menghitung nilai masa depan (future value) dari suatu investasi dengan tingkat bunga yang sangat rendah atau sangat tinggi. Kita akan menggunakan rumus future value sebagai berikut:

\[FV=PV×(1+r)^n\]

Di mana:

\begin{itemize}
\tightlist
\item
  \(FV\) adalah nilai masa depan investasi.
\item
  \(PV\) adalah nilai saat ini investasi.
\item
  \(r\) adalah tingkat bunga (dalam bentuk desimal).
\item
  \(n\) adalah jumlah periode.
\end{itemize}

Ketika \(r\) sangat kecil (misalnya, mendekati 0), atau \(r\) sangat besar, perhitungan nilai \((1+r)^n\) bisa menjadi masalah. Tingkat bunga yang sangat rendah atau tinggi dapat menghasilkan nilai \((1+r)^n\) yang sangat dekat dengan 1 atau nilai yang sangat besar, yang dapat menyebabkan error pembulatan atau overflow dalam komputasi.

Contoh implementasi dalam R:

\begin{Shaded}
\begin{Highlighting}[]
\CommentTok{\# Contoh perhitungan Future Value dengan tingkat bunga rendah dan tinggi}
\NormalTok{PV }\OtherTok{\textless{}{-}} \DecValTok{1000}  \CommentTok{\# Nilai saat ini investasi}
\NormalTok{n }\OtherTok{\textless{}{-}} \DecValTok{10}     \CommentTok{\# Jumlah periode}

\CommentTok{\# Tingkat bunga rendah dan tinggi}
\NormalTok{r\_low }\OtherTok{\textless{}{-}} \FloatTok{0.0001}
\NormalTok{r\_high }\OtherTok{\textless{}{-}} \DecValTok{100}

\CommentTok{\# Hitung Future Value}
\NormalTok{FV\_low }\OtherTok{\textless{}{-}}\NormalTok{ PV }\SpecialCharTok{*}\NormalTok{ (}\DecValTok{1} \SpecialCharTok{+}\NormalTok{ r\_low)}\SpecialCharTok{\^{}}\NormalTok{n}
\NormalTok{FV\_high }\OtherTok{\textless{}{-}}\NormalTok{ PV }\SpecialCharTok{*}\NormalTok{ (}\DecValTok{1} \SpecialCharTok{+}\NormalTok{ r\_high)}\SpecialCharTok{\^{}}\NormalTok{n}

\CommentTok{\# Tampilkan hasil}
\FunctionTok{print}\NormalTok{(}\FunctionTok{paste}\NormalTok{(}\StringTok{"Future Value (Low Interest Rate):"}\NormalTok{, FV\_low))}
\end{Highlighting}
\end{Shaded}

\begin{verbatim}
## [1] "Future Value (Low Interest Rate): 1001.00045012002"
\end{verbatim}

\begin{Shaded}
\begin{Highlighting}[]
\FunctionTok{print}\NormalTok{(}\FunctionTok{paste}\NormalTok{(}\StringTok{"Future Value (High Interest Rate):"}\NormalTok{, FV\_high))}
\end{Highlighting}
\end{Shaded}

\begin{verbatim}
## [1] "Future Value (High Interest Rate): 1.1046221254112e+23"
\end{verbatim}

Dalam kasus ini, perhatikan bagaimana tingkat bunga rendah (r\_low) menghasilkan hasil yang sangat mendekati nilai awal investasi. Di sisi lain, tingkat bunga yang sangat tinggi (r\_high) menghasilkan nilai yang sangat besar dan mungkin menyebabkan overflow atau masalah numerik lainnya. Studi kasus ini menunjukkan pentingnya memperhatikan keterbatasan representasi angka di komputer saat melakukan perhitungan keuangan, terutama dengan angka yang sangat kecil atau sangat besar.

\hypertarget{jumlah-sampel}{%
\subsection{Jumlah Sampel}\label{jumlah-sampel}}

Mari kita lihat contoh studi kasus yang melibatkan error numerik dalam jumlah sampel dalam bahasa R. Kita akan menggunakan pendekatan Monte Carlo untuk memperkirakan nilai \(π\) (pi).

Pendekatan Monte Carlo adalah salah satu metode numerik yang mengandalkan pembangkitan angka acak untuk memperkirakan solusi numerik. Dalam hal ini, kita akan menggunakan pendekatan ini untuk memperkirakan nilai π dengan menghasilkan titik-titik acak di dalam sebuah lingkaran dan menghitung berapa banyak di antaranya jatuh di dalam lingkaran.

\begin{Shaded}
\begin{Highlighting}[]
\CommentTok{\# Jumlah total titik yang akan digunakan}
\NormalTok{total\_points }\OtherTok{\textless{}{-}} \DecValTok{1000000}

\CommentTok{\# Fungsi untuk menghitung estimasi nilai π}
\NormalTok{monte\_carlo\_pi }\OtherTok{\textless{}{-}} \ControlFlowTok{function}\NormalTok{(total\_points) \{}
\NormalTok{  points\_inside\_circle }\OtherTok{\textless{}{-}} \DecValTok{0}
  \ControlFlowTok{for}\NormalTok{ (i }\ControlFlowTok{in} \DecValTok{1}\SpecialCharTok{:}\NormalTok{total\_points) \{}
\NormalTok{    x }\OtherTok{\textless{}{-}} \FunctionTok{runif}\NormalTok{(}\DecValTok{1}\NormalTok{, }\AttributeTok{min =} \SpecialCharTok{{-}}\DecValTok{1}\NormalTok{, }\AttributeTok{max =} \DecValTok{1}\NormalTok{)}
\NormalTok{    y }\OtherTok{\textless{}{-}} \FunctionTok{runif}\NormalTok{(}\DecValTok{1}\NormalTok{, }\AttributeTok{min =} \SpecialCharTok{{-}}\DecValTok{1}\NormalTok{, }\AttributeTok{max =} \DecValTok{1}\NormalTok{)}
    \ControlFlowTok{if}\NormalTok{ (x}\SpecialCharTok{\^{}}\DecValTok{2} \SpecialCharTok{+}\NormalTok{ y}\SpecialCharTok{\^{}}\DecValTok{2} \SpecialCharTok{\textless{}=} \DecValTok{1}\NormalTok{) \{}
\NormalTok{      points\_inside\_circle }\OtherTok{\textless{}{-}}\NormalTok{ points\_inside\_circle }\SpecialCharTok{+} \DecValTok{1}
\NormalTok{    \}}
\NormalTok{  \}}
\NormalTok{  estimated\_pi }\OtherTok{\textless{}{-}} \DecValTok{4} \SpecialCharTok{*}\NormalTok{ points\_inside\_circle }\SpecialCharTok{/}\NormalTok{ total\_points}
  \FunctionTok{return}\NormalTok{(estimated\_pi)}
\NormalTok{\}}

\CommentTok{\# Hitung estimasi nilai π}
\NormalTok{estimated\_pi }\OtherTok{\textless{}{-}} \FunctionTok{monte\_carlo\_pi}\NormalTok{(total\_points)}

\CommentTok{\# Tampilkan hasil}
\FunctionTok{print}\NormalTok{(}\FunctionTok{paste}\NormalTok{(}\StringTok{"Estimasi Nilai π dengan"}\NormalTok{, total\_points, }\StringTok{"titik:"}\NormalTok{, estimated\_pi))}
\end{Highlighting}
\end{Shaded}

\begin{verbatim}
## [1] "Estimasi Nilai π dengan 1e+06 titik: 3.141972"
\end{verbatim}

\begin{Shaded}
\begin{Highlighting}[]
\FunctionTok{print}\NormalTok{(}\FunctionTok{paste}\NormalTok{(}\StringTok{"Error absolut:"}\NormalTok{, }\FunctionTok{abs}\NormalTok{(estimated\_pi }\SpecialCharTok{{-}}\NormalTok{ pi)))}
\end{Highlighting}
\end{Shaded}

\begin{verbatim}
## [1] "Error absolut: 0.000379346410206871"
\end{verbatim}

\begin{Shaded}
\begin{Highlighting}[]
\FunctionTok{print}\NormalTok{(}\FunctionTok{paste}\NormalTok{(}\StringTok{"Error relatif:"}\NormalTok{, }\FunctionTok{abs}\NormalTok{(estimated\_pi }\SpecialCharTok{{-}}\NormalTok{ pi) }\SpecialCharTok{/}\NormalTok{ pi }\SpecialCharTok{*} \DecValTok{100}\NormalTok{, }\StringTok{"\%"}\NormalTok{))}
\end{Highlighting}
\end{Shaded}

\begin{verbatim}
## [1] "Error relatif: 0.0120749712657179 %"
\end{verbatim}

Dalam contoh ini, kita menggunakan pendekatan Monte Carlo untuk memperkirakan nilai \(π\) dengan menghasilkan titik-titik acak di dalam lingkaran dengan radius 1 dan menghitung berapa banyak dari titik-titik ini jatuh di dalam lingkaran tersebut. Semakin banyak titik yang kita gunakan, semakin akurat perkiraan kita. Anda dapat mengubah nilai \texttt{total\_points} untuk melihat bagaimana error numerik berubah dengan jumlah sampel yang berbeda.

Perhatikan bahwa dalam kasus ini, error numerik terkait dengan pendekatan Monte Carlo dan jumlah sampel yang digunakan. Semakin banyak sampel yang digunakan, semakin mendekati perkiraan kita dengan nilai \(π\) yang sebenarnya.

\hypertarget{jumlah-iterasi}{%
\subsection{Jumlah Iterasi}\label{jumlah-iterasi}}

Mari kita lihat contoh studi kasus yang menggambarkan bagaimana error numerik dapat berkembang seiring dengan jumlah iterasi dalam sebuah metode numerik. Kita akan menggunakan metode iteratif untuk menghitung akar kuadrat dari 2 dan melihat bagaimana error berkembang seiring dengan meningkatnya jumlah iterasi.

\begin{Shaded}
\begin{Highlighting}[]
\CommentTok{\# Nilai eksak akar kuadrat dari 2}
\NormalTok{nilai\_eksak }\OtherTok{\textless{}{-}} \FunctionTok{sqrt}\NormalTok{(}\DecValTok{2}\NormalTok{)}

\CommentTok{\# Metode iteratif untuk menghitung akar kuadrat}
\NormalTok{iterative\_method }\OtherTok{\textless{}{-}} \ControlFlowTok{function}\NormalTok{(x, n) \{}
\NormalTok{  approx\_value }\OtherTok{\textless{}{-}}\NormalTok{ x}
  \ControlFlowTok{for}\NormalTok{ (i }\ControlFlowTok{in} \DecValTok{1}\SpecialCharTok{:}\NormalTok{n) \{}
\NormalTok{    approx\_value }\OtherTok{\textless{}{-}} \FloatTok{0.5} \SpecialCharTok{*}\NormalTok{ (approx\_value }\SpecialCharTok{+}\NormalTok{ x }\SpecialCharTok{/}\NormalTok{ approx\_value)}
\NormalTok{  \}}
  \FunctionTok{return}\NormalTok{(approx\_value)}
\NormalTok{\}}

\CommentTok{\# Jumlah iterasi yang akan diuji}
\NormalTok{jumlah\_iterasi }\OtherTok{\textless{}{-}} \FunctionTok{c}\NormalTok{(}\DecValTok{1}\NormalTok{, }\DecValTok{2}\NormalTok{, }\DecValTok{5}\NormalTok{, }\DecValTok{10}\NormalTok{, }\DecValTok{20}\NormalTok{, }\DecValTok{50}\NormalTok{, }\DecValTok{100}\NormalTok{, }\DecValTok{200}\NormalTok{)}

\CommentTok{\# Hitung nilai perkiraan akar kuadrat dari 2 dengan jumlah iterasi yang berbeda}
\NormalTok{nilai\_approximations }\OtherTok{\textless{}{-}} \FunctionTok{sapply}\NormalTok{(jumlah\_iterasi, }\ControlFlowTok{function}\NormalTok{(iter) }\FunctionTok{iterative\_method}\NormalTok{(}\DecValTok{2}\NormalTok{, iter))}

\CommentTok{\# Hitung absolute error untuk setiap iterasi}
\NormalTok{absolute\_errors }\OtherTok{\textless{}{-}} \FunctionTok{abs}\NormalTok{(nilai\_eksak }\SpecialCharTok{{-}}\NormalTok{ nilai\_approximations)}

\CommentTok{\# Tampilkan hasil}
\NormalTok{results }\OtherTok{\textless{}{-}} \FunctionTok{data.frame}\NormalTok{(}\AttributeTok{Jumlah\_Iterasi =}\NormalTok{ jumlah\_iterasi, }\AttributeTok{Perkiraan\_Akar =}\NormalTok{ nilai\_approximations, }\AttributeTok{Absolute\_Error =}\NormalTok{ absolute\_errors)}
\FunctionTok{print}\NormalTok{(results)}
\end{Highlighting}
\end{Shaded}

\begin{verbatim}
##   Jumlah_Iterasi Perkiraan_Akar Absolute_Error
## 1              1          1.500      8.579e-02
## 2              2          1.417      2.453e-03
## 3              5          1.414      2.220e-16
## 4             10          1.414      2.220e-16
## 5             20          1.414      2.220e-16
## 6             50          1.414      2.220e-16
## 7            100          1.414      2.220e-16
## 8            200          1.414      2.220e-16
\end{verbatim}

Dalam contoh ini, kita menggunakan metode Newton-Raphson iteratif untuk menghitung akar kuadrat dari 2. Kami menghitung perkiraan akar kuadrat untuk berbagai jumlah iterasi yang berbeda dan mengukur absolute error antara nilai perkiraan dan nilai eksak. Dengan melihat hasilnya, Anda dapat melihat bagaimana error numerik berkurang seiring dengan peningkatan jumlah iterasi. Semakin banyak iterasi, semakin mendekati hasil numerik dengan nilai eksak.

Anda dapat memodifikasi jumlah\_iterasi sesuai dengan preferensi Anda untuk melihat bagaimana error numerik berkembang dengan lebih banyak iterasi. Ingatlah bahwa, meskipun error numerik berkurang seiring dengan peningkatan jumlah iterasi, ada batasan akurasi yang dicapai karena keterbatasan representasi angka di komputer.

\hypertarget{referensi}{%
\section{Referensi}\label{referensi}}

\begin{enumerate}
\def\labelenumi{\arabic{enumi}.}
\tightlist
\item
  Howard, J.P. 2017. \textbf{Computational Methods for Numerical Analysis with R}. CRC Press.
\item
  Sutarno,H., Rachmatin,D. 2008. \textbf{Hands Out Metode Numerik}. Universitas Pendidikan Indonesia.
\end{enumerate}

\hypertarget{linearaljabar}{%
\chapter{Aljabar Linier}\label{linearaljabar}}

Pada \emph{chapter} ini penulis akan menjelaskan mengenai cara untuk menyelesaikan sistem persamaan linier. Adapun yang akan dibahas pada \emph{chapter} ini antara lain:

\begin{itemize}
\tightlist
\item
  operasi Vektor dan matriks
\item
  Metode Eliminasi Gauss
\item
  Metode Dekomposisi matriks
\item
  Studi Kasus
\end{itemize}

\hypertarget{vecmat}{%
\section{Vektor dan matriks}\label{vecmat}}

Pada Chapter \ref{vector} dan Chapter \ref{matriks} telah dijelaskan sekilas bagaimana cara melakukan operasi pada vektor dan matriks. Pada \emph{chapter} ini, penulis akan menambahkan operasi-operasi lain yang dapat dilakukan pada vektor dan matriks. Dasar-dasar operasi ini selanjutnya akan digunakan sebagai dasar menyusun algoritma penyelesaian sistem persamaan linier.

\hypertarget{operasivektor}{%
\subsection{Operasi Vektor}\label{operasivektor}}

Misalkan saja diberikan vektor \(u\) dan \(v\) yang ditunjukkan pada Persamaan \eqref{eq:vectoruv}.

\begin{equation}
u = \begin{bmatrix}
      u_1            \\[0.3em]
      u_2            \\[0.3em]
      \vdots         \\[0.3em] 
      u_n
     \end{bmatrix}
dan\ v\ = \begin{bmatrix}
      v_1            \\[0.3em]
      v_2            \\[0.3em]
      \vdots         \\[0.3em] 
      v_n
     \end{bmatrix}
  \label{eq:vectoruv}
\end{equation}

Jika kita menambahkan atau mengurangkan nilai elemen vektor dengan suatu skalar (konstanta yang hanya memiliki besaran), maka operasi penjumlahan/pengurangan akan dilakukan pada setiap elemen vektor.

\begin{equation}
u \pm x = \begin{bmatrix}
      u_1            \\[0.3em]
      u_2            \\[0.3em]
      \vdots         \\[0.3em] 
      u_n
     \end{bmatrix}
\pm x = \begin{bmatrix}
      u_1 \pm x            \\[0.3em]
      u_2 \pm x           \\[0.3em]
      \vdots         \\[0.3em] 
      u_n \pm x
     \end{bmatrix}
     \label{eq:addvector}
\end{equation}

Jika kita melakukan penjumlahan pada vektor \(u\) dan \(v\), maka operasi akan terjadi pada masing-masing elemen dengan indeks yang sama.

\begin{equation}
u \pm v = \begin{bmatrix}
      u_1            \\[0.3em]
      u_2            \\[0.3em]
      \vdots         \\[0.3em] 
      u_n
     \end{bmatrix}
\pm v\ = \begin{bmatrix}
      v_1            \\[0.3em]
      v_2            \\[0.3em]
      \vdots         \\[0.3em] 
      v_n
     \end{bmatrix}
= \begin{bmatrix}
      u_1 \pm v_1            \\[0.3em]
      u_2 \pm v_2           \\[0.3em]
      \vdots         \\[0.3em] 
      u_n \pm v_n
     \end{bmatrix}
     \label{eq:addvector2}
\end{equation}

Untuk lebih memahami operasi tersebut, berikut penulis berikan contoh penerapannya pada \texttt{R}:

\begin{Shaded}
\begin{Highlighting}[]
\NormalTok{u }\OtherTok{\textless{}{-}} \FunctionTok{seq}\NormalTok{(}\DecValTok{1}\NormalTok{,}\DecValTok{5}\NormalTok{)}
\NormalTok{v }\OtherTok{\textless{}{-}} \FunctionTok{seq}\NormalTok{(}\DecValTok{6}\NormalTok{,}\DecValTok{10}\NormalTok{)}

\CommentTok{\# penjumlahan}
\NormalTok{u}\SpecialCharTok{+}\NormalTok{v}
\end{Highlighting}
\end{Shaded}

\begin{verbatim}
## [1]  7  9 11 13 15
\end{verbatim}

\begin{Shaded}
\begin{Highlighting}[]
\CommentTok{\# penguranga}
\NormalTok{u}\SpecialCharTok{{-}}\NormalTok{v}
\end{Highlighting}
\end{Shaded}

\begin{verbatim}
## [1] -5 -5 -5 -5 -5
\end{verbatim}

Bagaimana jika kita melakukan operasi dua vektor, dimaana salah satu vektor memiliki penjang yang berbeda?. Untuk memnjawab hal tersebut, perhatikan sintaks berikut:

\begin{Shaded}
\begin{Highlighting}[]
\NormalTok{x }\OtherTok{\textless{}{-}} \FunctionTok{seq}\NormalTok{(}\DecValTok{1}\NormalTok{,}\DecValTok{2}\NormalTok{)}
\NormalTok{u}\SpecialCharTok{+}\NormalTok{x}
\end{Highlighting}
\end{Shaded}

\begin{verbatim}
## Warning in u + x: longer object length is not a
## multiple of shorter object length
\end{verbatim}

\begin{verbatim}
## [1] 2 4 4 6 6
\end{verbatim}

Berdasarkan contoh tersebut, \texttt{R} akan mengeluarkan peringatan yang menunjukkan operasi dilakukan pada vektor dengan panjang berbeda. \texttt{R} akan tetap melakukan perhitungan dengan menjumlahkan kembali vektor \(u\) yang belum dijumlahkan dengan vektor \(x\) sampai seluruh elemen vektor \(u\) dilakukan operasi penjumlahan.

Operasi lain yang dapat dilakukan pada vektor adalah menghitung \emph{inner product} dan panjang vektor. Inner product dihitung menggunakan Persamaan \eqref{eq:innerproduct}.

\begin{equation}
u.v=\sum_{i=1}^nu_1v_1+u_2v_2+\dots+u_nv_n
  \label{eq:innerproduct}
\end{equation}

Panjang vektor atau vektor yang telah dinormalisasi dihitung menggunakan Persamaan \eqref{eq:panjangvektor}

\begin{equation}
\left|u\right|=\sqrt{u_1^2+u_2^2+\dots+u_n^2}
  \label{eq:panjangvektor}
\end{equation}

Berikut adalah contoh bagaimana cara menghitung \texttt{inner\ product} dan panjang vektor menggunakan \texttt{R}:

\begin{Shaded}
\begin{Highlighting}[]
\CommentTok{\# inner product}
\NormalTok{u}\SpecialCharTok{\%*\%}\NormalTok{v}
\end{Highlighting}
\end{Shaded}

\begin{verbatim}
##      [,1]
## [1,]  130
\end{verbatim}

\begin{Shaded}
\begin{Highlighting}[]
\CommentTok{\# panjang vektor u}
\FunctionTok{sqrt}\NormalTok{(}\FunctionTok{sum}\NormalTok{(u}\SpecialCharTok{*}\NormalTok{u))}
\end{Highlighting}
\end{Shaded}

\begin{verbatim}
## [1] 7.416
\end{verbatim}

\hypertarget{operasimatrik}{%
\subsection{Operasi matriks}\label{operasimatrik}}

Misalkan kita memiliki 2 buah matriks \(A\) dan \(B\).

\begin{equation}
A = \begin{bmatrix}
       a_{1.1} & a_{1.2} &\cdots& a_{1.n}           \\[0.3em]
       a_{2.1} & a_{2.2} &\cdots& a_{2.n}           \\[0.3em]
       \vdots  & \vdots  &\ddots& \vdots            \\[0.3em]
       a_{m.1} & a_{m.2} &\cdots& a_{m.n}
     \end{bmatrix}
dan\ B = \begin{bmatrix}
      b_{1.1} & b_{1.2} &\cdots& b_{1.n}           \\[0.3em]
      b_{2.1} & b_{2.2} &\cdots& b_{2.n}           \\[0.3em]
      \vdots  & \vdots  &\ddots& \vdots            \\[0.3em]
      b_{m.1} & b_{m.2} &\cdots& b_{m.n}
     \end{bmatrix}
  \label{eq:matrikuv}
\end{equation}

Jika salah satu matriks tersebut dijumlahkan atau dikurangkan dengan skalar.

\begin{equation}
A \pm x = \begin{bmatrix}
       a_{1.1} & a_{1.2} &\cdots& a_{1.n}           \\[0.3em]
       a_{2.1} & a_{2.2} &\cdots& a_{2.n}           \\[0.3em]
       \vdots  & \vdots  &\ddots& \vdots            \\[0.3em]
       a_{m.1} & a_{m.2} &\cdots& a_{m.n}
     \end{bmatrix}
\pm x = \begin{bmatrix}
      a_{1.1}\pm x & a_{1.2}\pm x &\cdots& a_{1.n}\pm x           \\[0.3em]
      a_{2.1}\pm x & a_{2.2}\pm x &\cdots& a_{2.n}\pm x           \\[0.3em]
      \vdots  & \vdots  &\ddots& \vdots            \\[0.3em]
      a_{m.1}\pm x & a_{m.2}\pm x &\cdots& a_{m.n}\pm x
     \end{bmatrix}
  \label{eq:addmatrik}
\end{equation}

Jika kedua matriks \(A\) dan \(B\) saling dijumlahkan atau dikurangkan. Perlu diperhatikan bahwa penjumlahan dua buah matriks hanya dapat dilakukan pada matriks dengan ukuran yang seragam.

\begin{equation}
\begin{split}
A \pm B & = \begin{bmatrix}
       a_{1.1} & a_{1.2} &\cdots& a_{1.n}           \\[0.3em]
       a_{2.1} & a_{2.2} &\cdots& a_{2.n}           \\[0.3em]
       \vdots  & \vdots  &\ddots& \vdots            \\[0.3em]
       a_{m.1} & a_{m.2} &\cdots& a_{m.n}
     \end{bmatrix}
\pm \begin{bmatrix}
      b_{1.1} & b_{1.2} &\cdots& b_{1.n}           \\[0.3em]
      b_{2.1} & b_{2.2} &\cdots& b_{2.n}           \\[0.3em]
      \vdots  & \vdots  &\ddots& \vdots            \\[0.3em]
      b_{m.1} & b_{m.2} &\cdots& b_{m.n}
     \end{bmatrix} \\
& = \begin{bmatrix}
       a_{1.1}\pm b_{1.1} & a_{1.2}\pm b_{1.2} &\cdots& a_{1.n}\pm b_{1.n}           \\[0.3em]
       a_{2.1}\pm b_{2.1} & a_{2.2}\pm b_{2.2} &\cdots& a_{2.n}\pm b_{2.n}           \\[0.3em]
       \vdots  & \vdots  &\ddots& \vdots            \\[0.3em]
       a_{m.1}\pm b_{m.1} & a_{m.2}\pm b_{m.2} &\cdots& a_{m.n}\pm b_{m.n}
     \end{bmatrix}
\end{split}
  \label{eq:addmatrik2}
\end{equation}

Untuk lebih memahaminya, berikut disajikan contoh operasi penjumlahan pada matriks:

\begin{Shaded}
\begin{Highlighting}[]
\NormalTok{A }\OtherTok{\textless{}{-}} \FunctionTok{matrix}\NormalTok{(}\DecValTok{1}\SpecialCharTok{:}\DecValTok{9}\NormalTok{,}\DecValTok{3}\NormalTok{)}
\NormalTok{B }\OtherTok{\textless{}{-}} \FunctionTok{matrix}\NormalTok{(}\DecValTok{10}\SpecialCharTok{:}\DecValTok{18}\NormalTok{,}\DecValTok{3}\NormalTok{)}
\NormalTok{C }\OtherTok{\textless{}{-}} \FunctionTok{matrix}\NormalTok{(}\DecValTok{1}\SpecialCharTok{:}\DecValTok{6}\NormalTok{,}\DecValTok{3}\NormalTok{)}

\CommentTok{\# penjumlahan dengan skalar}
\NormalTok{A}\SpecialCharTok{+}\DecValTok{1}
\end{Highlighting}
\end{Shaded}

\begin{verbatim}
##      [,1] [,2] [,3]
## [1,]    2    5    8
## [2,]    3    6    9
## [3,]    4    7   10
\end{verbatim}

\begin{Shaded}
\begin{Highlighting}[]
\CommentTok{\# penjumlahan A+B}
\NormalTok{A}\SpecialCharTok{+}\NormalTok{B}
\end{Highlighting}
\end{Shaded}

\begin{verbatim}
##      [,1] [,2] [,3]
## [1,]   11   17   23
## [2,]   13   19   25
## [3,]   15   21   27
\end{verbatim}

\begin{Shaded}
\begin{Highlighting}[]
\CommentTok{\# penjumlahan}
\NormalTok{A}\SpecialCharTok{+}\NormalTok{C}
\end{Highlighting}
\end{Shaded}

Operasi pehitungan lain yang penting pada matriks adalah operasi perkalian matriks. Perlu diperhatikan bahwa untuk perkalian matriks, jumlah kolom matriks sebelah kiri harus sama dengan jumlah baris pada matriks sebelah kanan. Perkalian antara dua matriks disajikan pada Persamaan \eqref{eq:kalimatrik}.

\begin{equation}
A_{m.n}\times B_{n.r}=AB_{m.r}
  \label{eq:kalimatrik}
\end{equation}

Pada \texttt{R} perkalian matriks dilakukan menggunakan operator \texttt{\%*\%}. Berikut adalah contoh perkalian matriks pada \texttt{R}:

\begin{Shaded}
\begin{Highlighting}[]
\CommentTok{\# Perkalian matriks}
\NormalTok{A}\SpecialCharTok{\%*\%}\NormalTok{B}
\end{Highlighting}
\end{Shaded}

\begin{verbatim}
##      [,1] [,2] [,3]
## [1,]  138  174  210
## [2,]  171  216  261
## [3,]  204  258  312
\end{verbatim}

\hypertarget{rowoperation}{%
\section{Operasi Baris Elementer}\label{rowoperation}}

Terdapat tiga buah operasi dasar pada baris matriksoperasi baris elementer. Ketiga operasi ini akan menjadi dasar operasi \emph{sub-chapter} selanjutnya. Ketiga operasi dasar tersebut antara lain:

\begin{enumerate}
\def\labelenumi{\arabic{enumi}.}
\tightlist
\item
  \textbf{\emph{Row Scalling}}. Mengalikan baris matriks dengan konstanta bukan nol.
\item
  \textbf{\emph{Row Swaping}}. Menukar urutan baris pada sebuah matriks (contoh: menukar baris 1 dengan baris 2 dan sebaliknya).
\item
  \textbf{\emph{Row Replacement}}. Baris matriks diganti dengan hasil penjumlahan atau pengurangan baris matriks tersebut dengan baris matriks lainnya, dimana baris matriks lainnya yang akan dijumlahkan/dikurangkan dengan matriks tersebut telah dilakukan proses \emph{row scalling}. Luaran yang diperoleh pada umumnya adalah nilai nol pada baris matriks awal atau akhir.
\end{enumerate}

Ketiga proses tersebut akan terjadi secara berulang, khusunya jika kita hendak mengerjakan sistem persamaan linier menggunakan algoritma eliminasi Gauss. Untuk mempermudah proses tersebut, kita dapat membuat masing-masing fungsi untuk masing-masing operasi tersebut. Algoritma fungsi-fungsi tersebut selanjutnya menjadi dasar penyusunan algoritma fungsi-fungsi eliminasi Gauss dan dekomposisi matriks yang akan dijelaskan pada \emph{chapter} selanjutnya.

Fungsi \emph{row scalling} pada \texttt{R} dapat dituliskan pada sintaks berikut:

\begin{Shaded}
\begin{Highlighting}[]
\NormalTok{scale\_row }\OtherTok{\textless{}{-}} \ControlFlowTok{function}\NormalTok{(m, row, k)\{}
\NormalTok{ m[row, ] }\OtherTok{\textless{}{-}}\NormalTok{ m[row, ]}\SpecialCharTok{*}\NormalTok{k}
 \FunctionTok{return}\NormalTok{(m)}
\NormalTok{\}}
\end{Highlighting}
\end{Shaded}

Berikut adalah contoh penerapannya:

\begin{Shaded}
\begin{Highlighting}[]
\CommentTok{\# membuat matriks A}
\NormalTok{(A }\OtherTok{\textless{}{-}} \FunctionTok{matrix}\NormalTok{(}\DecValTok{1}\SpecialCharTok{:}\DecValTok{15}\NormalTok{, }\AttributeTok{nrow=}\DecValTok{5}\NormalTok{))}
\end{Highlighting}
\end{Shaded}

\begin{verbatim}
##      [,1] [,2] [,3]
## [1,]    1    6   11
## [2,]    2    7   12
## [3,]    3    8   13
## [4,]    4    9   14
## [5,]    5   10   15
\end{verbatim}

\begin{Shaded}
\begin{Highlighting}[]
\CommentTok{\# lakukan scaling pada row 2 dengan nilai 10}
\FunctionTok{scale\_row}\NormalTok{(}\AttributeTok{m=}\NormalTok{A, }\AttributeTok{row=}\DecValTok{2}\NormalTok{, }\DecValTok{10}\NormalTok{)}
\end{Highlighting}
\end{Shaded}

\begin{verbatim}
##      [,1] [,2] [,3]
## [1,]    1    6   11
## [2,]   20   70  120
## [3,]    3    8   13
## [4,]    4    9   14
## [5,]    5   10   15
\end{verbatim}

\begin{quote}
\textbf{Catatan:} Untuk menyimpan hasil perhitungan, simpan proses perhitungan dalam sebuah objek (lihat Chapter \ref{assigningvar}).
\end{quote}

\emph{Row swapping} merupakan proses yang berulang, kita perlu menyimpan terlebih dahulu baris matriks pertama kedalam sebuah objek. Baris matriks pertama selanjutnya diganti dengan baris matriks kedua, sedangkan baris matriks kedua selanjutnya akan diganti dengan baris matriks pertama yang telah terlebih dahulu disimpan dalam sebuah objek. Fungsi \emph{row swapping} pada \texttt{R} dapat dituliskan pada sintaks berikut:

\begin{Shaded}
\begin{Highlighting}[]
\NormalTok{swap\_row }\OtherTok{\textless{}{-}} \ControlFlowTok{function}\NormalTok{(m, row1, row2)\{}
\NormalTok{  row\_tmp }\OtherTok{\textless{}{-}}\NormalTok{ m[row1, ]}
\NormalTok{  m[row1, ] }\OtherTok{\textless{}{-}}\NormalTok{ m[row2, ]}
\NormalTok{  m[row2, ] }\OtherTok{\textless{}{-}}\NormalTok{ row\_tmp}
  \FunctionTok{return}\NormalTok{(m)}
\NormalTok{\}}
\end{Highlighting}
\end{Shaded}

Berikut merupakan contoh penerapan fungsi \texttt{swap\_row()}:

\begin{Shaded}
\begin{Highlighting}[]
\CommentTok{\# pertukarkan baris 2 dengan baris 5}
\FunctionTok{swap\_row}\NormalTok{(}\AttributeTok{m=}\NormalTok{A, }\AttributeTok{row1=}\DecValTok{2}\NormalTok{, }\AttributeTok{row2=}\DecValTok{5}\NormalTok{)}
\end{Highlighting}
\end{Shaded}

\begin{verbatim}
##      [,1] [,2] [,3]
## [1,]    1    6   11
## [2,]    5   10   15
## [3,]    3    8   13
## [4,]    4    9   14
## [5,]    2    7   12
\end{verbatim}

Pada proses \emph{row replacement}, proses perhitungan dilakukan dengan melakukan penjumlahan suatu baris matriks dengan baris matriks lainnya dengan terlebih dahulu melakukan \emph{row scalling} terhadap matriks lainnya. Berikut adalah fungsi \texttt{replace\_row()} yang ditulis pada \texttt{R}:

\begin{Shaded}
\begin{Highlighting}[]
\NormalTok{replace\_row }\OtherTok{\textless{}{-}} \ControlFlowTok{function}\NormalTok{(m, row1, row2, k)\{}
\NormalTok{  m[row2, ] }\OtherTok{\textless{}{-}}\NormalTok{ m[row2, ] }\SpecialCharTok{+}\NormalTok{ m[row1, ]}\SpecialCharTok{*}\NormalTok{k}
  \FunctionTok{return}\NormalTok{(m)}
\NormalTok{\}}
\end{Highlighting}
\end{Shaded}

Berikut adalah contoh penerapan fungsi \texttt{replace\_row()}:

\begin{Shaded}
\begin{Highlighting}[]
\FunctionTok{replace\_row}\NormalTok{(}\AttributeTok{m=}\NormalTok{A, }\AttributeTok{row1=}\DecValTok{1}\NormalTok{, }\AttributeTok{row2=}\DecValTok{3}\NormalTok{, }\AttributeTok{k=}\SpecialCharTok{{-}}\DecValTok{3}\NormalTok{)}
\end{Highlighting}
\end{Shaded}

\begin{verbatim}
##      [,1] [,2] [,3]
## [1,]    1    6   11
## [2,]    2    7   12
## [3,]    0  -10  -20
## [4,]    4    9   14
## [5,]    5   10   15
\end{verbatim}

\hypertarget{gausselimination}{%
\section{Eliminasi Gauss}\label{gausselimination}}

Pada \emph{sub-chapter} ini kita akan menggunakan operasi baris elementer yang telah dijelaskan pada Chapter \ref{assigningvar}. Terdapat dua topik yang akan dibahas pada \emph{sub-chapter} ini, yaitu: \emph{row echelon form} termasuk \emph{reduced row echelon form} dan matriks tridiagonal.

Eliminasi Gauss merupakan sebuah cara untuk mencari penyelesaian sistem persamaan linier. Ide dasar dari eliminasi Gauss adalah melakukan operasi matematika pada baris matriks (lihat Chapter \ref{assigningvar}) dan melanjutkannya sampai hanya tersisa satu variabel saja. Kita dapat melakukan lebih dari satu operasi baris elementer pada proses elmininasi ini (contoh: mengalikan sebuah baris dengan konstanta dan menjumlahkan hasilnya pada baris lain).

\hypertarget{rowechelonform}{%
\subsection{\texorpdfstring{\emph{Row Echelon Form}}{Row Echelon Form}}\label{rowechelonform}}

Sebuah matriks merupakan \emph{row echelon form} jika matriks tersebut memenuhi beberapa kondisi:

\begin{enumerate}
\def\labelenumi{\arabic{enumi}.}
\tightlist
\item
  Angka bukan nol pertama dari kiri (\emph{leading coefficient}) selalu di sebelah kanan angka bukan nol pertama pada baris di atasnya.
\item
  Baris yang terdiri dari semua nol ada di bagian bawah matriks.
\end{enumerate}

Misalkan terdapat persamaan linier seperti yang ditunjukkan pada Persamaan \eqref{eq:spl}.

\begin{equation}
\begin{matrix}
  a_{1.1}x_1+a_{1.2}x_2+a_{1.3}x_3+\cdots+a_{1.n}x_n=b_1 \\
  a_{2.1}x_1+a_{2.2}x_2+a_{2.3}x_3+\cdots+a_{2.n}x_n=b_2 \\
  a_{3.1}x_1+a_{3.2}x_2+a_{3.3}x_3+\cdots+a_{3.n}x_n=b_3 \\
  \cdots\cdots\cdots\cdots\cdots\cdots\cdots\cdots       \\
  a_{m.1}x_1+a_{m.2}x_2+a_{m.3}x_3+\cdots+a_{m.n}x_n=b_n
 \end{matrix}
  \label{eq:spl}
\end{equation}

dimana \(a_{i.j}\) untuk \(i=1\) sampai dengan \(m\) dan \(j=1\) sampai dengan \(n\) merupakan koefisien persamaan linier. \(x_i\) untuk \(i=1\) sampai dengan \(n\) merupakan variabel bebas pada sistem persamaan linier.

Persamaan linier pada Persamaan \eqref{eq:spl} dapat dinyatakan ke dalam bentuk matriks pada Persamaan \eqref{eq:matrikpl}.

\begin{equation}
\begin{bmatrix}
     a_{1.1} & a_{1.2} & a_{1.3} &\cdots& a_{1.n}           \\[0.3em]
     a_{2.1} & a_{2.2} & a_{2.3} &\cdots& a_{2.n}           \\[0.3em]
     a_{3.1} & a_{3.2} & a_{3.3} &\cdots& a_{3.n}           \\[0.3em]
     \vdots  & \vdots  & \vdots  &\ddots& \vdots            \\[0.3em]
     a_{m.1} & a_{m.2} & a_{m.3} &\cdots& a_{m.n}
     \end{bmatrix}
\begin{bmatrix}
     x_1                                          \\[0.3em]
     x_2                                          \\[0.3em]
     x_3                                          \\[0.3em]
     \cdots                                       \\[0.3em]
     x_n                                       
     \end{bmatrix}
= \begin{bmatrix}
     b_1                                          \\[0.3em]
     b_2                                          \\[0.3em]
     b_3                                          \\[0.3em]
     \cdots                                       \\[0.3em]
     b_n                                       
     \end{bmatrix}
  \label{eq:matrikpl}
\end{equation}

\begin{equation}
AX=B
  \label{eq:matrikpl2}
\end{equation}

dimana:

\begin{itemize}
\tightlist
\item
  matriks A merupakan matriks koefisien / Jacobian
\item
  vaktor X merupakan vaktor variabel
\item
  vektor B merupakan vektor konstanta
\end{itemize}

matriks pada Persamaan \eqref{eq:matrikpl} dapat diubah menjadi \emph{augmented matrix}, yaitu: perluasan matriks A dengan menambahkan vektor B pada kolom terakhirnya.

\begin{equation}
\begin{bmatrix}
     a_{1.1} & a_{1.2} & a_{1.3} &\cdots& a_{1.n} & b_1     \\[0.3em]
     a_{2.1} & a_{2.2} & a_{2.3} &\cdots& a_{2.n} & b_2     \\[0.3em]
     a_{3.1} & a_{3.2} & a_{3.3} &\cdots& a_{3.n} & b_3     \\[0.3em]
     \vdots  & \vdots  & \vdots &\ddots& \vdots            \\[0.3em]
     a_{m.1} & a_{m.2} & a_{m.3} &\cdots& a_{m.n} & b_n
     \end{bmatrix}
  \label{eq:augmatrik}
\end{equation}

\begin{equation}
A=\left[A|B\right]
  \label{eq:augmatrik2}
\end{equation}

\begin{theorem}[spltheorem]
\protect\hypertarget{thm:unnamed-chunk-25}{}\label{thm:unnamed-chunk-25}Suatu sistem persamaan linier mempunyai penyelesaian tunggal bila memenuhi syarat-syarat sebagai berikut:
\end{theorem}

\begin{itemize}
\tightlist
\item
  ukuran persamaan linier simultan bujursangkar (jumlah persamaan sama dengan jumlah variabel bebas).
\item
  sistem persamaan linier \emph{non-homogen} di mana minimal ada satu nilai vektor konstanta \(B\) tidak nol atau terdapat \(b_{n}\neq 0\).
\item
  Determinan dari matriks koefisiensistem persamaan linier tidak sama dengan nol.
\end{itemize}

Untuk memperoleh penyelesaian sistem persamaan linier, Persamaan \eqref{eq:augmatrik} perlu dilakukan operasi baris elementer. Hasil operasi baris dasar akan menghasilkan matriks \emph{row echelon form} yang disajikan pada Persamaan \eqref{eq:refeq}.

\begin{equation}
\begin{bmatrix}
     a_{1.1} & a_{1.2} & a_{1.3} &\cdots& a_{1.n} & b_1     \\[0.3em]
     a_{2.1} & a_{2.2} & a_{2.3} &\cdots& a_{2.n} & b_2     \\[0.3em]
     a_{3.1} & a_{3.2} & a_{3.3} &\cdots& a_{3.n} & b_3     \\[0.3em]
     \vdots  & \vdots  & \vdots &\ddots& \vdots             \\[0.3em]
     a_{m.1} & a_{m.2} & a_{m.3} &\cdots& a_{m.n} & b_n
     \end{bmatrix}
\implies
\begin{bmatrix}
     c_{1.1} & c_{1.2} & c_{1.3} &\cdots& c_{1.n} & d_1     \\[0.3em]
     0       & c_{2.2} & c_{2.3} &\cdots& c_{2.n} & d_2     \\[0.3em]
     0       & 0       & c_{3.3} &\cdots& c_{3.n} & d_3     \\[0.3em]
     \vdots  & \vdots  & \vdots &\ddots& \vdots             \\[0.3em]
     0       & 0       & 0      &\cdots& c_{m.n} & d_n
     \end{bmatrix}
  \label{eq:refeq}
\end{equation}

Sehingga penyelesaian sistem persamaan linier dapat diperoleh menggunakan Persamaan \eqref{eq:refsolution}.

\begin{equation}
\begin{matrix}
  x_n=\frac{d_n}{c_{m.n}} \\
  x_{n-1}=\frac{1}{C_{m-1.n-1}}\left(d_{n-1}-c_{m-1.n}x_n\right) \\
  \cdots\cdots\cdots\cdots\cdots\cdots\cdots\cdots       \\
  x_2=\frac{1}{c_{2.2}}\left(d_2-c_{2.3}x_3-c_{2.4}x_4-\dots-c_{2.n}x_n\right) \\
  x_1=\frac{1}{c_{1.1}}\left(d_1-c_{1.2}x_2-c_{1.3}x_3-\dots-c_{1.n}x_n\right)
   \end{matrix}
  \label{eq:refsolution}
\end{equation}

\begin{example}
\protect\hypertarget{exm:refexm}{}\label{exm:refexm}Selesaikan sistem persamaan berikut:
\end{example}

\[
\begin{matrix}
  x_1+x_2+x_3=6 \\
  x_1+2x_2-x_3=2 \\
  2x_1+x_2+2x_3=10 \\
\end{matrix}
\]

\textbf{Jawab}:

\emph{Augmented matrix} sistem persamaan linier tersebut adalah sebagai berikut:

\[
\begin{bmatrix}
     1 & 1 & 1 & 6     \\[0.3em]
     1 & 2 & -1 & 2     \\[0.3em]
     2 & 1 & 2 & 10
\end{bmatrix}
\]

Operasi baris elementer selanjutnya dilakukan pada matriks tersebut. Pada langkah pertama, baris ke-2 dikurangkan dengan baris ke-1 (\(B_2-B_1\)) dan baris ke-3 dikurangkan oleh dua kali baris ke-1 (\(B_3-2B_1\)).

\begin{equation*}
\begin{bmatrix}
     1 & 1 & 1 & 6     \\[0.3em]
     1 & 2 & -1 & 2     \\[0.3em]
     2 & 1 & 2 & 10
\end{bmatrix}
\begin{matrix}
  B_2-B_1 \\
  \implies \\
  B_3-2B_1
\end{matrix}
\begin{bmatrix}
     1 & 1 & 1 & 6     \\[0.3em]
     0 & 1 & -2 & -4     \\[0.3em]
     0 & -1 & 0 & -2
\end{bmatrix}
\end{equation*}

Hasil dari langkah pertama tersebut, selanjutnya menjadi input dari langkah selanjutnya. Pada langkah selanjutnya operasi baris elementer kembali dilanjutkan. Baris ke-3 dikurangkan denganbaris ke-2 (\(B_3-B_2\)).

\begin{equation*}
\begin{bmatrix}
     1 & 1 & 1 & 6     \\[0.3em]
     0 & 1 & -2 & -4     \\[0.3em]
     0 & -1 & 0 & -2
\end{bmatrix}
\begin{matrix}
  B_3-B_2 \\
  \implies
\end{matrix}
\begin{bmatrix}
     1 & 1 & 1 & 6     \\[0.3em]
     0 & 1 & -2 & -4     \\[0.3em]
     0 & 0 & -2 & -6
\end{bmatrix}
\end{equation*}

Setelah diperoleh matriks \emph{row echelon form} selanjutnya penyelesaian persamaan dapat dikerjakan menggunakan Persamaan \eqref{eq:refsolution}.

\begin{equation*}
\begin{matrix}
  x_3=\frac{-6}{-2}=3 \\
  x_2=\frac{1}{1}\left(-4-\left(2\right)3\right)=2 \\
  x_1=\frac{1}{1}\left(6-2-3\right)=1
\end{matrix}
\end{equation*}

\begin{center}\rule{0.5\linewidth}{0.5pt}\end{center}

\textbf{Algoritma Row Echelon Form}

\begin{enumerate}
\def\labelenumi{\arabic{enumi}.}
\tightlist
\item
  Masukkan matriks \(A\), dan vektor \(B\) beserta ukurannya \(n\)
\item
  Buat \emph{augmented matrix} \(\left[A|B\right]\) namakan dengan \(A\)
\item
  Untuk baris ke-\(i\) dimana \(i=1\) s/d \(n\), perhatikan apakah nilai \(a_{i,j}\) sama dengan nol. \textbf{a)} \textbf{Bila iya}, lakukan \emph{row swapping} antara baris ke-\(i\) dan baris ke-\(i+k\leq n\), dimana \(a_{i+k,j}\) tidak sama dengan nol. Bila tidak ada berarti perhitungan tidak bisa dilanjutkan dan proses dihentikan dengan tanpa penyelesaian, \textbf{b)} \textbf{Bila tidak}, lanjutkan.
\item
  Untuk baris ke-\(j\), dimana \(j=i+1\) s/d \(n\), lakukan operasi baris elementer:\textbf{a)} Hitung \(c=\frac{a_{j,i}}{a_{i,i}}\), \textbf{b)} untuk kolom \(k\), dimana \(k=1\) s/d \(n+1\), hitung \(a_{j,k}=a_{j,k}-c.a_{i,k}\).
\item
  Hitung akar, untuk \(i=n\) s/d 1 (bergerak dari baris pertama) menggunakan Persamaan \eqref{eq:refsolution}.
\end{enumerate}

\begin{center}\rule{0.5\linewidth}{0.5pt}\end{center}

Berdasarkan algoritma tersebut, kita dapat menyusun fungsi pada \texttt{R} untuk menyelesaikan sistem persamaan linier menggunakan matriks \emph{row echelon form}. Fungsi yang akan dibentuk hanya sampai pada algoritma ke-4. Proses substitusi akan dilakukan secara manual. Berikut adalah sintaks yang digunakan:

\begin{Shaded}
\begin{Highlighting}[]
\NormalTok{ref\_matrix }\OtherTok{\textless{}{-}} \ControlFlowTok{function}\NormalTok{(a)\{}
\NormalTok{  m }\OtherTok{\textless{}{-}} \FunctionTok{nrow}\NormalTok{(a)}
\NormalTok{  n }\OtherTok{\textless{}{-}} \FunctionTok{ncol}\NormalTok{(a)}
\NormalTok{  piv }\OtherTok{\textless{}{-}} \DecValTok{1}
  
\CommentTok{\# cek elemen diagonal apakah bernilai nol}
  \ControlFlowTok{for}\NormalTok{(row\_curr }\ControlFlowTok{in} \DecValTok{1}\SpecialCharTok{:}\NormalTok{m)\{}
    \ControlFlowTok{if}\NormalTok{(piv }\SpecialCharTok{\textless{}=}\NormalTok{ n)\{}
\NormalTok{      i }\OtherTok{\textless{}{-}}\NormalTok{ row\_curr}
      \ControlFlowTok{while}\NormalTok{(a[i, piv] }\SpecialCharTok{==} \DecValTok{0} \SpecialCharTok{\&\&}\NormalTok{ i }\SpecialCharTok{\textless{}}\NormalTok{ m)\{}
\NormalTok{        i }\OtherTok{\textless{}{-}}\NormalTok{ i}\SpecialCharTok{+}\DecValTok{1}
        \ControlFlowTok{if}\NormalTok{(i }\SpecialCharTok{\textgreater{}}\NormalTok{ m)\{}
\NormalTok{          i }\OtherTok{\textless{}{-}}\NormalTok{ row\_curr}
\NormalTok{          piv }\OtherTok{\textless{}{-}}\NormalTok{ piv}\SpecialCharTok{+}\DecValTok{1}
          \ControlFlowTok{if}\NormalTok{(piv }\SpecialCharTok{\textgreater{}}\NormalTok{ n)}
            \FunctionTok{return}\NormalTok{(a)}
\NormalTok{        \}}
\NormalTok{      \}}
      
\CommentTok{\# jika diagonal bernilai nol, lakukan row swapping}
    \ControlFlowTok{if}\NormalTok{(i }\SpecialCharTok{!=}\NormalTok{ row\_curr)}
\NormalTok{      a }\OtherTok{\textless{}{-}} \FunctionTok{swap\_row}\NormalTok{(a, i, row\_curr)}
    
\CommentTok{\# proses triangulasi untuk membentuk matriks segitiga atas}
    \ControlFlowTok{for}\NormalTok{(j }\ControlFlowTok{in}\NormalTok{ row\_curr}\SpecialCharTok{:}\NormalTok{m)}
      \ControlFlowTok{if}\NormalTok{(j }\SpecialCharTok{!=}\NormalTok{ row\_curr)\{}
\NormalTok{        c }\OtherTok{\textless{}{-}}\NormalTok{ a[j, piv]}\SpecialCharTok{/}\NormalTok{a[row\_curr, piv]}
\NormalTok{        a }\OtherTok{\textless{}{-}} \FunctionTok{replace\_row}\NormalTok{(a, row\_curr, j, }\SpecialCharTok{{-}}\NormalTok{c)}
\NormalTok{      \}}
\NormalTok{    piv }\OtherTok{\textless{}{-}}\NormalTok{ piv}\SpecialCharTok{+}\DecValTok{1}
\NormalTok{    \}}
\NormalTok{  \}}
  \FunctionTok{return}\NormalTok{(a)}
\NormalTok{\}}
\end{Highlighting}
\end{Shaded}

Dengan menggunakan fungsi \texttt{ref\_matrix()}, kita dapat membentuk matriks \emph{row echelon form} pada Contoh \ref{exm:refexm}.

\begin{Shaded}
\begin{Highlighting}[]
\NormalTok{am }\OtherTok{\textless{}{-}} \FunctionTok{c}\NormalTok{(}\DecValTok{1}\NormalTok{,}\DecValTok{1}\NormalTok{,}\DecValTok{2}\NormalTok{,}
        \DecValTok{1}\NormalTok{,}\DecValTok{2}\NormalTok{,}\DecValTok{1}\NormalTok{,}
        \DecValTok{1}\NormalTok{,}\SpecialCharTok{{-}}\DecValTok{1}\NormalTok{,}\DecValTok{2}\NormalTok{,}
        \DecValTok{6}\NormalTok{,}\DecValTok{2}\NormalTok{,}\DecValTok{10}\NormalTok{)}
\NormalTok{(m }\OtherTok{\textless{}{-}} \FunctionTok{matrix}\NormalTok{(am, }\AttributeTok{nrow=}\DecValTok{3}\NormalTok{))}
\end{Highlighting}
\end{Shaded}

\begin{verbatim}
##      [,1] [,2] [,3] [,4]
## [1,]    1    1    1    6
## [2,]    1    2   -1    2
## [3,]    2    1    2   10
\end{verbatim}

\begin{Shaded}
\begin{Highlighting}[]
\FunctionTok{ref\_matrix}\NormalTok{(m)}
\end{Highlighting}
\end{Shaded}

\begin{verbatim}
##      [,1] [,2] [,3] [,4]
## [1,]    1    1    1    6
## [2,]    0    1   -2   -4
## [3,]    0    0   -2   -6
\end{verbatim}

matriks yang diperoleh selanjutnya dapat diselesaikan menggunakan Persamaan \eqref{eq:refsolution}.

\begin{example}
\protect\hypertarget{exm:refexm2}{}\label{exm:refexm2}Dengan menggunakan fungsi \texttt{ref\_matrix()}, buatlah matriks \emph{row echelon form} dari sistem persamaan linier berikut:
\end{example}

\[
\begin{matrix}
  2x_1+x_2-x_3=1 \\
  3x_1+2x_2-2x_3=1 \\
  x_1-5x_2+4x_3=3 \\
\end{matrix}
\]

\textbf{Jawab}:

\emph{Augmented matrix} dari sistem persamaan tersebut adalah sebagai berikut:

\[
\begin{bmatrix}
     2 & 1 & -1 & 1     \\[0.3em]
     3 & 2 & -2 & 1     \\[0.3em]
     1 & -5 & 4 & 3
\end{bmatrix}
\]

Penyelesaian matriks tersebut adalah sebagai berikut:

\begin{Shaded}
\begin{Highlighting}[]
\NormalTok{(m }\OtherTok{\textless{}{-}} \FunctionTok{matrix}\NormalTok{(}\FunctionTok{c}\NormalTok{(}\DecValTok{2}\NormalTok{,}\DecValTok{3}\NormalTok{,}\DecValTok{1}\NormalTok{,}
              \DecValTok{1}\NormalTok{,}\DecValTok{2}\NormalTok{,}\SpecialCharTok{{-}}\DecValTok{5}\NormalTok{,}
              \SpecialCharTok{{-}}\DecValTok{1}\NormalTok{,}\SpecialCharTok{{-}}\DecValTok{2}\NormalTok{,}\DecValTok{4}\NormalTok{,}
              \DecValTok{1}\NormalTok{,}\DecValTok{1}\NormalTok{,}\DecValTok{3}\NormalTok{), }\AttributeTok{nrow=}\DecValTok{3}\NormalTok{))}
\end{Highlighting}
\end{Shaded}

\begin{verbatim}
##      [,1] [,2] [,3] [,4]
## [1,]    2    1   -1    1
## [2,]    3    2   -2    1
## [3,]    1   -5    4    3
\end{verbatim}

\begin{Shaded}
\begin{Highlighting}[]
\FunctionTok{ref\_matrix}\NormalTok{(m)}
\end{Highlighting}
\end{Shaded}

\begin{verbatim}
##      [,1] [,2] [,3] [,4]
## [1,]    2  1.0 -1.0  1.0
## [2,]    0  0.5 -0.5 -0.5
## [3,]    0  0.0 -1.0 -3.0
\end{verbatim}

Proses lebih lanjut akan menghasilkan penyelesaian sebagai berikut:

\[
\begin{matrix}
     x_1=1     \\[0.3em]
     x_2=2     \\[0.3em]
     x_3=3
\end{matrix}
\]

\hypertarget{redrowechelonform}{%
\subsection{Eliminasi Gauss-Jordan}\label{redrowechelonform}}

Berbeda dengan metode eliminasi Gauss yang telah dijelaskan pada Chapter \ref{rowechelonform}, metode eliminasi Gauss-Jordan membentuk matriks menjadi bentuk \emph{reduced row echelon form}. Metode ini merupakan pengembangan metode eliminasi Gauss, dimana matriks sebelah kiri \emph{augmented matrix} diubah menjadi matriks diagonal (lihat Persamaan \eqref{eq:gaussjordan}).

\begin{equation}
\begin{bmatrix}
     a_{1,1} & a_{1,2} & a_{1,3} &\cdots& a_{1,n} & b_1     \\[0.3em]
     a_{2,1} & a_{2,2} & a_{2,3} &\cdots& a_{2,n} & b_2     \\[0.3em]
     a_{3,1} & a_{3,2} & a_{3,3} &\cdots& a_{3,n} & b_3     \\[0.3em]
     \vdots  & \vdots  & \vdots &\ddots& \vdots             \\[0.3em]
     a_{m,1} & a_{m,2} & a_{m,3} &\cdots& a_{m,n} & b_n
     \end{bmatrix}
\implies
\begin{bmatrix}
     1       & 0       & 0       &\cdots& 0       & d_1     \\[0.3em]
     0       & 1       & 0       &\cdots& 0       & d_2     \\[0.3em]
     0       & 0       & 1       &\cdots& 0       & d_3     \\[0.3em]
     \vdots  & \vdots  & \vdots &\ddots& \vdots             \\[0.3em]
     0       & 0       & 0      &\cdots& 1        & d_n
     \end{bmatrix}
  \label{eq:gaussjordan}
\end{equation}

Sehingga penyelesaian persamaan linier tersebut adalah nilai \(d_1,d_2,d3,\dots,d_n\) dan atau:

\begin{equation}
\begin{matrix}
  x_1=d_1 \\
  x_2=d_2 \\
  x_3=d_3 \\
  \cdots\cdots\cdots\cdots\cdots\cdots\cdots\cdots       \\
  x_n=d_n
   \end{matrix}
  \label{eq:gaussjordansolution}
\end{equation}

\begin{example}
\protect\hypertarget{exm:gaussjordanexm}{}\label{exm:gaussjordanexm}Selesaikan sistem persamaan berikut:
\end{example}

\[
\begin{matrix}
  x_1+x_2=3 \\
  2x_1+4x_2=8 \\
\end{matrix}
\]

\textbf{Jawab}:

\emph{Augmented matrix} dari persamaan linier tersebut adalah sebagai berikut:

\[
\begin{bmatrix}
     1 & 1 & 3     \\[0.3em]
     2 & 4 & 8
\end{bmatrix}
\]

Operasi baris elementer selanjutnya dilakukan pada matriks tersebut.

\begin{equation*}
\begin{bmatrix}
     1 & 1 & 3     \\[0.3em]
     2 & 4 & 8
\end{bmatrix}
\begin{matrix}
  B_2-2B_1 \\
  \implies 
\end{matrix}
\begin{bmatrix}
     1 & 1 & 3     \\[0.3em]
     0 & 2 & 2
\end{bmatrix}
\end{equation*}

\begin{equation*}
\begin{bmatrix}
     1 & 1 & 3     \\[0.3em]
     0 & 2 & 2
\end{bmatrix}
\begin{matrix}
  \frac{B_2}{2} \\
  \implies
\end{matrix}
\begin{bmatrix}
     1 & 1 & 3     \\[0.3em]
     0 & 1 & 1
\end{bmatrix}
\end{equation*}

\begin{equation*}
\begin{bmatrix}
     1 & 1 & 3     \\[0.3em]
     0 & 1 & 1
\end{bmatrix}
\begin{matrix}
  B_1-B_2 \\
  \implies
\end{matrix}
\begin{bmatrix}
     1 & 0 & 2     \\[0.3em]
     0 & 1 & 1
\end{bmatrix}
\end{equation*}

Penyelesaian persamaan linier tersebut adalah sebagai berikut:

\[
x_1=2\ dan\ x_2=1
\]

\begin{center}\rule{0.5\linewidth}{0.5pt}\end{center}

\textbf{Algoritma Metode Eliminasi Gauss-Jordan}

\begin{enumerate}
\def\labelenumi{\arabic{enumi}.}
\tightlist
\item
  Masukkan matriks \(A\) dan vektor \(B\) beserta ukurannya \(n\)
\item
  Buat \emph{augmented matrix} \(\left[A|B\right]\) namakan dengan \(A\)
\item
  Untuk baris ke-\(i\) dimana \(i=1\) s/d \(n\)
\end{enumerate}

\begin{itemize}
\item
  Perhatikan apakah nilai \(a_{i.i}\) sama dengan nol:

  \begin{itemize}
  \tightlist
  \item
    \textbf{Bila ya}: pertukakan baris ke-\(i\) dan baris ke-\(i+k\le n\), dimana \(a_{i+k.i}\) tidak sama dengan nol, bila tidak ada berarti perhitungan tidak bisa dilanjutkan dan proses dihentikan dengan tanpa penyelesaian.
  \item
    \textbf{Bila tidak}: lanjutkan
  \end{itemize}
\item
  Jadikan nilai diagonalnya menjadi satu dengan cara untuk setiap kolom \(k\) dimana \(k=1\) s/d \(n+1\), hitung \(a_{i.k}=\frac{a_{i.k}}{a_{i.i}}\)
\end{itemize}

\begin{enumerate}
\def\labelenumi{\arabic{enumi}.}
\setcounter{enumi}{3}
\tightlist
\item
  Untuk baris ke-\(j\), dimana \(j=i+1\) s/d \(n\). Lakukan operasi baris elementer untuk kolom \(k\) dimana \(k=1\) s/d \(n\).
\end{enumerate}

\begin{itemize}
\tightlist
\item
  Hitung \(c=a_{j.i}\)
\item
  Hitung \(a_{j.k}=a_{j.k}-c.a_{i.k}\)
\end{itemize}

\begin{enumerate}
\def\labelenumi{\arabic{enumi}.}
\setcounter{enumi}{4}
\tightlist
\item
  Penyelesaian untuk \(i=n\) s/d 1 disajikan pada Persamaan \eqref{eq:gaussjordansolution}.
\end{enumerate}

\begin{center}\rule{0.5\linewidth}{0.5pt}\end{center}

Dari algoritma tersebut, kita dapat membangun sebuah fungsi menggunakan \texttt{R}. Fungsi tersebut adalah sebagai berikut:

\begin{Shaded}
\begin{Highlighting}[]
\NormalTok{gauss\_jordan }\OtherTok{\textless{}{-}} \ControlFlowTok{function}\NormalTok{ (a)\{}
\NormalTok{    m }\OtherTok{\textless{}{-}} \FunctionTok{nrow}\NormalTok{ (a)}
\NormalTok{    n }\OtherTok{\textless{}{-}} \FunctionTok{ncol}\NormalTok{ (a)}
\NormalTok{    piv }\OtherTok{\textless{}{-}} \DecValTok{1}
    
\CommentTok{\# cek elemen diagonal utama apakah bernilai nol}
    \ControlFlowTok{for}\NormalTok{(row\_curr }\ControlFlowTok{in} \DecValTok{1}\SpecialCharTok{:}\NormalTok{m)\{}
        \ControlFlowTok{if}\NormalTok{(piv }\SpecialCharTok{\textless{}=}\NormalTok{ n)\{}
\NormalTok{            i }\OtherTok{\textless{}{-}}\NormalTok{ row\_curr}
            \ControlFlowTok{while}\NormalTok{(a[i, piv] }\SpecialCharTok{==} \DecValTok{0} \SpecialCharTok{\&\&}\NormalTok{ i }\SpecialCharTok{\textless{}}\NormalTok{ m)\{}
\NormalTok{                i }\OtherTok{\textless{}{-}}\NormalTok{ i }\SpecialCharTok{+} \DecValTok{1}
                \ControlFlowTok{if}\NormalTok{(i }\SpecialCharTok{\textgreater{}}\NormalTok{ m)\{}
\NormalTok{                    i }\OtherTok{\textless{}{-}}\NormalTok{ row\_curr}
\NormalTok{                    piv }\OtherTok{\textless{}{-}}\NormalTok{ piv }\SpecialCharTok{+} \DecValTok{1}
                    \ControlFlowTok{if}\NormalTok{(piv }\SpecialCharTok{\textgreater{}}\NormalTok{ n)}
                        \FunctionTok{return}\NormalTok{ (a)}
\NormalTok{                \}}
\NormalTok{            \}}

\CommentTok{\# jika diagonal utama bernilai nol,lakukan row swapping}
            \ControlFlowTok{if}\NormalTok{(i }\SpecialCharTok{!=}\NormalTok{ row\_curr)}
\NormalTok{                a }\OtherTok{\textless{}{-}} \FunctionTok{swap\_row}\NormalTok{(a, i, row\_curr)}
            
\CommentTok{\# proses pembentukan matriks reduced row echelon form}
\NormalTok{            piv\_val }\OtherTok{\textless{}{-}}\NormalTok{ a[row\_curr , piv]}
\NormalTok{            a }\OtherTok{\textless{}{-}} \FunctionTok{scale\_row}\NormalTok{ (a, row\_curr , }\DecValTok{1} \SpecialCharTok{/}\NormalTok{ piv\_val)}
            \ControlFlowTok{for}\NormalTok{(j }\ControlFlowTok{in} \DecValTok{1}\SpecialCharTok{:}\NormalTok{ m)\{}
                \ControlFlowTok{if}\NormalTok{(j }\SpecialCharTok{!=}\NormalTok{ row\_curr)\{}
\NormalTok{                    k }\OtherTok{\textless{}{-}}\NormalTok{ a[j, piv]}\SpecialCharTok{/}\NormalTok{a[row\_curr, piv]}
\NormalTok{                    a }\OtherTok{\textless{}{-}} \FunctionTok{replace\_row}\NormalTok{ (a, row\_curr, j, }\SpecialCharTok{{-}}\NormalTok{k)}
\NormalTok{                \}}
\NormalTok{            \}}
\NormalTok{            piv }\OtherTok{\textless{}{-}}\NormalTok{ piv }\SpecialCharTok{+} \DecValTok{1}
\NormalTok{        \}}
\NormalTok{    \}}
    \FunctionTok{return}\NormalTok{ (a)}
\NormalTok{\}}
\end{Highlighting}
\end{Shaded}

Dengan menggunakan fungsi \texttt{gauss\_jordan()}, sistem persamaan linier pada Contoh \ref{exm:gaussjordanexm}:

\begin{Shaded}
\begin{Highlighting}[]
\NormalTok{(m }\OtherTok{\textless{}{-}} \FunctionTok{matrix}\NormalTok{(}\FunctionTok{c}\NormalTok{(}\DecValTok{1}\NormalTok{,}\DecValTok{2}\NormalTok{,}\DecValTok{1}\NormalTok{,}\DecValTok{4}\NormalTok{,}\DecValTok{3}\NormalTok{,}\DecValTok{8}\NormalTok{), }\AttributeTok{nrow=}\DecValTok{2}\NormalTok{))}
\end{Highlighting}
\end{Shaded}

\begin{verbatim}
##      [,1] [,2] [,3]
## [1,]    1    1    3
## [2,]    2    4    8
\end{verbatim}

\begin{Shaded}
\begin{Highlighting}[]
\FunctionTok{gauss\_jordan}\NormalTok{(m)}
\end{Highlighting}
\end{Shaded}

\begin{verbatim}
##      [,1] [,2] [,3]
## [1,]    1    0    2
## [2,]    0    1    1
\end{verbatim}

\begin{example}
\protect\hypertarget{exm:gaussjordanexm2}{}\label{exm:gaussjordanexm2}Dengan menggunakan fungsi \texttt{gauss\_jordan()}, carilah penyelesaian sistem persamaan linier pada Contoh \ref{exm:refexm} dan Contoh \ref{exm:refexm2}:
\end{example}

\textbf{Jawab}:

Untuk Contoh \ref{exm:refexm}:

\begin{Shaded}
\begin{Highlighting}[]
\NormalTok{am }\OtherTok{\textless{}{-}} \FunctionTok{c}\NormalTok{(}\DecValTok{1}\NormalTok{,}\DecValTok{1}\NormalTok{,}\DecValTok{2}\NormalTok{,}
        \DecValTok{1}\NormalTok{,}\DecValTok{2}\NormalTok{,}\DecValTok{1}\NormalTok{,}
        \DecValTok{1}\NormalTok{,}\SpecialCharTok{{-}}\DecValTok{1}\NormalTok{,}\DecValTok{2}\NormalTok{,}
        \DecValTok{6}\NormalTok{,}\DecValTok{2}\NormalTok{,}\DecValTok{10}\NormalTok{)}
\NormalTok{m }\OtherTok{\textless{}{-}} \FunctionTok{matrix}\NormalTok{(am, }\AttributeTok{nrow=}\DecValTok{3}\NormalTok{)}

\FunctionTok{gauss\_jordan}\NormalTok{(m)}
\end{Highlighting}
\end{Shaded}

\begin{verbatim}
##      [,1] [,2] [,3] [,4]
## [1,]    1    0    0    1
## [2,]    0    1    0    2
## [3,]    0    0    1    3
\end{verbatim}

Untuk Contoh \ref{exm:refexm2}:

\begin{Shaded}
\begin{Highlighting}[]
\NormalTok{m }\OtherTok{\textless{}{-}} \FunctionTok{matrix}\NormalTok{(}\FunctionTok{c}\NormalTok{(}\DecValTok{2}\NormalTok{,}\DecValTok{3}\NormalTok{,}\DecValTok{1}\NormalTok{,}\DecValTok{1}\NormalTok{,}\DecValTok{2}\NormalTok{,}\SpecialCharTok{{-}}\DecValTok{5}\NormalTok{,}
              \SpecialCharTok{{-}}\DecValTok{1}\NormalTok{,}\SpecialCharTok{{-}}\DecValTok{2}\NormalTok{,}\DecValTok{4}\NormalTok{,}\DecValTok{1}\NormalTok{,}\DecValTok{1}\NormalTok{,}\DecValTok{3}\NormalTok{), }
            \AttributeTok{nrow=}\DecValTok{3}\NormalTok{)}
\FunctionTok{gauss\_jordan}\NormalTok{(m)}
\end{Highlighting}
\end{Shaded}

\begin{verbatim}
##      [,1] [,2] [,3] [,4]
## [1,]    1    0    0    1
## [2,]    0    1    0    2
## [3,]    0    0    1    3
\end{verbatim}

\hypertarget{matriktridiagonal}{%
\subsection{Matrik Tridiagonal}\label{matriktridiagonal}}

Metode eliminasi Gauss merupakan metode yang sederhana untuk digunakan khususnya jika semua koefisien bukan nol berkumpul pada diagonal utama dan beberapa diagonal sekitarnya. Suatu sistem yang bersifat demikian disebut sebagai \emph{banded} dan banyaknya diagonal yang memuat koefisien bukan nol disebut sebagai \emph{bandwidth}. Contoh khusus yang sering dijumpai adalah matriks tridiagonal yang memiliki \emph{bandwidth} tiga.

Proses eliminasi untuk matriks tridiagonal bersifat trivial karena dengan membentuk sebuah subdiagonal tambahan, proses substitusi mundur segera dapat dilakukan. Bentuk matriks tridiagonal disajikan pada Persamaan \eqref{eq:matrikstridiagonal}.

\begin{equation}
\begin{bmatrix}
     a_{1,1} & a_{1,2} & 0       &\cdots& 0                 \\[0.3em]
     a_{2,1} & a_{2,2} & a_{2,3} &\cdots& 0             \\[0.3em]
     0       & a_{3,2} & a_{3,3} &\cdots& 0             \\[0.3em]
     \vdots  & \vdots  & \vdots  &\ddots& \vdots            \\[0.3em]
     0       & 0       & 0       &\cdots& a_{m,n}
     \end{bmatrix}
\begin{bmatrix}
     x_1                                          \\[0.3em]
     x_2                                          \\[0.3em]
     x_3                                          \\[0.3em]
     \cdots                                       \\[0.3em]
     x_n                                       
     \end{bmatrix}
= \begin{bmatrix}
     b_1                                          \\[0.3em]
     b_2                                          \\[0.3em]
     b_3                                          \\[0.3em]
     \cdots                                       \\[0.3em]
     b_n                                       
     \end{bmatrix}
  \label{eq:matrikstridiagonal}
\end{equation}

Penyelesaian persamaan tersebut disajikan pada Persamaan \eqref{eq:solusimatrikstridiagonal}.

\begin{equation}
x_n=\frac{b_n}{a_{m.n}};\ x_i=\frac{b_i-a_{i,j+1}x_{i+1}}{a_{i,j}}
  \label{eq:solusimatrikstridiagonal}
\end{equation}

dimana \(i=n-1,n-2,\dots,1\).

Pada beberapa \emph{textbook}, diagonal matriks sering dilambangkan dengan \(l\)(diagonal bawah), \(d\)(diagonal tengah), dan \(u\) (diagonal atas). Bentuk matriksnya disajikan pada Persamaan \eqref{eq:matrikstridiagonal2}.

\begin{equation}
\begin{bmatrix}
     d_{1} & u_{2} & 0       &\cdots& 0                 \\[0.3em]
     l_{2} & d_{2} & u_{3} &\cdots& 0             \\[0.3em]
     0       & l_{3} & d_{3} &\cdots& 0             \\[0.3em]
     \vdots  & \vdots  & \vdots  &\ddots& \vdots            \\[0.3em]
     0       & 0       & 0       &\cdots& d_{n}
     \end{bmatrix}
\begin{bmatrix}
     x_1                                          \\[0.3em]
     x_2                                          \\[0.3em]
     x_3                                          \\[0.3em]
     \cdots                                       \\[0.3em]
     x_n                                       
     \end{bmatrix}
= \begin{bmatrix}
     b_1                                          \\[0.3em]
     b_2                                          \\[0.3em]
     b_3                                          \\[0.3em]
     \cdots                                       \\[0.3em]
     b_n                                       
     \end{bmatrix}
  \label{eq:matrikstridiagonal2}
\end{equation}

\begin{center}\rule{0.5\linewidth}{0.5pt}\end{center}

\textbf{Algoritma Penyelesaian Matrik Tridiagonal}

\begin{enumerate}
\def\labelenumi{\arabic{enumi}.}
\tightlist
\item
  Bentuk sistem persamaan linier menjadi matriks pada Persamaan \eqref{eq:matrikstridiagonal2}.
\item
  Lakukan \emph{foward sweep}. Setiap elemen diagonal \(l\) dieliminasi menggunakan reduksi baris.
\end{enumerate}

\begin{itemize}
\item
  Untuk \(i=1\)

  \begin{itemize}
  \tightlist
  \item
    Hitung \(u_1=\frac{u_1}{d_1}\)
  \item
    Hitung \(b_1=\frac{b_1}{d_1}\)
  \end{itemize}
\item
  Untuk \(i=2\) s/d \(n-1\)

  \begin{itemize}
  \tightlist
  \item
    Hitung \(u_i=\frac{u_i}{d_i-l_i\times u_{i-1}}\)
  \item
    Hitung \(b_i=\frac{b_i-l_i\times u_{i-1}}{d_i-l_i\times u_{i-1}}\)
  \end{itemize}
\item
  Hitung \(b_n=\frac{b_n-l_n\times u_{n-1}}{d_n-l_n\times u_{n-1}}\)
\end{itemize}

\begin{enumerate}
\def\labelenumi{\arabic{enumi}.}
\setcounter{enumi}{2}
\tightlist
\item
  Lakukan \emph{backward sweep}. Setiap elemen diagonal \(u\) dilakukan eliminasi.
\end{enumerate}

\begin{itemize}
\item
  Untuk \(i=n-1\) s/d \(1\)

  \begin{itemize}
  \tightlist
  \item
    Hitung \(x_n=b_i-u_i\times x_{i+1}\)
  \end{itemize}
\item
  Hitung \(x_n=b_n\)
\end{itemize}

\begin{center}\rule{0.5\linewidth}{0.5pt}\end{center}

Berdasarkan algoritma tersebut, kita dapat membangun sebuah fungsi pada \texttt{R}. Fungsi penyelesaian matriks tridiagonal disajikan sebagai berikut:

\begin{Shaded}
\begin{Highlighting}[]
\NormalTok{tridiagmatrix }\OtherTok{\textless{}{-}} \ControlFlowTok{function}\NormalTok{ (L, D, U, b)\{}
\NormalTok{  n }\OtherTok{\textless{}{-}} \FunctionTok{length}\NormalTok{ (D)}
\NormalTok{  L }\OtherTok{\textless{}{-}} \FunctionTok{c}\NormalTok{(}\ConstantTok{NA}\NormalTok{ , L)}
  
  \DocumentationTok{\#\# forward sweep}
\NormalTok{  U[}\DecValTok{1}\NormalTok{] }\OtherTok{\textless{}{-}}\NormalTok{ U[}\DecValTok{1}\NormalTok{] }\SpecialCharTok{/}\NormalTok{ D[}\DecValTok{1}\NormalTok{]}
\NormalTok{  b[}\DecValTok{1}\NormalTok{] }\OtherTok{\textless{}{-}}\NormalTok{ b[}\DecValTok{1}\NormalTok{] }\SpecialCharTok{/}\NormalTok{ D[}\DecValTok{1}\NormalTok{]}
  \ControlFlowTok{for}\NormalTok{(i }\ControlFlowTok{in} \DecValTok{2}\SpecialCharTok{:}\NormalTok{(n }\SpecialCharTok{{-}} \DecValTok{1}\NormalTok{))\{}
\NormalTok{      U[i] }\OtherTok{\textless{}{-}}\NormalTok{ U[i] }\SpecialCharTok{/}\NormalTok{ (D[i] }\SpecialCharTok{{-}}\NormalTok{ L[i] }\SpecialCharTok{*}\NormalTok{ U[i }\SpecialCharTok{{-}} \DecValTok{1}\NormalTok{])}
\NormalTok{      b[i] }\OtherTok{\textless{}{-}}\NormalTok{ (b[i] }\SpecialCharTok{{-}}\NormalTok{ L[i] }\SpecialCharTok{*}\NormalTok{ b[i }\SpecialCharTok{{-}} \DecValTok{1}\NormalTok{]) }\SpecialCharTok{/}
\NormalTok{      (D[i] }\SpecialCharTok{{-}}\NormalTok{ L[i] }\SpecialCharTok{*}\NormalTok{ U[i }\SpecialCharTok{{-}} \DecValTok{1}\NormalTok{])}
\NormalTok{  \}}
\NormalTok{  b[n] }\OtherTok{\textless{}{-}}\NormalTok{ (b[n] }\SpecialCharTok{{-}}\NormalTok{ L[n] }\SpecialCharTok{*}\NormalTok{ b[n }\SpecialCharTok{{-}} \DecValTok{1}\NormalTok{])}\SpecialCharTok{/}\NormalTok{(D[n] }\SpecialCharTok{{-}}\NormalTok{ L[n] }\SpecialCharTok{*}\NormalTok{ U[n }\SpecialCharTok{{-}} \DecValTok{1}\NormalTok{])}
  
  \DocumentationTok{\#\# backward sweep}
\NormalTok{  x }\OtherTok{\textless{}{-}} \FunctionTok{rep.int}\NormalTok{ (}\DecValTok{0}\NormalTok{, n)}
\NormalTok{  x[n] }\OtherTok{\textless{}{-}}\NormalTok{ b[n]}
  \ControlFlowTok{for}\NormalTok{(i }\ControlFlowTok{in}\NormalTok{ (n }\SpecialCharTok{{-}} \DecValTok{1}\NormalTok{) }\SpecialCharTok{:}\DecValTok{1}\NormalTok{)}
\NormalTok{      x[i] }\OtherTok{\textless{}{-}}\NormalTok{ b[i] }\SpecialCharTok{{-}}\NormalTok{ U[i] }\SpecialCharTok{*}\NormalTok{ x[i }\SpecialCharTok{+} \DecValTok{1}\NormalTok{]}
  \FunctionTok{return}\NormalTok{ (x)}
\NormalTok{\}}
\end{Highlighting}
\end{Shaded}

\begin{example}
\protect\hypertarget{exm:tridiagexm}{}\label{exm:tridiagexm}Selesaikan sistem persamaan berikut menggunakan fungsi \texttt{tridiagmatrix()} dan fungsi \texttt{gauss\_jordan()}!
\end{example}

\[
\begin{matrix}
  3x_1+4x_2=20 \\
  4x_1+5x_2-2x_3=28 \\
  2x_2+5x_3-3x_4=18 \\
  3x_3+5x_4=18
\end{matrix}
\]

\textbf{Jawab}:

Langkah pertama untuk menyelesaikannya, kita harus merubah persamaan tersebut kedalam bentuk matriks

\begin{equation*}
\begin{bmatrix}
     3 & 4 & 0  & 0     \\[0.3em]
     4 & 5 & 2 & 0     \\[0.3em]
     0 & 2 & 5 & 3     \\[0.3em]
     0 & 0 & 3 & 5
\end{bmatrix}
x = \begin{bmatrix}
     20     \\[0.3em]
     28     \\[0.3em]
     18     \\[0.3em]
     18
\end{bmatrix}
\end{equation*}

Untuk menyelesaikan persamaan tersebut menggunakan fungsi \texttt{tridiagmatrix()}, kita perlu membentuk vektor diagonal \(l\), \(d\), \(u\), dan \(b\).

\begin{Shaded}
\begin{Highlighting}[]
\NormalTok{l }\OtherTok{\textless{}{-}}\NormalTok{ u }\OtherTok{\textless{}{-}} \FunctionTok{c}\NormalTok{(}\DecValTok{4}\NormalTok{, }\DecValTok{2}\NormalTok{, }\DecValTok{3}\NormalTok{); d }\OtherTok{\textless{}{-}} \FunctionTok{c}\NormalTok{(}\DecValTok{3}\NormalTok{, }\DecValTok{5}\NormalTok{, }\DecValTok{5}\NormalTok{, }\DecValTok{5}\NormalTok{)}
\NormalTok{b }\OtherTok{\textless{}{-}} \FunctionTok{c}\NormalTok{(}\DecValTok{20}\NormalTok{, }\DecValTok{28}\NormalTok{, }\DecValTok{18}\NormalTok{, }\DecValTok{18}\NormalTok{)}
\end{Highlighting}
\end{Shaded}

Setelah terbentuk, vektor tersebut dapat langsung dimasukkan ke dalam fungsi \texttt{tridiagmatrix()}.

\begin{Shaded}
\begin{Highlighting}[]
\FunctionTok{tridiagmatrix}\NormalTok{(}\AttributeTok{L=}\NormalTok{l, }\AttributeTok{D=}\NormalTok{d, }\AttributeTok{U=}\NormalTok{u, }\AttributeTok{b=}\NormalTok{b)}
\end{Highlighting}
\end{Shaded}

\begin{verbatim}
## [1] 4 2 1 3
\end{verbatim}

Untuk menyelesaikannya menggunakan fungsi \texttt{gauss\_jordan()}, kita perlu membentuk \emph{augmented matrix}-nya terlebih dahulu.

\begin{Shaded}
\begin{Highlighting}[]
\NormalTok{m }\OtherTok{\textless{}{-}} \FunctionTok{matrix}\NormalTok{(}\FunctionTok{c}\NormalTok{(}\DecValTok{3}\NormalTok{,}\DecValTok{4}\NormalTok{,}\DecValTok{0}\NormalTok{,}\DecValTok{0}\NormalTok{,}\DecValTok{4}\NormalTok{,}\DecValTok{5}\NormalTok{,}\DecValTok{2}\NormalTok{,}\DecValTok{0}\NormalTok{,}
              \DecValTok{0}\NormalTok{,}\DecValTok{2}\NormalTok{,}\DecValTok{5}\NormalTok{,}\DecValTok{3}\NormalTok{,}\DecValTok{0}\NormalTok{,}\DecValTok{0}\NormalTok{,}\DecValTok{3}\NormalTok{,}\DecValTok{5}\NormalTok{,}
              \DecValTok{20}\NormalTok{,}\DecValTok{28}\NormalTok{,}\DecValTok{18}\NormalTok{,}\DecValTok{18}\NormalTok{), }\AttributeTok{nrow=}\DecValTok{4}\NormalTok{)}
\FunctionTok{gauss\_jordan}\NormalTok{(m)}
\end{Highlighting}
\end{Shaded}

\begin{verbatim}
##      [,1] [,2] [,3] [,4] [,5]
## [1,]    1    0    0    0    4
## [2,]    0    1    0    0    2
## [3,]    0    0    1    0    1
## [4,]    0    0    0    1    3
\end{verbatim}

\hypertarget{solvefunc}{%
\subsection{\texorpdfstring{Penyelesaian Sistem Persamaan Linier Menggunakan Fungsi \texttt{solve()}}{Penyelesaian Sistem Persamaan Linier Menggunakan Fungsi solve()}}\label{solvefunc}}

\texttt{R} menyediakan fungsi bawaan \texttt{solve()} untuk menyelesaiakan sistem persamaan linier. Format fungsi \texttt{solve()} adalah sebagai berikut:

\begin{Shaded}
\begin{Highlighting}[]
\FunctionTok{solve}\NormalTok{(a,b)}
\end{Highlighting}
\end{Shaded}

\begin{quote}
\textbf{Catatan}:

\begin{itemize}
\tightlist
\item
  \textbf{a}: matriks koefisien atau matriks segiempat
\item
  \textbf{b}: vektor konstanta
\end{itemize}
\end{quote}

Berikut adalah contoh penerapan fungsi \texttt{solve()} pada sistem persamaan linier yang disajikan pada Contoh \ref{exm:refexm2}:

\begin{Shaded}
\begin{Highlighting}[]
\CommentTok{\# memecah matriks m menjadi matriks koefisien dan vektor konstanta}
\NormalTok{a }\OtherTok{\textless{}{-}} \FunctionTok{matrix}\NormalTok{(}\FunctionTok{c}\NormalTok{(}\DecValTok{2}\NormalTok{,}\DecValTok{3}\NormalTok{,}\DecValTok{1}\NormalTok{,}\DecValTok{1}\NormalTok{,}\DecValTok{2}\NormalTok{,}\SpecialCharTok{{-}}\DecValTok{5}\NormalTok{,}\SpecialCharTok{{-}}\DecValTok{1}\NormalTok{,}\SpecialCharTok{{-}}\DecValTok{2}\NormalTok{,}\DecValTok{4}\NormalTok{),}\AttributeTok{nrow=}\DecValTok{3}\NormalTok{)}
\NormalTok{b }\OtherTok{\textless{}{-}} \FunctionTok{c}\NormalTok{(}\DecValTok{1}\NormalTok{,}\DecValTok{1}\NormalTok{,}\DecValTok{3}\NormalTok{)}

\FunctionTok{solve}\NormalTok{(a,b)}
\end{Highlighting}
\end{Shaded}

\begin{verbatim}
## [1] 1 2 3
\end{verbatim}

Jika kita hanya memasukkan matriks persegi, maka output yang akan dihasilkan adalah invers dari matriks yang kita masukkan.

\begin{Shaded}
\begin{Highlighting}[]
\FunctionTok{solve}\NormalTok{(a)}
\end{Highlighting}
\end{Shaded}

\begin{verbatim}
##      [,1] [,2]       [,3]
## [1,]    2   -1  7.401e-17
## [2,]   14   -9 -1.000e+00
## [3,]   17  -11 -1.000e+00
\end{verbatim}

Jika kita mengalikan invers dengan matriks semula, maka akan dihasilkan output berupa matriks identitas.

\begin{Shaded}
\begin{Highlighting}[]
\NormalTok{a}\SpecialCharTok{\%*\%}\FunctionTok{solve}\NormalTok{(a)}
\end{Highlighting}
\end{Shaded}

\begin{verbatim}
##      [,1] [,2] [,3]
## [1,]    1    0    0
## [2,]    0    1    0
## [3,]    0    0    1
\end{verbatim}

\hypertarget{solvetridiagfunct}{%
\subsection{Penyelesaian Sistem Persamaan Linier Menggunakan Fungsi 'Solve.tridiag()`}\label{solvetridiagfunct}}

Penyelesaian matriks tridiagonal selain menggunakan fungsi \texttt{solve()}, juga dapat menggunakan fungsi \texttt{Solve.tridiag()} dari Paket \texttt{limSolve}. Untuk menginstall dan mengaktifkan Paket tersebut, jalankan sintaks berikut:

\begin{Shaded}
\begin{Highlighting}[]
\FunctionTok{install.packages}\NormalTok{(}\StringTok{"limSolve"}\NormalTok{)}
\end{Highlighting}
\end{Shaded}

\begin{Shaded}
\begin{Highlighting}[]
\FunctionTok{library}\NormalTok{(limSolve)}
\end{Highlighting}
\end{Shaded}

Fungsi \texttt{Solve.tridiag()} memiliki format sebagai berikut:

\begin{Shaded}
\begin{Highlighting}[]
\FunctionTok{Solve.tridiag}\NormalTok{ ( diam1, dia, diap1, }\AttributeTok{B=}\FunctionTok{rep}\NormalTok{(}\DecValTok{0}\NormalTok{,}\AttributeTok{times=}\FunctionTok{length}\NormalTok{(dia)))}
\end{Highlighting}
\end{Shaded}

\begin{quote}
\textbf{Catatan}:

\begin{itemize}
\tightlist
\item
  \textbf{diam1}: vektor bukan nol di bawah diagonal matriks
\item
  \textbf{dia}: vektor bukan nol pada diagonal matriks
\item
  \textbf{diap1}: vektor bukan nol di atas diagonal matriks
\item
  \textbf{B}: vektor konstanta
\end{itemize}
\end{quote}

Untuk memahami penerapannya, kita akan menggunakan kembali matriks yang ada pada Contoh \ref{exm:tridiagexm}.

\begin{Shaded}
\begin{Highlighting}[]
\NormalTok{l }\OtherTok{\textless{}{-}}\NormalTok{ u }\OtherTok{\textless{}{-}} \FunctionTok{c}\NormalTok{(}\DecValTok{4}\NormalTok{, }\DecValTok{2}\NormalTok{, }\DecValTok{3}\NormalTok{); d }\OtherTok{\textless{}{-}} \FunctionTok{c}\NormalTok{(}\DecValTok{3}\NormalTok{, }\DecValTok{5}\NormalTok{, }\DecValTok{5}\NormalTok{, }\DecValTok{5}\NormalTok{)}
\NormalTok{b }\OtherTok{\textless{}{-}} \FunctionTok{c}\NormalTok{(}\DecValTok{20}\NormalTok{, }\DecValTok{28}\NormalTok{, }\DecValTok{18}\NormalTok{, }\DecValTok{18}\NormalTok{)}
\FunctionTok{Solve.tridiag}\NormalTok{(}\AttributeTok{diam1=}\NormalTok{l, }\AttributeTok{dia=}\NormalTok{d, }\AttributeTok{diap1=}\NormalTok{u, }\AttributeTok{B=}\NormalTok{b)}
\end{Highlighting}
\end{Shaded}

\begin{verbatim}
##      [,1]
## [1,]    4
## [2,]    2
## [3,]    1
## [4,]    3
\end{verbatim}

\hypertarget{dekomposisimatriks}{%
\section{Dekomposisi Matriks}\label{dekomposisimatriks}}

Seringkali kita diminta untuk memperoleh nilai penyelesaian suatu persamaan linier \(Ax=B\), dimana nilai vektor \(B\) yang selalu berubah-ubah. Penggunaan metode eliminasi Gauss mengharuskan untuk menyelesaikan sistem persamaan linier \(Ax=B\) secara terpisah untuk setiap perubahan vektor \(B\). Untuk menghindari pekerjaan eliminasi yang selalu berulang-ulang, faktorisasi menjadi suatu hal yang dapat dilakukan untuk mempersingkat prosesnya. Faktorisasi atau dekomposisi matriks merupakan suatu algoritma untuk memecah matriks \(A\), hasil pemecahan ini selanjutnya digunakan untuk memperoleh penyelesaian sistem persamaan linier melalui perkalian antara vektor \(B\) dan hasil faktorisasi matriks \(A\).

\hypertarget{ludecomp}{%
\subsection{Dekomposisi LU}\label{ludecomp}}

Misalkan kita memiliki persamaan linier seperti yang ditunjukkan oleh Persamaan \eqref{eq:matrikpl2}. Pada metode dekomposisi LU, matriks \(A\) difaktorkan menjadi matriks \(L\) dan matriks \(U\), dimana ukuran kedua matriks tersebut harus sama dengan ukuran matriks \(A\) atau dapat kita tuliskan bahwa hasil perkalian kedua matriks tersebut akan menghasilkan matriks \(A\).

\begin{equation}
A=LU
 \label{eq:LUdecomp}
\end{equation}

Sehingga Persamaan \eqref{eq:matrikpl2} akan menjadi Persamaan \eqref{eq:LUdecomp2}.

\begin{equation}
LUx=b
 \label{eq:LUdecomp2}
\end{equation}

Langkah penyelesaian sistem persamaan linier, diawali dengan menghadirkan vektor \(t\) yang ditunjukkan pada Persamaan \eqref{eq:LUdecomp3}.

\begin{equation}
Ux=t
 \label{eq:LUdecomp3}
\end{equation}

Langkah pada Persamaan \eqref{eq:LUdecomp3} tidak dimaksudkan untuk menghitung vektor \(t\), melainkan untuk menghitung vektor \(x\). Vektor \(t\) diperoleh dengan menggunakan Persamaan \eqref{eq:LUdecomp4}.

\begin{equation}
Lx=t
 \label{eq:LUdecomp4}
\end{equation}

Kita dapat menyelesaikan sistem persamaan yang ditunjukkan pada Persamaan \eqref{eq:LUdecomp3} dan Persamaan \eqref{eq:LUdecomp4} menggunakan berbagai algoritma penyelesaian yang telah dibahas sebelumnya. Namun, karena matriks \(L\) merupakan matriks segitiga bawah dengan nilai nol berada pada bagian atas diagonal utama, penyelesaian \(t\) mengambil langkah yang lebih sedikit. Kondisi ini sama dengan kondisi penyelesaian matriks tridiagonal, dimana kita memanfaatkan sejumlah jalan pintas penyelesaiaannya guna mempercepat komputasi. Matriks segitia bawah \(L\) akan berupa matriks persegi dengan ukuran \(m\), di mana \(m\) merupakan jumlah baris matriks \(A\). Persamaan \eqref{eq:LUdecomp4} dalam bentuk matriks akan terlihat seperti Persamaan \eqref{eq:LUdecomp5}.

\begin{equation}
Lt=
\begin{bmatrix}
     1       & 0       & 0 &\cdots& 0           \\[0.3em]
     l_{2,1} & 1       & 0 &\cdots& 0           \\[0.3em]
     l_{3,1} & l_{3,2} & 1 &\cdots& 0           \\[0.3em]
     \vdots  & \vdots  & \vdots  &\ddots& \vdots            \\[0.3em]
     l_{m,1} & l_{m,2} & l_{m,3} &\cdots& 1
     \end{bmatrix}
\begin{bmatrix}
     t_1                                          \\[0.3em]
     t_2                                          \\[0.3em]
     t_3                                          \\[0.3em]
     \cdots                                       \\[0.3em]
     t_n                                       
     \end{bmatrix}
= \begin{bmatrix}
     b_1                                          \\[0.3em]
     b_2                                          \\[0.3em]
     b_3                                          \\[0.3em]
     \cdots                                       \\[0.3em]
     b_n                                       
     \end{bmatrix}
  \label{eq:LUdecomp5}
\end{equation}

Berdasarkan Persamaan \eqref{eq:LUdecomp5}, diketahui nilai \(t_1=b_1\). Nilai ini selanjutnya dapat digunakan untuk melakukan proses substitusi guna memperoleh seluruh nilai vektor \(t\). Proses ini disebut sebagai \emph{foward substitution}. Proses substitusi dapat dituliskan menggunakan Persamaan \eqref{eq:LUdecomp6}.

\begin{equation}
t_i=b_i-\sum_{j=1}^{i-1}l_{i,j}t_i
  \label{eq:LUdecomp6}
\end{equation}

Seteleh nilai vektor \(t\) dihitung, kita dapat menghitung nilai \(x\) pada Persamaan \eqref{eq:LUdecomp7}.

\begin{equation}
Ux=
\begin{bmatrix}
     u_{1,1} & u_{1,2} & u_{1,3} &\cdots& u_{1,n}           \\[0.3em]
     0       & u_{2,2} & u_{2,3} &\cdots& u_{2,n}           \\[0.3em]
     0       & 0       & u_{3,3} &\cdots& u_{3,n}           \\[0.3em]
     \vdots  & \vdots  & \vdots  &\ddots& \vdots            \\[0.3em]
     0       & 0      & 0        &\cdots& u_{m,n}
     \end{bmatrix}
\begin{bmatrix}
     x_1                                          \\[0.3em]
     x_2                                          \\[0.3em]
     x_3                                          \\[0.3em]
     \cdots                                       \\[0.3em]
     x_n                                       
     \end{bmatrix}
= \begin{bmatrix}
     t_1                                          \\[0.3em]
     t_2                                          \\[0.3em]
     t_3                                          \\[0.3em]
     \cdots                                       \\[0.3em]
     t_n                                       
     \end{bmatrix}
  \label{eq:LUdecomp7}
\end{equation}

Jika diperhatikan, kita dapat mengetahui mengetahui nilai \(x_n=\frac{t_n}{u_{m,n}}\). Nilai tersebut selanjutnya dapat digunakan untuk melakukan proses susbtitusi pada nilai lainnya. Proses substitusi ini disebut sebagai \emph{backward substitution}. Proses dekomposisi atau faktorisasi LU digambarkan pada Gambar \ref{fig:LUfig}.

\begin{figure}

{\centering \includegraphics[width=0.9\linewidth]{./images/LU} 

}

\caption{Tahapan dekomposisi LU.}\label{fig:LUfig}
\end{figure}

Dekomposisi LU didasarkan pada operasi baris elementer. Pertama, kita perlu menemukan matriks segitiga atas yang sesuai dengan matriks \(A\). Solusi untuk melakukan dekomposisi bisa jadi tak terhingga, namun solusi yang paling sederhana adalah mengubah matriks \(A\) menjadi matriks \emph{row echelon form}. Kedua, \(L\) harus menjadi matriks segitiga bawah yang mereduksi ke-\(l\) dengan mengikuti operasi baris yang sama yag menghasilkan \(U\). Kita dapat menggunakan algoritma Doolittle untuk menghasilkan \(L\), di mana nilai setiap entri dalam matriks segitiga bawah merupakan pengali yang digunakan untuk menghilangkan entri yang sesuai untuk setiap proses \emph{row replacement}.

Pada praktiknya, proses eliminasi Gauss untuk memperoleh matriks \(U\) kadang menghasilkan nol di kolom pivotnya. Kondisi tersebut mengharuskan kita untuk melakukan proses \emph{row swapping} atau pertukaran baris (biasanya dengan baris bawahnya) untuk pivot bukan nol. Jika proses tersebut berhasil dilakukan bisa jadi matriks \(A\) mungkin setara dengan matriks LU, tetapi tidak sama dalam hal urutan nilai pada tiap barisnya. Agar kita dapat memperoleh hasil yang sama (matriks A sama dengan matriks LU), diperlukan matriks ketiga, \(P\). Matriks ini merupakan matriks identitas dengan ukuran sama dengan matriks \(A\). Jika pertukaran baris dilakukan selama proses pembentukan matriks \(U\), maka pertukaran baris yang sama juga akan diimplemenntasikan pada matriks \(P\). oleh karena itu, dalam praktiknya matriks \(A=PLU\) dan perkalian dengan matriks \(P\) berfungsi untuk mengembalikan urutan baris.

\begin{example}
\protect\hypertarget{exm:LUexmp}{}\label{exm:LUexmp}Selesaikan sistem persamaan linier berikut menggunakan faktorisasi LU
\end{example}

\[
\begin{matrix}
  x_1+x_2+3x_4=4 \\
  2x_1+x_2-x_3+x_4=1 \\
  3x_1-x_2-x_3+2x_4=-3 \\
  -x_1-2x_2+3x_3-x_4=4
\end{matrix}
\]

\textbf{Jawab}:

Nayatakan sistem persamaan tersebut ke dalam bentuk matriks \(Ax=b\).

\begin{equation*}
Ux=
\begin{bmatrix}
     1       & 1       & 0       & 3           \\[0.3em]
     2       & 1       & -1      & 1           \\[0.3em]
     3       & -1      & -1      & 2           \\[0.3em]
     -1      & 2       & 3       & -1           
     \end{bmatrix}
\begin{bmatrix}
     x_1                                          \\[0.3em]
     x_2                                          \\[0.3em]
     x_3                                          \\[0.3em]
     x_4                                       
     \end{bmatrix}
= \begin{bmatrix}
     4                                          \\[0.3em]
     1                                          \\[0.3em]
     -3                                          \\[0.3em]
     4                                       
     \end{bmatrix}
\end{equation*}

Lakukan operasi baris elementer pada matriks \(A\) untuk memperoleh matriks \(U\). Urutan operasi baris elementer yang dilakukan adalah sebagai berikut:

\begin{itemize}
\tightlist
\item
  \(\left(B_2-2B_1\right)\to B_2 \to l_{2,1}=2\),
\item
  \(\left(B_3-3B_1\right)\to B_3 \to l_{3,1}=3\),
\item
  \(\left(B_4+B_1\right)\to B_4 \to l_{4,1}=-1\),
\item
  \(\left(B_3-4B_2\right)\to B_3\to l_{3,2}=4\),
\item
  \(\left(B_4+3B_2\right)\to B_4 \to l_{4,2}=-3\),
\item
  \(l_{4,3}=0\)
\end{itemize}

Simpan pengali tiap tahapan pada masing-masing elemen matriks \(L\). Hasil operasi tersebut akan menghasilkan matriks triangular \(U\).

\[
U=
\begin{bmatrix}
     1       & 1       & 0       & 3           \\[0.3em]
     0       & -1      & -1      & -5          \\[0.3em]
     0       & 0       & 3       & 13          \\[0.3em]
     0       & 0       & 0       & -13           
\end{bmatrix}
\]

Untuk matriks \(L\) sebagai berikut:

\[
L=
\begin{bmatrix}
     1         & 0       & 0       & 0           \\[0.3em]
     2         & 1       & 0       & 0           \\[0.3em]
     3         & 4       & 1       & 0           \\[0.3em]
     -1        & -3      & 0       & 1           
\end{bmatrix}
\]

Karena pada proses operasi baris elementer tidak terdapat operasi pertukaran baris, maka matriks \(P\) tidak mengalami perubahan:

\[
P=
\begin{bmatrix}
     1         & 0       & 0       & 0           \\[0.3em]
     0         & 1       & 0       & 0           \\[0.3em]
     0         & 0       & 1       & 0           \\[0.3em]
     0         & 0       & 0       & 1           
     \end{bmatrix}
\]

Lakukan operasi \emph{forward substitution} menggunakan Persamaan \eqref{eq:LUdecomp5}.

\begin{equation*}
\begin{bmatrix}
      1         & 0       & 0       & 0           \\[0.3em]
      2         & 1       & 0       & 0           \\[0.3em]
      3         & 4       & 1       & 0           \\[0.3em]
      -1        & -3      & 0       & 1
     \end{bmatrix}
\begin{bmatrix}
     t_1                                          \\[0.3em]
     t_2                                          \\[0.3em]
     t_3                                          \\[0.3em]
     t_4                                       
     \end{bmatrix}
= \begin{bmatrix}
     4                                          \\[0.3em]
     1                                          \\[0.3em]
     -3                                          \\[0.3em]
     4                                       
     \end{bmatrix}
\end{equation*}

Berdasarkan hasil perhitungan diperoleh nilai vektor \(t\).

\[
t_1=4, t_2=1, t_3=-3, t_4=4
\]

Operasi terakhir yang perlu dilakukan untuk memperoleh nilai \(x\) adalah dengan melakukan \emph{backward substitution} menggunakan nilai vektor \(t\) yang telah dihitung.

\begin{equation*}
\begin{bmatrix}
     1       & 1       & 0       & 3           \\[0.3em]
     0       & -1      & -1      & -5          \\[0.3em]
     0       & 0       & 3       & 13          \\[0.3em]
     0       & 0       & 0       & -13
     \end{bmatrix}
\begin{bmatrix}
     x_1                                          \\[0.3em]
     x_2                                          \\[0.3em]
     x_3                                          \\[0.3em]
     x_4                                       
     \end{bmatrix}
= \begin{bmatrix}
     4                                          \\[0.3em]
     7                                          \\[0.3em]
     13                                          \\[0.3em]
     -13                      
     \end{bmatrix}
\end{equation*}

Berdasarkan hasil perhitungan diperoleh nilai \(x\) sebagai berikut:

\[
x_1=-1, x_2=2, x_3=0, x_4=1
\]

\begin{center}\rule{0.5\linewidth}{0.5pt}\end{center}

\textbf{Algoritma Dekomposisi LU}

\begin{enumerate}
\def\labelenumi{\arabic{enumi}.}
\tightlist
\item
  Masukkan matriks \(A\), dan vektor \(B\) beserta ukurannya \(n\)
\item
  Lakukan langkah poin ke-4 s/d poin 5 untuk meperoleh matriks \(U\).
\item
  Untuk baris ke-\(i\) di mana \(i=1\) s/d \(n\), perhatikan apakah nilai \(a_{i,j}\) sama dengan nol.
\end{enumerate}

\begin{itemize}
\tightlist
\item
  \textbf{Bila iya}, lakukan \emph{row swapping} antara baris ke-\(i\) dan baris ke-\(i+k\leq n\), dimana \(a_{i+k,j}\) tidak sama dengan nol. Bila tidak ada berarti perhitungan tidak bisa dilanjutkan dan proses dihentikan dengan tanpa penyelesaian.
\item
  \textbf{Bila tidak}, lanjutkan.
\end{itemize}

\begin{enumerate}
\def\labelenumi{\arabic{enumi}.}
\setcounter{enumi}{4}
\tightlist
\item
  Untuk baris ke-\(j\), dimana \(j=i+1\) s/d \(n\), lakukan operasi baris elementer:
\end{enumerate}

\begin{itemize}
\tightlist
\item
  Hitung \(c=\frac{a_{j,i}}{a_{i,i}}\)
\item
  untuk kolom \(k\), dimana \(k=1\) s/d \(n+1\), hitung \(a_{j,k}=a_{j,k}-c_i.a_{i,k}\)
\end{itemize}

\begin{enumerate}
\def\labelenumi{\arabic{enumi}.}
\setcounter{enumi}{5}
\tightlist
\item
  Lakukan langkah poin ke-7 s/d poin 9 untuk memperoleh matriks \(L\)
\item
  Untuk diagonal matriks \(L\) isikan dengan nilai 1 dan elemen di atas diagonal dengan nilai nol.
\item
  Untuk elemen di bawah diagonal isikan dengan faktor pengali operasi baris elementer matriks \(U\).
\item
  Lakukan proses \emph{forward substitution} menggunakan Persamaan \eqref{eq:LUdecomp6} untuk memperoleh nilai vektor \(t\).
\item
  Lakukan \emph{backward substituion} menggunakan Persamaan \eqref{eq:refsolution}.
\end{enumerate}

\begin{center}\rule{0.5\linewidth}{0.5pt}\end{center}

Berdasarkan algoritma tersebut, kita dapat menyusun algoritma faktorisasi LU menggunakan \texttt{R}. Berikut adalah sintaks yang digunakan:

\begin{Shaded}
\begin{Highlighting}[]
\NormalTok{lu\_solve }\OtherTok{\textless{}{-}} \ControlFlowTok{function}\NormalTok{(a, }\AttributeTok{b=}\ConstantTok{NULL}\NormalTok{)\{}
\NormalTok{    m }\OtherTok{\textless{}{-}} \FunctionTok{nrow}\NormalTok{(a)}
\NormalTok{    n }\OtherTok{\textless{}{-}} \FunctionTok{ncol}\NormalTok{(a)}
\NormalTok{    piv }\OtherTok{\textless{}{-}} \DecValTok{1}

\CommentTok{\# membentuk matriks identitas P dan L}
\NormalTok{    P }\OtherTok{\textless{}{-}}\NormalTok{ L }\OtherTok{\textless{}{-}} \FunctionTok{diag}\NormalTok{(n)}

\CommentTok{\# cek elemen diagonal utama apakah bernilai nol}
    \ControlFlowTok{for}\NormalTok{(row\_curr }\ControlFlowTok{in} \DecValTok{1}\SpecialCharTok{:}\NormalTok{m)\{}
        \ControlFlowTok{if}\NormalTok{(piv }\SpecialCharTok{\textless{}=}\NormalTok{ n)\{}
\NormalTok{            i }\OtherTok{\textless{}{-}}\NormalTok{ row\_curr}
            \ControlFlowTok{while}\NormalTok{(a[i, piv] }\SpecialCharTok{==} \DecValTok{0} \SpecialCharTok{\&\&}\NormalTok{ i }\SpecialCharTok{\textless{}}\NormalTok{ m)\{}
\NormalTok{                i }\OtherTok{\textless{}{-}}\NormalTok{ i }\SpecialCharTok{+} \DecValTok{1}
                \ControlFlowTok{if}\NormalTok{(i }\SpecialCharTok{\textgreater{}}\NormalTok{ m)\{}
\NormalTok{                    i }\OtherTok{\textless{}{-}}\NormalTok{ row\_curr}
\NormalTok{                    piv }\OtherTok{\textless{}{-}}\NormalTok{ piv }\SpecialCharTok{+} \DecValTok{1}
                    \ControlFlowTok{if}\NormalTok{(piv }\SpecialCharTok{\textgreater{}}\NormalTok{ n)}
                        \FunctionTok{return}\NormalTok{(}\FunctionTok{list}\NormalTok{(}\AttributeTok{P =}\NormalTok{ P, }\AttributeTok{L =}\NormalTok{ L, }\AttributeTok{U =}\NormalTok{ a))}
\NormalTok{                \}}
\NormalTok{            \}}
            
\CommentTok{\# jika elemen diagonal utama bernilai nol,lakukan row swapping}
            \ControlFlowTok{if}\NormalTok{(i }\SpecialCharTok{!=}\NormalTok{ row\_curr)\{}
\NormalTok{                a }\OtherTok{\textless{}{-}} \FunctionTok{swap\_row}\NormalTok{(a, i, row\_curr)}
\NormalTok{                P }\OtherTok{\textless{}{-}} \FunctionTok{swap\_row}\NormalTok{(P, i, row\_curr)}
\NormalTok{            \}}
            
  \CommentTok{\# pembentukan matriks L dan U}
            \ControlFlowTok{for}\NormalTok{(j }\ControlFlowTok{in}\NormalTok{ row\_curr}\SpecialCharTok{:}\NormalTok{m)}
                \ControlFlowTok{if}\NormalTok{(j }\SpecialCharTok{!=}\NormalTok{ row\_curr)\{}
\NormalTok{                    k }\OtherTok{\textless{}{-}}\NormalTok{ a[j, piv]}\SpecialCharTok{/}\NormalTok{a[row\_curr, piv]}
                    
  \CommentTok{\# matriks U}
\NormalTok{                    a }\OtherTok{\textless{}{-}} \FunctionTok{replace\_row}\NormalTok{(a, row\_curr, j, }\SpecialCharTok{{-}}\NormalTok{k)}
                    
  \CommentTok{\# pengisian elemen matriks L}
\NormalTok{                    L[j, piv] }\OtherTok{\textless{}{-}}\NormalTok{ k}
\NormalTok{                \}}
\NormalTok{            piv }\OtherTok{\textless{}{-}}\NormalTok{ piv }\SpecialCharTok{+} \DecValTok{1}
\NormalTok{        \}}
\NormalTok{    \}}
    
\CommentTok{\# penyelesaian persamaan linier}
    \ControlFlowTok{if}\NormalTok{(}\FunctionTok{is.null}\NormalTok{(b))\{}
      \FunctionTok{return}\NormalTok{(}\FunctionTok{list}\NormalTok{(}\AttributeTok{P =}\NormalTok{ P, }\AttributeTok{L =}\NormalTok{ L, }\AttributeTok{U =}\NormalTok{ a))}
\NormalTok{    \}}\ControlFlowTok{else}\NormalTok{\{}
      
      \CommentTok{\# forward substitution}
\NormalTok{      t }\OtherTok{\textless{}{-}} \FunctionTok{forwardsolve}\NormalTok{(L, b)}
      
      \CommentTok{\# backward substitution}
\NormalTok{      x }\OtherTok{\textless{}{-}} \FunctionTok{backsolve}\NormalTok{(a, t)}
      \FunctionTok{return}\NormalTok{(}\FunctionTok{list}\NormalTok{(}\AttributeTok{P =}\NormalTok{ P, }\AttributeTok{L =}\NormalTok{ L, }\AttributeTok{U =}\NormalTok{ a, }\AttributeTok{result=}\NormalTok{x))}
\NormalTok{     \}}
\NormalTok{\}}
\end{Highlighting}
\end{Shaded}

Kita dapat menyelesaikan sistem persamaan linier pada Contoh \ref{exm:LUexmp} menggunakan fungsi yang telah kita buat.

\begin{Shaded}
\begin{Highlighting}[]
\CommentTok{\# membuat matriks a dan vektor b}
\NormalTok{a }\OtherTok{\textless{}{-}} \FunctionTok{matrix}\NormalTok{(}\FunctionTok{c}\NormalTok{(}\DecValTok{1}\NormalTok{,}\DecValTok{2}\NormalTok{,}\DecValTok{3}\NormalTok{,}\SpecialCharTok{{-}}\DecValTok{1}\NormalTok{,}\DecValTok{1}\NormalTok{,}\DecValTok{1}\NormalTok{,}\SpecialCharTok{{-}}\DecValTok{1}\NormalTok{,}\DecValTok{2}\NormalTok{,}
              \DecValTok{0}\NormalTok{,}\SpecialCharTok{{-}}\DecValTok{1}\NormalTok{,}\SpecialCharTok{{-}}\DecValTok{1}\NormalTok{,}\DecValTok{3}\NormalTok{,}\DecValTok{3}\NormalTok{,}\DecValTok{1}\NormalTok{,}\DecValTok{2}\NormalTok{,}\SpecialCharTok{{-}}\DecValTok{1}\NormalTok{),}
            \AttributeTok{nrow=}\DecValTok{4}\NormalTok{)}
\NormalTok{b }\OtherTok{\textless{}{-}} \FunctionTok{c}\NormalTok{(}\DecValTok{4}\NormalTok{,}\DecValTok{1}\NormalTok{,}\SpecialCharTok{{-}}\DecValTok{3}\NormalTok{,}\DecValTok{4}\NormalTok{)}

\CommentTok{\# penyelesaian}
\NormalTok{decomp}\OtherTok{\textless{}{-}}\FunctionTok{lu\_solve}\NormalTok{(a,b)}
\end{Highlighting}
\end{Shaded}

Untuk membentuk kembali matriks \(A\), kita dapat mengalikan matriks \(L\), \(U\), dan \(P\).

\begin{Shaded}
\begin{Highlighting}[]
\NormalTok{decomp}\SpecialCharTok{$}\NormalTok{L}\SpecialCharTok{\%*\%}\NormalTok{decomp}\SpecialCharTok{$}\NormalTok{U}\SpecialCharTok{\%*\%}\NormalTok{decomp}\SpecialCharTok{$}\NormalTok{P}
\end{Highlighting}
\end{Shaded}

\begin{verbatim}
##      [,1] [,2] [,3] [,4]
## [1,]    1    1    0    3
## [2,]    2    1   -1    1
## [3,]    3   -1   -1    2
## [4,]   -1    2    3   -1
\end{verbatim}

\begin{example}
\protect\hypertarget{exm:LUexmp2}{}\label{exm:LUexmp2}Lakukan dekomposisi LU pada matriks berikut dan lakukan pengecekan apakah perkalian hasil dekomposisi matriks akan menghasilkan matriks semula!
\end{example}

\[
\begin{bmatrix}
     0         & 1       & -1                  \\[0.3em]
     1         & 5       & 9                   \\[0.3em]
     7         & -1      & -5                  
\end{bmatrix}
\]

\textbf{Jawab}:

Lakukan proses dekomposisi menggunakan fungsi \texttt{lu\_solve()}.

\begin{Shaded}
\begin{Highlighting}[]
\CommentTok{\# membentuk matriks a}
\NormalTok{(A }\OtherTok{\textless{}{-}} \FunctionTok{matrix}\NormalTok{(}\FunctionTok{c}\NormalTok{(}\DecValTok{0}\NormalTok{, }\DecValTok{1}\NormalTok{, }\DecValTok{7}\NormalTok{, }\DecValTok{1}\NormalTok{, }\DecValTok{5}\NormalTok{, }\SpecialCharTok{{-}}\DecValTok{1}\NormalTok{, }\SpecialCharTok{{-}}\DecValTok{2}\NormalTok{, }\DecValTok{9}\NormalTok{, }\SpecialCharTok{{-}}\DecValTok{5}\NormalTok{), }\DecValTok{3}\NormalTok{))}
\end{Highlighting}
\end{Shaded}

\begin{verbatim}
##      [,1] [,2] [,3]
## [1,]    0    1   -2
## [2,]    1    5    9
## [3,]    7   -1   -5
\end{verbatim}

\begin{Shaded}
\begin{Highlighting}[]
\CommentTok{\# dekomposisi lu}
\NormalTok{decomp}\OtherTok{\textless{}{-}}\FunctionTok{lu\_solve}\NormalTok{(A)}
\end{Highlighting}
\end{Shaded}

Lakukan pengecekan apakah matriks hasil dekomposisi akan menghasilkan matriks \(A\).

\begin{Shaded}
\begin{Highlighting}[]
\NormalTok{decomp}\SpecialCharTok{$}\NormalTok{P }\SpecialCharTok{\%*\%}\NormalTok{ decomp}\SpecialCharTok{$}\NormalTok{L }\SpecialCharTok{\%*\%}\NormalTok{ decomp}\SpecialCharTok{$}\NormalTok{U}
\end{Highlighting}
\end{Shaded}

\begin{verbatim}
##      [,1] [,2] [,3]
## [1,]    0    1   -2
## [2,]    1    5    9
## [3,]    7   -1   -5
\end{verbatim}

Fungsi \texttt{lu()} pada Paket \texttt{Matrix} dapat digunakan untuk melakukan dekomposisi LU. Untuk meggunakan fungsi tersebut, kita harus menginstall dan mengaktifkan Paket \texttt{Matrix}.

\begin{Shaded}
\begin{Highlighting}[]
\FunctionTok{install.packages}\NormalTok{(}\StringTok{"Matrix"}\NormalTok{)}
\end{Highlighting}
\end{Shaded}

\begin{Shaded}
\begin{Highlighting}[]
\FunctionTok{library}\NormalTok{(Matrix)}
\end{Highlighting}
\end{Shaded}

Untuk dapat menggunakannya kita hanya perlu menginputkan matriks kedalam fungsi tersebut. Berikut adalah contoh penerapannya:

\begin{Shaded}
\begin{Highlighting}[]
\CommentTok{\# membuat matriks a }
\NormalTok{a }\OtherTok{\textless{}{-}}\NormalTok{ Matrix}\SpecialCharTok{::}\FunctionTok{Matrix}\NormalTok{(}\FunctionTok{round}\NormalTok{(}\FunctionTok{rnorm}\NormalTok{(}\DecValTok{9}\NormalTok{),}\DecValTok{2}\NormalTok{), }\AttributeTok{nrow=}\DecValTok{3}\NormalTok{)}

\CommentTok{\# dekomposisi}
\NormalTok{lum }\OtherTok{\textless{}{-}}\NormalTok{ Matrix}\SpecialCharTok{::}\FunctionTok{lu}\NormalTok{(a)}
\NormalTok{lum}
\end{Highlighting}
\end{Shaded}

\begin{verbatim}
## 'MatrixFactorization' of Formal class 'denseLU' [package "Matrix"] with 4 slots
##   ..@ Dimnames:List of 2
##   .. ..$ : NULL
##   .. ..$ : NULL
##   ..@ x       : num [1:9] -1.33 0.586 -0.977 0.08 -1.547 ...
##   ..@ perm    : int [1:3] 2 2 3
##   ..@ Dim     : int [1:2] 3 3
\end{verbatim}

Untuk menampilkan hasil dekomposisi, jalankan fungsi \texttt{expand()}.

\begin{Shaded}
\begin{Highlighting}[]
\NormalTok{decomp }\OtherTok{\textless{}{-}}\NormalTok{ Matrix}\SpecialCharTok{::}\FunctionTok{expand}\NormalTok{(lum)}
\NormalTok{decomp}
\end{Highlighting}
\end{Shaded}

\begin{verbatim}
## $L
## 3 x 3 Matrix of class "dtrMatrix" (unitriangular)
##      [,1]    [,2]    [,3]   
## [1,]  1.0000       .       .
## [2,]  0.5865  1.0000       .
## [3,] -0.9774 -0.7810  1.0000
## 
## $U
## 3 x 3 Matrix of class "dtrMatrix"
##      [,1]   [,2]   [,3]  
## [1,] -1.330  0.080  1.470
## [2,]      . -1.547 -1.692
## [3,]      .      .  1.775
## 
## $P
## 3 x 3 sparse Matrix of class "pMatrix"
##           
## [1,] . | .
## [2,] | . .
## [3,] . . |
\end{verbatim}

\hypertarget{dekomposisi-cholesky}{%
\subsection{Dekomposisi Cholesky}\label{dekomposisi-cholesky}}

Dekomposisi Cholesky memberikan faktorisasi matriks alternatif sehingga \(A = LL^T\), di mana \(L^T\) merupakan transpose konjugat dari matriks \(L\). Dalam kasus ini, penulis hanya bekerja dengan matriks rill dengan nilai rill dan bagian imajiner nol. Jadi untuk tujuan \emph{sub-chapter} ini, matriks \(L^T\) hanyalah transpose dari matriks \(L\).

Seperti dekomposisi LU, dekomposisi Cholesky dapat digunakan untuk menyelesaikan sistem persamaan linier. Kelebihannya, Menemukan dekomposisi Cholesky jauh lebih cepat daripada dekomposisi LU. Namun, dekomposisi ini hanya terbatas pada matriks tertentu saja. Dekomposisi Cholesky hanya dapat digunakan pada matriks definit positif dan simetris. Matriks simeteris merupakan matriks yang nilai di atas dan di bawah diagonalnya simetris atau sama; secara matematis, untuk semua \(i\) dan \(j\) pada matriks \(A\), \(a_{i;j}=a_{j;i}\). Definit positif berarti bahwa setiap entri pivot (nilai elemen diagonal utama) selelu bernilai positif. Selain itu, untuk matriks definit positif, hubungan \(xAx>0\) untuk semua vektor, \(x\).

Karena \(L^∗\) transpose dari matriks \(L\), maka \(l^{T}_{i,j} = l_{j,i}\) untuk semua nilai \(i\) dan \(j\). Tanpa kendala (\emph{constraint}) ini, dekomposisi Cholesky akan mirip dekomposisi LU. Tetapi dengan kendala ini, nilai elemen matriks \(L\) dan \(L^T\) harus dipilih dengan cermat sehingga hubungan \(A = LL^T\) berlaku. Bentuk dekomposisi Cholesky disajikan pada Persamaan \eqref{eq:cholesky}.

\begin{equation}
\begin{bmatrix}
     a_{1,1} & a_{1,2} & a_{1,3} &\cdots& a_{1,m}           \\[0.3em]
     a_{2,1} & a_{2,2} & a_{2,3} &\cdots& a_{2,m}           \\[0.3em]
     a_{3,1} & a_{3,2} & a_{3,3} &\cdots& a_{3,m}           \\[0.3em]
     \vdots  & \vdots  & \vdots  &\ddots& \vdots            \\[0.3em]
     a_{m,1} & a_{m,2} & a_{m,3} &\cdots& a_{m,m}
     \end{bmatrix}
=
\begin{bmatrix}
     l_{1,1} & 0       & 0       &\cdots& 0           \\[0.3em]
     l_{2,1} & l_{2,2} & 0       &\cdots& 0           \\[0.3em]
     l_{3,1} & l_{3,2} & l_{3,3} &\cdots& 0           \\[0.3em]
     \vdots  & \vdots  & \vdots  &\ddots& \vdots            \\[0.3em]
     l_{m,1} & l_{m,2} & l_{m,3} &\cdots& l_{m,m}
     \end{bmatrix}
\begin{bmatrix}
     l_{1,1} & l_{1,2} & l_{1,3} &\cdots& l_{1,m}           \\[0.3em]
     0       & l_{2,2} & l_{2,3} &\cdots& l_{2,m}           \\[0.3em]
     0       & 0       & l_{3,3} &\cdots& l_{3,m}           \\[0.3em]
     \vdots  & \vdots  & \vdots  &\ddots& \vdots            \\[0.3em]
     0       & 0       & 0       &\cdots& l_{m,m}
     \end{bmatrix}
  \label{eq:cholesky}
\end{equation}

Untuk setiap elemen matriks \(A\) memiliki hubungan yang dituliskan pada Persamaan \eqref{eq:cholesky2}.

\begin{equation}
a_{i,j}=\sum_{k=1}^mL_{i,k}L_{k,j}
  \label{eq:cholesky2}
\end{equation}

Berdasarkan Persamaan \eqref{eq:cholesky}, sejumlah nilai elemen \(L_{i,k}\) dan \(L_{k,j}\) adalah nol. Nilai tiap elemen diagonal utama yang tidak bernilai nol dihitung menggunakan Persamaan \eqref{eq:cholesky3}.

\begin{equation}
l_{i,i}=\sqrt{\left(a_{i,i}-\sum_{k=1}^{i-1}l_{i,k}^2\right)}
  \label{eq:cholesky3}
\end{equation}

Elemen diagonal dihitung menggunakan Persamaan \eqref{eq:cholesky5}

\begin{equation}
l_{i,j}=\frac{1}{l_{i,i}}\left(a_{i,j}-\sum_{k=1}^{i-1}l_{i,k}l_{j,k}\right)
  \label{eq:cholesky5}
\end{equation}

\begin{center}\rule{0.5\linewidth}{0.5pt}\end{center}

\textbf{Algoritma Dekomposisi Cholesky}

\begin{enumerate}
\def\labelenumi{\arabic{enumi}.}
\tightlist
\item
  Masukkan matriks \(A\), dan vektor \(B\) beserta ukurannya \(n\).
\item
  Untuk elemen matriks \(L\), hitung menggunakan Persamaan \eqref{eq:cholesky5}.
\item
  Untuk nilai diagonal utama matriks \(L\), hitung menggunakan Persamaan \eqref{eq:cholesky3}.
\item
  Untuk memperoleh matriks \(L^T\), lakukan transpose pada matriks \(L\).
\item
  Untuk memperoleh nilai \(x\),
\end{enumerate}

\begin{itemize}
\tightlist
\item
  Hitung vektor \(t\) menggunakan Persamaan \eqref{eq:LUdecomp4}.
\item
  Hitung vektor \(x\) menggunakan Persamaan \eqref{eq:LUdecomp3}, dimana matriks \(U=L^T\).
\end{itemize}

\begin{center}\rule{0.5\linewidth}{0.5pt}\end{center}

Berdasarkan algoritma tersebut, kita dapat menyusun fungsi pada \texttt{R} untuk melakukan dekomposisi Cholesky. Fungsi tersebut disajikan pada sintaks berikut:

\begin{Shaded}
\begin{Highlighting}[]
\NormalTok{cholesky\_solve }\OtherTok{\textless{}{-}} \ControlFlowTok{function}\NormalTok{(a, }\AttributeTok{b=}\ConstantTok{NULL}\NormalTok{)\{}
\NormalTok{    m }\OtherTok{\textless{}{-}} \FunctionTok{nrow}\NormalTok{(a)}
   
\CommentTok{\# membentuk matriks L dengan elemen nol}
\NormalTok{    L }\OtherTok{=} \FunctionTok{diag}\NormalTok{(}\DecValTok{0}\NormalTok{,m)}

\CommentTok{\# Perhitungan elemen matriks L}
    \ControlFlowTok{for}\NormalTok{(i }\ControlFlowTok{in} \DecValTok{1}\SpecialCharTok{:}\NormalTok{m)\{}
        \ControlFlowTok{for}\NormalTok{(k }\ControlFlowTok{in} \DecValTok{1}\SpecialCharTok{:}\NormalTok{i)\{}
\NormalTok{            p\_sum }\OtherTok{\textless{}{-}} \DecValTok{0}
            \ControlFlowTok{for}\NormalTok{(j }\ControlFlowTok{in} \DecValTok{1}\SpecialCharTok{:}\NormalTok{k)}
\NormalTok{                p\_sum }\OtherTok{\textless{}{-}}\NormalTok{ p\_sum }\SpecialCharTok{+}\NormalTok{ L[j,i]}\SpecialCharTok{*}\NormalTok{L[j,k]}
                
\CommentTok{\# Pehitungan elemen diagonal utama}
            \ControlFlowTok{if}\NormalTok{(i}\SpecialCharTok{==}\NormalTok{k)}
\NormalTok{                L[k,i]}\OtherTok{\textless{}{-}}\FunctionTok{sqrt}\NormalTok{(a[i,i]}\SpecialCharTok{{-}}\NormalTok{p\_sum)}
            \ControlFlowTok{else}
\NormalTok{                L[k,i]}\OtherTok{\textless{}{-}}\NormalTok{(a[k,i]}\SpecialCharTok{{-}}\NormalTok{p\_sum)}\SpecialCharTok{/}\NormalTok{L[k,k]}
\NormalTok{        \}}
\NormalTok{    \}}
    
\CommentTok{\# Perhitungan elemn matriks L*}
\NormalTok{    tL }\OtherTok{\textless{}{-}} \FunctionTok{t}\NormalTok{(L)}
    
\CommentTok{\# penyelesaian persamaan linier}
    \ControlFlowTok{if}\NormalTok{(}\FunctionTok{is.null}\NormalTok{(b))\{}
      \FunctionTok{return}\NormalTok{(}\FunctionTok{list}\NormalTok{(}\AttributeTok{L =}\NormalTok{ L, }\AttributeTok{tL =}\NormalTok{ tL, }\AttributeTok{a =}\NormalTok{ a))}
\NormalTok{    \}}\ControlFlowTok{else}\NormalTok{\{}
      
      \CommentTok{\# forward substitution}
\NormalTok{      t }\OtherTok{\textless{}{-}} \FunctionTok{forwardsolve}\NormalTok{(L, b)}
      
      \CommentTok{\# backward substitution}
\NormalTok{      x }\OtherTok{\textless{}{-}} \FunctionTok{backsolve}\NormalTok{(tL, t)}
      \FunctionTok{return}\NormalTok{(}\FunctionTok{list}\NormalTok{(}\AttributeTok{L =}\NormalTok{ L, }\AttributeTok{tL =}\NormalTok{ tL, }\AttributeTok{a =}\NormalTok{ a, }\AttributeTok{result=}\NormalTok{x))}
\NormalTok{     \}}
\NormalTok{\}}
\end{Highlighting}
\end{Shaded}

\begin{example}
\protect\hypertarget{exm:cholexm2}{}\label{exm:cholexm2}Dengan menggunakan fungsi \texttt{cholesky\_solve()}, lakukan dekomposisi pada matriks berikut! Lakukan pengecekan pada hasil dekomposisi apakah hasil kali matriks dekomposisi akan menghasilkan matriks semula!
\end{example}

\[
\begin{bmatrix}
     9         & -3       & 6                  \\[0.3em]
     -3         & 17       & -10                   \\[0.3em]
     6         & -10      & 12                  
\end{bmatrix}
\]

\textbf{Jawab}:

Dekomposisi Cholesky menggunakan fungsi \texttt{cholesky\_solve()}, disajikan pada sintaks berikut:

\begin{Shaded}
\begin{Highlighting}[]
\NormalTok{a }\OtherTok{\textless{}{-}} \FunctionTok{matrix}\NormalTok{(}\FunctionTok{c}\NormalTok{(}\DecValTok{9}\NormalTok{,}\SpecialCharTok{{-}}\DecValTok{3}\NormalTok{,}\DecValTok{6}\NormalTok{,}\SpecialCharTok{{-}}\DecValTok{3}\NormalTok{,}\DecValTok{17}\NormalTok{,}\SpecialCharTok{{-}}\DecValTok{10}\NormalTok{,}\DecValTok{6}\NormalTok{,}\SpecialCharTok{{-}}\DecValTok{10}\NormalTok{,}\DecValTok{12}\NormalTok{),}\DecValTok{3}\NormalTok{)}

\CommentTok{\# dekomposisi Cholesky}
\NormalTok{(decomp}\OtherTok{\textless{}{-}}\FunctionTok{cholesky\_solve}\NormalTok{(a))}
\end{Highlighting}
\end{Shaded}

\begin{verbatim}
## $L
##      [,1] [,2] [,3]
## [1,]    3   -1    2
## [2,]    0    4   -2
## [3,]    0    0    2
## 
## $tL
##      [,1] [,2] [,3]
## [1,]    3    0    0
## [2,]   -1    4    0
## [3,]    2   -2    2
## 
## $a
##      [,1] [,2] [,3]
## [1,]    9   -3    6
## [2,]   -3   17  -10
## [3,]    6  -10   12
\end{verbatim}

\begin{Shaded}
\begin{Highlighting}[]
\CommentTok{\# mengecek hasil dekomposisi}
\NormalTok{decomp}\SpecialCharTok{$}\NormalTok{tL }\SpecialCharTok{\%*\%}\NormalTok{ decomp}\SpecialCharTok{$}\NormalTok{L}
\end{Highlighting}
\end{Shaded}

\begin{verbatim}
##      [,1] [,2] [,3]
## [1,]    9   -3    6
## [2,]   -3   17  -10
## [3,]    6  -10   12
\end{verbatim}

Fungsi lain yang dapat digunakan untuk melakukan dekomposisi Cholesky adalah menggunakan fungsi \texttt{chol()} pada Paket \texttt{Matrix}. Pada fungsi tersebut, kita hanya perlu menginputkan objek matrik kedalamnya. Berikut adalah contoh penerapan fungsi tersebut menggunakan matriks pada Contoh \ref{exm:cholexm2}.

\begin{Shaded}
\begin{Highlighting}[]
\FunctionTok{chol}\NormalTok{(a)}
\end{Highlighting}
\end{Shaded}

\begin{verbatim}
##      [,1] [,2] [,3]
## [1,]    3   -1    2
## [2,]    0    4   -2
## [3,]    0    0    2
\end{verbatim}

\begin{quote}
\textbf{Penting!!!}

Fungsi \texttt{chol()} hanya menampilkan matriks \(L^T\). Untuk menampilkan matriks \(L\), kita perlu melakukan transpose
\end{quote}

\hypertarget{othersdecomp}{%
\subsection{Dekomposisi Lainnya}\label{othersdecomp}}

Terdapat beberapa algoritma lain yang telah dikembangkan untuk melakukan dekomposisi matriks. Pada buku ini hanya akan dijelaskan secara singkat terkait fungsi yang digunakan dalam melakukan dekomposisi matriks. Algoritma yang akan dijelaskan pada \emph{sub-chapter} ini antara lain: QR, \emph{singular value decomposition} (SVD), dan dekomposisi eigen. Untuk algoritma lainnya, pembaca dapat membaca buku terkait atau mengecek dokumentasinya pada Paket \texttt{base}.

\hypertarget{qrdecomp}{%
\subsubsection{Dekomposisi QR}\label{qrdecomp}}

Dekomposisi QR merupakan dekomposisi yang penting dalam menyelesaikan sistem persamaan linier. Dekomposisi ini juga berperan penting untuk menghitung koefisien regresi dan pengaplikasian algoritma Newton-Raphson.

Untuk memperoleh informasi terkait dekomposisi ini, pembaca dapat mengetikkan sintaks berikut pada \texttt{R}:

\begin{Shaded}
\begin{Highlighting}[]
\NormalTok{?qr}
\end{Highlighting}
\end{Shaded}

Berikut merupakan contoh penerapan fungsi \texttt{qr()} untuk menyelesaikan sistem persamaan linier:

\begin{Shaded}
\begin{Highlighting}[]
\CommentTok{\# membuat matriks A dan B}
\FunctionTok{set.seed}\NormalTok{(}\DecValTok{123}\NormalTok{)}
\NormalTok{A }\OtherTok{\textless{}{-}} \FunctionTok{matrix}\NormalTok{((}\DecValTok{1}\SpecialCharTok{:}\DecValTok{12}\NormalTok{)}\SpecialCharTok{+}\FunctionTok{rnorm}\NormalTok{(}\DecValTok{12}\NormalTok{), }\AttributeTok{nrow=}\DecValTok{4}\NormalTok{)}
\NormalTok{b }\OtherTok{\textless{}{-}} \DecValTok{2}\SpecialCharTok{:}\DecValTok{5}

\CommentTok{\# dekomposisi matriks A}
\FunctionTok{qr}\NormalTok{(A)}
\end{Highlighting}
\end{Shaded}

\begin{verbatim}
## $qr
##         [,1]     [,2]     [,3]
## [1,] -6.3778 -12.1257 -19.8501
## [2,]  0.2775  -6.3105  -7.9392
## [3,]  0.7148  -0.6461   2.3512
## [4,]  0.6382  -0.5654   0.2767
## 
## $rank
## [1] 3
## 
## $qraux
## [1] 1.069 1.513 1.961
## 
## $pivot
## [1] 1 2 3
## 
## attr(,"class")
## [1] "qr"
\end{verbatim}

\begin{Shaded}
\begin{Highlighting}[]
\CommentTok{\# memperoleh penyelesaian SPL}
\FunctionTok{qr.solve}\NormalTok{(A,b)}
\end{Highlighting}
\end{Shaded}

\begin{verbatim}
## [1]  0.3046 -0.1111  0.3237
\end{verbatim}

\hypertarget{svddecomp}{%
\subsubsection{\texorpdfstring{\emph{Singular Value Decomposition}}{Singular Value Decomposition}}\label{svddecomp}}

\emph{Singular value decomposition} (SVD) merupakan algoritma faktorisasi matriks yang mendekomposisi matriks segiempat menjadi matriks \(UDV_H\), dimana \(D\) merupakan mmatriks diagonal non negatif, \(U\) dan \(V\) merupakan matriks \emph{unitary}, dan \(V_H\) merupakan matriks tanspose konjugat dari matriks \(V\). Algoritma ini banyak digunakan dalam analisis \emph{principal component}.

Pada \texttt{R}, SVD dapat dilakukan menggunakan fungsi \texttt{svd()} dari Paket \texttt{base}. Berikut adalah sintaks untuk memperoleh informasi terkait fungsi tersebut:

\begin{Shaded}
\begin{Highlighting}[]
\NormalTok{?svd}
\end{Highlighting}
\end{Shaded}

Berikut adalah contoh penerapan fungsi \texttt{svd()}:

\begin{Shaded}
\begin{Highlighting}[]
\CommentTok{\# dekomposisi matriks A}
\FunctionTok{svd}\NormalTok{(A)}
\end{Highlighting}
\end{Shaded}

\begin{verbatim}
## $d
## [1] 26.094  2.727  1.330
## 
## $u
##         [,1]    [,2]    [,3]
## [1,] -0.3685 -0.5661  0.6651
## [2,] -0.4707 -0.5703 -0.6113
## [3,] -0.5740  0.4324 -0.2590
## [4,] -0.5596  0.4089  0.3419
## 
## $v
##         [,1]     [,2]    [,3]
## [1,] -0.2257  0.87169 -0.4350
## [2,] -0.5202 -0.48536 -0.7028
## [3,] -0.8237  0.06764  0.5630
\end{verbatim}

\hypertarget{eigendecomp}{%
\subsubsection{Dekomposisi Eigen}\label{eigendecomp}}

Proses umum yang digunakan untuk menemukan nilai eigen dan vektor eigen suatu matriks segiempat dapat dilihat sebagai proses dari dekomposisi eigen. Proses ini akan mendekomposisi matriks menjadi \(VDV^{-1}\), dimana \(D\) merupakan matriks diagonal yang terbentuk dari nilai eigen, dan \(V\) merupakan vektor eigen. Proses dekomposisi ini akan berguna bagi pembaca yang ingin mempelajari \emph{principal component analysis}.

Fungsi \texttt{eigen()} pada Paket \texttt{base} dapat digunakan untuk melakukan dekomposisi eigen. Untuk mempelajari lebih jauh terkait fungsi ini, pambaca dapat menjalankan sintaks berikut:

\begin{Shaded}
\begin{Highlighting}[]
\NormalTok{?eigen}
\end{Highlighting}
\end{Shaded}

Berikut adalah contoh sintaks untuk melakukan dekomposisi eigen:

\begin{Shaded}
\begin{Highlighting}[]
\NormalTok{A }\OtherTok{\textless{}{-}} \FunctionTok{matrix}\NormalTok{(}\FunctionTok{c}\NormalTok{(}\DecValTok{2}\NormalTok{,}\SpecialCharTok{{-}}\DecValTok{1}\NormalTok{,}\DecValTok{0}\NormalTok{,}\SpecialCharTok{{-}}\DecValTok{1}\NormalTok{,}\DecValTok{2}\NormalTok{,}\SpecialCharTok{{-}}\DecValTok{1}\NormalTok{,}\DecValTok{0}\NormalTok{,}\SpecialCharTok{{-}}\DecValTok{1}\NormalTok{,}\DecValTok{2}\NormalTok{), }\AttributeTok{nrow=}\DecValTok{3}\NormalTok{)}

\CommentTok{\# dekomposisi matriks A}
\FunctionTok{eigen}\NormalTok{(A)}
\end{Highlighting}
\end{Shaded}

\begin{verbatim}
## eigen() decomposition
## $values
## [1] 3.4142 2.0000 0.5858
## 
## $vectors
##         [,1]       [,2]   [,3]
## [1,] -0.5000 -7.071e-01 0.5000
## [2,]  0.7071  1.099e-15 0.7071
## [3,] -0.5000  7.071e-01 0.5000
\end{verbatim}

\hypertarget{iteratif}{%
\section{Metode Iterasi}\label{iteratif}}

Pada Chapter \ref{iteratif} kita akan membahas penyelesaian persamaan linier dengan menggunakan metode iterasi. Terdapat dua metode iterasi yang akan dibahas yaitu iterasi Jacobi dan Gauss-Seidel.

Metode iterasi dimulai dengan estimasi nilai akhir. Setelah menerapkan beberapa perlakuan pada nilai estimasi, hasil perlakuan selanjutnya menjadi nilai estimasi untuk iterasi berikutnya. Proses tersebut akan berlangsung secara terus-menerus hingga ambang batas dipenuhi. Nilai ambang batas dapat berupa jumlah iterasi maksimum atau selisih antara nilai estimasi baru dan estimasi semula lebih kecil dari suatu nilai toleransi yang ditetapkan.

Jumlah kuadrat merupakan metode yang sering digunakan untuk mengecek apakah selisih nilai estimasi baru terhadap estimasi lama lebih kecil dari nilai toleransi yang ditetapkan. Persamaan \eqref{eq:toleransi} menampilkan hubungan antara jumlah kuadrat dan nilai toleransi pada proses iterasi.

\begin{equation}
\sqrt{\sum_{i=1}^n\left(x_i^{n+1}-x_i^{n}\right)^2}<t_0
 \label{eq:toleransi}
\end{equation}

dimana \(x^{n}\) merupakan iterasi ke-\(n\) dari algoritma dan \(t_0\) merupakan nilai toleransi maksimum yang diterima.

\hypertarget{jacobiiter}{%
\subsection{Iterasi Jacobi}\label{jacobiiter}}

Untuk menyelesaikan matriks menggunakan metode iterasi, kita dapat mulai dengan premis terdapat matriks \(A\) dan vektor \(x\) dan b, sehingga \(Ax = b\). Dengan menggunakan metode Jacobi, pertama-tama kita dapat amati bahwa terdapat matriks \(R\) dan \(D\) yang memiliki hubungan \(A = R + D\). Berdasarkan kedua hubungan tersebut, dapat diturunkan operasi matriks melalui persamaan berikut:

\begin{equation}
Ax=b
 \label{eq:jacobi}
\end{equation}

\begin{equation}
Rx+Dx=b
 \label{eq:jacobi2}
\end{equation}

\begin{equation}
Dx=b-Rx
 \label{eq:jacobi3}
\end{equation}

\begin{equation}
x=D^{-1}\left(b-Rx\right)
 \label{eq:jacobi4}
\end{equation}

Persamaan \eqref{eq:jacobi4} merupakan persamaan yang dapat kita gunakan untuk memperoleh nilai \(x\). Jika kita menulis kembali persamaan tersebut, maka kita akan memperoleh persamaan yang digunakan sebagai acuan iterasi Jacobi.

\begin{equation}
x^{n+1}=D^{-1}\left(b-Rx^{n}\right)
 \label{eq:jacobi5}
\end{equation}

dimana \(D\) merupakan matriks diagonal dengan nilai elemen diagonal berupa diagonal utama matriks \(A\). Invers dari matriks \(D\) secara sederhana sebagai matriks diagonal sama dengan satu dibagi dengan elemen diagonal utama matriks \(A\). Matriks \(R\) identik dengan matriks \(A\). Namun, diagonal utamanya bernilai nol. Suatu iterasi dikatakan konvergen jika jumlah kuadrat dari vektor \(x^{\left(n+1\right)}\) dan vektor \(x^{\left(n\right)}\) semakin mengecil.

Suatu persamaan linier yang hendak diselesaikan dengan menggunakan metode iterasi Jacobi harus memenuhi syarat nilai elemen diagonal utama matriks harus lebih dominan. Maksudnya adalah nilai absolut diagonal utama matriks harus lebih besar dari jumlah nilai absolut elemen matriks lainnya pada satu kolom.

\begin{example}
\protect\hypertarget{exm:jacobiexm}{}\label{exm:jacobiexm}Selesaikan sistem persamaan berikut menggunakan iterasi Jacobi!
\end{example}

\begin{equation*}
\begin{bmatrix}
     5 & 2 & 3     \\[0.3em]
     2 & 7 & 4     \\[0.3em]
     1 & 3 & 8
\end{bmatrix}
x = \begin{bmatrix}
     40     \\[0.3em]
     39     \\[0.3em]
     55
\end{bmatrix}
\end{equation*}

\textbf{Jawab}:

Berdasarkan matriks \(A\) (matriks koefisien), kita dapat memastikan bahwa matriks tersebut memiliki nilai dominan pada elemen diagonal utama. Sebagai contoh:

\[
\left|5\right|>\left|2\right|+\left|1\right|\ \ \ \left(kolom\ 1\right)
\]

\[
\left|7\right|>\left|2\right|+\left|3\right|\ \ \ \left(kolom\ 2\right)
\]
Untuk mempermudah proses iterasi, kita akan menggunakan bantuan \texttt{R} untuk melakukan komputasi. Langkah pertama yang perlu dilakukan adalah menyiapkan matriks \(A\), vektor \(b\), dan vektor \(x\) (nilai taksiran awal).

\begin{Shaded}
\begin{Highlighting}[]
\NormalTok{(A }\OtherTok{\textless{}{-}} \FunctionTok{matrix}\NormalTok{(}\FunctionTok{c}\NormalTok{(}\DecValTok{5}\NormalTok{,}\DecValTok{2}\NormalTok{,}\DecValTok{1}\NormalTok{,}\DecValTok{2}\NormalTok{,}\DecValTok{7}\NormalTok{,}\DecValTok{3}\NormalTok{,}\DecValTok{3}\NormalTok{,}\DecValTok{4}\NormalTok{,}\DecValTok{8}\NormalTok{), }\DecValTok{3}\NormalTok{))}
\end{Highlighting}
\end{Shaded}

\begin{verbatim}
##      [,1] [,2] [,3]
## [1,]    5    2    3
## [2,]    2    7    4
## [3,]    1    3    8
\end{verbatim}

\begin{Shaded}
\begin{Highlighting}[]
\NormalTok{(b }\OtherTok{\textless{}{-}} \FunctionTok{c}\NormalTok{(}\DecValTok{40}\NormalTok{,}\DecValTok{39}\NormalTok{,}\DecValTok{55}\NormalTok{))}
\end{Highlighting}
\end{Shaded}

\begin{verbatim}
## [1] 40 39 55
\end{verbatim}

\begin{Shaded}
\begin{Highlighting}[]
\NormalTok{(x }\OtherTok{\textless{}{-}} \FunctionTok{rep}\NormalTok{(}\DecValTok{0}\NormalTok{,}\DecValTok{3}\NormalTok{))}
\end{Highlighting}
\end{Shaded}

\begin{verbatim}
## [1] 0 0 0
\end{verbatim}

Langkah selanjutnya adalah memperoleh invers matriks \(D\).

\begin{Shaded}
\begin{Highlighting}[]
\NormalTok{(Dinv }\OtherTok{\textless{}{-}} \FunctionTok{diag}\NormalTok{(}\DecValTok{1}\SpecialCharTok{/}\FunctionTok{diag}\NormalTok{(A)))}
\end{Highlighting}
\end{Shaded}

\begin{verbatim}
##      [,1]   [,2]  [,3]
## [1,]  0.2 0.0000 0.000
## [2,]  0.0 0.1429 0.000
## [3,]  0.0 0.0000 0.125
\end{verbatim}

Persiapan terakhir sebelum iterasi dilakukan adalah menyiapkan matriks \(R\).

\begin{Shaded}
\begin{Highlighting}[]
\NormalTok{(R}\OtherTok{\textless{}{-}}\NormalTok{A}\SpecialCharTok{{-}}\FunctionTok{diag}\NormalTok{(}\FunctionTok{diag}\NormalTok{(A)))}
\end{Highlighting}
\end{Shaded}

\begin{verbatim}
##      [,1] [,2] [,3]
## [1,]    0    2    3
## [2,]    2    0    4
## [3,]    1    3    0
\end{verbatim}

Iterasi selanjutnya dilakukan menggunakan Persamaan \eqref{eq:jacobi5}.

\textbf{iterasi 1}

\begin{Shaded}
\begin{Highlighting}[]
\NormalTok{(x1 }\OtherTok{\textless{}{-}}\NormalTok{ Dinv }\SpecialCharTok{\%*\%}\NormalTok{ (b}\SpecialCharTok{{-}}\NormalTok{R}\SpecialCharTok{\%*\%}\NormalTok{x))}
\end{Highlighting}
\end{Shaded}

\begin{verbatim}
##       [,1]
## [1,] 8.000
## [2,] 5.571
## [3,] 6.875
\end{verbatim}

\textbf{iterasi 2}

\begin{Shaded}
\begin{Highlighting}[]
\NormalTok{(x2 }\OtherTok{\textless{}{-}}\NormalTok{ Dinv }\SpecialCharTok{\%*\%}\NormalTok{ (b}\SpecialCharTok{{-}}\NormalTok{R}\SpecialCharTok{\%*\%}\NormalTok{x1))}
\end{Highlighting}
\end{Shaded}

\begin{verbatim}
##         [,1]
## [1,]  1.6464
## [2,] -0.6429
## [3,]  3.7857
\end{verbatim}

\textbf{iterasi 3}

\begin{Shaded}
\begin{Highlighting}[]
\NormalTok{(x3 }\OtherTok{\textless{}{-}}\NormalTok{ Dinv }\SpecialCharTok{\%*\%}\NormalTok{ (b}\SpecialCharTok{{-}}\NormalTok{R}\SpecialCharTok{\%*\%}\NormalTok{x2))}
\end{Highlighting}
\end{Shaded}

\begin{verbatim}
##       [,1]
## [1,] 5.986
## [2,] 2.938
## [3,] 6.910
\end{verbatim}

Selama proses iterasi,jumlah akar jumlah kuadrat dihitung. Sebagai contoh berikut disajikan akar jumlah kuadrat pada iterasi ke-3:

\begin{Shaded}
\begin{Highlighting}[]
\FunctionTok{sqrt}\NormalTok{(}\FunctionTok{sum}\NormalTok{(x3}\SpecialCharTok{{-}}\NormalTok{x2)}\SpecialCharTok{\^{}}\DecValTok{2}\NormalTok{)}
\end{Highlighting}
\end{Shaded}

\begin{verbatim}
## [1] 11.04
\end{verbatim}

Selama proses iterasi nilai tersebut terus mengecil. Iterasi dihantikan jika nilai akar jumlah kuadrat tersebut lebih kecil dari nilai toleransi. Pada contoh ini digunakan nilai toleransi \(10^{-7}\).

Proses iterasi berlangsung sampai dengan iterasi ke-62 dengan nilai \(x\) akhir sebagai berikut:

\[
x = \begin{bmatrix}
     4     \\[0.3em]
     1     \\[0.3em]
     6
\end{bmatrix}
\]

\begin{center}\rule{0.5\linewidth}{0.5pt}\end{center}

\textbf{Algoritma Iterasi Jacobi}

\begin{enumerate}
\def\labelenumi{\arabic{enumi}.}
\tightlist
\item
  Masukkan matriks \(A\), dan vektor \(B\) beserta ukurannya \(n\).
\item
  Hitung invers matriks \(D\), dimana nilai invernya merupakan matriks diagonal dari satu per diagonal utama matriks \(A\).
\item
  Hitung matriks \(R\), dimana \(R\) merupakan selisih matriks \(A\) dikurangi dengan matriks diagonal dengan entri dari diagonal utama matriks \(A\).
\item
  Tetapkan vektor \(x\) estimasi.
\item
  Tetapkan nilai toleransi maksimum yang dapat diterima.
\item
  Lakukan iterasi menggunakan Persamaan \eqref{eq:jacobi5}.
\item
  Hitung akar jumlah kuadrat dari vektor \(x^{n+1}\) dan vektor \(x^n\).
\item
  Jadikan nilai \(x^{n+1}\) sebagai nilai taksiran \(x\) untuk iterasi berikutnya.
\item
  Hentikan proses iterasi jika telah memenuhi syarat yang ditampilkan pada Persamaan \eqref{eq:toleransi}.
\end{enumerate}

\begin{center}\rule{0.5\linewidth}{0.5pt}\end{center}

Berdasarkan algoritma tersebut, kita dapat menyusun fungsi sebuah fungsi untuk melakukan iterasi Jacobi. Berikut sintaks yang digunakan:

\begin{Shaded}
\begin{Highlighting}[]
\NormalTok{jacobi }\OtherTok{\textless{}{-}} \ControlFlowTok{function}\NormalTok{(a, b, }\AttributeTok{tol=}\FloatTok{1e{-}7}\NormalTok{, }\AttributeTok{maxiter=}\DecValTok{100}\NormalTok{)\{}
\NormalTok{  n }\OtherTok{\textless{}{-}} \FunctionTok{length}\NormalTok{(b)}
\NormalTok{  iter }\OtherTok{\textless{}{-}} \DecValTok{0}
  
\NormalTok{  Dinv }\OtherTok{\textless{}{-}} \FunctionTok{diag}\NormalTok{(}\DecValTok{1}\SpecialCharTok{/}\FunctionTok{diag}\NormalTok{(a))}
\NormalTok{  R }\OtherTok{\textless{}{-}}\NormalTok{ a}\SpecialCharTok{{-}}\FunctionTok{diag}\NormalTok{(}\FunctionTok{diag}\NormalTok{(a))}
\NormalTok{  x }\OtherTok{\textless{}{-}} \FunctionTok{rep}\NormalTok{(}\DecValTok{0}\NormalTok{,n)}
\NormalTok{  x\_new }\OtherTok{\textless{}{-}} \FunctionTok{rep}\NormalTok{(tol, n)}
  
  \ControlFlowTok{while}\NormalTok{(}\FunctionTok{sqrt}\NormalTok{(}\FunctionTok{sum}\NormalTok{(x\_new}\SpecialCharTok{{-}}\NormalTok{x)}\SpecialCharTok{\^{}}\DecValTok{2}\NormalTok{)}\SpecialCharTok{\textgreater{}}\NormalTok{tol)\{}
            \ControlFlowTok{if}\NormalTok{(iter}\SpecialCharTok{\textgreater{}}\NormalTok{maxiter)\{}
              \FunctionTok{warning}\NormalTok{(}\StringTok{"iterasi maksimum tercapai"}\NormalTok{)}
              \ControlFlowTok{break}
\NormalTok{            \}}
\NormalTok{            x }\OtherTok{\textless{}{-}}\NormalTok{ x\_new}
\NormalTok{            x\_new }\OtherTok{\textless{}{-}}\NormalTok{ Dinv }\SpecialCharTok{\%*\%}\NormalTok{ (b }\SpecialCharTok{{-}}\NormalTok{ R }\SpecialCharTok{\%*\%}\NormalTok{ x)}
\NormalTok{            iter }\OtherTok{\textless{}{-}}\NormalTok{ iter}\SpecialCharTok{+}\DecValTok{1}
\NormalTok{  \}}
  \FunctionTok{return}\NormalTok{(}\FunctionTok{list}\NormalTok{(}\AttributeTok{X =}\NormalTok{ x\_new, }\AttributeTok{iter=}\NormalTok{iter))}
  
\NormalTok{\}}
\end{Highlighting}
\end{Shaded}

Berikut adalah penerpan fungsi \texttt{jacobi()} tersebut:

\begin{Shaded}
\begin{Highlighting}[]
\FunctionTok{jacobi}\NormalTok{(A,b)}
\end{Highlighting}
\end{Shaded}

\begin{verbatim}
## $X
##      [,1]
## [1,]    4
## [2,]    1
## [3,]    6
## 
## $iter
## [1] 62
\end{verbatim}

\begin{example}
\protect\hypertarget{exm:jacobiexm2}{}\label{exm:jacobiexm2}Selesaikan sistem persamaan berikut menggunakan fungsi \texttt{jacobi()}
\end{example}

\[
\begin{bmatrix}
     27 & 6 & -1     \\[0.3em]
     6 & 15 & 2     \\[0.3em]
     1 & 1 & 54
\end{bmatrix}
x = \begin{bmatrix}
     85     \\[0.3em]
     72     \\[0.3em]
     110
\end{bmatrix}
\]

\textbf{Jawab}:

Matriks \(A\) (matriks koefisien) berdasarkan sistem persamaan linier tersebut telah memenuhi syarat dari algoritma Jacobi (nilai diagonal utama dominan dibanding nilai lainnya pada satu kolom). Penyelesaian sistem persamaan tersebut, sebagai berikut:

\begin{Shaded}
\begin{Highlighting}[]
\NormalTok{A }\OtherTok{\textless{}{-}} \FunctionTok{matrix}\NormalTok{(}\FunctionTok{c}\NormalTok{(}\DecValTok{27}\NormalTok{,}\DecValTok{6}\NormalTok{,}\DecValTok{1}\NormalTok{,}\DecValTok{6}\NormalTok{,}\DecValTok{15}\NormalTok{,}\DecValTok{1}\NormalTok{,}\SpecialCharTok{{-}}\DecValTok{1}\NormalTok{,}\DecValTok{2}\NormalTok{,}\DecValTok{54}\NormalTok{), }\DecValTok{3}\NormalTok{)}
\NormalTok{b }\OtherTok{\textless{}{-}} \FunctionTok{c}\NormalTok{(}\DecValTok{85}\NormalTok{,}\DecValTok{72}\NormalTok{,}\DecValTok{110}\NormalTok{)}

\FunctionTok{jacobi}\NormalTok{(A,b)}
\end{Highlighting}
\end{Shaded}

\begin{verbatim}
## $X
##       [,1]
## [1,] 2.425
## [2,] 3.573
## [3,] 1.926
## 
## $iter
## [1] 17
\end{verbatim}

Nilai vektor \(x\) sesungguhnya dapat diperoleh menggunakan fungsi \texttt{solve()}.

\begin{Shaded}
\begin{Highlighting}[]
\FunctionTok{solve}\NormalTok{(A,b)}
\end{Highlighting}
\end{Shaded}

\begin{verbatim}
## [1] 2.425 3.573 1.926
\end{verbatim}

Berdasarkan hasil perhitungan, vektor \(x\) hasil iterasi memiliki nilai identik dengan nilai penyelesaian yang sebenarnya.

Perlu diperhatikan dalam penggunaan fungsi \texttt{jacobi()} syarat utama matriks haruslah terpenuhi, seperti: nilai diagonal matriks \(A\) lebih besar dari nilai elemen lainnya pada satu kolom. Selain itu, nilai diagonal matriks \(D\) tidak boleh sama dengan nol agar inver matriks \(D\) dapat diperoleh. Jika syarat-syarat tersebut terpenuhi, maka metode Jacobi dapat diterapkan. Jika tidak terpenuhi, maka penyelesaian yang konvergen mungkin masih dapat diperoleh meskipun penulis tidak dapat menjamin hal tersebut dapat terjadi.

\hypertarget{seideliter}{%
\subsection{Iterasi Gauss-Seidel}\label{seideliter}}

Metode iterasi Gauss-Seidel melakukan dekomposisi pada matriks \(A\) menjadi matriks segitiga atas \(U\) dan matriks segitiga bawah \(L\). Dekomposisi ini tidak sama dengan dekomposisi LU pada Chapter \ref{ludecomp}. Matriks \(U\) pada metode Gauss-Seidel merupakan elemen (entri) matriks \(A\) pada bagian atas diagonal utama, sedangkan matriks \(L\) merupakan elemen diagonal utama dan bagian bawah diagonal utama matriks \(A\). Elemen selain yang penulis sebutkan pada kedua matriks tersebut akan bernilai nol. Persamaan iterasi Gauss-Seidel ditampilkan pada Persamaan \eqref{eq:gaussseidel}.

\begin{equation}
x^{n+1}=L^{-1}\left(b-Ux^{n}\right)
 \label{eq:gaussseidel}
\end{equation}

Syarat agar suatu sistem persamaan linier dapat diselesaikan menggunakan metode Gauss-Seidel adalah matriks harus memiliki nilai diagonal utama yang dominan. Maksudnya, nilai absolut diagonal utama lebih besar dari jumlah nilai absolut elemen lainnya dalam satu kolom. Jika syarat ini tidak terpenuhi maka metode ini tidak akan memperoleh penyelesaian yang konvergen.

\begin{example}
\protect\hypertarget{exm:gaussseidelexm}{}\label{exm:gaussseidelexm}Selesaikan sistem persamaan pada Contoh \ref{exm:jacobiexm2} menggunakan iterasi Gauss-Seidel!
\end{example}

\textbf{Jawab}:

Kita akan kembali menggunakan bantuan \texttt{R} untuk melakukan kalkulasi pada proses iterasi Gauss-Seidel. Kita telah melakukan pengecekan pada sistem persamaan linier pada contoh tersebut dan menghasilkan kesimpulan bahwa persamaan linier tersebut dapat diselesaikan dengan metode Gauss-Seidel. Langkah selanjutnya adalah membentuk matriks \(L\) dan \(U\).

\begin{Shaded}
\begin{Highlighting}[]
\CommentTok{\# membentuk matriks U dan L dari matriks A}
\NormalTok{(L }\OtherTok{\textless{}{-}}\NormalTok{ U }\OtherTok{\textless{}{-}}\NormalTok{ A)}
\end{Highlighting}
\end{Shaded}

\begin{verbatim}
##      [,1] [,2] [,3]
## [1,]   27    6   -1
## [2,]    6   15    2
## [3,]    1    1   54
\end{verbatim}

\begin{Shaded}
\begin{Highlighting}[]
\CommentTok{\# membentuk matriks L dari entri bagian bawah diagonal utama matriks A}
\NormalTok{L[}\FunctionTok{upper.tri}\NormalTok{(A, }\AttributeTok{diag=}\ConstantTok{FALSE}\NormalTok{)]}\OtherTok{\textless{}{-}}\DecValTok{0}
\NormalTok{L}
\end{Highlighting}
\end{Shaded}

\begin{verbatim}
##      [,1] [,2] [,3]
## [1,]   27    0    0
## [2,]    6   15    0
## [3,]    1    1   54
\end{verbatim}

\begin{Shaded}
\begin{Highlighting}[]
\CommentTok{\# membentuk matriks U dari entri bagian atas diagonal utama matriks A}
\NormalTok{U[}\FunctionTok{lower.tri}\NormalTok{(A, }\AttributeTok{diag=}\ConstantTok{TRUE}\NormalTok{)]}\OtherTok{\textless{}{-}}\DecValTok{0}
\NormalTok{U}
\end{Highlighting}
\end{Shaded}

\begin{verbatim}
##      [,1] [,2] [,3]
## [1,]    0    6   -1
## [2,]    0    0    2
## [3,]    0    0    0
\end{verbatim}

Selanjutya lakukan invers terhadap matriks \(L\) menggunakn fungsi \texttt{solve()}.

\begin{Shaded}
\begin{Highlighting}[]
\NormalTok{(Linv }\OtherTok{\textless{}{-}} \FunctionTok{solve}\NormalTok{(L))}
\end{Highlighting}
\end{Shaded}

\begin{verbatim}
##            [,1]      [,2]    [,3]
## [1,]  0.0370370  0.000000 0.00000
## [2,] -0.0148148  0.066667 0.00000
## [3,] -0.0004115 -0.001235 0.01852
\end{verbatim}

Tetapkan nilai estimasi awal dan nilai toleransi yang dikehendaki. Nilai toleransi pada proses ini ditetapkan sebesar \(10^-7\).

\begin{Shaded}
\begin{Highlighting}[]
\CommentTok{\# tebakan awal nilai x}
\NormalTok{(x }\OtherTok{\textless{}{-}} \FunctionTok{rep}\NormalTok{(}\DecValTok{0}\NormalTok{, }\FunctionTok{length}\NormalTok{(b)))}
\end{Highlighting}
\end{Shaded}

\begin{verbatim}
## [1] 0 0 0
\end{verbatim}

Lakukan iterasi menggunakan Persamaan \eqref{eq:gaussseidel}.

\textbf{Iterasi 1}

\begin{Shaded}
\begin{Highlighting}[]
\NormalTok{(x1 }\OtherTok{\textless{}{-}}\NormalTok{ Linv }\SpecialCharTok{\%*\%}\NormalTok{ (b }\SpecialCharTok{{-}}\NormalTok{ U }\SpecialCharTok{\%*\%}\NormalTok{ x))}
\end{Highlighting}
\end{Shaded}

\begin{verbatim}
##       [,1]
## [1,] 3.148
## [2,] 3.541
## [3,] 1.913
\end{verbatim}

\begin{Shaded}
\begin{Highlighting}[]
\CommentTok{\# akar jumlah kuadrat}
\FunctionTok{sqrt}\NormalTok{(}\FunctionTok{sum}\NormalTok{(x1}\SpecialCharTok{{-}}\NormalTok{x)}\SpecialCharTok{\^{}}\DecValTok{2}\NormalTok{)}
\end{Highlighting}
\end{Shaded}

\begin{verbatim}
## [1] 8.602
\end{verbatim}

\textbf{Iterasi 2}

\begin{Shaded}
\begin{Highlighting}[]
\NormalTok{(x2 }\OtherTok{\textless{}{-}}\NormalTok{ Linv }\SpecialCharTok{\%*\%}\NormalTok{ (b }\SpecialCharTok{{-}}\NormalTok{ U }\SpecialCharTok{\%*\%}\NormalTok{ x1))}
\end{Highlighting}
\end{Shaded}

\begin{verbatim}
##       [,1]
## [1,] 2.432
## [2,] 3.572
## [3,] 1.926
\end{verbatim}

\begin{Shaded}
\begin{Highlighting}[]
\CommentTok{\# akar jumlah kuadrat}
\FunctionTok{sqrt}\NormalTok{(}\FunctionTok{sum}\NormalTok{(x2}\SpecialCharTok{{-}}\NormalTok{x1)}\SpecialCharTok{\^{}}\DecValTok{2}\NormalTok{)}
\end{Highlighting}
\end{Shaded}

\begin{verbatim}
## [1] 0.672
\end{verbatim}

Iterasi terus dilakukan sampai dengan nilai akar jumlah kuadrat lebih kecil dari nilai toleransi. Setelah iterasi ke-7 diperoleh nilai vektor \(x\) sebesar:

\[
x = \begin{bmatrix}
     2,425476     \\[0.3em]
     3,573016     \\[0.3em]
     1,925954
\end{bmatrix}
\]

\begin{center}\rule{0.5\linewidth}{0.5pt}\end{center}

\textbf{Algoritma Iterasi Gauss-Seidel}

\begin{enumerate}
\def\labelenumi{\arabic{enumi}.}
\tightlist
\item
  Masukkan matriks \(A\), dan vektor \(B\) beserta ukurannya \(n\).
\item
  Lakukan dekomposisi LU, dimana matriks \(L\) merupakan matriks segitiga bawah dengan nilai entri diagonal utama matriks \(A\) dan bagian bawah diagonalnya dan matriks \(U\) merupakan matriks segitiga atas dengan entri berasal dari elemen atas diagonal utama matriks \(A\). Isi elemen lain yang tidak disebut pada kedua matriks tersebut dengan nol.
\item
  Tetapkan vektor \(x\) estimasi.
\item
  Tetapkan nilai toleransi maksimum yang dapat diterima.
\item
  Lakukan iterasi menggunakan Persamaan \eqref{eq:gaussseidel}.
\item
  Hitung akar jumlah kuadrat dari vektor \(x^{n+1}\) dan vektor \(x^n\).
\item
  Jadikan nilai \(x^{n+1}\) sebagai nilai taksiran \(x\) untuk iterasi berikutnya.
\item
  Hentikan proses iterasi jika telah memenuhi syarat yang ditampilkan pada Persamaan \eqref{eq:toleransi}.
\end{enumerate}

\begin{center}\rule{0.5\linewidth}{0.5pt}\end{center}

Berdasarkan algoritma tersebut, kita dapat menyusun fungsi sebuah fungsi untuk melakukan iterasi Gauss-Seidel. Berikut sintaks yang digunakan:

\begin{Shaded}
\begin{Highlighting}[]
\NormalTok{gauss\_seidel }\OtherTok{\textless{}{-}} \ControlFlowTok{function}\NormalTok{(a, b, }\AttributeTok{tol=}\FloatTok{1e{-}7}\NormalTok{, }\AttributeTok{maxiter=}\DecValTok{100}\NormalTok{)\{}
\NormalTok{  n }\OtherTok{\textless{}{-}} \FunctionTok{length}\NormalTok{(b)}
\NormalTok{  iter }\OtherTok{\textless{}{-}} \DecValTok{0}
  
  
\NormalTok{  L }\OtherTok{\textless{}{-}}\NormalTok{ U }\OtherTok{\textless{}{-}}\NormalTok{ a}
\NormalTok{  L[}\FunctionTok{upper.tri}\NormalTok{(a, }\AttributeTok{diag=}\ConstantTok{FALSE}\NormalTok{)] }\OtherTok{\textless{}{-}} \DecValTok{0}
\NormalTok{  U[}\FunctionTok{lower.tri}\NormalTok{(a, }\AttributeTok{diag=}\ConstantTok{TRUE}\NormalTok{)] }\OtherTok{\textless{}{-}} \DecValTok{0}
\NormalTok{  Linv }\OtherTok{\textless{}{-}} \FunctionTok{solve}\NormalTok{(L)}
  
\NormalTok{  x }\OtherTok{\textless{}{-}} \FunctionTok{rep}\NormalTok{(}\DecValTok{0}\NormalTok{,n)}
\NormalTok{  x\_new }\OtherTok{\textless{}{-}} \FunctionTok{rep}\NormalTok{(tol, n)}
  
  \ControlFlowTok{while}\NormalTok{(}\FunctionTok{sqrt}\NormalTok{(}\FunctionTok{sum}\NormalTok{(x\_new}\SpecialCharTok{{-}}\NormalTok{x)}\SpecialCharTok{\^{}}\DecValTok{2}\NormalTok{)}\SpecialCharTok{\textgreater{}}\NormalTok{tol)\{}
            \ControlFlowTok{if}\NormalTok{(iter}\SpecialCharTok{\textgreater{}}\NormalTok{maxiter)\{}
              \FunctionTok{warning}\NormalTok{(}\StringTok{"iterasi maksimum tercapai"}\NormalTok{)}
              \ControlFlowTok{break}
\NormalTok{            \}}
\NormalTok{            x }\OtherTok{\textless{}{-}}\NormalTok{ x\_new}
\NormalTok{            x\_new }\OtherTok{\textless{}{-}}\NormalTok{ Linv }\SpecialCharTok{\%*\%}\NormalTok{ (b }\SpecialCharTok{{-}}\NormalTok{ U }\SpecialCharTok{\%*\%}\NormalTok{ x)}
\NormalTok{            iter }\OtherTok{\textless{}{-}}\NormalTok{ iter}\SpecialCharTok{+}\DecValTok{1}
\NormalTok{  \}}
      \FunctionTok{return}\NormalTok{(}\FunctionTok{list}\NormalTok{(}\AttributeTok{X =}\NormalTok{ x\_new, }\AttributeTok{iter=}\NormalTok{iter))}
\NormalTok{\}}
\end{Highlighting}
\end{Shaded}

\begin{example}
\protect\hypertarget{exm:gaussseidelexm2}{}\label{exm:gaussseidelexm2}Selesaikan sistem persamaan pada Contoh \ref{exm:jacobiexm2} menggunakan fungsi \texttt{gauss\_seidel()}!
\end{example}

\textbf{Jawab}:

Penyelesaiansistem persamaan linier tersebut menggunakan fungsi \texttt{gauss\_seidel()} disajikan pada sintaks berikut:

\begin{Shaded}
\begin{Highlighting}[]
\FunctionTok{gauss\_seidel}\NormalTok{(A,b)}
\end{Highlighting}
\end{Shaded}

\begin{verbatim}
## $X
##       [,1]
## [1,] 2.425
## [2,] 3.573
## [3,] 1.926
## 
## $iter
## [1] 7
\end{verbatim}

\hypertarget{studikasus}{%
\section{Studi Kasus}\label{studikasus}}

Aljabar linier banyak diaplikasikan baik dalam bidang \emph{engineering}, fisika, sampai dengan statistika. Pada \emph{sub-chapter} ini penulis akan menjelaskan penerapan aljabar linier pada metode kuadrat terkecil dan aliran massa dalam reaktor. Untuk penerapan lainnya pembaca dapat membaca buku lainnya terkait aljabar linier.

\hypertarget{leastsquare}{%
\subsection{Metode Kuadrat Terkecil}\label{leastsquare}}

Metode kuadrat terkecil merupakan salah satu aplikasi penerapan aljabar linier yang paling populer. Intuisi dibalik metode ini adalah bagaimana kita meminimalkan jarak antara sejumlah titik dengan garis regresi. Misalkan kita menggambarkan scatterplot antara dua buah variabel. Pola yang terbentuk dari plot tersebut adalah terjadi korelasi positif antara variabel pada sumbu \(x\) dan sumbu \(y\). Kita ingin menggambarkan garis regresi terbaik yang dapat menangkap seluruh pola tersebut. Garis regresi terbaik terjadi ketika jumlah kuadrat jarak antara titik observasi dan garis regresi yang terbentuk seminimal mungkin.

Untuk lebih memahaminya kita akan melakukan latihan menggunakan dataset \texttt{trees} yang berisi data hasil pengukuran kayu dari pohon yang ditebang. Pada dataset ini terdapat 31 observasi dan 3 buah kolom. Keterangan dari ketiga buah kolom tersebut adalah sebagai berikut:

\begin{itemize}
\tightlist
\item
  \texttt{Girth}: diameter pohon dalam satuan \emph{inch}.
\item
  \texttt{Height}: tinggi pohon dalam satuan \emph{feet}.
\item
  \texttt{Volume}: volume kayu dalam satuan \emph{cubic feet}.
\end{itemize}

Untuk mengecek 6 observasi pertama dan struktur data, jalankan sintaks berikut:

\begin{Shaded}
\begin{Highlighting}[]
\FunctionTok{head}\NormalTok{(trees)}
\end{Highlighting}
\end{Shaded}

\begin{verbatim}
##   Girth Height Volume
## 1   8.3     70   10.3
## 2   8.6     65   10.3
## 3   8.8     63   10.2
## 4  10.5     72   16.4
## 5  10.7     81   18.8
## 6  10.8     83   19.7
\end{verbatim}

\begin{Shaded}
\begin{Highlighting}[]
\FunctionTok{str}\NormalTok{(trees)}
\end{Highlighting}
\end{Shaded}

\begin{verbatim}
## 'data.frame':    31 obs. of  3 variables:
##  $ Girth : num  8.3 8.6 8.8 10.5 10.7 10.8 11 11 11.1 11.2 ...
##  $ Height: num  70 65 63 72 81 83 66 75 80 75 ...
##  $ Volume: num  10.3 10.3 10.2 16.4 18.8 19.7 15.6 18.2 22.6 19.9 ...
\end{verbatim}

Scatterplot matriks sangat bagus untuk mengecek korelasi antar variabel dalam dataset tersebut. Berikut adalah sintaks untuk membuatnya:

\begin{figure}

{\centering \includegraphics[width=0.9\linewidth]{Algoritma_Pemrograman_R_files/figure-latex/trees, LUfig-1} 

}

\caption{Scatterplot matriks dataset trees}(\#fig:trees, LUfig)
\end{figure}

Kita ingin membuat sebuah model linier untuk memprediksi \texttt{Volume} kayu berdasarkan variabel \texttt{Girth} dan \texttt{Heiht} atau volume sebagia fungsi dari variabel \texttt{Girth} dan \texttt{Heiht}. Kita dapat menuliskan relasi antara variabel volume sebagai fungsi dari variabel \texttt{Girth} dan \texttt{Heiht} menggunakan Persamaan \eqref{eq:trees}.

\begin{equation}
Volume=\beta_{girth} Girth+\beta_{height}Height+\beta_0
 \label{eq:trees}
\end{equation}

dimana \(\beta_0\) merupakan intersep persamaan regresi linier dan nilai \(\beta\) lainnya merupakan koefisien dari variabel \texttt{Girth} dan \texttt{Heiht}. Variabel \texttt{Volume} disebut sebagai variabel respon, sedangkan variabel \texttt{Girth} dan \texttt{Heiht} disebut sebagai variabel prediktor.

Metode kuadrat terkecil berusaha memperoleh seluruh koefisien variabel dan intersep dari persamaan regresi linier. Berdasarkan yang telah penulis jelaskan garis regresi terbaik adalah garis yang memiliki nilai kuadrat terkecil jarak antara titik observasi dan garis regresi. Dasar dari metode kuadrat terkecil merupakan persamaan yang relatif sederhana yang ditunjukkan pada Persamaan \eqref{eq:trees2}.

\begin{equation}
A^{T}A=A^{T}b
 \label{eq:trees2}
\end{equation}

dimana \(b\) merupakan vektor dari variabel respon (\texttt{Volume}) dan matrik \(A\) merupakan matriks variabel prediktor (variabel \texttt{Girth} dan \texttt{Heiht}).

Untuk menginputkan intercept kedalam persamaan linier kita perlu menmabhakan satu kolom di awal matriks \(A\) yang berisi nilai 1. Berikut adalah sintaks yang digunakan untuk membentuk matriks \(A\):

\begin{Shaded}
\begin{Highlighting}[]
\CommentTok{\# membentuk matriks A}
\NormalTok{pred }\OtherTok{\textless{}{-}} \FunctionTok{cbind}\NormalTok{(}\AttributeTok{intercept=}\DecValTok{1}\NormalTok{, }\AttributeTok{Girth=}\NormalTok{trees}\SpecialCharTok{$}\NormalTok{Girth, }\AttributeTok{Height=}\NormalTok{trees}\SpecialCharTok{$}\NormalTok{Height)}
\FunctionTok{head}\NormalTok{(A)}
\end{Highlighting}
\end{Shaded}

\begin{verbatim}
##      [,1] [,2] [,3]
## [1,]   27    6   -1
## [2,]    6   15    2
## [3,]    1    1   54
\end{verbatim}

Langkah selanjutnya adalah membentuk matriks \(b\). Berikut adalah sintaks yang digunakan:

\begin{Shaded}
\begin{Highlighting}[]
\NormalTok{resp}\OtherTok{\textless{}{-}}\NormalTok{ trees}\SpecialCharTok{$}\NormalTok{Volume}
\FunctionTok{head}\NormalTok{(resp)}
\end{Highlighting}
\end{Shaded}

\begin{verbatim}
## [1] 10.3 10.3 10.2 16.4 18.8 19.7
\end{verbatim}

Untuk memperoleh koefisien \(\beta\), kita dapat mencarinya dengan cara menyelesaikan Persamaan \eqref{eq:trees2}. Berikut adalah sintaks yang digunakan:

\begin{Shaded}
\begin{Highlighting}[]
\NormalTok{A }\OtherTok{\textless{}{-}} \FunctionTok{t}\NormalTok{(pred) }\SpecialCharTok{\%*\%}\NormalTok{ pred}
\NormalTok{b }\OtherTok{\textless{}{-}} \FunctionTok{t}\NormalTok{(pred) }\SpecialCharTok{\%*\%}\NormalTok{ resp}

\NormalTok{Ab }\OtherTok{\textless{}{-}} \FunctionTok{cbind}\NormalTok{(A,b)}
\NormalTok{(x }\OtherTok{\textless{}{-}} \FunctionTok{gauss\_jordan}\NormalTok{(Ab))}
\end{Highlighting}
\end{Shaded}

\begin{verbatim}
##           intercept Girth Height         
## intercept         1     0      0 -57.9877
## Girth             0     1      0   4.7082
## Height            0     0      1   0.3393
\end{verbatim}

Berdasarkan hasil yang diperoleh, persamaan linier yang terbentuk disajikan pada Persamaan \eqref{eq:trees3}.

\begin{equation}
Volume=4.7081605 Girth + 0.3392512 Height  -57.9876589
 \label{eq:trees3}
\end{equation}

Pembaca juga dapat menggunakan fungsi lain untuk memperoleh nilai koefisien tersebut, seperti: \texttt{lu\_solve()}dan\texttt{solve()}. untuk fungsi \texttt{jacobi()} dan \texttt{gauss\_seidel()}, kita harus pastikan syarat-syarat terkait metode tersebut. Berikut adalah contoh penyelesaian menggunakan sintaks lainnya:

\begin{Shaded}
\begin{Highlighting}[]
\CommentTok{\# metode LU}
\FunctionTok{lu\_solve}\NormalTok{(A,b)}
\end{Highlighting}
\end{Shaded}

\begin{verbatim}
## $P
##      [,1] [,2] [,3]
## [1,]    1    0    0
## [2,]    0    1    0
## [3,]    0    0    1
## 
## $L
##       [,1]  [,2] [,3]
## [1,]  1.00 0.000    0
## [2,] 13.25 1.000    0
## [3,] 76.00 1.054    1
## 
## $U
##           intercept Girth Height
## intercept        31 410.7 2356.0
## Girth             0 295.4  311.5
## Height            0   0.0  889.6
## 
## $result
##          [,1]
## [1,] -57.9877
## [2,]   4.7082
## [3,]   0.3393
\end{verbatim}

\begin{Shaded}
\begin{Highlighting}[]
\CommentTok{\# fungsi solve()}
\FunctionTok{solve}\NormalTok{(A,b)}
\end{Highlighting}
\end{Shaded}

\begin{verbatim}
##               [,1]
## intercept -57.9877
## Girth       4.7082
## Height      0.3393
\end{verbatim}

\texttt{R} juga menyediakan fungsi untuk membentuk model regresi linier. Fungsi yang digunakan adalah \texttt{lm()}. Berikut sintaks yang digunakan untuk membentuk model linier menggunakan fungsi \texttt{lm()}:

\begin{Shaded}
\begin{Highlighting}[]
\FunctionTok{lm}\NormalTok{(Volume}\SpecialCharTok{\textasciitilde{}}\NormalTok{Girth}\SpecialCharTok{+}\NormalTok{Height, }\AttributeTok{data=}\NormalTok{trees)}
\end{Highlighting}
\end{Shaded}

\begin{verbatim}
## 
## Call:
## lm(formula = Volume ~ Girth + Height, data = trees)
## 
## Coefficients:
## (Intercept)        Girth       Height  
##     -57.988        4.708        0.339
\end{verbatim}

\hypertarget{reaktorkimia}{%
\subsection{Aliran Massa Dalam Reaktor}\label{reaktorkimia}}

Pada \emph{sub-chapter} ini penulis akan memberikan penerapan aljabar linier untuk menghitung konsentrasi suatu zat atau parameter lingkungan dalam reaktor yang saling terhubung. Pada contoh kasus kali ini diasumsikan terdapat lima buah reaktor yang saling terhubung satu sama lain sesuai Gambar \ref{fig:reaktor}. Debit air (\(\frac{m^{3}}{detik}\)) dan konsentrasi zat pencemar (\(\frac{mg}{m^3}\)) disajikan pula diagram alir tersebut. Diasumsikan kelima buah reaktor tersebut dalam kondisi \emph{steady} dan volume reaktor diasumsikan sama. Kesetimbangan massa persatuan waktu dalam kondisi \emph{steady} disajikan pada Persamaan \eqref{eq:steady}.

\begin{figure}

{\centering \includegraphics[width=0.9\linewidth]{./images/reaktor} 

}

\caption{Aliran massa dalam reaktor.}\label{fig:reaktor}
\end{figure}

\begin{equation}
m_in=m_out
 \label{eq:steady}
\end{equation}

\begin{equation}
Q_{in}C_{in}= Q_{out}C_{out}
 \label{eq:steady2}
\end{equation}

Berdasarkan Gambar \ref{fig:reaktor}, dapat dibentuk lima buah sistem persamaan linier. Persamaan linier yang terbentuk disajikan sebagai berikut:

\[
\begin{matrix}
  6c_{1}-c_{3}=50 \\
  -3c_{1}+3c_{2}=0 \\
  -c_{2}+9c_{3}=160 \\
  -c_{2}-8c_{3}+11c_{4}-2c_{5}=0 \\
  -3c_{1}-c_{2}+4c_{5}=0
\end{matrix}
\]

Untuk menyelesaiakan sistem persamaan linier tersebut dan memperoleh nilai \(c\) dari masing-masing reaktor, kiat perlu mengubahnya dulu kedalam bentuk matriks \(Ax=b\). Berikut adalah matriks yang terbentuk:

\begin{equation*}
\begin{bmatrix}
      6 &  0 & -1 & 0  & 0    \\[0.3em]
     -3 &  3 & 0  & 0  & 0    \\[0.3em]
      0 & -1 & 9  & 0  & 0    \\[0.3em]
      0 & -1 & -8 & 11 & -2   \\[0.3em]
     -3 & -1 & 0  & 0  & 4
\end{bmatrix}
c = \begin{bmatrix}
     50     \\[0.3em]
     0      \\[0.3em]
     160    \\[0.3em]
     0      \\[0.3em]
     0
\end{bmatrix}
\end{equation*}

Kita akan menyelesaikannya dengan menggunakan metode elminasi Gauss-Jordan, dekomposisi LU, iterasi Jacobi, dan iterasi Gauss-Seidel. Untuk dapat menyelesaikannya menggunakan metode-metode tersebut pada \texttt{R}, kita perlu membentuk matriksnya terlebih dahulu:

\begin{Shaded}
\begin{Highlighting}[]
\NormalTok{(A }\OtherTok{\textless{}{-}} \FunctionTok{matrix}\NormalTok{(}\FunctionTok{c}\NormalTok{(}\DecValTok{6}\NormalTok{,}\SpecialCharTok{{-}}\DecValTok{3}\NormalTok{,}\DecValTok{0}\NormalTok{,}\DecValTok{0}\NormalTok{,}\SpecialCharTok{{-}}\DecValTok{3}\NormalTok{,}
              \DecValTok{0}\NormalTok{,}\DecValTok{3}\NormalTok{,}\SpecialCharTok{{-}}\DecValTok{1}\NormalTok{,}\SpecialCharTok{{-}}\DecValTok{1}\NormalTok{,}\SpecialCharTok{{-}}\DecValTok{1}\NormalTok{,}
              \SpecialCharTok{{-}}\DecValTok{1}\NormalTok{,}\DecValTok{0}\NormalTok{,}\DecValTok{9}\NormalTok{,}\SpecialCharTok{{-}}\DecValTok{8}\NormalTok{,}\DecValTok{0}\NormalTok{,}
              \DecValTok{0}\NormalTok{,}\DecValTok{0}\NormalTok{,}\DecValTok{0}\NormalTok{,}\DecValTok{11}\NormalTok{,}\DecValTok{0}\NormalTok{,}
              \DecValTok{0}\NormalTok{,}\DecValTok{0}\NormalTok{,}\DecValTok{0}\NormalTok{,}\SpecialCharTok{{-}}\DecValTok{2}\NormalTok{,}\DecValTok{4}\NormalTok{),}\AttributeTok{nrow=}\DecValTok{5}\NormalTok{))}
\end{Highlighting}
\end{Shaded}

\begin{verbatim}
##      [,1] [,2] [,3] [,4] [,5]
## [1,]    6    0   -1    0    0
## [2,]   -3    3    0    0    0
## [3,]    0   -1    9    0    0
## [4,]    0   -1   -8   11   -2
## [5,]   -3   -1    0    0    4
\end{verbatim}

\begin{Shaded}
\begin{Highlighting}[]
\NormalTok{(b }\OtherTok{\textless{}{-}} \FunctionTok{c}\NormalTok{(}\DecValTok{50}\NormalTok{,}\DecValTok{0}\NormalTok{,}\DecValTok{160}\NormalTok{,}\DecValTok{0}\NormalTok{,}\DecValTok{0}\NormalTok{))}
\end{Highlighting}
\end{Shaded}

\begin{verbatim}
## [1]  50   0 160   0   0
\end{verbatim}

\textbf{Metode Eliminasi Gauss-Jordan}

\begin{Shaded}
\begin{Highlighting}[]
\FunctionTok{gauss\_jordan}\NormalTok{(}\FunctionTok{cbind}\NormalTok{(A,b))}
\end{Highlighting}
\end{Shaded}

\begin{verbatim}
##                    b
## [1,] 1 0 0 0 0 11.51
## [2,] 0 1 0 0 0 11.51
## [3,] 0 0 1 0 0 19.06
## [4,] 0 0 0 1 0 17.00
## [5,] 0 0 0 0 1 11.51
\end{verbatim}

\textbf{Metode Dekomposisi LU}

\begin{Shaded}
\begin{Highlighting}[]
\FunctionTok{lu\_solve}\NormalTok{(A,b)}
\end{Highlighting}
\end{Shaded}

\begin{verbatim}
## $P
##      [,1] [,2] [,3] [,4] [,5]
## [1,]    1    0    0    0    0
## [2,]    0    1    0    0    0
## [3,]    0    0    1    0    0
## [4,]    0    0    0    1    0
## [5,]    0    0    0    0    1
## 
## $L
##      [,1]    [,2]     [,3] [,4] [,5]
## [1,]  1.0  0.0000  0.00000    0    0
## [2,] -0.5  1.0000  0.00000    0    0
## [3,]  0.0 -0.3333  1.00000    0    0
## [4,]  0.0 -0.3333 -0.92453    1    0
## [5,] -0.5 -0.3333 -0.07547    0    1
## 
## $U
##      [,1] [,2]       [,3] [,4] [,5]
## [1,]    6    0 -1.000e+00    0    0
## [2,]    0    3 -5.000e-01    0    0
## [3,]    0    0  8.833e+00    0    0
## [4,]    0    0  0.000e+00   11   -2
## [5,]    0    0  1.110e-16    0    4
## 
## $result
## [1] 11.51 11.51 19.06 17.00 11.51
\end{verbatim}

\textbf{Metode Iterasi Jacobi}

\begin{Shaded}
\begin{Highlighting}[]
\FunctionTok{jacobi}\NormalTok{(A,b, }\AttributeTok{maxiter=}\DecValTok{100}\NormalTok{)}
\end{Highlighting}
\end{Shaded}

\begin{verbatim}
## $X
##       [,1]
## [1,] 11.51
## [2,] 11.51
## [3,] 19.06
## [4,] 17.00
## [5,] 11.51
## 
## $iter
## [1] 17
\end{verbatim}

\textbf{Metode Iterasi Gauss-Seidel}

\begin{Shaded}
\begin{Highlighting}[]
\FunctionTok{gauss\_seidel}\NormalTok{(A,b, }\AttributeTok{maxiter=}\DecValTok{200}\NormalTok{)}
\end{Highlighting}
\end{Shaded}

\begin{verbatim}
## $X
##       [,1]
## [1,] 11.51
## [2,] 11.51
## [3,] 19.06
## [4,] 17.00
## [5,] 11.51
## 
## $iter
## [1] 7
\end{verbatim}

Berdasarkan seluruh metode tersebut, diperoleh konsentrasi zat pencemar pada masing-masing reaktor adalah sebagai berikut:

\[
\begin{matrix}
  c_{1}=11,50943 \frac{mg}{m^3} \\
  c_{2}=11,50943 \frac{mg}{m^3} \\
  c_{3}=19,05660 \frac{mg}{m^3} \\
  c_{4}=16,99828 \frac{mg}{m^3} \\
  c_{5}=11.50943 \frac{mg}{m^3}
\end{matrix}
\]

\hypertarget{referensi-1}{%
\section{Referensi}\label{referensi-1}}

\begin{enumerate}
\def\labelenumi{\arabic{enumi}.}
\tightlist
\item
  Bloomfield, V.A. 2014. \textbf{Using R for Numerical Analysis in Science and Engineering}. CRC Press
\item
  Howard, J.P. 2017. \textbf{Computational Methods for Numerical Analysis with R}. CRC Press.
\item
  Kreyszig, E. 2011. \textbf{Advanced Engineering Mathematics, 10th Edition}. John Wiley \& Sons.
\item
  Primartha, R. 2018. \textbf{Belajar Machine Learning Teori dan Praktik}. Penerbit Informatika : Bandung.
\item
  Sanjaya, M. 2015. \textbf{Metode Numerik Berbasis Phython}. Penerbit Gava Media: Yogyakarta.
\item
  Suparno, S. 2008. \textbf{Komputasi untuk Sains dan Teknik Edisi II}. Departemen Fisika-FMIPA Universitas Indonesia.
\end{enumerate}

\hypertarget{latihan}{%
\section{Latihan}\label{latihan}}

\begin{enumerate}
\def\labelenumi{\arabic{enumi}.}
\tightlist
\item
  Selesaikan sistem persamaan linier berikut menggunakan eliminasi Gauss!
\end{enumerate}

\[
\begin{matrix}
  -4x+4y=-1 \\
  -2x+2y-3z=-3 \\
  3x+1y-3z=-3
\end{matrix}
\]

\begin{enumerate}
\def\labelenumi{\arabic{enumi}.}
\setcounter{enumi}{1}
\tightlist
\item
  Carilah penyelesaian dari sistem persamaan linier soal no.1 menggunakan algoritma dekomposisi LU!
\item
  Tunjukan 5 iterasi pertama sistem persamaan linier berikut menggunakan algoritma Jacobi dan Gauss-Seidel!
\end{enumerate}

\[
\begin{matrix}
  3x+2y-1z=-3 \\
  -3x-3y-3z=9 \\
  1y-1z=-1
\end{matrix}
\]

\begin{enumerate}
\def\labelenumi{\arabic{enumi}.}
\setcounter{enumi}{3}
\tightlist
\item
  Gunakan fungsi \texttt{jacobi()} dan \texttt{gauss\_seidel()} untuk menyelesaikan sistem persamaan linier pada soal no.4 dan tentukan metode mana yang paling cepat memperoleh penyelesaian? (\textbf{petunjuk}: gunakan fungsi \texttt{system.time()} dan jumlah iterasi yang diperlukan untuk memperoleh hasil yang konvergen)
\item
  Apakah yang terjadi jika kita menginputkan matriks segiempat \(A\) kedalam fungsi \texttt{solve()} dan apa yang akan terjadi jika selanjutnya argumen pada fungsi tersebut juga menyertakan vektor \(b\)?
\item
  Dengan menggunakan dataset \texttt{mtcars} buatlah persamaan linier variabel \texttt{mpg} sebagai fungsi dari variabel \texttt{wt}, \texttt{hp}, dan \texttt{qsec} menggunakan algoritma dekomposisi LU?
\end{enumerate}

\hypertarget{rootfinding}{%
\chapter{Akar Persamaan Non-Linier}\label{rootfinding}}

Persamaan non-linier dapat diartikan sebagai persamaan yang tidak mengandung syarat seperti persamaan linier, sehingga persamaan non-linier dapat merupakan:

\begin{enumerate}
\def\labelenumi{\alph{enumi}.}
\tightlist
\item
  Persamaan yang memiliki pangkat selain satu (misal: \(x^2\))
\item
  Persamaan yang mempunyai produk dua variabel (misal: \(xy\))
\end{enumerate}

Dalam penyelesaian persamaan non-linier diperlukan akar-akar persamaan non-linier, dimana akar sebuah persamaan non-linier \(f\left(x\right)=0\) merupakan nilai \(x\) yang menyebabkan nilai \(f\left(x\right)\) sama dengan nol. Dalam hal ini dapat disimpulkan bahwa akar-akar penyelesaian persamaan non-linier merupakan titik potong antara kurva \(f\left(x\right)\) dengan sumbu \(x\). Ilustrasi penjelasan tersebut ditampilkan pada Gambar \ref{fig:root}.

\begin{figure}

{\centering \includegraphics[width=0.8\linewidth]{./images/root} 

}

\caption{Penyelesaian persamaan non-linier.}\label{fig:root}
\end{figure}

Contoh sederhana dari penentuan akar persamaan non-linier adalah penentuan akar persamaan kuadratik. Secara analitik penentuan akar persamaan kuadratik dapat dilakukan menggunakan Persamaan \eqref{eq:kuadratik}.

\begin{equation}
x_{1,2}=\frac{-b\pm\sqrt{b^2-4a}}{2a}
  \label{eq:kuadratik}
\end{equation}

Untuk masalah yang lebih rumit, penyelesaian analitik sudah tidak mungkin dilakukan. Metode numerik dapat digunakan untuk menyelesaikan masalah yang lebih kompleks. Untuk mengetahui apakah suatu persamaan non-linier memiliki akar-akar penyelesaian atau tidak, diperlukan analisa menggunakan Teorema berikut:

\begin{theorem}[root]
\protect\hypertarget{thm:unnamed-chunk-97}{}\label{thm:unnamed-chunk-97}Suatu range x={[}a,b{]} mempunyai akar bila f(a) dan f(b) berlawanan tanda atau memenuhi f(a).f(b)\textless0
\end{theorem}

Untuk memahami teorema tersebut perhatikan ilustrasi pada Gambar \ref{fig:Bolzano}.

\begin{figure}

{\centering \includegraphics[width=0.8\linewidth]{./images/Bolzano} 

}

\caption{Ilustrasi teorema Bolzano.}\label{fig:Bolzano}
\end{figure}

Pada Chapter \ref{rootfinding} ini, akan dilakukan sejumlah pembahasan antara lain:

\begin{itemize}
\tightlist
\item
  penentuan akar persamaan dengan metode tertutup
\item
  penentuan akar persamaan dengan metode terbuka
\item
  fungsi-fungsi \texttt{R} untuk mementukan akar persamaan non-linier
\item
  studi kasus
\end{itemize}

\hypertarget{bracketing}{%
\section{Metode Tertutup}\label{bracketing}}

Metode tertutup disebut juga metode \emph{bracketing}. Disebut sebagai metode tertutup karena dalam pencarian akar-akar persamaan non-linier dilakukan dalam suatu selang \(\left[a,b \right]\).

\hypertarget{table}{%
\subsection{Metode Tabel}\label{table}}

Penyelesaian persamaan non-linier menggunakan metode tabel dilakukan dengan membagi persamaan menjadi beberapa area, dimana untuk \(x=\left[a,b \right]\) dibagi sebanyak \(N\) bagian dan pada masing-masing bagian dihitung nilai \(f\left(x \right)\) sehingga diperoleh nilai \(f\left(x \right)\) pada setian \(N\) bagian.

Bila nilai \(f\left(x_k \right)=0\) atau mendekati nol, dimana \(a \le k \le b\), maka dikatakan bahwa \(x_k\) adalah penyelesaian persamaan \(f\left(x \right)\). Bila tidak ditemukan, dicari nilai \(f\left(x_k \right)\) dan \(f\left(x_{k+1} \right)\) yang berlawanan tanda. Bila tidak ditemukan, maka persamaan tersebut dapat dikatakan tidak mempunyai akar untuk rentang \(\left[a,b \right]\).

Bila akar persamaan tidak ditemukan, maka ada dua kemungkinan untuk menentukan akar persamaan, yaitu:

\begin{enumerate}
\def\labelenumi{\alph{enumi}.}
\tightlist
\item
  Akar persamaan ditentukan oleh nilai mana yang lebih dekat. Bila \(f\left(x_k\right)\le f\left(x_{k+1}\right)\), maka akarnya \(x_k\). Bila \(f\left(x_{k+1}\right)\le f\left(x_{k}\right)\), maka akarnya \(x_{k+1}\).
\item
  Perlu dicari lagi menggunakan rentang \(x=\left[x_{k}, x_{k+1} \right]\).
\end{enumerate}

Secara grafis penyelesaian persamaan non-linier menggunakan metode table disajikan pada Gambar \ref{fig:tabelviz}.

\begin{figure}

{\centering \includegraphics[width=0.8\linewidth]{./images/tabelviz} 

}

\caption{Ilustrasi metode tabel.}\label{fig:tabelviz}
\end{figure}

\begin{center}\rule{0.5\linewidth}{0.5pt}\end{center}

\textbf{Algoritma Metode Tabel}

\begin{enumerate}
\def\labelenumi{\arabic{enumi}.}
\tightlist
\item
  Definisikan fungsi \(f\left(x \right)\)
\item
  Tentukan rentang untuk \(x\) yang berupa batas bawah \(a\) dan batas atas \(b\).
\item
  Tentukan jumlah pembagi \(N\)
\item
  Hitung step pembagi
\end{enumerate}

\begin{equation}
h=\frac{b+a}{N}
  \label{eq:tabel1}
\end{equation}

\begin{enumerate}
\def\labelenumi{\arabic{enumi}.}
\setcounter{enumi}{4}
\tightlist
\item
  Untuk \(i=0\) s/d \(N\), hitung:
\end{enumerate}

\begin{equation}
x_i=a+i.h
  \label{eq:tabel2}
\end{equation}

\begin{equation}
y_i=f\left(x_i \right)
  \label{eq:tabel3}
\end{equation}

\begin{enumerate}
\def\labelenumi{\arabic{enumi}.}
\setcounter{enumi}{5}
\tightlist
\item
  Untuk \(i=0\) s/d \(N\), dimana
\end{enumerate}

\begin{itemize}
\item
  Bila \(f\left(x \right)=0\), maka akarnya \(x_k\)
\item
  Bila \(f\left(a \right) f\left(b \right) <0\), maka:

  \begin{itemize}
  \tightlist
  \item
    \(f\left(x_k\right)\le f\left(x_{k+1}\right)\), maka akarnya \(x_k\)
  \item
    Bila tida, \(x_{k+1}\) adalah penyelesaian atau dapat dikatakan penyelesaian berada diantara \(x_k\) dan \(x_{k+1}\).
  \end{itemize}
\end{itemize}

\begin{center}\rule{0.5\linewidth}{0.5pt}\end{center}

Kita dapat membuat suatu fungsi pada \texttt{R} untuk melakukan proses iterasi pada metode Tabel. Fungsi \texttt{root\_table()} akan melakukan iterasi berdasarkan step algoritma 1 sampai 5. Berikut adalah sintaks yang digunakan:

\begin{Shaded}
\begin{Highlighting}[]
\NormalTok{root\_table }\OtherTok{\textless{}{-}} \ControlFlowTok{function}\NormalTok{(f, a, b, }\AttributeTok{N=}\DecValTok{20}\NormalTok{)\{}
\NormalTok{    h }\OtherTok{\textless{}{-}} \FunctionTok{abs}\NormalTok{((a}\SpecialCharTok{+}\NormalTok{b)}\SpecialCharTok{/}\NormalTok{N)}
\NormalTok{    x }\OtherTok{\textless{}{-}} \FunctionTok{seq}\NormalTok{(}\AttributeTok{from=}\NormalTok{a, }\AttributeTok{to=}\NormalTok{b, }\AttributeTok{by=}\NormalTok{h)}
\NormalTok{    fx }\OtherTok{\textless{}{-}} \FunctionTok{rep}\NormalTok{(}\DecValTok{0}\NormalTok{, N}\SpecialCharTok{+}\DecValTok{1}\NormalTok{)}
    \ControlFlowTok{for}\NormalTok{(i }\ControlFlowTok{in} \DecValTok{1}\SpecialCharTok{:}\NormalTok{(N}\SpecialCharTok{+}\DecValTok{1}\NormalTok{))\{}
\NormalTok{      fx[i] }\OtherTok{\textless{}{-}} \FunctionTok{f}\NormalTok{(x[i])}
\NormalTok{    \}}
\NormalTok{    data }\OtherTok{\textless{}{-}} \FunctionTok{data.frame}\NormalTok{(}\AttributeTok{x=}\NormalTok{x, }\AttributeTok{fx=}\NormalTok{fx)}
    \FunctionTok{return}\NormalTok{(data)}
\NormalTok{\}}
\end{Highlighting}
\end{Shaded}

\begin{example}
\protect\hypertarget{exm:tabelexmp}{}\label{exm:tabelexmp}Carilah akar persamaan \(f\left(x \right)=x+e^{x}\) pada rentang \(x=\left[-1,0 \right]\)?
\end{example}

\textbf{Jawab}:

Sebagai permulaan, jumlah pembagi yang digunakan adalah \(N=10\). Dengan menggunakan fungsi \texttt{root\_table()} diperoleh hasil yang disajikan pada Tabel \ref{tab:tabeltabel}.

\begin{Shaded}
\begin{Highlighting}[]
\NormalTok{tabel }\OtherTok{\textless{}{-}} \FunctionTok{root\_table}\NormalTok{(}\AttributeTok{f=}\ControlFlowTok{function}\NormalTok{(x)\{x}\SpecialCharTok{+}\FunctionTok{exp}\NormalTok{(x)\},}
                     \AttributeTok{a=}\SpecialCharTok{{-}}\DecValTok{1}\NormalTok{, }\AttributeTok{b=}\DecValTok{0}\NormalTok{, }\AttributeTok{N=}\DecValTok{10}\NormalTok{)}
\end{Highlighting}
\end{Shaded}

\begin{table}

\caption{\label{tab:tabeltabel}Penyelesaian persamaan x+exp(x)=0}
\centering
\begin{tabular}[t]{r|r}
\hline
x & fx\\
\hline
-1.0 & -0.6321\\
\hline
-0.9 & -0.4934\\
\hline
-0.8 & -0.3507\\
\hline
-0.7 & -0.2034\\
\hline
-0.6 & -0.0512\\
\hline
-0.5 & 0.1065\\
\hline
-0.4 & 0.2703\\
\hline
-0.3 & 0.4408\\
\hline
-0.2 & 0.6187\\
\hline
-0.1 & 0.8048\\
\hline
0.0 & 1.0000\\
\hline
\end{tabular}
\end{table}

Berdasarkan Tabel \ref{tab:tabeltabel} diperoleh penyelesaian di antara \(-0,6\) dan \(-0,5\) dengan nilai \(f\left(x \right)\) masing-masing sebesar \(-0,0512\) dan \(-0,1065\), sehingga dapat diambil penyelesaian \(x=-0,6\). Kita dapat terus melakukan iterasi sampai memperoleh nilai \(f\left(x \right)\) \textless{} nilai toleransi dengan terus merubah rentang yang diberikan. Iterasi berikutnya dengan nilai pembagi sama dan rentang nilai \(x=\left[-0,6;-0,5\right]\) diperoleh nilai \(x=-0,57\) dan \(f\left(x \right)=0,00447\).

Untuk melihat gambaran lokasi akar, kita dapat pulang mengeplotkan data menggunakan fungsi plot. Berikut adalah fungsi yang digunakan:

\begin{figure}

{\centering \includegraphics[width=0.8\linewidth]{Algoritma_Pemrograman_R_files/figure-latex/tabelplot-1} 

}

\caption{Plot fungsi x+exp(x) pada rentang -1 sampai 0.}\label{fig:tabelplot}
\end{figure}

Untuk mengetahui lokasi akar dengan lebih jelas, kita dapat memperkecil lagi rentang nilai yang dimasukkan dalam fungsi \texttt{curve()}.

Metode tabel pada dasarnya memiliki kelemahan yaitu cukup sulit untuk memdapatkan error penyelesaian yang cukup kecil, sehingga metode ini jarang sekali digunakan untuk menyelesaikan persamaan non-linier. Namun, metode ini cukup baik digunakan dalam menentukan area penyelesaian sehingga dapat dijadikan acuan metode lain yang lebih baik.

\hypertarget{bisection}{%
\subsection{Metode Biseksi}\label{bisection}}

Prinsip metode bagi dua adalah mengurung akar fungsi pada interval \(x=\left[a,b \right]\) atau pada nilai \(x\) batas bawah \(a\) dan batas atas \(b\). Selanjutnya interval tersebut terus menerus dibagi 2 hingga sekecil mungkin, sehingga nilai hampiran yang dicari dapat ditentukan dengan tingkat toleransi tertentu. Untuk lebih memahami metode biseksi, perhatikan visualisasi pada Gambar \ref{fig:biseksi}.

\begin{figure}

{\centering \includegraphics[width=0.8\linewidth]{./images/biseksi} 

}

\caption{Ilustrasi metode biseksi.}\label{fig:biseksi}
\end{figure}

Metode biseksi merupakan metode yang paling mudah dan paling sederhana dibanding metode lainnya. Adapun sifat metode ini antara lain:

\begin{enumerate}
\def\labelenumi{\arabic{enumi}.}
\tightlist
\item
  Konvergensi lambat
\item
  Caranya mudah
\item
  Tidak dapat digunakan untuk mencari akar imaginer
\item
  Hanya dapat mencari satu akar pada satu siklus.
\end{enumerate}

\begin{center}\rule{0.5\linewidth}{0.5pt}\end{center}

\textbf{Algoritma Metode Biseksi}

\begin{enumerate}
\def\labelenumi{\arabic{enumi}.}
\tightlist
\item
  Definisikan fungsi \(f\left(x \right)\)
\item
  Tentukan rentang untuk \(x\) yang berupa batas bawah \(a\) dan batas atas \(b\).
\item
  Tentukan nilai toleransi \(e\) dan iterasi maksimum \(N\)
\item
  Hitung \(f\left(a \right)\) dan \(f\left(b \right)\)
\item
  Hitung:
\end{enumerate}

\begin{equation}
x=\frac{a+b}{2}
  \label{eq:biseksi1}
\end{equation}

\begin{enumerate}
\def\labelenumi{\arabic{enumi}.}
\setcounter{enumi}{5}
\tightlist
\item
  Hitung \(f\left(x \right)\)
\item
  Bila \(f\left(x \right).f\left(a \right)<0\), maka \(b=x\) dan \(f\left(b \right)=f\left(x \right)\). Bila tidak, \(a=x\) dan \(f\left(a \right)=f\left(x \right)\)
\item
  Bila \(\left|b-a \right|<e\) atau iterasi maksimum maka proses dihentikan dan didapatkan akar=\(x\), dan bila tidak ulangi langkah 6.
\item
  Jika sudah diperoleh nilai dibawah nilai toleransi, nilai akar selanjutnya dihitung berdasarkan Persamaan \eqref{eq:biseksi1} dengan nilai \(a\) dan \(b\) merupakan nilai baru yang diperoleh dari proses iterasi.
\end{enumerate}

\begin{center}\rule{0.5\linewidth}{0.5pt}\end{center}

Berdasarkan algoritma tersebut, kita dapat menyusun suatu fungsi pada \texttt{R} yang dapat digunakan untuk melakukan iterasi tersebut. Fungsi \texttt{root\_bisection()} merupakan fungsi yang telah penulis susun untuk melakukan iterasi menggunakan metode biseksi. Berikut adalah sintaks dari fungsi tersebut:

\begin{Shaded}
\begin{Highlighting}[]
\NormalTok{root\_bisection }\OtherTok{\textless{}{-}} \ControlFlowTok{function}\NormalTok{(f, a, b, }\AttributeTok{tol=}\FloatTok{1e{-}7}\NormalTok{, }\AttributeTok{N=}\DecValTok{100}\NormalTok{)\{}
\NormalTok{  iter }\OtherTok{\textless{}{-}} \DecValTok{0}
\NormalTok{  fa }\OtherTok{\textless{}{-}} \FunctionTok{f}\NormalTok{(a)}
\NormalTok{  fb }\OtherTok{\textless{}{-}} \FunctionTok{f}\NormalTok{(b)}
  
  \ControlFlowTok{while}\NormalTok{(}\FunctionTok{abs}\NormalTok{(b}\SpecialCharTok{{-}}\NormalTok{a)}\SpecialCharTok{\textgreater{}}\NormalTok{tol)\{}
\NormalTok{    iter }\OtherTok{\textless{}{-}}\NormalTok{ iter}\SpecialCharTok{+}\DecValTok{1}
    \ControlFlowTok{if}\NormalTok{(iter}\SpecialCharTok{\textgreater{}}\NormalTok{N)\{}
      \FunctionTok{warning}\NormalTok{(}\StringTok{"iterations maximum exceeded"}\NormalTok{)}
      \ControlFlowTok{break}
\NormalTok{    \}}
\NormalTok{    x }\OtherTok{\textless{}{-}}\NormalTok{ (a}\SpecialCharTok{+}\NormalTok{b)}\SpecialCharTok{/}\DecValTok{2}
\NormalTok{    fx }\OtherTok{\textless{}{-}} \FunctionTok{f}\NormalTok{(x)}
    \ControlFlowTok{if}\NormalTok{(fa}\SpecialCharTok{*}\NormalTok{fx}\SpecialCharTok{\textgreater{}}\DecValTok{0}\NormalTok{)\{}
\NormalTok{      a }\OtherTok{\textless{}{-}}\NormalTok{ x}
\NormalTok{      fa }\OtherTok{\textless{}{-}}\NormalTok{ fx}
\NormalTok{    \} }\ControlFlowTok{else}\NormalTok{\{}
\NormalTok{      b }\OtherTok{\textless{}{-}}\NormalTok{ x}
\NormalTok{      fb }\OtherTok{\textless{}{-}}\NormalTok{ fx}
\NormalTok{    \}}
\NormalTok{  \}}
  
  \CommentTok{\# iterasi nilai x sebagai return value}
\NormalTok{  root }\OtherTok{\textless{}{-}}\NormalTok{ (a}\SpecialCharTok{+}\NormalTok{b)}\SpecialCharTok{/}\DecValTok{2}
  \FunctionTok{return}\NormalTok{(}\FunctionTok{list}\NormalTok{(}\StringTok{\textasciigrave{}}\AttributeTok{function}\StringTok{\textasciigrave{}}\OtherTok{=}\NormalTok{f, }\AttributeTok{root=}\NormalTok{root, }\AttributeTok{iter=}\NormalTok{iter))}
\NormalTok{\}}
\end{Highlighting}
\end{Shaded}

\begin{example}
\protect\hypertarget{exm:biseksexmp}{}\label{exm:biseksexmp}Carilah akar persamaan \(f\left(x \right)=xe^{-x}+1\) pada rentang \(x=\left[-1,0 \right]\) dengan nilai toleransi sebesar \(10^-7\)?
\end{example}

\textbf{Jawab}:

Langkah pertama dalam penghitungan adalah menghitung nilai \(x\) menggunakan Persamaan \eqref{eq:biseksi1}.

\[
x=\frac{-1+0}{2}=-0,5
\]

Hitung nilai \(f\left(x \right)\) dan \(f\left(a \right)\).

\[
f\left(x \right)=-0,5.e^{0,5}+1=0,175639
\]

\[
f\left(a \right)=-1.e^{1}+1=-1,71828
\]

Berdasarkan hasil perhitungan diperoleh:

\[
f\left(x \right).f\left(a \right)<0
\]

Sehingga \(b=x\) dan \(f\left(b \right)=f\left(x \right)\). Iterasi dilakukan kembali dengan menggunakan nilai \(b\) tersebut.

Untuk mempersingkat waktu iterasi kita akan menggunakan fungsi \texttt{root\_bisection()} pada \texttt{R}. Berikut adalah sintaks yang digunakan:

\begin{Shaded}
\begin{Highlighting}[]
\FunctionTok{root\_bisection}\NormalTok{(}\ControlFlowTok{function}\NormalTok{(x)\{x}\SpecialCharTok{*}\FunctionTok{exp}\NormalTok{(}\SpecialCharTok{{-}}\NormalTok{x)}\SpecialCharTok{+}\DecValTok{1}\NormalTok{\},}
               \AttributeTok{a=}\SpecialCharTok{{-}}\DecValTok{1}\NormalTok{, }\AttributeTok{b=}\DecValTok{0}\NormalTok{)}
\end{Highlighting}
\end{Shaded}

\begin{verbatim}
## $`function`
## function(x){x*exp(-x)+1}
## <bytecode: 0x000001ef80501cd0>
## 
## $root
## [1] -0.5671
## 
## $iter
## [1] 24
\end{verbatim}

Berdasarkan hasil iterasi diperoleh akar persamaan \(x=-2.980232e-08\) dan iterasi yang diperlukan untuk memperolehnya sebanyak \(24\) iterasi.

\hypertarget{regulafalsi}{%
\subsection{Metode Regula Falsi}\label{regulafalsi}}

Metode regula falsi merupakan metode yang menyerupai metode biseksi, dimana iterasi dilakukan dengan terus melakukan pembaharuan rentang untuk memperoleh akar persamaan. Hal yang membedakan metode ini dengan metode biseksi adalah pencarian akar didasarkan pada slope (kemiringan) dan selisih tinggi dari kedua titik rentang. Titik pendekatan pada metode regula-falsi disajikan pada Persamaan \eqref{eq:rf}.

\begin{equation}
x=\frac{f\left(b\right).a-f\left(a\right).b}{f\left(b\right)-f\left(a\right)}
  \label{eq:rf}
\end{equation}

Ilustrasi dari metode regula falsi disajikan pada Gambar \ref{fig:regula}.

\begin{figure}

{\centering \includegraphics[width=0.8\linewidth]{./images/regula} 

}

\caption{Ilustrasi metode regula falsi.}\label{fig:regula}
\end{figure}

\begin{center}\rule{0.5\linewidth}{0.5pt}\end{center}

\textbf{Algoritma Metode Regula Falsi}

\begin{enumerate}
\def\labelenumi{\arabic{enumi}.}
\tightlist
\item
  Definisikan fungsi \(f\left(x \right)\)
\item
  Tentukan rentang untuk \(x\) yang berupa batas bawah \(a\) dan batas atas \(b\).
\item
  Tentukan nilai toleransi \(e\) dan iterasi maksimum \(N\)
\item
  Hitung \(f\left(a \right)\) dan \(f\left(b \right)\)
\item
  Untuk iterasi \(i=1\) s/d \(N\)
\end{enumerate}

\begin{itemize}
\tightlist
\item
  Hitung nilai \(x\) berdasarkan Persamaan \eqref{eq:rf}
\item
  Hitung \(f\left(x \right)\)
\item
  Hitung \(error=\left|f\left(x \right) \right|\)
\item
  Jika \(f\left(x \right).f\left(a \right)<0\), maka \(b=x\) dan \(f\left(b \right)=f\left(x \right)\). Jika tidak, \(a=x\) dan \(f\left(a \right)=f\left(x \right)\).
\end{itemize}

\begin{enumerate}
\def\labelenumi{\arabic{enumi}.}
\setcounter{enumi}{5}
\tightlist
\item
  Akar persamaan adalah \(x\)
\end{enumerate}

\begin{center}\rule{0.5\linewidth}{0.5pt}\end{center}

Fungsi \texttt{root\_rf()} didasarkan pada langkah-langkah di atas. Sintaks fungsi tersebut adalah sebagai berikut:

\begin{Shaded}
\begin{Highlighting}[]
\NormalTok{root\_rf }\OtherTok{\textless{}{-}} \ControlFlowTok{function}\NormalTok{(f, a, b, }\AttributeTok{tol=}\FloatTok{1e{-}7}\NormalTok{, }\AttributeTok{N=}\DecValTok{100}\NormalTok{)\{}
\NormalTok{  iter }\OtherTok{\textless{}{-}} \DecValTok{1}
\NormalTok{  fa }\OtherTok{\textless{}{-}} \FunctionTok{f}\NormalTok{(a)}
\NormalTok{  fb }\OtherTok{\textless{}{-}} \FunctionTok{f}\NormalTok{(b)}
\NormalTok{  x }\OtherTok{\textless{}{-}}\NormalTok{ ((fb}\SpecialCharTok{*}\NormalTok{a)}\SpecialCharTok{{-}}\NormalTok{(fa}\SpecialCharTok{*}\NormalTok{b))}\SpecialCharTok{/}\NormalTok{(fb}\SpecialCharTok{{-}}\NormalTok{fa)}
\NormalTok{  fx }\OtherTok{\textless{}{-}} \FunctionTok{f}\NormalTok{(x)}
  
  \ControlFlowTok{while}\NormalTok{(}\FunctionTok{abs}\NormalTok{(fx)}\SpecialCharTok{\textgreater{}}\NormalTok{tol)\{}
\NormalTok{    iter }\OtherTok{\textless{}{-}}\NormalTok{ iter}\SpecialCharTok{+}\DecValTok{1}
    \ControlFlowTok{if}\NormalTok{(iter}\SpecialCharTok{\textgreater{}}\NormalTok{N)\{}
      \FunctionTok{warning}\NormalTok{(}\StringTok{"iterations maximum exceeded"}\NormalTok{)}
      \ControlFlowTok{break}
\NormalTok{    \}}
    \ControlFlowTok{if}\NormalTok{(fa}\SpecialCharTok{*}\NormalTok{fx}\SpecialCharTok{\textgreater{}}\DecValTok{0}\NormalTok{)\{}
\NormalTok{      a }\OtherTok{\textless{}{-}}\NormalTok{ x}
\NormalTok{      fa }\OtherTok{\textless{}{-}}\NormalTok{ fx}
\NormalTok{    \} }\ControlFlowTok{else}\NormalTok{\{}
\NormalTok{      b }\OtherTok{\textless{}{-}}\NormalTok{ x}
\NormalTok{      fb }\OtherTok{\textless{}{-}}\NormalTok{ fx}
\NormalTok{    \}}
\NormalTok{    x }\OtherTok{\textless{}{-}}\NormalTok{ (fb}\SpecialCharTok{*}\NormalTok{a}\SpecialCharTok{{-}}\NormalTok{fa}\SpecialCharTok{*}\NormalTok{b)}\SpecialCharTok{/}\NormalTok{(fb}\SpecialCharTok{{-}}\NormalTok{fa)}
\NormalTok{    fx }\OtherTok{\textless{}{-}} \FunctionTok{f}\NormalTok{(x)}
\NormalTok{  \}}
  
  \CommentTok{\# iterasi nilai x sebagai return value}
\NormalTok{  root }\OtherTok{\textless{}{-}}\NormalTok{ x}
  \FunctionTok{return}\NormalTok{(}\FunctionTok{list}\NormalTok{(}\StringTok{\textasciigrave{}}\AttributeTok{function}\StringTok{\textasciigrave{}}\OtherTok{=}\NormalTok{f, }\AttributeTok{root=}\NormalTok{root, }\AttributeTok{iter=}\NormalTok{iter))}
\NormalTok{\}}
\end{Highlighting}
\end{Shaded}

\begin{example}
\protect\hypertarget{exm:rfexmp}{}\label{exm:rfexmp}Selesaikan persamaan non-linier pada Contoh \ref{exm:biseksexmp} menggunakan metode regula falsi pada rentang \(x=\left[-1,0 \right]\) dengan nilai toleransi sebesar \(10^-7\)?
\end{example}

\textbf{Jawab}:

Langkah pertama penyelesaian dilakukan dengan mencari nilai \(f\left(a \right)\) dan \(f\left(b \right)\).

\[
f\left(a \right)=-1.e^{1}+1=-1,71828
\]
\[
f\left(b \right)=0.e^{0}+1=1
\]
Hitung nilai \(x\) dan \(f\left(x \right)\).

\[
x=\frac{\left(1.-1\right)-\left(-1,71828.0\right)}{1+1,71828}=-0.36788
\]

\[
f\left(x \right)=-0.36788.e^{0.36788}+1=0.468536
\]

Berdasarkan hasil perhitungan diperoleh:

\[
f\left(x \right).f\left(a \right)<0
\]

Sehingga \(b=x\) dan \(f\left(b \right)=f\left(x \right)\). Iterasi dilakukan kembali dengan menggunakan nilai \(b\) tersebut.

Untuk mempercepat proses iterasi, kita dapat pula menggunakan fungsi \texttt{root\_rf()} pada \texttt{R}. Berikut adalah sintaks yang digunakan:

\begin{Shaded}
\begin{Highlighting}[]
\FunctionTok{root\_rf}\NormalTok{(}\ControlFlowTok{function}\NormalTok{(x)\{x}\SpecialCharTok{*}\FunctionTok{exp}\NormalTok{(}\SpecialCharTok{{-}}\NormalTok{x)}\SpecialCharTok{+}\DecValTok{1}\NormalTok{\},}
               \AttributeTok{a=}\SpecialCharTok{{-}}\DecValTok{1}\NormalTok{, }\AttributeTok{b=}\DecValTok{0}\NormalTok{)}
\end{Highlighting}
\end{Shaded}

\begin{verbatim}
## $`function`
## function(x){x*exp(-x)+1}
## <bytecode: 0x000001ef81ae2d40>
## 
## $root
## [1] -0.5671
## 
## $iter
## [1] 15
\end{verbatim}

Berdasarkan hasil perhitungan diperoleh nilai \(x=-0,5671433\) dan jumlah iterasi yang diperlukan adalah \(15\). Jumlah ini lebih sedikit dari jumlah iterasi yang diperlukan pada metode iterasi biseksi yang juga menunjukkan metode ini lebih cepat memperoleh persamaan dibandingkan metode biseksi.

\hypertarget{openmethod}{%
\section{Metode Terbuka}\label{openmethod}}

Metode terbuka merupakan metode yang menggunakan satu atau dua tebakan awal yang tidak memerlukan rentang sejumlah nilai. Metode terbuka terdiri dari beberapa jenis yaitu metode iterasi titik tetap, metode Newton-Raphson, dan metode Secant.

\hypertarget{fixpoint}{%
\subsection{Metode Iterasi Titik Tetap}\label{fixpoint}}

Metode iterasi titik tetap merupakan metode penyelesaian persamaan non-linier dengan cara menyelesaikan setiap variabel \(x\) yang ada dalam suatu persamaan dengan sebagian yang lain sehingga diperoleh \(x=g\left(x \right)\) untuk masing-masing variabel \(x\). Sebagai contoh, untuk menyelesaikan persamaan \(x+e^{x}=0\), maka persamaan tersebut perlu diubah menjadi \(x=e^x\) atau \(g\left(x \right)=e^x\). Secara grafis metode ini diilustrasikan seperti Gambar \ref{fig:fixpointiter}.

\begin{figure}

{\centering \includegraphics[width=0.8\linewidth]{./images/fixpointiter} 

}

\caption{Ilustrasi metode iterasi titik tetap.}\label{fig:fixpointiter}
\end{figure}

\begin{center}\rule{0.5\linewidth}{0.5pt}\end{center}

\textbf{Algoritma Metode Iterasi Titik Tetap}

\begin{enumerate}
\def\labelenumi{\arabic{enumi}.}
\tightlist
\item
  Definisikan \(f\left(x \right)\) dan \(g\left(x \right)\)
\item
  Tentukan nilai toleransi \(e\) dan iterasi masimum (N)
\item
  Tentukan tebakan awal \(x_0\)
\item
  Untuk iterasi \(i=1\) s/d \(N\) atau \(f\left(x_iterasi \right)\ge e \to x_i=g\left(x_{i-1} \right)\), Hitung \(f\left(x_i \right)\)
\item
  Akar persamaan adalah \(x\) terakhir yang diperoleh
\end{enumerate}

\begin{center}\rule{0.5\linewidth}{0.5pt}\end{center}

FUngsi \texttt{root\_fpi()} dapat digunakan untuk melakukan iterasi dengan argumen fungsi berupa persamaan non-linier, nilai tebakan awal, nilai toleransi, dan jumlah iterasi maksimum. Berikut adalah sintaks fungsi tersebut:

\begin{Shaded}
\begin{Highlighting}[]
\NormalTok{root\_fpi }\OtherTok{\textless{}{-}} \ControlFlowTok{function}\NormalTok{(f, x0, }\AttributeTok{tol=}\FloatTok{1e{-}7}\NormalTok{, }\AttributeTok{N=}\DecValTok{100}\NormalTok{)\{}
\NormalTok{  iter }\OtherTok{\textless{}{-}} \DecValTok{1}
\NormalTok{  xold }\OtherTok{\textless{}{-}}\NormalTok{ x0}
\NormalTok{  xnew }\OtherTok{\textless{}{-}} \FunctionTok{f}\NormalTok{(xold)}
  
  \ControlFlowTok{while}\NormalTok{(}\FunctionTok{abs}\NormalTok{(xnew}\SpecialCharTok{{-}}\NormalTok{xold)}\SpecialCharTok{\textgreater{}}\NormalTok{tol)\{}
\NormalTok{    iter }\OtherTok{\textless{}{-}}\NormalTok{ iter}\SpecialCharTok{+}\DecValTok{1}
    \ControlFlowTok{if}\NormalTok{(iter}\SpecialCharTok{\textgreater{}}\NormalTok{N)\{}
      \FunctionTok{stop}\NormalTok{(}\StringTok{"No solutions found"}\NormalTok{)}
\NormalTok{    \}}
\NormalTok{    xold }\OtherTok{\textless{}{-}}\NormalTok{ xnew}
\NormalTok{    xnew }\OtherTok{\textless{}{-}} \FunctionTok{f}\NormalTok{(xold)}
\NormalTok{  \}}
  
\NormalTok{  root }\OtherTok{\textless{}{-}}\NormalTok{ xnew}
  \FunctionTok{return}\NormalTok{(}\FunctionTok{list}\NormalTok{(}\StringTok{\textasciigrave{}}\AttributeTok{function}\StringTok{\textasciigrave{}}\OtherTok{=}\NormalTok{f, }\AttributeTok{root=}\NormalTok{root, }\AttributeTok{iter=}\NormalTok{iter))}
\NormalTok{\}}
\end{Highlighting}
\end{Shaded}

\begin{example}
\protect\hypertarget{exm:fixexmp}{}\label{exm:fixexmp}Selesaikan persamaan non-linier pada Contoh \ref{exm:biseksexmp} menggunakan metode iterasi titik tetap?
\end{example}

\textbf{Jawab}:

Untuk menyelesaikan persamaan non-linier tersebut kita perlu mentransformasi persamaan non-linier tersebut terlebih dahulu.

\[
xe^{-x}+1=0\ \to x=-\frac{1}{e^{-x}}
\]

Untuk tebakan awal digunakan nilai \(x=-1\)

\[
x_1 = -\frac{1}{e^{1}}=-2,718282
\]

Nilai \(x\) tersebut selanjutnya dijadikan nilai input pada iterasi selanjutnya:

\[
x_2 = -\frac{1}{e^{2,718282}}=-0,06598802
\]

iterasi terus dilakukan sampai diperoleh \(\left| x_{i+1}-x_i \right|\le e\).

Untuk mempercepat proses iterasi kita dapat menggunakan bantuan fungsi \texttt{root\_fpi()}. Berikut adalah sintaks yang digunakan:

\begin{Shaded}
\begin{Highlighting}[]
\FunctionTok{root\_fpi}\NormalTok{(}\ControlFlowTok{function}\NormalTok{(x)\{}\SpecialCharTok{{-}}\DecValTok{1}\SpecialCharTok{/}\FunctionTok{exp}\NormalTok{(}\SpecialCharTok{{-}}\NormalTok{x)\}, }\AttributeTok{x0=}\SpecialCharTok{{-}}\DecValTok{1}\NormalTok{)}
\end{Highlighting}
\end{Shaded}

\begin{verbatim}
## $`function`
## function(x){-1/exp(-x)}
## <bytecode: 0x000001effdcd1668>
## 
## $root
## [1] -0.5671
## 
## $iter
## [1] 29
\end{verbatim}

Berdasarkan hasil iterasi diperoleh nilai \(x=-0,5671433\) dengan jumlah iterasi yang diperlukan sebanyak \(29\) kali. Jumlah iterasi akan bergantung dengan nilai tebakan awal yang kita berikan. Semakin dekat nilai tersebut dengan akar, semakin cepat nilai akar diperoleh.

\hypertarget{newtonraphson}{%
\subsection{Metode Newton-Raphson}\label{newtonraphson}}

Metode Newton-Raphson merupakan metode penyelesaian persamaan non-linier dengan menggunakan pendekatan satu titik awal dan mendekatinya dengan memperhatikan slope atau gradien. titik pendekatan dinyatakan pada Persamaan \eqref{eq:rootnr}.

\begin{equation}
x_{n+1}=x_n-\frac{f\left(x_n\right)}{f'\left(x_n\right)}
  \label{eq:rootnr}
\end{equation}

Ilustrasi metode Newton-Raphson disajikan pada Gambar \ref{fig:nrviz}.

\begin{figure}

{\centering \includegraphics[width=0.95\linewidth]{./images/nrviz} 

}

\caption{Ilustrasi metode Newton-Raphson.}\label{fig:nrviz}
\end{figure}

\begin{center}\rule{0.5\linewidth}{0.5pt}\end{center}

\textbf{Algoritma Metode Newton-Raphson}

\begin{enumerate}
\def\labelenumi{\arabic{enumi}.}
\tightlist
\item
  Definisikan \(f\left(x \right)\) dan \(f'\left(x \right)\)
\item
  Tentukan nilai toleransi \(e\) dan iterasi masimum (N)
\item
  Tentukan tebakan awal \(x_0\)
\item
  Hitung \(f\left(x_0 \right)\) dan \(f'\left(x_0 \right)\)
\item
  Untuk iterasi \(i=1\) s/d \(N\) atau \(\left|f\left(x \right) \right|\ge e\), hitung \(x\) menggunakan Persamaan \eqref{eq:rootnr}
\item
  Akar persamaan merupakan nilai \(x_i\) terakhir yang diperoleh.
\end{enumerate}

\begin{center}\rule{0.5\linewidth}{0.5pt}\end{center}

Fungsi \texttt{root\_newton()} merupakan fungsi yang dibuat menggunakan algoritma di atas. Fungsi tersebut dituliskan pada sintaks berikut:

\begin{Shaded}
\begin{Highlighting}[]
\NormalTok{root\_newton }\OtherTok{\textless{}{-}} \ControlFlowTok{function}\NormalTok{(f, fp, x0, }\AttributeTok{tol=}\FloatTok{1e{-}7}\NormalTok{, }\AttributeTok{N=}\DecValTok{100}\NormalTok{)\{}
\NormalTok{  iter }\OtherTok{\textless{}{-}} \DecValTok{0}
\NormalTok{  xold}\OtherTok{\textless{}{-}}\NormalTok{x0}
\NormalTok{  xnew }\OtherTok{\textless{}{-}}\NormalTok{ xold }\SpecialCharTok{+} \DecValTok{10}\SpecialCharTok{*}\NormalTok{tol}
  
  \ControlFlowTok{while}\NormalTok{(}\FunctionTok{abs}\NormalTok{(xnew}\SpecialCharTok{{-}}\NormalTok{xold)}\SpecialCharTok{\textgreater{}}\NormalTok{tol)\{}
\NormalTok{    iter }\OtherTok{\textless{}{-}}\NormalTok{ iter}\SpecialCharTok{+}\DecValTok{1}
    \ControlFlowTok{if}\NormalTok{(iter}\SpecialCharTok{\textgreater{}}\NormalTok{N)\{}
      \FunctionTok{stop}\NormalTok{(}\StringTok{"No solutions found"}\NormalTok{)}
\NormalTok{    \}}
\NormalTok{    xold}\OtherTok{\textless{}{-}}\NormalTok{xnew}
\NormalTok{    xnew }\OtherTok{\textless{}{-}}\NormalTok{ xold }\SpecialCharTok{{-}} \FunctionTok{f}\NormalTok{(xold)}\SpecialCharTok{/}\FunctionTok{fp}\NormalTok{(xold)  }
\NormalTok{  \}}
  
\NormalTok{  root}\OtherTok{\textless{}{-}}\NormalTok{xnew}
  \FunctionTok{return}\NormalTok{(}\FunctionTok{list}\NormalTok{(}\StringTok{\textasciigrave{}}\AttributeTok{function}\StringTok{\textasciigrave{}}\OtherTok{=}\NormalTok{f, }\AttributeTok{root=}\NormalTok{root, }\AttributeTok{iter=}\NormalTok{iter))}
\NormalTok{\}}
\end{Highlighting}
\end{Shaded}

\begin{example}
\protect\hypertarget{exm:newtonraphsonexmp}{}\label{exm:newtonraphsonexmp}Selesaikan persamaan non-linier \(x-e^{-x}=0\) menggunakan metode Newton-Raphson?
\end{example}

\textbf{Jawab}:

Untuk dapat menggunakan metode Newton-Raphson, terlebih dahulu kita perlu memperoleh turunan pertama dari persamaan tersebut.

\[
f\left(x\right)=x-e^{-x}\to f'\left(x\right)=1+e^{-x}
\]

Tebakan awal yang digunakan adalah \(x=0\).

\[
f\left(x_0\right)=0-e^{-0}=-1
\]
\[
f'\left(x_0\right)=1+e^{-0}=2
\]

Hitung nilai \(x\) baru:

\[
x_1=x_0-\frac{f\left(x_0\right)}{f'\left(x_0\right)}=0-\frac{-1}{2}=0,5
\]

Untuk mempercepat proses iterasi, kita dapat menggunakan fungsi \texttt{root\_newton()}. Berikut adalah sintaks yang digunakan:

\begin{Shaded}
\begin{Highlighting}[]
\FunctionTok{root\_newton}\NormalTok{(}\ControlFlowTok{function}\NormalTok{(x)\{x}\SpecialCharTok{{-}}\FunctionTok{exp}\NormalTok{(}\SpecialCharTok{{-}}\NormalTok{x)\},}
            \ControlFlowTok{function}\NormalTok{(x)\{}\DecValTok{1}\SpecialCharTok{+}\FunctionTok{exp}\NormalTok{(}\SpecialCharTok{{-}}\NormalTok{x)\},}
              \AttributeTok{x0=}\DecValTok{0}\NormalTok{)}
\end{Highlighting}
\end{Shaded}

\begin{verbatim}
## $`function`
## function(x){x-exp(-x)}
## <bytecode: 0x000001eff54e8c80>
## 
## $root
## [1] 0.5671
## 
## $iter
## [1] 5
\end{verbatim}

Berdasarkan hasil iterasi diperoleh akar penyelesaian persamaan non-linier adalah \(x=0,5671433\) dengan jumlah iterasi yang diperlukan adalah \(5\) iterasi.

Dalam penerapannya metode Newton-Raphson dapat mengalami kendala. Kendala yang dihadapi adalah sebagai berikut:

\begin{enumerate}
\def\labelenumi{\arabic{enumi}.}
\tightlist
\item
  titik pendekatan tidak dapat digunakan jika merupakan titik ekstrim atau titik puncak. Hal ini disebabkan pada titik ini nilai \(f'\left(x \right)=0\). Untuk memahaminya perhatikan ilustasi yang disajikan pada Gambar \ref{fig:nrviz2}. Untuk menatasi kendala ini biasanya titik pendekatan akan digeser.
\end{enumerate}

\begin{figure}

{\centering \includegraphics[width=0.9\linewidth]{./images/nrviz2} 

}

\caption{Ilustrasi titik pendekatan di titik puncak.}\label{fig:nrviz2}
\end{figure}

\begin{enumerate}
\def\labelenumi{\arabic{enumi}.}
\setcounter{enumi}{1}
\tightlist
\item
  Sulit memperoleh penyelesaian ketika titik pendekatan berada diantara 2 titik stasioner. Untuk memahami kendala ini perhatikan Gambar \ref{fig:nrviz3}. Untuk menghindarinya, penentuan titik pendekatan dapat menggunakan bantuan metode tabel.
\end{enumerate}

\begin{figure}

{\centering \includegraphics[width=0.9\linewidth]{./images/nrviz3} 

}

\caption{Ilustrasi titik pendekatan diantara 2 titik stasioner.}\label{fig:nrviz3}
\end{figure}

\begin{enumerate}
\def\labelenumi{\arabic{enumi}.}
\setcounter{enumi}{2}
\tightlist
\item
  Turunan persamaan sering kali sulit untuk diperoleh (tidak dapat dikerjakan dengan metode analitik).
\end{enumerate}

\hypertarget{secant}{%
\subsection{Metode Secant}\label{secant}}

Metode Secant merupakan perbaikan dari metode regula-falsi dan Newton Raphson, dimana kemiringan dua titik dinyatakan secara diskrit dengan mengambil bentuk garis lurus yang melalui satu titik. Persamaan yang dihasilkan disajikan pada Persamaan \eqref{eq:secant}.

\begin{equation}
y-y_0=m\left(x-x_0\right)
  \label{eq:secant}
\end{equation}

Nilai \(m\) merupakan transformasi persamaan tersebut.

\begin{equation}
m_n=\frac{f\left(x_n\right)-f\left(x_{n-1}\right)}{x_n-x_{n-1}}
  \label{eq:secant2}
\end{equation}

Bila \(y=f\left(x\right)\) dan \(y_n\) dan \(x_n\) diketahui, maka titik ke \(n+1\) adalah:

\begin{equation}
y_{n+1}-y_n=m_n\left(x_{n+1}-x_n\right)
  \label{eq:secant3}
\end{equation}

Bila titik \(x_{n+1}\) dianggap akar persamaan maka nilai \(y_{n+1}=0\), sehingga diperoleh:

\begin{equation}
-y_n=m_n\left(x_{n+1}-x_n\right)
  \label{eq:secant4}
\end{equation}

\begin{equation}
\frac{m_nx_n-y_n}{m_n}=x_{n+1}
  \label{eq:secant5}
\end{equation}

atau

\begin{equation}
x_{n+1}=x_n-y_n\frac{1}{m_n}
  \label{eq:secant6}
\end{equation}

\begin{equation}
x_{n+1}=x_n-f\left(x_n\right)\frac{x_n-x_{n+1}}{f\left(x_n\right)-f\left(x_{n+1}\right)}
  \label{eq:secant7}
\end{equation}

Berdasarkan Persamaan \eqref{eq:secant7} diketahui bahwa untuk memperoleh akar persamaan diperlukan 2 buah titik pendekatan. Dalam buku ini akan digunakan titik pendekatan kedua merupakan titik pendekatan pertama ditambah sepuluh kali nilai toleransi.

\begin{equation}
x_1=x_0+10*tol
  \label{eq:secant8}
\end{equation}

\begin{center}\rule{0.5\linewidth}{0.5pt}\end{center}

\textbf{Algoritma Metode Secant}

\begin{enumerate}
\def\labelenumi{\arabic{enumi}.}
\tightlist
\item
  Definisikan \(f\left(x \right)\) dan \(f'\left(x \right)\)
\item
  Tentukan nilai toleransi \(e\) dan iterasi masimum (N)
\item
  Tentukan tebakan awal \(x_0\) dan \(x_1\)
\item
  Hitung \(f\left(x_0 \right)\) dan \(f\left(x_1 \right)\)
\item
  Untuk iterasi \(i=1\) s/d \(N\) atau \(\left|f\left(x \right) \right|\ge e\), hitung \(x\) menggunakan Persamaan \eqref{eq:secant7}
\item
  Akar persamaan adalah nilai x yang terakhir.
\end{enumerate}

\begin{center}\rule{0.5\linewidth}{0.5pt}\end{center}

Fungsi \texttt{root\_secant()} merupakan fungsi yang penulis buat untuk melakukan iterasi menggunakan metode Secant. Berikut merupakan sintaks dari fungsi tersebut:

\begin{Shaded}
\begin{Highlighting}[]
\NormalTok{root\_secant }\OtherTok{\textless{}{-}} \ControlFlowTok{function}\NormalTok{(f, x, }\AttributeTok{tol=}\FloatTok{1e{-}7}\NormalTok{, }\AttributeTok{N=}\DecValTok{100}\NormalTok{)\{}
\NormalTok{  iter }\OtherTok{\textless{}{-}} \DecValTok{0}
  
\NormalTok{  xold }\OtherTok{\textless{}{-}}\NormalTok{ x}
\NormalTok{  fxold }\OtherTok{\textless{}{-}} \FunctionTok{f}\NormalTok{(x)}
\NormalTok{  x }\OtherTok{\textless{}{-}}\NormalTok{ xold}\SpecialCharTok{+}\DecValTok{10}\SpecialCharTok{*}\NormalTok{tol}
  
  \ControlFlowTok{while}\NormalTok{(}\FunctionTok{abs}\NormalTok{(x}\SpecialCharTok{{-}}\NormalTok{xold)}\SpecialCharTok{\textgreater{}}\NormalTok{tol)\{}
\NormalTok{    iter }\OtherTok{\textless{}{-}}\NormalTok{ iter}\SpecialCharTok{+}\DecValTok{1}
    \ControlFlowTok{if}\NormalTok{(iter}\SpecialCharTok{\textgreater{}}\NormalTok{N)}
      \FunctionTok{stop}\NormalTok{(}\StringTok{"No solutions found"}\NormalTok{)}
    
\NormalTok{    fx }\OtherTok{\textless{}{-}} \FunctionTok{f}\NormalTok{(x)}
\NormalTok{    xnew }\OtherTok{\textless{}{-}}\NormalTok{ x }\SpecialCharTok{{-}}\NormalTok{ fx}\SpecialCharTok{*}\NormalTok{((x}\SpecialCharTok{{-}}\NormalTok{xold)}\SpecialCharTok{/}\NormalTok{(fx}\SpecialCharTok{{-}}\NormalTok{fxold))}
\NormalTok{    xold }\OtherTok{\textless{}{-}}\NormalTok{ x}
\NormalTok{    fxold }\OtherTok{\textless{}{-}}\NormalTok{ fx}
\NormalTok{    x }\OtherTok{\textless{}{-}}\NormalTok{ xnew}
\NormalTok{  \}}
  
\NormalTok{  root}\OtherTok{\textless{}{-}}\NormalTok{xnew}
  \FunctionTok{return}\NormalTok{(}\FunctionTok{list}\NormalTok{(}\StringTok{\textasciigrave{}}\AttributeTok{function}\StringTok{\textasciigrave{}}\OtherTok{=}\NormalTok{f, }\AttributeTok{root=}\NormalTok{root, }\AttributeTok{iter=}\NormalTok{iter))}
\NormalTok{\}}
\end{Highlighting}
\end{Shaded}

\begin{example}
\protect\hypertarget{exm:secantexmp}{}\label{exm:secantexmp}Selesaikan persamaan non-linier pada Contoh \ref{exm:newtonraphsonexmp} menggunakan metode Secant?
\end{example}

\textbf{Jawab}:

Untuk menyelesaikan persamaan tersebut digunakan nilai pendekatan awal \(x_0=0\) dan \(x_1=0+10*10^{-7}=10^{-6}\).

\[
f\left(x_0 \right)=0-e^{-0}=-1
\]

\[
f\left(x_1 \right)=10^{-6}-e^{-10^{-6}}=-0,999998
\]

Hitung nilai \(x_2\) dan \(f\left(x_2 \right)\).

\[
x_2=0+0,999998\frac{10^{-6}-0}{-0,999998+1}=0,499999
\]

Untuk mempercepat proses iterasi kita dapat menggunakan fungsi \texttt{root\_secant()} pada \texttt{R}. Berikut sintaks yang digunakan:

\begin{Shaded}
\begin{Highlighting}[]
\FunctionTok{root\_secant}\NormalTok{(}\ControlFlowTok{function}\NormalTok{(x)\{x}\SpecialCharTok{{-}}\FunctionTok{exp}\NormalTok{(}\SpecialCharTok{{-}}\NormalTok{x)\}, }\AttributeTok{x=}\DecValTok{0}\NormalTok{)}
\end{Highlighting}
\end{Shaded}

\begin{verbatim}
## $`function`
## function(x){x-exp(-x)}
## <bytecode: 0x000001effe747840>
## 
## $root
## [1] 0.5671
## 
## $iter
## [1] 6
\end{verbatim}

Berdasarkan hasil iterasi diperoleh nilai akar penyelesaian adalah \(x=0,5671433\) dengan iterasi dilakukan sebanyak \(6\) kali.

Secara umum metode Secant menawarkan sejumlah keuntungan dibanding metode lainnya. Pertama, seperti metode Newton-Raphson dan tidak seperti metode tertutup lainnya, metode ini tidak memerlukan rentang pencarian akar penyelesaian. Kedua, tidak seperti metode Newton-Raphson, metode ini tidak memerlukan pencarian turunan pertama persamaan non-linier secara analitik, dimana tidak dapat dilakukan otomasi pada setiap kasus.

Adapun kerugian dari metode ini adalah berpotensi menghasilkan hasil yang tidak konvergen sama seperti metode terbuka lainnya. Selain itu, kecepatan konvergensinya lebih lambat dibanding metode Newton-Raphson.

\hypertarget{uniroot}{%
\section{\texorpdfstring{Penyelesaian Persamaan Non-Linier Menggunakan Fungsi \texttt{uniroot} dan \texttt{uniroot.all}}{Penyelesaian Persamaan Non-Linier Menggunakan Fungsi uniroot dan uniroot.all}}\label{uniroot}}

Paket \texttt{base} pada \texttt{R} menyediakan fungsi \texttt{uniroot()} untuk mencari akar persamaan suatu fungsi pada rentang spesifik. Fungsi ini menggunakan metode Brent yaitu kombinasi antara \emph{root bracketing}, biseksi, dan interpolasi invers kuadrat. Format fungsi tersebut secara sederhana adalah sebagai berikut:

\begin{Shaded}
\begin{Highlighting}[]
\FunctionTok{uniroot}\NormalTok{(f, interval, }\AttributeTok{tol=}\NormalTok{.Machine}\SpecialCharTok{$}\NormalTok{double.eps}\SpecialCharTok{\^{}}\FloatTok{0.25}\NormalTok{, }
        \AttributeTok{maxiter=}\DecValTok{1000}\NormalTok{)}
\end{Highlighting}
\end{Shaded}

\begin{quote}
\textbf{Catatan}:

\begin{itemize}
\tightlist
\item
  \textbf{f}: persamaan non-linier
\item
  \textbf{interval}: vektor interval batas bawah dan atas
\item
  \textbf{tol}: nilai toleransi
\item
  \textbf{maxiter}: iterasi maksimum
\end{itemize}
\end{quote}

Berikut adalah contoh penerapan fungsi \texttt{uniroot()}:

\begin{Shaded}
\begin{Highlighting}[]
\FunctionTok{uniroot}\NormalTok{(}\ControlFlowTok{function}\NormalTok{(x)\{x}\SpecialCharTok{*}\FunctionTok{exp}\NormalTok{(}\SpecialCharTok{{-}}\NormalTok{x)}\SpecialCharTok{+}\DecValTok{1}\NormalTok{\},}
        \AttributeTok{interval=}\FunctionTok{c}\NormalTok{(}\SpecialCharTok{{-}}\DecValTok{1}\NormalTok{,}\DecValTok{0}\NormalTok{), }\AttributeTok{tol=}\FloatTok{1e{-}7}\NormalTok{)}
\end{Highlighting}
\end{Shaded}

\begin{verbatim}
## $root
## [1] -0.5671
## 
## $f.root
## [1] 1.533e-08
## 
## $iter
## [1] 7
## 
## $init.it
## [1] NA
## 
## $estim.prec
## [1] 5e-08
\end{verbatim}

Berdasarkan hasil iterasi diperoleh akar persamaan tersebut adalah \(-0,5671433\) dengan jumlah iterasi sebanyak \(7\) iterasi dan tingkat presisi sebesar \(5e-08\).

Fungsi lain yang dapat digunakan untuk mencari akar persamaan adalah \texttt{uniroot.all()} dari paket \texttt{rootSolve}. Fungsi ini mengatasi kelemahan dari \texttt{uniroot()}, dimana \texttt{uniroot()} tidak bekerja jika fungsi hanya menyentuh dan tidak melewati sumbu nol \(y=0\). Untuk memahaminya perhatikan contoh berikut:

\begin{Shaded}
\begin{Highlighting}[]
\FunctionTok{uniroot}\NormalTok{(}\ControlFlowTok{function}\NormalTok{(x)\{}\FunctionTok{sin}\NormalTok{(x)}\SpecialCharTok{+}\DecValTok{1}\NormalTok{\}, }\FunctionTok{c}\NormalTok{(}\SpecialCharTok{{-}}\NormalTok{pi,}\DecValTok{0}\NormalTok{))}
\end{Highlighting}
\end{Shaded}

Bandingkan dengan sintaks berikut:

\begin{Shaded}
\begin{Highlighting}[]
\FunctionTok{uniroot}\NormalTok{(}\ControlFlowTok{function}\NormalTok{(x)\{}\FunctionTok{sin}\NormalTok{(x)}\SpecialCharTok{+}\DecValTok{1}\NormalTok{\}, }\FunctionTok{c}\NormalTok{(}\SpecialCharTok{{-}}\NormalTok{pi,}\SpecialCharTok{{-}}\NormalTok{pi}\SpecialCharTok{/}\DecValTok{2}\NormalTok{))}
\end{Highlighting}
\end{Shaded}

\begin{verbatim}
## $root
## [1] -1.571
## 
## $f.root
## [1] 0
## 
## $iter
## [1] 0
## 
## $init.it
## [1] NA
## 
## $estim.prec
## [1] 0
\end{verbatim}

Untuk menggunakan fungsi \texttt{uniroot.all()}, jalankan sintaks berikut:

\begin{Shaded}
\begin{Highlighting}[]
\FunctionTok{library}\NormalTok{(rootSolve)}
\end{Highlighting}
\end{Shaded}

Jalankan kembali fungsi dan rentang di mana \texttt{uniroot()} tidak dapat bekerja:

\begin{Shaded}
\begin{Highlighting}[]
\FunctionTok{uniroot.all}\NormalTok{(}\ControlFlowTok{function}\NormalTok{(x)\{}\FunctionTok{sin}\NormalTok{(x)}\SpecialCharTok{+}\DecValTok{1}\NormalTok{\}, }\FunctionTok{c}\NormalTok{(}\SpecialCharTok{{-}}\NormalTok{pi,}\DecValTok{0}\NormalTok{))}
\end{Highlighting}
\end{Shaded}

\begin{verbatim}
## [1] -1.571
\end{verbatim}

\hypertarget{akar-persamaan-polinomial-menggunakan-fungsi-polyroot}{%
\section{\texorpdfstring{Akar Persamaan Polinomial Menggunakan Fungsi \texttt{polyroot}}{Akar Persamaan Polinomial Menggunakan Fungsi polyroot}}\label{akar-persamaan-polinomial-menggunakan-fungsi-polyroot}}

Fungsi \texttt{polyroot()} pada paket \texttt{base} dapat digunakan untuk memperoleh akar dari suatu polinomial. Algortima yang digunakan dalam fungsi tersebut adalah algoritma Jenkins dan Traub.

Untuk dapat menggunakannya kita hanya perlu memasukkan vektor koefisien dari polinomial. Pengisian elemen dalam vektor dimulai dari variabel dengan pangkat tertinggi menuju variabel dengan pangkat terendah. Berikut adalah contoh bagaimana fungsi \texttt{polyroot()} digunakan untuk mencari akar polinomial \(f\left(x\right)=x^2+1\):

\begin{Shaded}
\begin{Highlighting}[]
\FunctionTok{polyroot}\NormalTok{(}\FunctionTok{c}\NormalTok{(}\DecValTok{1}\NormalTok{,}\DecValTok{0}\NormalTok{,}\DecValTok{1}\NormalTok{))}
\end{Highlighting}
\end{Shaded}

\begin{verbatim}
## [1] 0+1i 0-1i
\end{verbatim}

Contoh lainnya adalah mencari akar polinomial \(f\left(x\right)=4x^2+5x+6\):

\begin{Shaded}
\begin{Highlighting}[]
\FunctionTok{polyroot}\NormalTok{(}\FunctionTok{c}\NormalTok{(}\DecValTok{4}\NormalTok{,}\DecValTok{5}\NormalTok{,}\DecValTok{6}\NormalTok{))}
\end{Highlighting}
\end{Shaded}

\begin{verbatim}
## [1] -0.4167+0.7022i -0.4167-0.7022i
\end{verbatim}

Pembaca dapat mencoba membuktikan hasil yang diperoleh tersebut menggunakan metode analitik.

\hypertarget{studi-kasus}{%
\section{Studi Kasus}\label{studi-kasus}}

Penerapan penyelesaian sistem persamaan non-linier banyak dijumpai dalam berbagai kasus di bidang lingkungan. Pada bagian ini penulis tidak akan menjelaskan seluruhnya. Penulis hanya akan menjelaskan penerapannya pada sebuah persamaan yaitu Hukum Bernoulli.

\hypertarget{persamaan-van-der-walls}{%
\subsection{Persamaan Van Der Walls}\label{persamaan-van-der-walls}}

\hypertarget{hukum-bernoulli}{%
\subsection{Hukum Bernoulli}\label{hukum-bernoulli}}

Misalkan terdapat sebuah saluran dengan penampang sesuai dengan Gambar \ref{fig:bernoulli}.

\begin{figure}

{\centering \includegraphics[width=0.8\linewidth]{./images/bernoulli} 

}

\caption{Aliran fluida pada sebuah pipa.}\label{fig:bernoulli}
\end{figure}

Berdasarkan hukum Bernoulli, maka diperoleh persamaan berikut:

\begin{equation}
\frac{Q^2}{2gb^2h_0^2}+h_0=\frac{Q^2}{2gb^2h^2}+h+H
  \label{eq:bernoullieq}
\end{equation}

Persamaan tersebut dapat dilakukan transformasi menjadi persamaan berikut:

\begin{equation}
f\left(h\right)=h^3+\left(H-\frac{Q^2}{2gb^2h_0^2}-h_0\right)h^2+\frac{Q^2}{2gb^2}=0
  \label{eq:bernoullieq2}
\end{equation}

Data-data terkait saluran tersebut adalah sebagai berikut:

\begin{itemize}
\tightlist
\item
  \(Q=1,2\ \frac{m^3}{\det\ }\) = volume aliran fluida tiap satuan waktu
\item
  \(g=9,81\ \frac{m}{s^2}\) =percepatan gravitasi
\item
  \(b=1,8\ m\) =lebar pipa
\item
  \(h_0=0,6\ m\) =ketinggian air maksimum
\item
  \(H=0,075\ m\) =tinggi pelebaran pipa
\item
  \(h\) = ketinggian air
\end{itemize}

Kita dapat menggunakan pendekatan numerik untuk menentukan \(h\). Pada studi kasus ini tidak dijelaskan lokasi dimana akar penyelesaian berada, sehingga metode terbuka seperti Secant cukup sesuai untuk menyelesaikannya:

Berikut adalah persamaan yang baru setelah seluruh data dimasukkan kedalam tiap variabelnya:

\[
f\left(h\right)=h^3+\left(0,075-\frac{1,2^2}{2\times9,81\times1,8^2\times0,6^2}-0,6\right)h^2+\frac{1,2^2}{2\times9,81\times1,8^2}=0
\]

Untuk penyelesaiannya penulis akan memberikan tebakan awal nilai \(h=h_0=0,6\). Berikut adalah sintaks penyelesaian menggunakan metode secant:

\begin{Shaded}
\begin{Highlighting}[]
\NormalTok{f }\OtherTok{\textless{}{-}} \ControlFlowTok{function}\NormalTok{(h)\{}
\NormalTok{  (h}\SpecialCharTok{\^{}}\DecValTok{3}\NormalTok{) }\SpecialCharTok{+}\NormalTok{ ((}\FloatTok{0.075}\SpecialCharTok{{-}}\NormalTok{((}\FloatTok{1.2}\SpecialCharTok{\^{}}\DecValTok{2}\NormalTok{)}\SpecialCharTok{/}\NormalTok{(}\DecValTok{2}\SpecialCharTok{*}\FloatTok{9.81}\SpecialCharTok{*}\NormalTok{(}\FloatTok{1.8}\SpecialCharTok{\^{}}\DecValTok{2}\NormalTok{)}\SpecialCharTok{*}\NormalTok{(}\FloatTok{0.6}\SpecialCharTok{\^{}}\DecValTok{2}\NormalTok{))))}\SpecialCharTok{*}\NormalTok{h}\SpecialCharTok{\^{}}\DecValTok{2}\NormalTok{)}\SpecialCharTok{+}\NormalTok{ (}\FloatTok{1.2}\SpecialCharTok{\^{}}\DecValTok{2}\SpecialCharTok{/}\NormalTok{(}\DecValTok{2}\SpecialCharTok{*}\FloatTok{9.81}\SpecialCharTok{*}\NormalTok{(}\FloatTok{1.8}\SpecialCharTok{\^{}}\DecValTok{2}\NormalTok{)))}
\NormalTok{\}}
\FunctionTok{root\_secant}\NormalTok{(f, }\FloatTok{0.6}\NormalTok{)}
\end{Highlighting}
\end{Shaded}

\begin{verbatim}
## $`function`
## function(h){
##   (h^3) + ((0.075-((1.2^2)/(2*9.81*(1.8^2)*(0.6^2))))*h^2)+ (1.2^2/(2*9.81*(1.8^2)))
## }
## <bytecode: 0x000001ef8014a028>
## 
## $root
## [1] -0.287
## 
## $iter
## [1] 26
\end{verbatim}

Berdasarkan hasil perhitungan diperoleh nilai \(h=-0,2870309\) atau ketinggian air sekitar \(0,3\ m\) dengan jumlah iterasi sebanyak \(26\) kali.

Pembaca dapat mencoba menggunakan metode lain seperti metode tertutup. Untuk dapat melakukannya, pembaca perlu memperoleh rentang lokasi akar persamaan tersebut berada menggunakan metode tabel.

\hypertarget{referensi-2}{%
\section{Referensi}\label{referensi-2}}

\begin{enumerate}
\def\labelenumi{\arabic{enumi}.}
\tightlist
\item
  Atmika, I.K.A. 2016. \textbf{Diktat Mata Kuliah: Metode Numerik}. Jurusan Teknik Mesin Universitas Udayana.
\item
  Bloomfield, V.A. 2014. \textbf{Using R for Numerical Analysis in Science and Engineering}. CRC Press
\item
  Howard, J.P. 2017. \textbf{Computational Methods for Numerical Analysis with R}. CRC Press.
\item
  Jones, O. Maillardet, R. Robinson, A. 2014. \textbf{Introduction to Scientific Programming and Simulation Using R}. CRC Press
\item
  Kreyszig, E. 2011. \textbf{Advanced Engineering Mathematics, 10th Edition}. John Wiley \& Sons.
\item
  Sanjaya, M. 2015. \textbf{Metode Numerik Berbasis Phython}. Penerbit Gava Media: Yogyakarta.
\item
  Sudiadi dan Teguh R. 2015. \textbf{Metode Numerik}. STMIK
\end{enumerate}

\hypertarget{latihan-1}{%
\section{Latihan}\label{latihan-1}}

\begin{enumerate}
\def\labelenumi{\arabic{enumi}.}
\tightlist
\item
  Temukan akar persamaan dari persamaan non-linier \(f\left(x\right)=x^3-2x+2\) menggunakan metode terbuka dengan \(x_0=0\) dan \(x_0=\frac{1}{2}\)!
\item
  Apakah kelebihan dari metode tertutup (contoh: metode biseksi) dibanding metode terbuka (contoh: Newton-Raphson)? (\textbf{catatan}: pembaca dapat pula mencari dari referensi lainnya)
\item
  Temukan akar persamaan dari persamaan \(f\left(x\right)=\frac{sin\left(x\right)}{x}\) dengan rentang pencarian \(x=0,5\) dan \(x=1\)!
\item
  Pada kondisi apakah metode Secant lebih dipilih dibanding metode Newton-Raphson?
\item
  Modifikasilah fungsi \texttt{root\_bisection()} dan \texttt{root\_rf()} sehingga kita tidak perlu memasukkan argumen \texttt{a} dan \texttt{b} dan hanya perlu memasukkan satu vektor \texttt{interval} kedalam fungsi tersebut! (\textbf{contoh}: \texttt{interval=c(a,b)})
\end{enumerate}

\hypertarget{interpolation}{%
\chapter{Interpolasi dan Ekstrapolasi}\label{interpolation}}

Pada dunia nyata, data sering kali tidak tersaji secara lengkap. Seringkali terdapat nilai data yang hilang (\emph{missing value}). Terdapat banyak penyebab dari kondisi tersebut, baik akibat kesalahan manusianya maupun keterbatasan kemampuan alat ukur.

Kondisi lain yang muncul dari data yang kita miliki adalah adanya \emph{outlier} atau nilai yang berbeda jauh dengan mayoritas data yang kita miliki. Nilai tersebut akan menentukan hasil analisis atau uji statistik yang kita lakukan, terlebih lagi jika uji statistik yang kita lakukan menggunakan metode parametrik.

Terdapat banyak cara untuk menangani kondisi-kondisi tersebut. Sejumlah peneliti memilih untuk menghapus data tersebut. Hal ini dapat dilakukan jika jumlah data yang kita miliki cukup besar. Bagaimana jika data yang kita miliki sedikit dan pengukuran ulang cukup mahal atau cukup sulit dilakukan?. Salah satu cara yang dapat dilakukan adalah dengan melakukan interpolasi terhadap data.

Interpolasi dan ekstrapolasi adalah proses ``menebak'' nilai data dengan memperhatikan data lain yang kita miliki. Interpolasi merupakan teknik untuk mencari nilai suatu variabel yang hilang pada rentang data yang diketahui, sedangkan ektrapolasi merupakan teknik menemukan nilai suatu variabel diluar rentang data yang telah diketahui. Data lain yang kita miliki seringkali memiliki sejumlah pola. Pola yang terbentuk dapat berupa polinomial atau mengelompok. Tiap pola akan memiliki metode pendekatan yang berbeda-beda. Terdapat kemungkinan tak terbatas dari pola data tersebut. Penilaian profesional atau ahli diperlukan untuk menentukan metode mana yang sesuai berdasarkan riwayat penelitian atau pekerjaan yang pernah dilakukan sebelumnya.

Pada Chapter \ref{interpolation} penulis akan menjelaskan teknik-teknik interpolasi yang dapat kita lakukan. Adapun yang akan dibahas pada \emph{Chapter} ini adalah sebagai berikut:

\begin{itemize}
\tightlist
\item
  Teknik interpolasi polinomial
\item
  Teknik interpolasi piecewise
\item
  Studi kasus penerapan teknik interpolasi
\end{itemize}

\hypertarget{poliinterpolation}{%
\section{Interpolasi Polinomial}\label{poliinterpolation}}

Interpolasi polinomial merupakan teknik interpolasi dengan mengasumsikan pola data yang kita miliki mengikuti pola polinomial baik berderajat satu (linier) maupun berderajat tinggi. Interpolasi dengan metode ini dilakukan dengan terlebih dahulu membentuk persamaan polinomial. Persamaan polinomial yang terbentuk selanjutnya digunakan untuk melakukan interpolasi dari nilai yang diketahui atau ekstrapolasi (prediksi) dari nilai diluar rentang data yang diketahui.

Pada Chapter \ref{poliinterpolation} pembahasan akan dibagi menjadi 3 bagian. Bagian pertama kita akan mengulang kembali teknik evaluasi polinomial, sedangkan dua bagian selanjutnya akan membahas teknik interpolasi linier dan polinomial orde tinggi dengan menjadikan pembahasan bagian pertama sebagai dasar pada dua bagian berikutnya.

\hypertarget{mengevaluasi-polinomial}{%
\subsection{Mengevaluasi Polinomial}\label{mengevaluasi-polinomial}}

Pada \emph{Chapter} ini pembaca akan mempelajari teknik untuk melakukan substitusi nilai \(x\) pada persamaan polinomial untuk memperoleh nilai \(y\). Terdapat berbagai pendekatan dalam melakukan proses tersebut, mulai dari metode naive maupun metode Horner. Kedua metode akan menghasilkan hasil yang sama namun dengan proses komputasi yang berbeda. Metode naive cenderung lambat dalam proses komputasi karena jumlah proses yang dilakukan dalam sekali proses lebih banyak dari pada metode Horner.

Untuk memahami metode-metode evaluasi polinomial yang telah disebutkan tersebut, secara umum persamaan polinomial disajikan pada Persamaan \eqref{eq:evalpoly}.

\begin{equation}
f\left(x\right)=a_nx^n+a_{n-1}x^{n-1}+\cdots+a_1x+a_0
  \label{eq:evalpoly}
\end{equation}

dimana \(a\) merupakan koefisien polinomial, \(x\) merupakan variabel, dan \(n\) merupakan indeks dan pangkat polinomial.

Pada metode naive kita melakukan evaluasi polinomial sama dengan cara kita melakukan evaluasi polinomial saat kita SMA. Nilai \(x\) akan disubstitusikan pada masing-masing elemen persamaan polinomial. Masing-masing elemen polinomial selanjutnya dijumkahkan untuk menghitung \(y\).

Pada \texttt{R} kita dapat menuliskan sebuah fungsi untuk melakukan evaluasi polinomial menggunakan metode naive tersebut. Pada fungsi tersebut, koefisien polinomial akan disimpan kedalam sebuah vektor dengan urutan pengisian mulai dari koefisien dengan pangkat \(x\) terendah ke tertinggi.

\begin{Shaded}
\begin{Highlighting}[]
\NormalTok{naive\_poly }\OtherTok{\textless{}{-}} \ControlFlowTok{function}\NormalTok{(x, coeff)\{}
\NormalTok{  n }\OtherTok{\textless{}{-}} \FunctionTok{length}\NormalTok{(x)}
\NormalTok{  y }\OtherTok{\textless{}{-}} \FunctionTok{rep}\NormalTok{(}\DecValTok{0}\NormalTok{, n)}
  
  \ControlFlowTok{for}\NormalTok{(i }\ControlFlowTok{in} \DecValTok{1}\SpecialCharTok{:}\FunctionTok{length}\NormalTok{(coeff))\{}
\NormalTok{    y }\OtherTok{\textless{}{-}}\NormalTok{ y }\SpecialCharTok{+}\NormalTok{ coeff[i]}\SpecialCharTok{*}\NormalTok{(x}\SpecialCharTok{\^{}}\NormalTok{(i}\DecValTok{{-}1}\NormalTok{))}
\NormalTok{  \}}
  
  \FunctionTok{return}\NormalTok{(y)}
\NormalTok{\}}
\end{Highlighting}
\end{Shaded}

\begin{example}
\protect\hypertarget{exm:polievalexmp}{}\label{exm:polievalexmp}Hitung nilai \(y\) pada persamaan \(f\left(x\right)=x^4+3x^3-15x^2-19x+30\), jika diketahui nilai \(x\) adalah -1, 0, dan 1!
\end{example}

\textbf{Jawab}:

Untuk dapat menghitung nilai \(y\) menggunakan fungsi \texttt{naive\_poly()} pada persamaan tersebut dengan nilai \(x\) yang diketahui, kita perlu merubah koefisien persamaan tersebut dan nilai \(x\) yang diketahui menjadi vektor:

\begin{Shaded}
\begin{Highlighting}[]
\NormalTok{x }\OtherTok{\textless{}{-}} \FunctionTok{c}\NormalTok{(}\SpecialCharTok{{-}}\DecValTok{1}\NormalTok{,}\DecValTok{0}\NormalTok{,}\DecValTok{1}\NormalTok{)}
\NormalTok{coeff }\OtherTok{\textless{}{-}} \FunctionTok{c}\NormalTok{(}\DecValTok{30}\NormalTok{,}\SpecialCharTok{{-}}\DecValTok{19}\NormalTok{,}\SpecialCharTok{{-}}\DecValTok{15}\NormalTok{,}\DecValTok{3}\NormalTok{,}\DecValTok{1}\NormalTok{)}
\end{Highlighting}
\end{Shaded}

Masukkan vektor-vektor yang telah terbentuk tersebut kedalam fungsi \texttt{naive\_poly()}.

\begin{Shaded}
\begin{Highlighting}[]
\FunctionTok{naive\_poly}\NormalTok{(x, coeff)}
\end{Highlighting}
\end{Shaded}

\begin{verbatim}
## [1] 32 30  0
\end{verbatim}

Berdasarkan hasil perhitungan diperoleh nilai \(y\) masing-masing sebesar 32, 30, dan 0. Pembaca dapat mengeceknya sendiri hasil perhitungan tersebut menggunakan cara manual.

Kita dapat meningkatkan efisiensi proses perhitungan pada fungsi \texttt{naive\_poly()} tersebut. Sebagai contoh, setiap kali kita melakukan loop untuk menghitung nilai \(y\) pada polinomial, kita dapat memperoleh eksponensial dari \(x\). Namun untuk setiap koefisien \(a_i\), nilai eksponensial yang terkait \(x_i\) merupakan seri produk.

\begin{equation}
x^i=\prod_{j=0}^ix
  \label{eq:evalpoly2}
\end{equation}

Untuk \(x^i\), terdapat sebanyak \(i\) koefisien \(x\) dikalikan bersamaan. Namun, untuk setiap \(x^{i-1}\) terdapat lebih sedikit 1 perkalian dibanding koefisien \(x\) dengan pangkat yang lebih besar dan seterusnya.

Berdasarkan ilustrasi tersebut, kita dapat membentuk fungsi \texttt{better\_poly()} sebagai perbaikan dari fungsi \texttt{naive\_poly()}. Berikut adalah sintaks yang digunakan:

\begin{Shaded}
\begin{Highlighting}[]
\NormalTok{better\_poly }\OtherTok{\textless{}{-}} \ControlFlowTok{function}\NormalTok{(x, coeff)\{}
\NormalTok{  n }\OtherTok{\textless{}{-}} \FunctionTok{length}\NormalTok{(x)}
\NormalTok{  y }\OtherTok{\textless{}{-}} \FunctionTok{rep}\NormalTok{(}\DecValTok{0}\NormalTok{, n)}
\NormalTok{  cached\_x }\OtherTok{\textless{}{-}} \DecValTok{1}
  
  \ControlFlowTok{for}\NormalTok{(i }\ControlFlowTok{in} \DecValTok{1}\SpecialCharTok{:}\FunctionTok{length}\NormalTok{(coeff))\{}
\NormalTok{    y }\OtherTok{\textless{}{-}}\NormalTok{ y }\SpecialCharTok{+}\NormalTok{ coeff[i]}\SpecialCharTok{*}\NormalTok{cached\_x}
\NormalTok{    cached\_x }\OtherTok{\textless{}{-}}\NormalTok{ cached\_x }\SpecialCharTok{*}\NormalTok{ x}
\NormalTok{  \}}
  \FunctionTok{return}\NormalTok{(y)}
\NormalTok{\}}
\end{Highlighting}
\end{Shaded}

\begin{example}
\protect\hypertarget{exm:polievalexmp2}{}\label{exm:polievalexmp2}Hitung nilai \(y\) pada persamaan yang disajikan pada Contoh \ref{exm:polievalexmp} menggunakan nilai \(x\) yang telah diketahui pada soal tersebut menggunakan fungsi \texttt{better\_poly()}?
\end{example}

\textbf{Jawab}:

\begin{Shaded}
\begin{Highlighting}[]
\FunctionTok{better\_poly}\NormalTok{(x, coeff)}
\end{Highlighting}
\end{Shaded}

\begin{verbatim}
## [1] 32 30  0
\end{verbatim}

Sejauh ini kita telah membentuk 2 fungsi yaitu \texttt{naive\_poly()} dan \texttt{better\_poly()}. Kedua fungsi tersebut memiliki perbedaan proses menghitung yang mempengaruhi efisiensinya masing-masing. Sebagai contoh jika diberikan polinomial berderajat 10, fungsi \texttt{naive\_poly()} akan mengejakan perkalian dalam proses \emph{loop} sebanyak 55 kali, sedangkan fungsi \texttt{better\_poly()} akan melakukannya sebanyak 20 kali (\(2n\) perkalian).

Metode lain yang lebih efisien dalam melakukan evaluasi polinomial adalah metode Horner. Metode ini oleh William Horner pada abad ke-18. Dalam metode Horner, bentuk polinomial pada Persamaan \eqref{eq:evalpoly} akan ditransformasi menjadi Persamaan \eqref{eq:evalpoly3}.

\begin{equation}
\begin{split}
f\left(x\right)&=a_nx^n+a_{n-1}x^{n-1}+\cdots+a_1x+a_0\\
&= a_0+a_1x+\cdots+a_{n-1}x^{n-1}+a_nx^n\\
&= a_0+x\left(a_1+\cdots+a_{n-1}x^{n-2}+a_nx^{n-1}\right)\\
&= a_0+x\left(a_1+\cdots+x\left(a_{n-1}+x\left(a_n\right)\right)\cdots\right)
\end{split}
  \label{eq:evalpoly3}
\end{equation}

Berdasarkan Persamaan \eqref{eq:evalpoly3}, jika kita melakukan perhitungan pada persamaan polinomial berderajat 10, kita dapat mereduksi perhitungan menjadi 10 perkalian dan 10 penjumlahan. Jumlah tersebut sangat kecil dibandingkan kedua metode sebelumnya dan dapat dikatakan lebih efisien dibandingkan metode lainnya.

FUngsi \texttt{horner\_poly()} merupakan fungsi yang dibentuk bedasarkan Persamaan \eqref{eq:evalpoly3}. Sintaks fungsi tersebut adalah sebagai berikut:

\begin{Shaded}
\begin{Highlighting}[]
\NormalTok{horner\_poly }\OtherTok{\textless{}{-}} \ControlFlowTok{function}\NormalTok{(x, coeff)\{}
\NormalTok{  n }\OtherTok{\textless{}{-}} \FunctionTok{length}\NormalTok{(x)}
\NormalTok{  y }\OtherTok{\textless{}{-}} \FunctionTok{rep}\NormalTok{(}\DecValTok{0}\NormalTok{, n)}
  
  \ControlFlowTok{for}\NormalTok{(i }\ControlFlowTok{in} \FunctionTok{length}\NormalTok{(coeff)}\SpecialCharTok{:}\DecValTok{1}\NormalTok{)\{}
\NormalTok{    y }\OtherTok{\textless{}{-}}\NormalTok{ coeff[i] }\SpecialCharTok{+}\NormalTok{ x }\SpecialCharTok{*}\NormalTok{ y}
\NormalTok{  \}}
  \FunctionTok{return}\NormalTok{(y)}
\NormalTok{\}}
\end{Highlighting}
\end{Shaded}

\begin{example}
\protect\hypertarget{exm:polievalexmp3}{}\label{exm:polievalexmp3}Kerjakan kembali Contoh \ref{exm:polievalexmp2} fungsi \texttt{horner\_poly()}?
\end{example}

\textbf{Jawab}:

\begin{Shaded}
\begin{Highlighting}[]
\FunctionTok{horner\_poly}\NormalTok{(x, coeff)}
\end{Highlighting}
\end{Shaded}

\begin{verbatim}
## [1] 32 30  0
\end{verbatim}

\hypertarget{lininterpol}{%
\subsection{Interpolasi Linier}\label{lininterpol}}

Misalkan kita memiliki 3 buah data dengan dua buah variabel kita misalkan variabel \(x\) dan variabel \(y\). Pada salah satu data terdapat data yang hilang pada. Agar ketiga data tersebut tetap dapat digunakan dalam iterasi diperlukan interpolasi untuk ``menebak'' nilai dari data yang hilang.

Berdasarkan pengukuran yang sebelumnya pernah dilakukan diketahui bahwa pola data yang terbentuk variabel \(x\) dan \(y\) divisualisasikan menggunakan scatterplot adalah pola linier. Berdasarkan hal tersebut interpolasi dilakukan dengan menggunakan metode linier.

Interpolasi linier dilakukan dengan terlebih dahulu membentuk fungsi linier. Dengan kata lain kita perlu mencari nilai slope \(m\) dan \emph{intercept} \(b\). Nilai \(m\) dihitung sebagai rasio selisih jarak dua titik pada sumbu \(y\) dan sumbu \(x\) yang dapat dituliskan melalui Persamaan \eqref{eq:lineinter}.

\begin{equation}
m=\frac{y_2-y_1}{x_2-x_1}
  \label{eq:lineinter}
\end{equation}

Nilai \emph{intercept} (titik potong pada sumbu \(y\)) dihitung menggunakan Persamaan \eqref{eq:lineinter2}.

\begin{equation}
b=y_2-mx_2
  \label{eq:lineinter2}
\end{equation}

\begin{center}\rule{0.5\linewidth}{0.5pt}\end{center}

\textbf{Algoritma Interpolasi Linier}

\begin{enumerate}
\def\labelenumi{\arabic{enumi}.}
\tightlist
\item
  Tentukan dua buah titik \(\left(x,y\right)\) sebagai dasar pembentukan persaman linier.
\item
  Hitung \(m\) menggunakan Persamaan \eqref{eq:lineinter}
\item
  Hitung \(b\) menggunakan Persamaan \eqref{eq:lineinter2}
\item
  Definiskan fungsi linier berdasarkan nilai \(m\) dan \(b\)
\item
  Hitung \(y\) dengan cara substitusi nilai \(x\) pada persamaan linier untuk melakukan interpolasi atau ekstrapolasi nilai \(y\) yang ingin dicari.
\end{enumerate}

\begin{center}\rule{0.5\linewidth}{0.5pt}\end{center}

Algoritma poin 1 sampai 4 tersebut, kita dapat membentuk fungsi pembentuk persamaan linier dari 2 titik yang diketahui. Fungsi tersebut disajikan pada sintaks berikut:

\begin{Shaded}
\begin{Highlighting}[]
\NormalTok{linear\_inter }\OtherTok{\textless{}{-}} \ControlFlowTok{function}\NormalTok{(x, y)\{}
\NormalTok{  m }\OtherTok{\textless{}{-}}\NormalTok{ (y[}\DecValTok{2}\NormalTok{]}\SpecialCharTok{{-}}\NormalTok{y[}\DecValTok{1}\NormalTok{]) }\SpecialCharTok{/}\NormalTok{ (x[}\DecValTok{2}\NormalTok{]}\SpecialCharTok{{-}}\NormalTok{x[}\DecValTok{1}\NormalTok{])}
\NormalTok{  b }\OtherTok{\textless{}{-}}\NormalTok{ y[}\DecValTok{2}\NormalTok{] }\SpecialCharTok{{-}}\NormalTok{ m}\SpecialCharTok{*}\NormalTok{x[}\DecValTok{2}\NormalTok{]}
  
  \FunctionTok{return}\NormalTok{(}\FunctionTok{c}\NormalTok{(b, m))}
\NormalTok{\}}
\end{Highlighting}
\end{Shaded}

\begin{example}
\protect\hypertarget{exm:linterpexmp}{}\label{exm:linterpexmp}Diketahui koordinat 2 buah titik yaitu (0,-1) dan (2,3). Jika diketahui titik ketiga memiliki koordinat sumbu \(x\) sebesar 1. Lakukan interpolasi untuk menentukan koordinat sumbu \(y\) titik ketiga tersebut!
\end{example}

\textbf{Jawab}:

Berdasarkan data-data yang terdapat pada soal terserbut, kita dapat menghitung nilai \(m\) dan \(b\). Nilai \(m\) dapat dihitung sebagai berikut:

\[
m=\frac{-1-3}{0-2}=2
\]

Dengan menggunakan nilai \(m\) tersebut kita dapat menghitung nilai \(b\).

\[
b=3-2*3=-1
\]

Berdasarkan hasil perhitungan diperoleh persamaan linier yang terbentuk adalah sebagai berikut:

\[
y=2x-1
\]

Berdasarkan persamaan linier tersebut nilai \(y\) dapat dihitung.

\[
y=2*1-1=1
\]

Kita dapat pula membentuk persamaan linier menggunakan fungsi \texttt{linear\_inter()}. Berikut adalah sintaks yang digunakan:

\begin{Shaded}
\begin{Highlighting}[]
\NormalTok{x }\OtherTok{\textless{}{-}} \FunctionTok{c}\NormalTok{(}\DecValTok{0}\NormalTok{,}\DecValTok{2}\NormalTok{)}
\NormalTok{y }\OtherTok{\textless{}{-}} \FunctionTok{c}\NormalTok{(}\SpecialCharTok{{-}}\DecValTok{1}\NormalTok{,}\DecValTok{3}\NormalTok{)}

\NormalTok{(coeff }\OtherTok{\textless{}{-}} \FunctionTok{linear\_inter}\NormalTok{(x,y))}
\end{Highlighting}
\end{Shaded}

\begin{verbatim}
## [1] -1  2
\end{verbatim}

Setelah diperoleh koefisien persamaan linier berdasarkan dua titik tersebut, kita akan menggunakan fungsi \texttt{horner\_poly()} untuk memperoleh nilai \(y\). Berikut adalah sintaks yang digunakan:

\begin{Shaded}
\begin{Highlighting}[]
\FunctionTok{horner\_poly}\NormalTok{(}\DecValTok{1}\NormalTok{, coeff)}
\end{Highlighting}
\end{Shaded}

\begin{verbatim}
## [1] 1
\end{verbatim}

Hasil interpolasi yang diperoleh dapat dikatakan sesuai dengan lokasi kedua titik data yang ditunjukkan pada Gambar \ref{fig:linviz}. Berdasarkan hal tersebut, kita dapat yakin bahwa hasil interpolasi yang telah kita lakukan telah sesuai.

\begin{figure}

{\centering \includegraphics[width=0.8\linewidth]{./images/linviz} 

}

\caption{Interpolasi linier dua titik 
(Sumber:Howard, 2017).}\label{fig:linviz}
\end{figure}

Metode interpolasi linier dapat dibilang merupakan metode interpolasi yang sangat sederhana. Disamping kemudahannya, metode ini memiliki potensi error numerik jika jarak antara kedua titik cukup berdekatan terlebih lagi jika selisih penyebut (\(x_2-x_1\)) sangat kecil sehingga akan menghasilkan nilai \(y\) yang sangat besar.

Disamping adanya potensi error numerik tersebut, metode ini menjadi dasar bagi metode interpolasi lain yang lebih kompleks. Metode selanjutnya merupakan pengembangan dari metode interpolasi ini.

\hypertarget{hopoliinter}{%
\subsection{Interpolasi Polinomial Orde Tinggi}\label{hopoliinter}}

Dengan menggunakan dua titik, kita dapat membentuk garis lurus (linier) yang tepat pada dua titik tersebut. Masalah timbul jika selisih nilai \(x\) kedua titik tersebut sangat kecil atau kedua titik tersebut memiliki nilai \(x\) yang sama. Hal ini akan menyebabkan slope yang dihasilkan menjadi tidak terhingga atau garis yang terbentuk adalah garis vertikal tegak lurus.

Bagaimana jika terdapat tiga buah titik? apakah kita masih bisa menggunakan interpolasi linie?. Ya, asalkan ketiga titik tersebut membentuk pola linier atau terletak pada satu garis yang sama. Pada kenyatannya kondisi tersebut jarang terjadi, sehingga pendekatan menggunakan polinomial orde lebih tinggi diperlukan. Persamaan kuadratik (polinomial orde dua) dapat digunakan untuk membentuk persamaan polinomial pada ketiga titik tersebut, sehingga iterasi dapat dilakukan. Untuk 4 buah titik data, polinomial orde tiga dapat digunakan untuk melakukan interpolasi. Secara umum berdasarkan penjelasan tersebut, untuk \(n\) titik data interpolasi dapat dilakukan menggunakan persamaan polinomial orde \(n-1\).

Diberikan set data berpasangan yang telah diurutkan \(\left(x_i,y_i\right)\), fungsi interpolasi harus memenuhi persyaratan berikut:

\begin{equation}
p\left(x_i\right)=y_i
  \label{eq:hointereq}
\end{equation}

Untuk setiap \(i\). Sebagai tambahan, fungsi interpolasi berupa fungsi polinomial dengan bentuk umum sebagai berikut:

\begin{equation}
y_i=\beta_nx_i^n+\beta_{n-1}x_i^{n-1}+\cdots+\beta_1x_i+\beta_0
  \label{eq:hointereq2}
\end{equation}

Persamaan \eqref{eq:hointereq2} dapat dituliskan kedalam bentuk matriks yang ditampilkan pada Persamaan \eqref{eq:hointereq3}.

\begin{equation}
\begin{bmatrix}
     x_1^n   & x_1^{n-1} & \cdots & x_1  & 1            \\[0.3em]
     x_2^n   & x_2^{n-1} & \cdots & x_2  & 1           \\[0.3em]
     \vdots  & \vdots    & \vdots &\ddots& \vdots            \\[0.3em]
     x_n^n   & x_n^{n-1} & \cdots & x_n  & 1
     \end{bmatrix}
\begin{bmatrix}
     \beta_n                                          \\[0.3em]
     \beta_{n-1}                                      \\[0.3em]
     \vdots                                           \\[0.3em]
     \beta_0                                       
     \end{bmatrix}
= \begin{bmatrix}
     y_1                                          \\[0.3em]
     y_2                                          \\[0.3em]
     \vdots                                       \\[0.3em]
     y_n                                       
     \end{bmatrix}
  \label{eq:hointereq3}
\end{equation}

Persamaan matrik tersebut dapat dituliskan sebagai \(X\beta = y\). Untuk menyelesaikan persamaan tersebut (memperoleh nilai \(\beta\)), pembaca dapat membaca kembali Chapter \ref{linearaljabar}. Matriks \(X\) disebut sebagai matriks Vandemonde dan matriks tersebut mengandung sejumlah nilai \(x\) dengan pangkat sampai dengan \(n\).

\begin{center}\rule{0.5\linewidth}{0.5pt}\end{center}

\textbf{Algoritma Interpolasi Polinomial Orde Tinggi}

\begin{enumerate}
\def\labelenumi{\arabic{enumi}.}
\tightlist
\item
  Tentukan set titik berpasangan \(\left(x,y\right)\) yang telah diurutkan.
\item
  Bentuk matriks Vandermonde sesuai dengan Persamaan \eqref{eq:hointereq3}.
\item
  Definiskan persamaan matriks \(X\beta=y\)
\item
  Selesaikan persamaan matriks pada poin 3 untuk memperoleh nilai \(\beta\)
\item
  Definisikan persamaan polinomial berdasarkan koefisien \(\beta\) yang diperoleh
\item
  Lakukan substitusi \(x\) persamaan polinomial pada poin 5 untuk memperoleh nilai \(y\)
\end{enumerate}

\begin{center}\rule{0.5\linewidth}{0.5pt}\end{center}

Berdasarkan algoritma poin 1 sampai 5, kita dapat membentuk suatu fungsi untuk membentuk persamaan polinomial berdasarkan Persamaan \eqref{eq:hointereq3}. Berikut adalah sintaks fungsi tersebut:

\begin{Shaded}
\begin{Highlighting}[]
\NormalTok{poly\_inter }\OtherTok{\textless{}{-}} \ControlFlowTok{function}\NormalTok{(x, y)\{}
  \ControlFlowTok{if}\NormalTok{(}\FunctionTok{length}\NormalTok{(x) }\SpecialCharTok{!=} \FunctionTok{length}\NormalTok{(y))}
    \FunctionTok{stop}\NormalTok{(}\StringTok{"Lenght of x and y vectors must be the same"}\NormalTok{)}
  
\NormalTok{  n }\OtherTok{\textless{}{-}} \FunctionTok{length}\NormalTok{(x)}\SpecialCharTok{{-}}\DecValTok{1}
\NormalTok{  vandermonde }\OtherTok{\textless{}{-}} \FunctionTok{rep}\NormalTok{(}\DecValTok{1}\NormalTok{, }\FunctionTok{length}\NormalTok{(x))}
  \ControlFlowTok{for}\NormalTok{(i }\ControlFlowTok{in} \DecValTok{1}\SpecialCharTok{:}\NormalTok{n)\{}
\NormalTok{    xi }\OtherTok{\textless{}{-}}\NormalTok{ x}\SpecialCharTok{\^{}}\NormalTok{i}
\NormalTok{    vandermonde }\OtherTok{\textless{}{-}} \FunctionTok{cbind}\NormalTok{(vandermonde, xi)}
\NormalTok{  \}}
\NormalTok{  beta }\OtherTok{\textless{}{-}} \FunctionTok{solve}\NormalTok{(vandermonde, y)}
  
  \FunctionTok{names}\NormalTok{(beta) }\OtherTok{\textless{}{-}} \ConstantTok{NULL}
  \FunctionTok{return}\NormalTok{(beta)}
\NormalTok{\}}
\end{Highlighting}
\end{Shaded}

\begin{example}
\protect\hypertarget{exm:hopoliexmp}{}\label{exm:hopoliexmp}Diketahui koordinat 3 buah titik yaitu (-1,-2), (1,2) dan (0,1). Jika diketahui titik keempat memiliki koordinat sumbu \(x\) sebesar -2. Lakukan interpolasi untuk menentukan koordinat sumbu \(y\) titik keempat tersebut!
\end{example}

\textbf{Jawab}:

Untuk menyelesaiakn contoh soal tersebut, kita perlu terlebih dahulu membentuk matriks sesuai dengan Persamaan \eqref{eq:hointereq3}. Berdasarkan soal tersebut, terdapat tiga buah titik data yang diketahui, sehingga polinomial yang hendak dibentuk selanjutnya adalah polinomial berderajat 2.

\begin{equation*}
\begin{bmatrix}
     -1^2   & -1^{1}  & 1            \\[0.3em]
     1^2    & 1^{1}   & 1            \\[0.3em]
     0^2    & 0^{1}   & 1
     \end{bmatrix}
\begin{bmatrix}
     \beta_2                                          \\[0.3em]
     \beta_1                                          \\[0.3em]
     \beta_0                                       
     \end{bmatrix}
= \begin{bmatrix}
     -2                                          \\[0.3em]
     2                                           \\[0.3em]
     1                                       
     \end{bmatrix}
\end{equation*}

Setelah matriks tersebut terbentuk, pembaca dapat menyelesaikannya menggunakan berbagai metode yang telah penulis jelaskan pada Chapter \ref{linearaljabar} untuk memperoleh nilai \(\beta\).

Untuk menyelesaikan contoh soal tersebut pada \texttt{R}, kiat perlu membentuk matriks \(x\) dan \(y\) terlebih dahulu.

\begin{Shaded}
\begin{Highlighting}[]
\NormalTok{x }\OtherTok{\textless{}{-}} \FunctionTok{c}\NormalTok{(}\SpecialCharTok{{-}}\DecValTok{1}\NormalTok{, }\DecValTok{1}\NormalTok{, }\DecValTok{0}\NormalTok{)}
\NormalTok{y }\OtherTok{\textless{}{-}} \FunctionTok{c}\NormalTok{(}\SpecialCharTok{{-}}\DecValTok{2}\NormalTok{, }\DecValTok{2}\NormalTok{, }\SpecialCharTok{{-}}\DecValTok{1}\NormalTok{)}
\end{Highlighting}
\end{Shaded}

Koefisien persamaan polinomial dihitung menggunakan fungsi \texttt{poly\_inter()}.

\begin{Shaded}
\begin{Highlighting}[]
\NormalTok{(coeff }\OtherTok{\textless{}{-}} \FunctionTok{poly\_inter}\NormalTok{(x, y))}
\end{Highlighting}
\end{Shaded}

\begin{verbatim}
## [1] -1  2  1
\end{verbatim}

Berdasarkan hasil perhitungan, diperoleh nilai \(\beta\). Nilai tersebut selanjutnya digunakan untuk membentuk persamaan polinomial. Berikut merupakan persamaan polinomial yang terbentuk:

\[
f\left(x\right)=1x^2+2x-1
\]

Fungsi \texttt{horner\_poly()} selanjutnya digunakan untuk mengevaluasi polinomial tersebut. Berikut adalah hasil substitusi \(x\) pada persamaan tersebut:

\begin{Shaded}
\begin{Highlighting}[]
\FunctionTok{horner\_poly}\NormalTok{(}\SpecialCharTok{{-}}\DecValTok{2}\NormalTok{, coeff)}
\end{Highlighting}
\end{Shaded}

\begin{verbatim}
## [1] -1
\end{verbatim}

Hasil yang diperoleh terlihat cukup sesuai jika kita perhatikan visualisasi ketiga titik tersebut pada Gambar \ref{fig:hopoliviz}.

\begin{figure}

{\centering \includegraphics[width=0.8\linewidth]{./images/hopoliviz} 

}

\caption{Interpolasi kuadratik tiga titik 
(Sumber:Howard, 2017).}\label{fig:hopoliviz}
\end{figure}

\begin{example}
\protect\hypertarget{exm:hopoliexmp2}{}\label{exm:hopoliexmp2}Dengan menggunakan data pada Contoh \ref{exm:linterpexmp}, lakukan proses perhitungan untuk membentuk persamaan polinomial menggunakan fungsi \texttt{poly\_inter()}
\end{example}

\textbf{Jawab}:

Berdasarkan data pada Contoh \ref{exm:linterpexmp}, polinomial yang terbentuk merupakan polinomial derajat 1 (linier). Berikut adalah nilai koefisien yang dihasilkan dari perhitungan menggunakan fungsi \texttt{poly\_inter()}.

\begin{Shaded}
\begin{Highlighting}[]
\NormalTok{x }\OtherTok{\textless{}{-}} \FunctionTok{c}\NormalTok{(}\DecValTok{2}\NormalTok{, }\DecValTok{0}\NormalTok{)}
\NormalTok{y }\OtherTok{\textless{}{-}} \FunctionTok{c}\NormalTok{(}\DecValTok{3}\NormalTok{, }\SpecialCharTok{{-}}\DecValTok{1}\NormalTok{)}
\FunctionTok{poly\_inter}\NormalTok{(x, y)}
\end{Highlighting}
\end{Shaded}

\begin{verbatim}
## [1] -1  2
\end{verbatim}

Meskipun proses perhitungan menggunakan fungsi \texttt{poly\_inter()} lebih rumit dibandingkan dengan fungsi \texttt{linear\_poly()}, hasil perhitungan keduanya menggunakan data pada Contoh \ref{exm:linterpexmp} menghasilkan hasil yang sama.

\hypertarget{pwinter}{%
\section{Interpolasi Piecewise}\label{pwinter}}

Interpolasi dengan polinomial sering memberikan hasil yang tidak dapat diterima. Interpolasi polinomial yang dihasilkan dari sejumlah besar data titik biasanya berderajat tinggi. Polinomial berderajat tinggi pada umumnya bersifat osilatif (grafiknya naik turun secara cepat). Akibatnya, perubahan data pada interval kecil dapat menyebabkan fluktuasi besar pada keseluruhan interval. Karena alasan ini, biasanya interpolasi hanya menggunakan polinomial berderajat rendah.

Interpolasi piecewise menawarkan alternatif lain. Pada interpolasi piecewise, pada titik yang berbeda sepanjang kurva, nilai fungsi lebih mungkin lebih baik didekati menggunakan dua atau lebih interpolasi. Pada metode ini kita akan membuat fungsi interpolasi ditiap antara dua titik observasi.

Pada sub-Chapter ini akan dijelaskan 2 buah metode interpolasi piecewise, yaitu: interpolasi linier piecewise dan interpolasi kubik spline. Interpolasi pertama dilakukan menggunakan persamaan linier, sehingga kurva yang terbentuk bukan merupakan kurva kontinu. Interpolasi selanjutnya dilakukan menggunakan persamaan polinomial berderajat tinggi sehingga kurva yang dihasilkan lebih halus (tidak ada sudut siku pada setiap titik).

\hypertarget{pwlininter}{%
\subsection{Interpolasi Linier Piecewise}\label{pwlininter}}

Interpolasi linier piecewise merupakan interpolasi yang menggunakan pendekatan interpolasi linier. Fungsi linier akan dibentuk pada setiap dua titik observasi. Untuk lebih memahaminya perhatikan kembali Gambar \ref{fig:hopoliviz}. Pada gambar tersebut sebelumnya kita telah membentuk persamaan kudratik untuk menghubungkan titik-titik tersebut. Dibanding menggunakan persamaan polinomial seperti kuadratik tersebut, interpolasi piecewise akan menghubungkan tiap dua titik observasi tersebut dengan garis lurus.

\begin{center}\rule{0.5\linewidth}{0.5pt}\end{center}

\textbf{Algoritma Interpolasi Linier Piecewise}

\begin{enumerate}
\def\labelenumi{\arabic{enumi}.}
\tightlist
\item
  Tentukan set titik berpasangan \(\left(x,y\right)\) yang telah diurutkan berdasarkan nilai sumbu \(x\).
\item
  Hitung \(m\) pada setiap dua titik berdekatan menggunakan Persamaan \eqref{eq:lineinter}
\item
  Hitung \(b\) pada setiap dua titik berdekatan menggunakan Persamaan \eqref{eq:lineinter2}
\item
  Definiskan fungsi linier berdasarkan nilai \(m\) dan \(b\)
\item
  Hitung \(y\) dengan cara substitusi nilai \(x\) pada persamaan linier untuk melakukan interpolasi nilai \(y\) yang ingin dicari.
\item
  Untuk melakukan ekstrapolasi dengan titik observasi diluar rentang titik diketahui, gunakan persamaan linier yang berada pada bagian ujung terdekat dengan nilai \(x\) yang hendak dicari nilai \(y\)-nya.
\end{enumerate}

\begin{center}\rule{0.5\linewidth}{0.5pt}\end{center}

Berdasarkan algoritma tersebut, kita dapat menyusun fungsi pada \texttt{R} untuk membentuk persamaan linier piecewise. Berikut adalah sintaks yang digunakan:

\begin{Shaded}
\begin{Highlighting}[]
\NormalTok{pwise\_linterp }\OtherTok{\textless{}{-}} \ControlFlowTok{function}\NormalTok{(x, y)\{}
\NormalTok{  n }\OtherTok{\textless{}{-}} \FunctionTok{length}\NormalTok{(x)}\SpecialCharTok{{-}}\DecValTok{1}
  
\NormalTok{  y }\OtherTok{\textless{}{-}}\NormalTok{ y[}\FunctionTok{order}\NormalTok{(x)]}
\NormalTok{  x }\OtherTok{\textless{}{-}}\NormalTok{ x[}\FunctionTok{order}\NormalTok{(x)]}
  
\NormalTok{  m\_vec }\OtherTok{\textless{}{-}}\NormalTok{ b\_vec }\OtherTok{\textless{}{-}} \FunctionTok{c}\NormalTok{()}
  
  \ControlFlowTok{for}\NormalTok{(i }\ControlFlowTok{in} \DecValTok{1}\SpecialCharTok{:}\NormalTok{n)\{}
\NormalTok{    m }\OtherTok{\textless{}{-}}\NormalTok{ (y[i}\SpecialCharTok{+}\DecValTok{1}\NormalTok{]}\SpecialCharTok{{-}}\NormalTok{y[i]) }\SpecialCharTok{/}\NormalTok{ (x[i}\SpecialCharTok{+}\DecValTok{1}\NormalTok{]}\SpecialCharTok{{-}}\NormalTok{x[i])}
\NormalTok{    b }\OtherTok{\textless{}{-}}\NormalTok{ y[i}\SpecialCharTok{+}\DecValTok{1}\NormalTok{] }\SpecialCharTok{{-}}\NormalTok{ m}\SpecialCharTok{*}\NormalTok{x[i}\SpecialCharTok{+}\DecValTok{1}\NormalTok{]}
\NormalTok{    m\_vec }\OtherTok{\textless{}{-}} \FunctionTok{c}\NormalTok{(m\_vec, m)}
\NormalTok{    b\_vec }\OtherTok{\textless{}{-}} \FunctionTok{c}\NormalTok{(b\_vec, b)}
\NormalTok{  \}}
  
  \FunctionTok{return}\NormalTok{(}\FunctionTok{list}\NormalTok{(}\AttributeTok{b =}\NormalTok{ b\_vec, }\AttributeTok{m =}\NormalTok{ m\_vec))}
\NormalTok{\}}
\end{Highlighting}
\end{Shaded}

\begin{example}
\protect\hypertarget{exm:pwiselinexm}{}\label{exm:pwiselinexm}Tentukan persamaan-persamaan linier yang dihasilkan dari titik observasi yang ditampilkan pada Contoh \ref{exm:hopoliexmp}?
\end{example}

\textbf{Jawab}:

Untuk menentukan persamaan-persamaan linier yang menghubungkan setiap titik, kita akan menggunakan fungsi \texttt{pwise\_linterp()} yang telah kita buat. Berikut adalah sintaks yang digunakan:

\begin{Shaded}
\begin{Highlighting}[]
\NormalTok{x }\OtherTok{\textless{}{-}} \FunctionTok{c}\NormalTok{(}\SpecialCharTok{{-}}\DecValTok{1}\NormalTok{, }\DecValTok{1}\NormalTok{, }\DecValTok{0}\NormalTok{, }\SpecialCharTok{{-}}\DecValTok{2}\NormalTok{)}
\NormalTok{y }\OtherTok{\textless{}{-}} \FunctionTok{c}\NormalTok{(}\SpecialCharTok{{-}}\DecValTok{2}\NormalTok{, }\DecValTok{2}\NormalTok{, }\SpecialCharTok{{-}}\DecValTok{1}\NormalTok{, }\SpecialCharTok{{-}}\DecValTok{1}\NormalTok{)}

\FunctionTok{pwise\_linterp}\NormalTok{(x, y)}
\end{Highlighting}
\end{Shaded}

\begin{verbatim}
## $b
## [1] -3 -1 -1
## 
## $m
## [1] -1  1  3
\end{verbatim}

Jika persamaan-persamaan yang terbentuk tersebut divisualisasikan akan terlihat seperti pada Gambar \ref{fig:pwiselinviz}.

\begin{figure}

{\centering \includegraphics[width=0.8\linewidth]{./images/pwiselinviz} 

}

\caption{Interpolasi linier piecewise empat titik 
(Sumber:Howard, 2017).}\label{fig:pwiselinviz}
\end{figure}

\texttt{R} juga menyediakan fungsi untuk melakukan interpolasi linier piecewise. Fungsi \texttt{approxfun()} dapat digunakan untuk melakukan interpolasi tersebut. Fungsi \texttt{approxfun()} hanya memerlukan dua input yaitu vektor \(x\) dan vektor \(x\). Untuk lebih memahaminya, kita akan menggunakan kembali data yang disajikan pada Contoh \ref{exm:hopoliexmp} untuk membuat fungsi linier piecewise. Berikut adalah sintaks yang digunakan:

\begin{Shaded}
\begin{Highlighting}[]
\NormalTok{f }\OtherTok{\textless{}{-}} \FunctionTok{approxfun}\NormalTok{(x, y)}

\CommentTok{\# tentukan nilai y jika x= 0}
\FunctionTok{f}\NormalTok{(}\DecValTok{0}\NormalTok{)}
\end{Highlighting}
\end{Shaded}

\begin{verbatim}
## [1] -1
\end{verbatim}

\begin{Shaded}
\begin{Highlighting}[]
\CommentTok{\# tentukan nilai y jika x = 0.5}
\FunctionTok{f}\NormalTok{(}\FloatTok{0.5}\NormalTok{)}
\end{Highlighting}
\end{Shaded}

\begin{verbatim}
## [1] 0.5
\end{verbatim}

\hypertarget{cubicspline}{%
\subsection{Interpolasi Spline Kubik}\label{cubicspline}}

Jika menggunakan interpolasi polinomial berderajat satu (sebuah garis) lebih dari beberapa interval merupakan peningkatan dari satu baris interpolasi, dan jika menggunakan polinomial berderajat tinggi juga merupakan peningkatan dari satu garis interpolasi tunggal, maka dapat disimpulkan bahwa penggunaan polinomial berderajat tinggi pada selang beberapa interval juga akan menjadi peningkatan dalam proses interpolasi. Dalam beberapa kasus, hal tersebut benar, tetapi kita masih menghadapi sudut tajam di mana masing-masing kurva interpolasi tergabung (interpolasi linier piecewise). Sudut tajam ini mencegah diferensiasi dan pada prakteknya tidak dapat digunakan untuk memodelkan beberapa fungsi di dunia nyata, seperti \emph{roller coaster span}.

Interpolasi spline kubik memecahkan masalah ini. Interpolasi ini akan memberikan kurva tergabung yang halus. Hal tersebut juga membuat spline terintegrasi. Karena setiap bagian individu diwakili oleh kurva kubik (polinomial derajat 3), maka masing-masing bagian individu juga dapat dianalisis sebagai kurva kubik. Dengan asumsi ada \(n\) titik data untuk interpolasi, kita akan mendefinisikan \(S_i\) sebagai fungsi polinomial kubik yang mewakili kurva pada domain \(\left[x_i; x_{i + 1}\right]\). Kemudian untuk \(n\) titik observasi, ada \(n - 1\) interpolasi polinomial kubik.

Bentuk umum seri polinomial dituliskan pada Persamaan \eqref{eq:pwisequb}.

\begin{equation}
S_i=d_i\left(x-x_i\right)^3+c_i\left(x-x_i\right)^2+b_i\left(x-x_i\right)+a_i
  \label{eq:pwisequb}
\end{equation}

Ini mengarah ke 4 nilai yang tidak diketahui tidak diketahui, yaitu: \(a_i\), \(b_i\), \(c_i\), dan \(d_i\) pada setiap Persamaan \eqref{eq:pwisequb}. Oleh karena itu, ada \(4 \left(n - 1\right) = 4n - 4\) nilai yang tidak diketahui. Karena kita ingin spline membentuk garis kontinu dan dapat didiferensiasi, ada satu set persamaan yang menentukan spline kubik:

\begin{equation}
S_i\left(x_i\right)=y_i,\ \ \ \ \ \ \ \ \ \ i=1,\dots\dots,n-1
  \label{eq:pwisequb2}
\end{equation}

\begin{equation}
S_i\left(x_{i+1}\right)=y_{i+1},\ \ \ \ \ \ \ \ \ \ i=1,\dots\dots,n-1
  \label{eq:pwisequb3}
\end{equation}

\begin{equation}
S_i'\left(x_{i+1}\right)=S'_{i+1}\left(x_i\right),\ \ \ \ \ \ \ \ \ \ i=1,\dots\dots,n-2
  \label{eq:pwisequb4}
\end{equation}

\begin{equation}
S_i''\left(x_{i+1}\right)=S''_{i+1}\left(x_i\right),\ \ \ \ \ \ \ \ \ \ i=1,\dots\dots,n-2
  \label{eq:pwisequb5}
\end{equation}

Persamaan \eqref{eq:pwisequb2} dan \eqref{eq:pwisequb3} sudah cukup jelas. Persyaratan ini memastikan bahwa jika kita mengevaluasi spline di salah satu node internal, hasil yang akan kita peroleh merupakan jawaban yang telah ditentukan, yaitu spline yang dievaluasi pada \(x_i\) untuk beberapa \(i\) adalah \(y_i\), dan setiap komponen spline bergabung dengan rapi. Persamaan \eqref{eq:pwisequb4} memastikan bahwa kita memiliki turunan pertama yang berkelanjutan di setiap simpul internal. Ini mencegah terbentuknya sudut tajam pada tiap node. Persamaan \eqref{eq:pwisequb5} memastikan turunan kedua juga kontinu, dimana kondisi ini menguntungkan karena itu berarti turunan pertama itu sendiri dapat didiferensiasi juga.

Kondisi ini menyebabkan ada \(4n − 6\) kondisi yang harus kita penuhi. Jika \(4n − 4\) kita yang tidak diketahui dipecahkan sebagai sebuah matriks, dan akhirnya matriks tersebut akan terpecahkan, matriks akan menjadi kurang ditentukan. Kita dapat menyelesaikan kondisi tersebut dengan memasukkan dua ketentuan tambahan. Dengan splines kubik, secara normal adalah menentukan akhir di kedua ujung untuk mencapai dua kondisi tambahan. Untuk contoh ini, dua kondisi yang akan kita tambahkan adalah \(S_i''\left(x_{i}\right)= 0\) dan \(S_i''\left(x_{n}\right)= 0\). Kedua kondisi ini memastikan bahwa pada titik akhir, turunan pertamanya linier dan oleh karena itu fungsi spline berlanjut ke arah yang sudah berjalan. Interpolasi spline kubik ini disebut juga sebagai ``natural spline''.

Awalnya, kita dapat melihat bahwa setiap polinomial kubik \(S_i\) digeser ke kanan oleh unit \(x_i\) atau bergeser ke kiri jika \(x_i\) negatif. Pada \(x_i\) nilai fungsi adalah \(a_i\) yang berarti \(a_i = y_i\) untuk setiap nilai \(i\). Menyelesaikan sisa koefisien yang ada akan lebih kompleks, tetapi sekarang \(3n\) tidak diketahui dengan \(3n\) kondisi. Derivasi penuh tersedia dari berbagai sumber, tetapi secara garis besar dari Persamaan \eqref{eq:pwisequb4} kita mendapatkan \(S_i'\left(x_{i+1}\right)=S'_{i+1}\), kemudian \(S'_{i + 1} - S'_i \left(x\right) = 0\), dan kita dapat mensubstitusi Persamaan \eqref{eq:pwisequb} ke dalam kedua komponen, pemecahan untuk \(d_i\) dalam hal ini \(x_i\), \(y_i\), dan \(c_i\). Proses yang sama dapat direplikasi dengan Persamaan \eqref{eq:pwisequb5} dan \(b_i\). Hasilnya adalah matriks tridiagonal, seperti yang ada pada Chapter \ref{matriktridiagonal}, yang dapat dipecahkan untuk menemukan koefisien. Terdapat sebuah matriks, A, sedemikian rupa sehingga,

\begin{equation}
AC=V
  \label{eq:pwisequb6}
\end{equation}

dimana \(A\) merupakan matriks tridiagonal. Pada matriks tridiagonal ini \(\left(u_1,u_2,\dots,u_{n-1}\right)\) merupakan vektor \(U\), dan vektor \(M\), \(L\), dan \(V\) ditentukan dengan cara serupa. Matriks ini sedemikian rupa,

\begin{equation}
u_i=l_i=x_{i+1}-x_i
  \label{eq:pwisequb7}
\end{equation}

kecuali pada \(u_0=l_n\). Lebih jauh, diagonal utama \(D\) ditentukan menggunakan Persamaan \eqref{eq:pwisequb8}.

\begin{equation}
m_i=2\left(x_{i+1}-x_i+x_i-x_{i-1}\right)
  \label{eq:pwisequb8}
\end{equation}

kecuali pada \(d_0=d_n=1\). Akhirnya vektor \(V\) ditentukan dengan Persamaan \eqref{eq:pwisequb9}.

\begin{equation}
v_i=3\left(\frac{y_{i+1}-y_i}{x_{i+1}-x_i}-\frac{y_i-y_{i-1}}{x_i-x_{i-1}}\right)
  \label{eq:pwisequb9}
\end{equation}

kecuali pada \(v_0=v_n=0\). Sehingga,

\begin{equation}
\begin{bmatrix}
     m_{1} & u_{1} & 0      &     0  & 0             \\[0.3em]
     l_{1} & m_{2} & u_{2}  & 0      & 0             \\[0.3em]
     0     & l_{2} & \ddots & \ddots & 0             \\[0.3em]
     0     & 0     & \ddots & \ddots & u_{n-1}       \\[0.3em]
     0     & 0     & 0      &l_{n-1} & m_{n}
     \end{bmatrix}
C
= \begin{bmatrix}
     v_1                                          \\[0.3em]
     v_2                                          \\[0.3em]
     \vdots                                       \\[0.3em]
     \vdots                                       \\[0.3em]
     v_n                                       
     \end{bmatrix}
  \label{eq:pwisequb10}
\end{equation}

Penyelesaian Persamaan \eqref{eq:pwisequb10} akan menghasilkan vektor koefisien \(c\), sehingga koefisien \(b\) dapat dihitung menggunakan Persamaan \eqref{eq:pwisequb11}.

\begin{equation}
b_i=\frac{y_{i+1}-y_i}{x_{i+1}-x_i}-\frac{x_{i+1}-x_i}{3}\left(2c_i+c_{i+1}\right)
  \label{eq:pwisequb11}
\end{equation}

Dengan menggunakan vektor \(C\) yang sudah diketahui, koefisien \(d\) dapat dihitung menggunakan Persamaan \eqref{eq:pwisequb12}.

\begin{equation}
d_i=\frac{c_{i+1}-c_i}{3\left(x_{i+1}-x_i\right)}
  \label{eq:pwisequb12}
\end{equation}

\begin{center}\rule{0.5\linewidth}{0.5pt}\end{center}

\textbf{Algoritma Interpolasi Spline Kubik}

\begin{enumerate}
\def\labelenumi{\arabic{enumi}.}
\tightlist
\item
  Tentukan set titik berpasangan \(\left(x,y\right)\).
\item
  Tentukan koefisien \(a_i\) menggunakan Persamaan \eqref{eq:pwisequb2}, dimana \(a_i = y_i\).
\item
  Hitung elemen diagonal bawah (\(l_i\)) dan elemen diagonal atas (\(u_i\)) matriks tridiagonal menggunakan Persamaan \eqref{eq:pwisequb7}, dimana \(u_0=l_n=0\).
\item
  Hitung elemen diagonal utama (\(m_i\)) menggunakan Persamaan \eqref{eq:pwisequb8}, dimana \(d_0=d_n=1\).
\item
  Hitung elemen vektor \(V\) menggunakan Persamaan \eqref{eq:pwisequb9}, dimana \(v_0=v_n=0\).
\item
  Susunlah elemen \(l_i\), \(u_i\), dan \(m_i\) menjadi matriks tridiagonal \(A\).
\item
  Deifinisikan persamaan linier seperti pada Persamaan \eqref{eq:pwisequb6}
\item
  Selesaikan sistem persamaan linier pada Persamaan \eqref{eq:pwisequb6} sehingga diperoleh vektor \(C\) yang merupakan kumpulan koefisien \(c\).
\item
  Hitung koefisien \(b_i\) menggunakan Persamaan \eqref{eq:pwisequb11}
\item
  Hitung koefisien \(d_i\) menggunakan Persamaan \eqref{eq:pwisequb12}.
\item
  Bentuk seri persamaan polinomial menggunakan elemen koefisien \(a\), \(b\), \(c\), dan \(d\) yang telah dihitung.
\item
  Untuk melakukan ekstrapolasi dengan titik observasi diluar rentang titik diketahui, gunakan persamaan polinomial yang berada pada bagian ujung terdekat dengan nilai \(x\) yang hendak dicari nilai \(y\)-nya.
\end{enumerate}

\begin{center}\rule{0.5\linewidth}{0.5pt}\end{center}

Berdasarkan algoritma tersebut, kita dapat menyusun suatu fungsi pada \texttt{R} untuk mencari seri persamaan polinomial derajat tiga. Fungsi tersebut adalah sebagai berikut:

\begin{Shaded}
\begin{Highlighting}[]
\NormalTok{cubic\_spline }\OtherTok{\textless{}{-}} \ControlFlowTok{function}\NormalTok{(x, y)\{}
\NormalTok{  n }\OtherTok{\textless{}{-}} \FunctionTok{length}\NormalTok{(x)}
\NormalTok{  d\_vec }\OtherTok{\textless{}{-}}\NormalTok{ b\_vec }\OtherTok{\textless{}{-}}\NormalTok{ a\_vec }\OtherTok{\textless{}{-}} \FunctionTok{rep}\NormalTok{(}\DecValTok{0}\NormalTok{, n}\DecValTok{{-}1}\NormalTok{)}
\NormalTok{  vec }\OtherTok{\textless{}{-}} \FunctionTok{rep}\NormalTok{(}\DecValTok{0}\NormalTok{, n)}
\NormalTok{  delta\_x }\OtherTok{\textless{}{-}}\NormalTok{ delta\_y }\OtherTok{\textless{}{-}} \FunctionTok{rep}\NormalTok{(}\DecValTok{0}\NormalTok{, n}\DecValTok{{-}1}\NormalTok{)}
  
  \DocumentationTok{\#\# Menghitung nilai selisih dan vektor A}
  \ControlFlowTok{for}\NormalTok{(i }\ControlFlowTok{in} \DecValTok{1}\SpecialCharTok{:}\NormalTok{(n}\DecValTok{{-}1}\NormalTok{))\{}
\NormalTok{    a\_vec[i] }\OtherTok{\textless{}{-}}\NormalTok{ y[i]}
\NormalTok{    delta\_x[i] }\OtherTok{\textless{}{-}}\NormalTok{ x[i}\SpecialCharTok{+}\DecValTok{1}\NormalTok{] }\SpecialCharTok{{-}}\NormalTok{ x[i]}
\NormalTok{    delta\_y[i] }\OtherTok{\textless{}{-}}\NormalTok{ y[i}\SpecialCharTok{+}\DecValTok{1}\NormalTok{] }\SpecialCharTok{{-}}\NormalTok{ y[i]}
\NormalTok{  \}}
  
  \DocumentationTok{\#\# Menyusun matriks tridiagona}
\NormalTok{  Au }\OtherTok{\textless{}{-}} \FunctionTok{c}\NormalTok{(}\DecValTok{0}\NormalTok{, delta\_x[}\DecValTok{2}\SpecialCharTok{:}\NormalTok{(n}\DecValTok{{-}1}\NormalTok{)])}
\NormalTok{  Am }\OtherTok{\textless{}{-}} \FunctionTok{c}\NormalTok{(}\DecValTok{1}\NormalTok{, }\DecValTok{2}\SpecialCharTok{*}\NormalTok{(delta\_x[}\DecValTok{1}\SpecialCharTok{:}\NormalTok{(n}\DecValTok{{-}2}\NormalTok{)]}\SpecialCharTok{+}\NormalTok{delta\_x[}\DecValTok{2}\SpecialCharTok{:}\NormalTok{(n}\DecValTok{{-}1}\NormalTok{)]), }\DecValTok{1}\NormalTok{)}
\NormalTok{  Al }\OtherTok{\textless{}{-}} \FunctionTok{c}\NormalTok{(delta\_x[}\DecValTok{1}\SpecialCharTok{:}\NormalTok{(n}\DecValTok{{-}2}\NormalTok{)], }\DecValTok{0}\NormalTok{)}
  
\NormalTok{  vec[}\DecValTok{0}\NormalTok{] }\OtherTok{\textless{}{-}}\NormalTok{ vec[n] }\OtherTok{\textless{}{-}} \DecValTok{0}
  \ControlFlowTok{for}\NormalTok{(i }\ControlFlowTok{in} \DecValTok{2}\SpecialCharTok{:}\NormalTok{(n}\DecValTok{{-}1}\NormalTok{))}
\NormalTok{    vec[i] }\OtherTok{\textless{}{-}} \DecValTok{3} \SpecialCharTok{*}\NormalTok{ (delta\_y[i]}\SpecialCharTok{/}\NormalTok{delta\_x[i] }\SpecialCharTok{{-}} 
\NormalTok{                   delta\_y[i}\DecValTok{{-}1}\NormalTok{]}\SpecialCharTok{/}\NormalTok{delta\_x[i}\DecValTok{{-}1}\NormalTok{])}
  
  \DocumentationTok{\#\# penyelesaian tridiagonal matriks}
\NormalTok{  nm }\OtherTok{\textless{}{-}} \FunctionTok{length}\NormalTok{(Am)}
\NormalTok{  Al }\OtherTok{\textless{}{-}} \FunctionTok{c}\NormalTok{(}\ConstantTok{NA}\NormalTok{ , Al)}
  
  \DocumentationTok{\#\#\# forward sweep}
\NormalTok{  Au[}\DecValTok{1}\NormalTok{] }\OtherTok{\textless{}{-}}\NormalTok{ Au[}\DecValTok{1}\NormalTok{] }\SpecialCharTok{/}\NormalTok{ Am[}\DecValTok{1}\NormalTok{]}
\NormalTok{  vec[}\DecValTok{1}\NormalTok{] }\OtherTok{\textless{}{-}}\NormalTok{ vec[}\DecValTok{1}\NormalTok{] }\SpecialCharTok{/}\NormalTok{ Am[}\DecValTok{1}\NormalTok{]}
  \ControlFlowTok{for}\NormalTok{(i }\ControlFlowTok{in} \DecValTok{2}\SpecialCharTok{:}\NormalTok{(n }\SpecialCharTok{{-}} \DecValTok{1}\NormalTok{))\{}
\NormalTok{      Au[i] }\OtherTok{\textless{}{-}}\NormalTok{ Au[i] }\SpecialCharTok{/}\NormalTok{ (Am[i] }\SpecialCharTok{{-}}\NormalTok{ Al[i] }\SpecialCharTok{*}\NormalTok{ Au[i }\SpecialCharTok{{-}} \DecValTok{1}\NormalTok{])}
\NormalTok{      vec[i] }\OtherTok{\textless{}{-}}\NormalTok{ (vec[i] }\SpecialCharTok{{-}}\NormalTok{ Al[i] }\SpecialCharTok{*}\NormalTok{ vec[i }\SpecialCharTok{{-}} \DecValTok{1}\NormalTok{]) }\SpecialCharTok{/}
\NormalTok{                (Am[i] }\SpecialCharTok{{-}}\NormalTok{ Al[i] }\SpecialCharTok{*}\NormalTok{ Au[i }\SpecialCharTok{{-}} \DecValTok{1}\NormalTok{])}
\NormalTok{  \}}
\NormalTok{  vec[n] }\OtherTok{\textless{}{-}}\NormalTok{ (vec[n] }\SpecialCharTok{{-}}\NormalTok{ Al[n] }\SpecialCharTok{*}\NormalTok{ vec[n }\SpecialCharTok{{-}} \DecValTok{1}\NormalTok{])}\SpecialCharTok{/}
\NormalTok{            (Am[n] }\SpecialCharTok{{-}}\NormalTok{ Al[n] }\SpecialCharTok{*}\NormalTok{ Au[n }\SpecialCharTok{{-}} \DecValTok{1}\NormalTok{])}
  
  \DocumentationTok{\#\#\# backward sweep}
\NormalTok{  c\_vec }\OtherTok{\textless{}{-}} \FunctionTok{rep.int}\NormalTok{ (}\DecValTok{0}\NormalTok{, n)}
\NormalTok{  c\_vec[n] }\OtherTok{\textless{}{-}}\NormalTok{ vec[n]}
  \ControlFlowTok{for}\NormalTok{(i }\ControlFlowTok{in}\NormalTok{ (n }\SpecialCharTok{{-}} \DecValTok{1}\NormalTok{) }\SpecialCharTok{:}\DecValTok{1}\NormalTok{)}
\NormalTok{      c\_vec[i] }\OtherTok{\textless{}{-}}\NormalTok{ vec[i] }\SpecialCharTok{{-}}\NormalTok{ Au[i] }\SpecialCharTok{*}\NormalTok{ c\_vec[i }\SpecialCharTok{+} \DecValTok{1}\NormalTok{]}
  
 \DocumentationTok{\#\# Hitung vektor B dan D dari vektor C}
 \ControlFlowTok{for}\NormalTok{(i }\ControlFlowTok{in} \DecValTok{1}\SpecialCharTok{:}\NormalTok{(n}\DecValTok{{-}1}\NormalTok{))\{}
\NormalTok{   b\_vec[i] }\OtherTok{\textless{}{-}}\NormalTok{ (delta\_y[i]}\SpecialCharTok{/}\NormalTok{delta\_x[i])}\SpecialCharTok{{-}}
\NormalTok{               (delta\_x[i]}\SpecialCharTok{/}\DecValTok{3}\NormalTok{)}\SpecialCharTok{*}\NormalTok{(}\DecValTok{2}\SpecialCharTok{*}\NormalTok{c\_vec[i]}\SpecialCharTok{+}\NormalTok{c\_vec[i}\SpecialCharTok{+}\DecValTok{1}\NormalTok{])}
\NormalTok{   d\_vec[i] }\OtherTok{\textless{}{-}}\NormalTok{ (c\_vec[i}\SpecialCharTok{+}\DecValTok{1}\NormalTok{]}\SpecialCharTok{{-}}\NormalTok{c\_vec[i]) }\SpecialCharTok{/}\NormalTok{ (}\DecValTok{3}\SpecialCharTok{*}\NormalTok{delta\_x[i])}
\NormalTok{ \}}
  
 \FunctionTok{return}\NormalTok{(}\FunctionTok{list}\NormalTok{(}\AttributeTok{a =}\NormalTok{ a\_vec, }\AttributeTok{b =}\NormalTok{ b\_vec, }\AttributeTok{c =}\NormalTok{ c\_vec[}\SpecialCharTok{{-}}\NormalTok{n], }\AttributeTok{d =}\NormalTok{ d\_vec))}
\NormalTok{\}}
\end{Highlighting}
\end{Shaded}

\begin{example}
\protect\hypertarget{exm:cubicsplineexm}{}\label{exm:cubicsplineexm}Tentukan persamaan-persamaan spline kubik yang dihasilkan dari titik observasi yang ditampilkan pada Contoh \ref{exm:hopoliexmp}?
\end{example}

\textbf{Jawab}:

Untuk menentukan persamaan-persamaan linier yang menghubungkan setiap titik, kita akan menggunakan fungsi \texttt{cubic\_spline()} yang telah kita buat. Berikut adalah sintaks yang digunakan:

\begin{Shaded}
\begin{Highlighting}[]
\NormalTok{x }\OtherTok{\textless{}{-}} \FunctionTok{c}\NormalTok{(}\SpecialCharTok{{-}}\DecValTok{2}\NormalTok{, }\SpecialCharTok{{-}}\DecValTok{1}\NormalTok{, }\DecValTok{0}\NormalTok{, }\DecValTok{1}\NormalTok{)}
\NormalTok{y }\OtherTok{\textless{}{-}} \FunctionTok{c}\NormalTok{(}\SpecialCharTok{{-}}\DecValTok{1}\NormalTok{, }\SpecialCharTok{{-}}\DecValTok{2}\NormalTok{, }\SpecialCharTok{{-}}\DecValTok{1}\NormalTok{, }\DecValTok{2}\NormalTok{)}

\FunctionTok{cubic\_spline}\NormalTok{(x, y)}
\end{Highlighting}
\end{Shaded}

\begin{verbatim}
## $a
## [1] -1 -2 -1
## 
## $b
## [1] -1.4 -0.2  2.2
## 
## $c
## [1] 0.0 1.2 1.2
## 
## $d
## [1]  0.4  0.0 -0.4
\end{verbatim}

Sebagai contoh untuk domain \(\left[0,1\right]\), persamaan spline kubiknya adalah \(-0,4x^3+1,2x^2+2,2x-1\). Jika persamaan-persamaan yang terbentuk tersebut divisualisasikan akan terlihat seperti pada Gambar \ref{fig:cubicsplineviz}.

\begin{figure}

{\centering \includegraphics[width=0.8\linewidth]{./images/cubicsplineviz} 

}

\caption{Interpolasi spline kubik empat titik 
(Sumber:Howard, 2017).}\label{fig:cubicsplineviz}
\end{figure}

Pada \texttt{R} juga terdapat fungsi \texttt{splinefun()} untuk melakukan interpolasi spline. Format fungsi tersebut adalah sebagai berikut:

\begin{Shaded}
\begin{Highlighting}[]
\FunctionTok{splinefun}\NormalTok{(x, }\AttributeTok{y =} \ConstantTok{NULL}\NormalTok{,}
          \AttributeTok{method =} \FunctionTok{c}\NormalTok{(}\StringTok{"fmm"}\NormalTok{, }\StringTok{"periodic"}\NormalTok{, }\StringTok{"natural"}\NormalTok{, }\StringTok{"monoH.FC"}\NormalTok{, }\StringTok{"hyman"}\NormalTok{),}
          \AttributeTok{ties =}\NormalTok{ mean)}
\end{Highlighting}
\end{Shaded}

\begin{quote}
\textbf{Catatan:}

\begin{itemize}
\tightlist
\item
  \texttt{x,y} : vektor titik yang akan dilakukan interpolasi
\item
  \texttt{method} : spesifikasi jenis spline yang digunakan. Nilai yang mungkin antara lain: \texttt{"fmm"}, \texttt{"natural"}, \texttt{"periodic"}, \texttt{"monoH.FC"} dan \texttt{"hyman"}.
\item
  \texttt{ties} : metode untuk menangani nilai imbang pada titik yang akan diinterpolasi.
\end{itemize}
\end{quote}

Untuk melakukan interpolasi spline kubik, metode yang digunakan adalah \texttt{"natural"}. Berikut adalah contoh interpolasi menggunakan kembali data pada Contoh \ref{exm:hopoliexmp}:

\begin{Shaded}
\begin{Highlighting}[]
\NormalTok{spline }\OtherTok{\textless{}{-}} \FunctionTok{splinefun}\NormalTok{(x, y, }\AttributeTok{method=}\StringTok{"natural"}\NormalTok{)}

\CommentTok{\# uji dengan nilai x yang sama}
\FunctionTok{spline}\NormalTok{(x)}
\end{Highlighting}
\end{Shaded}

\begin{verbatim}
## [1] -1 -2 -1  2
\end{verbatim}

FUngsi \texttt{splinefun()} menghasilkan nilai yang sama persis dengan proses interpolasi yang telah kita lakukan.

\hypertarget{studi-kasus-1}{%
\section{Studi Kasus}\label{studi-kasus-1}}

Pada studi kasus kali ini, kita akan membahas teknik mengisi nilai yang hilang pada data runtun waktu. Terdapat banyak teknik untuk yang dapat digunakan untuk mengisi data yang hilang. Salah satu teknik yang dapat digunakan adalah dengan melakukan interpolasi.

\hypertarget{interpolasi-data-runtun-waktu}{%
\subsection{Interpolasi Data Runtun Waktu}\label{interpolasi-data-runtun-waktu}}

Pengisian data hilang pada data runtun waktu (\emph{time series}) dapat dilakukan dengan berbagai cara sesuai dengan situasi yang dihadapi. Pengisian dapat menggunakan nilai rata-rata jika data memiliki pola \emph{white noise}, observasi terakhir atau observasi dimasa mendatang, dan interpolasi linier.

Data yang digunakan pada contoh kasus kali ini adalah data \texttt{airquality}. Dataset tersebut merupakan data kualitas udara bulan Mei sampai September 1973 yang ada di New York. Berikut adalah ringkasan data \texttt{airquality} tersebut:

\begin{Shaded}
\begin{Highlighting}[]
\FunctionTok{summary}\NormalTok{(airquality)}
\end{Highlighting}
\end{Shaded}

\begin{verbatim}
##      Ozone          Solar.R         Wind      
##  Min.   :  1.0   Min.   :  7   Min.   : 1.70  
##  1st Qu.: 18.0   1st Qu.:116   1st Qu.: 7.40  
##  Median : 31.5   Median :205   Median : 9.70  
##  Mean   : 42.1   Mean   :186   Mean   : 9.96  
##  3rd Qu.: 63.2   3rd Qu.:259   3rd Qu.:11.50  
##  Max.   :168.0   Max.   :334   Max.   :20.70  
##  NA's   :37      NA's   :7                    
##       Temp          Month           Day      
##  Min.   :56.0   Min.   :5.00   Min.   : 1.0  
##  1st Qu.:72.0   1st Qu.:6.00   1st Qu.: 8.0  
##  Median :79.0   Median :7.00   Median :16.0  
##  Mean   :77.9   Mean   :6.99   Mean   :15.8  
##  3rd Qu.:85.0   3rd Qu.:8.00   3rd Qu.:23.0  
##  Max.   :97.0   Max.   :9.00   Max.   :31.0  
## 
\end{verbatim}

\begin{Shaded}
\begin{Highlighting}[]
\FunctionTok{str}\NormalTok{(airquality)}
\end{Highlighting}
\end{Shaded}

\begin{verbatim}
## 'data.frame':    153 obs. of  6 variables:
##  $ Ozone  : int  41 36 12 18 NA 28 23 19 8 NA ...
##  $ Solar.R: int  190 118 149 313 NA NA 299 99 19 194 ...
##  $ Wind   : num  7.4 8 12.6 11.5 14.3 14.9 8.6 13.8 20.1 8.6 ...
##  $ Temp   : int  67 72 74 62 56 66 65 59 61 69 ...
##  $ Month  : int  5 5 5 5 5 5 5 5 5 5 ...
##  $ Day    : int  1 2 3 4 5 6 7 8 9 10 ...
\end{verbatim}

\begin{Shaded}
\begin{Highlighting}[]
\FunctionTok{head}\NormalTok{(airquality)}
\end{Highlighting}
\end{Shaded}

\begin{verbatim}
##   Ozone Solar.R Wind Temp Month Day
## 1    41     190  7.4   67     5   1
## 2    36     118  8.0   72     5   2
## 3    12     149 12.6   74     5   3
## 4    18     313 11.5   62     5   4
## 5    NA      NA 14.3   56     5   5
## 6    28      NA 14.9   66     5   6
\end{verbatim}

Pada contoh kasus kali ini kita akan mencoba melakukan pengisian data hilang pada data \texttt{Solar.R} pada dataset \texttt{airquality}. Langkah pertama yang perlu dilakukan adalah membuat objek data runtun waktu pada data tersebut.

\begin{Shaded}
\begin{Highlighting}[]
\NormalTok{Solar }\OtherTok{\textless{}{-}} \FunctionTok{ts}\NormalTok{(airquality[,}\StringTok{"Solar.R"}\NormalTok{])}
\end{Highlighting}
\end{Shaded}

Visualisasi data tersebut disajikan pada Gambar \ref{fig:timeviz}.

\begin{figure}

{\centering \includegraphics[width=0.8\linewidth]{Algoritma_Pemrograman_R_files/figure-latex/timeviz-1} 

}

\caption{Visualisasi variabel radiasi matahari pada dataset airquality.}\label{fig:timeviz}
\end{figure}

Berdasarkan visualisasi tersebut terdapat garis yang terputus yang menunjukkan data yang hilang. Agar garis tersebut dapat tersambung, kita perlu melakukan pengisian nilai yang hilang pada data tersebut. Berikut adalah sintaks yang digunakan untuk melakukan pengisian nilai hilang tersebut:

\begin{Shaded}
\begin{Highlighting}[]
\FunctionTok{library}\NormalTok{(xts)}
\end{Highlighting}
\end{Shaded}

\begin{verbatim}
## Loading required package: zoo
\end{verbatim}

\begin{verbatim}
## 
## Attaching package: 'zoo'
\end{verbatim}

\begin{verbatim}
## The following objects are masked from 'package:base':
## 
##     as.Date, as.Date.numeric
\end{verbatim}

\begin{Shaded}
\begin{Highlighting}[]
\CommentTok{\# metode nilai rata{-}rata}
\NormalTok{Solar\_mean }\OtherTok{\textless{}{-}} \FunctionTok{na.fill}\NormalTok{(Solar, }\AttributeTok{fill=}\FunctionTok{mean}\NormalTok{(Solar, }\AttributeTok{na.rm=}\ConstantTok{TRUE}\NormalTok{))}

\CommentTok{\# metode last observation carried forward}
\NormalTok{Solar\_locf }\OtherTok{\textless{}{-}} \FunctionTok{na.locf}\NormalTok{(Solar)}

\CommentTok{\# metode next observation caried backward}
\NormalTok{Solar\_nocb }\OtherTok{\textless{}{-}} \FunctionTok{na.locf}\NormalTok{(Solar, }\AttributeTok{fromLast =} \ConstantTok{TRUE}\NormalTok{)}

\CommentTok{\# metode interpolasi linier}
\NormalTok{Solar\_linterp }\OtherTok{\textless{}{-}} \FunctionTok{na.approx}\NormalTok{(Solar)}
\end{Highlighting}
\end{Shaded}

Berikut adalah visualisasi menggunakan metode nilai rata-rata:

\begin{figure}

{\centering \includegraphics[width=0.8\linewidth]{Algoritma_Pemrograman_R_files/figure-latex/timeviz2-1} 

}

\caption{Visualisasi data radiasi matahari menggunakan metode nilai rata-rata.}\label{fig:timeviz2}
\end{figure}

Secara berturut-turut berikut adalah visualisasi dari metode locf, nocb, dan interpolasi linier:

\begin{figure}

{\centering \includegraphics[width=0.8\linewidth]{Algoritma_Pemrograman_R_files/figure-latex/timeviz3-1} 

}

\caption{Visualisasi data radiasi matahari menggunakan metode locf.}\label{fig:timeviz3}
\end{figure}

\begin{figure}

{\centering \includegraphics[width=0.8\linewidth]{Algoritma_Pemrograman_R_files/figure-latex/timeviz4-1} 

}

\caption{Visualisasi data radiasi matahari menggunakan metode nocb.}\label{fig:timeviz4}
\end{figure}

\begin{figure}

{\centering \includegraphics[width=0.8\linewidth]{Algoritma_Pemrograman_R_files/figure-latex/timeviz5-1} 

}

\caption{Visualisasi data radiasi matahari menggunakan metode interpolasi linier.}\label{fig:timeviz5}
\end{figure}

Pemilihan metode interpolasi mana yang sesuai akan berbeda pada setiap situasi dan jenis data yang akan dilakukan interpolasi. Interpolasi data runtun waktu pada bidang lingkungan umumnya menggunakan metode interpolasi nilai rata-rata dan linier dan tidak menutup kemungkinan interpolasi dengan metode lain yang telah dijelaskan pada buku ini dapat pula digunakan.

Pada situasi dimana data membentuk pola \emph{white noises} (pola acak disekitar nilai rata-rata dan memiliki varians yang konstan) seperti yang ditunjukkan variabel \texttt{Solar.R}, interpolaasi dengan nilai rata-rata cukup sesuai untuk digunakan untuk mengisi nilai hilang (\emph{missing value}) pada data runtun waktu tersebut.

\hypertarget{referensi-3}{%
\section{Referensi}\label{referensi-3}}

\begin{enumerate}
\def\labelenumi{\arabic{enumi}.}
\tightlist
\item
  Howard, J.P. 2017. \textbf{Computational Methods for Numerical Analysis with R}. CRC Press.
\item
  Kreyszig, E. 2011. \textbf{Advanced Engineering Mathematics, 10th Edition}. John Wiley \& Sons.
\item
  Suparno, S. 2008. \textbf{Komputasi untuk Sains dan Teknik Edisi II}. Departemen Fisika-FMIPA Universitas Indonesia.
\end{enumerate}

\hypertarget{latihan-2}{%
\section{Latihan}\label{latihan-2}}

\begin{enumerate}
\def\labelenumi{\arabic{enumi}.}
\tightlist
\item
  Diberikan data titik (3,5), (0,-2), dan (4,1). Tentukan persamaan polinomial untuk melakukan interpolasi pada ketiga titik tersebut!
\item
  Jika diberikan 13 titik observasi, apakah saudara cenderung akan menggunakan interpolasi polinomial atau spline? jelaskan alasan saudara?
\item
  Bentuklah kembali fungsi \texttt{poly\_inter()} dengan menambahkan metode evaluasi polinomial ke dalam fungsi tersebut!
\item
  Pada Latihan No.1, bentuklah persamaan spline kubik menggunakan titik observasi tersebut!
\end{enumerate}

\hypertarget{diffinteg}{%
\chapter{Diferensiasi dan Integrasi Numerik}\label{diffinteg}}

Pada Chapter \ref{diffinteg}, penulis akan menjabarkan mengenai metode numerik untuk melakukan diferensiasi dan integrasi pada suatu fungsi. Adapun yang akan dibahas pada \emph{Chapter} ini antara lain:

\begin{itemize}
\tightlist
\item
  Metode Beda Hingga
\item
  Metode Integrasi Newton-Cotes
\item
  Metode Integrasi Kudratur Gauss
\item
  Metode Integrasi Adaptif
\item
  Metode Integrasi Romberg
\item
  Metode Integrasi Monte Carlo
\item
  Studi Kasus
\end{itemize}

\hypertarget{finitediff}{%
\section{Metode Beda Hingga}\label{finitediff}}

Diferensiasi merupakan proses mencari slope suatu garis pada titik yang diberikan. Secara umum proses diferensiasi dinyatakan melalui Persamaan \eqref{eq:diff}.

\begin{equation}
f'\left(x\right) \approx \frac{f\left(x+h\right)-f\left(x\right)}{h}
  \label{eq:diff}
\end{equation}

Kita dapat menyatakan secara formal proses diferensiasi sebagai limit Persamaan \eqref{eq:diff} dimana \(h\) mendekati nol. Jadi kita ingin membuat nilai \(h\) sekecil mungkin untuk memperoleh pendekatan terbaik terhadap nilai turunan suatu fungsi. Kita membatasi nilai \(h\) pada sejumlah nilai yang masuk akal untuk mencegah pembagian dengan nilai yang tidak biasa. Kita juga harus memastikan \(f\left(x\right)\) dan \(f\left(x+h\right)\) terpisah cukup jauh untuk mencegah \emph{floating point round off error} mempengaruhi proses substraksi.

Terdapat 3 buah metode untuk memperoleh turunan pertama suatu fungsi dengan menggunakan metode numerik, yaitu: metode selisih maju, metode selisih mundur, dan metode selisih tengah. Error pada ketiga metode numerik tersebut ditaksir menggunakan deret Taylor. Persamaan \eqref{eq:diff2} dan Persamaan \eqref{eq:diff3} menunjukkan persamaan untuk memperoleh turunan pertama dan taksiran error menggunakan metode selisih maju dan metode selisih mundur.

\begin{equation}
f'\left(x\right) = \frac{f\left(x+h\right)-f\left(x\right)}{h} - \frac{h}{2} f''\left(c\right)
  \label{eq:diff2}
\end{equation}

\begin{equation}
f'\left(x\right) = \frac{f\left(x\right)-f\left(x-h\right)}{h} - \frac{h}{2} f''\left(c\right)
  \label{eq:diff3}
\end{equation}

Metode nilai tengah menggunakan ukuran langkah \(h\) dua kali dibandingkan dengan 2 metode lainnya. Error yang dihasilkan juga berbeda dengan kedua metode sebelumnya, dimana error dihasilkan dari pemotongan turunan ketiga pada deret Taylor. Secara umum metode selisih tengah memiliki akurasi yang lebih baik dibandingkan kedua metode sebelumnya karena metode ini mempertimbangkan dua sisi untuk memeriksa nilai \(x\). Persamaan \eqref{eq:diff4} merupakan persamaan untuk memperoleh nilai turunan pertama suatu fungsi dan estimasi error menggunakan deret Taylor.

\begin{equation}
f'\left(x\right) = \frac{f\left(x+h\right)-f\left(x-h\right)}{2h} - \frac{h^2}{6} f'''\left(c\right)
  \label{eq:diff4}
\end{equation}

Bagaimana menentukan \(h\)? beberapa literatur menggunakan pendekatan \emph{machine error} \(\epsilon\) berdasarkan program yang digunakan untuk melakukan proses perhitungan. Metode selisih maju dan selisih mundur menggunakan pendekatan yang ditunjukkan pada Persamaan \eqref{eq:diff5}.

\begin{equation}
h^{\ast}=x\sqrt{\epsilon}
  \label{eq:diff5}
\end{equation}

Untuk metode selisih tengah pendekatan nilai \(h\) menggunakan Persamaan \eqref{eq:diff6}.

\begin{equation}
h^{\ast}=x\sqrt[3]{\epsilon}
  \label{eq:diff6}
\end{equation}

Kita dapat menggunakan Persamaan \eqref{eq:diff2} sampai Persamaan \eqref{eq:diff4} untuk membentuk sebuah program yang digunakan untuk menghitung turunan pertama suatu fungsi. Sintaks yang digunakan adalah sebagai berikut:

\begin{Shaded}
\begin{Highlighting}[]
\NormalTok{findiff }\OtherTok{\textless{}{-}} \ControlFlowTok{function}\NormalTok{(f, x, h, }\AttributeTok{method=}\ConstantTok{NULL}\NormalTok{)\{}
  \ControlFlowTok{if}\NormalTok{(}\FunctionTok{is.null}\NormalTok{(method))\{}
    \FunctionTok{warning}\NormalTok{(}\StringTok{"please select a method"}\NormalTok{)}
\NormalTok{  \}}\ControlFlowTok{else}\NormalTok{\{}
    \ControlFlowTok{if}\NormalTok{(method }\SpecialCharTok{==} \StringTok{"forward"}\NormalTok{)\{}
      \FunctionTok{return}\NormalTok{((}\FunctionTok{f}\NormalTok{(x}\SpecialCharTok{+}\NormalTok{h)}\SpecialCharTok{{-}}\FunctionTok{f}\NormalTok{(x))}\SpecialCharTok{/}\NormalTok{h)}
\NormalTok{    \}}\ControlFlowTok{else} \ControlFlowTok{if}\NormalTok{(method}\SpecialCharTok{==}\StringTok{"backward"}\NormalTok{)\{}
      \FunctionTok{return}\NormalTok{((}\FunctionTok{f}\NormalTok{(x)}\SpecialCharTok{{-}}\FunctionTok{f}\NormalTok{(x}\SpecialCharTok{{-}}\NormalTok{h))}\SpecialCharTok{/}\NormalTok{h)}
\NormalTok{    \}}\ControlFlowTok{else} \ControlFlowTok{if}\NormalTok{(method}\SpecialCharTok{==}\StringTok{"central"}\NormalTok{)\{}
      \FunctionTok{return}\NormalTok{((}\FunctionTok{f}\NormalTok{(x}\SpecialCharTok{+}\NormalTok{h)}\SpecialCharTok{{-}}\FunctionTok{f}\NormalTok{(x}\SpecialCharTok{{-}}\NormalTok{h))}\SpecialCharTok{/}\NormalTok{(}\DecValTok{2}\SpecialCharTok{*}\NormalTok{h))}
\NormalTok{    \}}\ControlFlowTok{else}\NormalTok{\{}
      \FunctionTok{warning}\NormalTok{(}\StringTok{"you can use method: forward, bacward, or central"}\NormalTok{)}
\NormalTok{    \}}
\NormalTok{  \}}
\NormalTok{\}}
\end{Highlighting}
\end{Shaded}

\begin{example}
\protect\hypertarget{exm:diffexm}{}\label{exm:diffexm}Hitunglah turunan pertama persamaan berikut menggunakan metode selisih titik tengah pada x =1 dan nilai h=0,05!
\end{example}

\[
f\left(x\right)=e^{-x}\sin\left(2x\right)+1
\]

\textbf{Jawab}:

Untuk menghitung turunan pertama menggunakan metode selisih tengah, kita dapat menggunakan Persamaan \eqref{eq:diff4}. Berikut adalah proses perhitungannya:

\[
f'\left(1\right) = \frac{f\left(1+0.05\right)-f\left(1-0,05\right)}{2\times0.05}=--0.6390352
\]

Dengan menggunakan fungsi \texttt{findiff()}, hasil yang diperoleh adalah sebagai berikut:

\begin{Shaded}
\begin{Highlighting}[]
\FunctionTok{findiff}\NormalTok{(}\ControlFlowTok{function}\NormalTok{(x)}
  \FunctionTok{exp}\NormalTok{(}\SpecialCharTok{{-}}\NormalTok{x)}\SpecialCharTok{*}\FunctionTok{sin}\NormalTok{(}\DecValTok{2}\SpecialCharTok{*}\NormalTok{x)}\SpecialCharTok{+}\DecValTok{1}\NormalTok{, }\AttributeTok{x=}\DecValTok{1}\NormalTok{, }\AttributeTok{h=}\FloatTok{0.05}\NormalTok{,}
  \AttributeTok{method=}\StringTok{"central"}\NormalTok{)}
\end{Highlighting}
\end{Shaded}

\begin{verbatim}
## [1] -0.639
\end{verbatim}

Kita dapat memperkecil nilai \(h\) untuk memperoleh akurasi yang lebih baik berdasarkan pendekatan Persamaan \eqref{eq:diff6}.

\begin{Shaded}
\begin{Highlighting}[]
\FunctionTok{findiff}\NormalTok{(}\ControlFlowTok{function}\NormalTok{(x)\{  }\FunctionTok{exp}\NormalTok{(}\SpecialCharTok{{-}}\NormalTok{x)}\SpecialCharTok{*}\FunctionTok{sin}\NormalTok{(}\DecValTok{2}\SpecialCharTok{*}\NormalTok{x)}\SpecialCharTok{+}\DecValTok{1}\NormalTok{\}, }\AttributeTok{x=}\DecValTok{1}\NormalTok{,}
        \AttributeTok{h=}\DecValTok{1}\SpecialCharTok{*}\NormalTok{.Machine}\SpecialCharTok{$}\NormalTok{double.eps}\SpecialCharTok{\^{}}\NormalTok{(}\DecValTok{1}\SpecialCharTok{/}\DecValTok{3}\NormalTok{), }
        \AttributeTok{method=}\StringTok{"central"}\NormalTok{)}
\end{Highlighting}
\end{Shaded}

\begin{verbatim}
## [1] -0.6407
\end{verbatim}

Penyelesaian persamaan matematik dalam bidang Teknik Lingkungan pada umumnya tidak hanya melibatkan turunan pertama, pada penyelesaian persamaan difusi umumnya menggunakan turunan kedua. Persamaan \eqref{eq:diff7} merupakan pendekatan numerik untuk memperoleh nilai turunan kedua suatu persamaan dengan pendekatan deret Taylor.

\begin{equation}
f''\left(x\right)=\frac{f\left(x+h\right)-2f\left(x\right)+f\left(x-h\right)}{h^2}-\frac{h^2}{12}f^{\left(4\right)}\left(c\right)
  \label{eq:diff7}
\end{equation}

Fungsi \texttt{findiff2()} merupakan fungsi yang digunakan untuk menghitung turunan kedua suatu persamaan yang didasarkan pada Persamaan \eqref{eq:diff7}.

\begin{Shaded}
\begin{Highlighting}[]
\NormalTok{findiff2 }\OtherTok{\textless{}{-}} \ControlFlowTok{function}\NormalTok{(f, x, h)\{}
  \FunctionTok{return}\NormalTok{((}\FunctionTok{f}\NormalTok{(x}\SpecialCharTok{+}\NormalTok{h)}\SpecialCharTok{{-}}\DecValTok{2}\SpecialCharTok{*}\FunctionTok{f}\NormalTok{(x)}\SpecialCharTok{+}\FunctionTok{f}\NormalTok{(x}\SpecialCharTok{{-}}\NormalTok{h))}\SpecialCharTok{/}\NormalTok{(h}\SpecialCharTok{\^{}}\DecValTok{2}\NormalTok{))}
\NormalTok{\}}
\end{Highlighting}
\end{Shaded}

Kita dapat menghitung kembali turunan kedua fungsi pada Contoh \ref{exm:diffexm} menggunakan fungsi \texttt{findiff2()}. Berikut adalah sintaks yang digunakan:

\begin{Shaded}
\begin{Highlighting}[]
\FunctionTok{findiff2}\NormalTok{(}\ControlFlowTok{function}\NormalTok{(x)\{}
  \FunctionTok{exp}\NormalTok{(}\SpecialCharTok{{-}}\NormalTok{x)}\SpecialCharTok{*}\FunctionTok{sin}\NormalTok{(}\DecValTok{2}\SpecialCharTok{*}\NormalTok{x)}\SpecialCharTok{+}\DecValTok{1}
\NormalTok{\}, }\AttributeTok{x=}\DecValTok{1}\NormalTok{, }\AttributeTok{h=}\FloatTok{0.05}\NormalTok{)}
\end{Highlighting}
\end{Shaded}

\begin{verbatim}
## [1] -0.3924
\end{verbatim}

\hypertarget{diffother}{%
\section{Diferensiasi Menggunakan Fungsi Lainnya di R}\label{diffother}}

Terdapat sejumlah fungsi R yang dapat digunakan untuk menghitung turunan suatu persamaan matematik. Fungsi-fungsi tersebut tersedia dalam sejumlah Paket, baik \emph{base package} maupun yang berasal dari Paket lainnya.

\hypertarget{diferensiasi-metode-titik-pusat-mengggunakan-fungsidiff}{%
\subsection{\texorpdfstring{Diferensiasi Metode Titik Pusat Mengggunakan Fungsi\texttt{diff()}}{Diferensiasi Metode Titik Pusat Mengggunakan Fungsidiff()}}\label{diferensiasi-metode-titik-pusat-mengggunakan-fungsidiff}}

Fungsi \texttt{diff()} pada Paket \emph{base} dapat digunakan untuk menghitung turunan suatu persamaan menggunakan metode titik pusat. Fungsi ini pada umumnya digunakan untuk menghitung \emph{lag} suatu data runtun waktu. Agar fungsi tersebut dapat digunakan untuk menghitung turunan pertama suatu persamaan, kita dapat menggunakan argumen \texttt{lag\ =\ 2}. Berikut adalah contoh penerapan fungsi \texttt{diff()} untuk memperoleh turunan pertama persamaan matematik pada Contoh \ref{exm:diffexm}:

\begin{Shaded}
\begin{Highlighting}[]
\NormalTok{f }\OtherTok{\textless{}{-}} \ControlFlowTok{function}\NormalTok{(x)\{}\FunctionTok{exp}\NormalTok{(}\SpecialCharTok{{-}}\NormalTok{x)}\SpecialCharTok{*}\FunctionTok{sin}\NormalTok{(}\DecValTok{2}\SpecialCharTok{*}\NormalTok{x)}\SpecialCharTok{+}\DecValTok{1}\NormalTok{\}}
\NormalTok{x }\OtherTok{\textless{}{-}} \DecValTok{1}
\NormalTok{h }\OtherTok{\textless{}{-}}\NormalTok{ x}\SpecialCharTok{*}\NormalTok{.Machine}\SpecialCharTok{$}\NormalTok{double.eps}\SpecialCharTok{\^{}}\NormalTok{(}\DecValTok{1}\SpecialCharTok{/}\DecValTok{3}\NormalTok{)}
\NormalTok{xvec }\OtherTok{\textless{}{-}} \FunctionTok{seq}\NormalTok{(x}\SpecialCharTok{{-}}\NormalTok{h, x}\SpecialCharTok{+}\NormalTok{h, h)}

\CommentTok{\# turunan pertama}
\FunctionTok{diff}\NormalTok{(}\FunctionTok{f}\NormalTok{(xvec), }\AttributeTok{lag=}\DecValTok{2}\NormalTok{)}\SpecialCharTok{/}\NormalTok{(}\DecValTok{2}\SpecialCharTok{*}\NormalTok{h)}
\end{Highlighting}
\end{Shaded}

\begin{verbatim}
## [1] -0.6407
\end{verbatim}

\hypertarget{diferensiasi-menggunakan-paket-numderiv}{%
\subsection{\texorpdfstring{Diferensiasi Menggunakan Paket \texttt{numDeriv}}{Diferensiasi Menggunakan Paket numDeriv}}\label{diferensiasi-menggunakan-paket-numderiv}}

Paket standar yang sering digunakan untuk melakukan taksiran numerik turunan suatu fungsi adalah Paket \texttt{numDeriv}. Pada Paket tersebut terdapat fungsi \texttt{grad()} yang digunakan untuk menaksir turunan pertama suatu persamaan. Format fungsi tersebut adalah sebagai berikut:

\begin{Shaded}
\begin{Highlighting}[]
\FunctionTok{grad}\NormalTok{(func, x, }\AttributeTok{method=}\StringTok{"Richardson"}\NormalTok{, }\AttributeTok{method.args=}\FunctionTok{list}\NormalTok{(), ...)}
\end{Highlighting}
\end{Shaded}

\begin{quote}
\textbf{Catatan:}

\begin{itemize}
\tightlist
\item
  \textbf{func}: Fungsi persamaan matematik yang akan dicari turunannya.
\item
  \textbf{x}: Lokasi atau titik yang akan dicari gradiennya
\item
  \textbf{method}: Metode estimasi yang digunakan. Metode yang dapat digunakan antara lain:
\item
  ``simple'': metode selisih maju dengan \(h = 10^{-4}\)
\item
  ``Richardson'': metode interpolasi Richardson
\item
  ``complex'': \emph{complex-step derivative approach} dan dapat digunakan untuk persamaan \emph{complex-differentiable}.
\item
  \textbf{method.args}: argumen tambahan yang digunakan bersama dengan argumen \textbf{method}. Jika metode yang digunakan adalah ``simple'', nilai \(h\) dapat dispesifikasikan pada \textbf{method.args} jika diinginkan nilai \(h\) lainnya.
\end{itemize}
\end{quote}

Berikut adalah contoh penerapan fungsi \texttt{grad()} untuk memperoleh turunan pertama persamaan matematik pada Contoh \ref{exm:diffexm}:

\begin{Shaded}
\begin{Highlighting}[]
\NormalTok{numDeriv}\SpecialCharTok{::}\FunctionTok{grad}\NormalTok{(}\ControlFlowTok{function}\NormalTok{(x)\{}\FunctionTok{exp}\NormalTok{(}\SpecialCharTok{{-}}\NormalTok{x)}\SpecialCharTok{*}\FunctionTok{sin}\NormalTok{(}\DecValTok{2}\SpecialCharTok{*}\NormalTok{x)}\SpecialCharTok{+}\DecValTok{1}\NormalTok{\},}
     \AttributeTok{x=}\DecValTok{1}\NormalTok{, }\AttributeTok{method =} \StringTok{"simple"}\NormalTok{, }
     \AttributeTok{method.args =} \FunctionTok{list}\NormalTok{(}\AttributeTok{eps=}\DecValTok{1}\SpecialCharTok{*}\FunctionTok{sqrt}\NormalTok{(.Machine}\SpecialCharTok{$}\NormalTok{double.eps)))}
\end{Highlighting}
\end{Shaded}

\begin{verbatim}
## [1] -0.6407
\end{verbatim}

\hypertarget{diferensiasi-menggunakan-paket-pracma}{%
\subsection{\texorpdfstring{DIferensiasi Menggunakan Paket \texttt{pracma}}{DIferensiasi Menggunakan Paket pracma}}\label{diferensiasi-menggunakan-paket-pracma}}

Terdapat sejumlah fungsi pada Paket \texttt{pracma} yang dapat digunakan untuk melakukan diferensiasi suatu persamaan matematik. FUngsi-fungsi yang dapat digunakan untuk melakukan diferensiasi sederhana antara lain: \texttt{fderiv()}, \texttt{numderiv()}, \texttt{numdiff()}, dan \texttt{grad()}.

Fungsi \texttt{fderiv()} dapat digunakan untuk melakukan diferensiasi orde pertama sampai dengan orde tinggi. Perlu dicatat bahwa diferensiasi cenderung kurang akurat jika orde diferensiasi semakin tinggi. Format yang digunakan untuk melakukan diferensiasi menggunakan fungsi \texttt{fderiv()} adalah sebagai berikut:

\begin{Shaded}
\begin{Highlighting}[]
\FunctionTok{fderiv}\NormalTok{(f, x, }\AttributeTok{n =} \DecValTok{1}\NormalTok{, }\AttributeTok{h =} \DecValTok{0}\NormalTok{,}
        \AttributeTok{method =} \FunctionTok{c}\NormalTok{(}\StringTok{"central"}\NormalTok{, }\StringTok{"forward"}\NormalTok{, }\StringTok{"backward"}\NormalTok{), }
\NormalTok{       ...)}
\end{Highlighting}
\end{Shaded}

\begin{quote}
\textbf{Catatan:}

\begin{itemize}
\tightlist
\item
  \textbf{f}: Fungsi persamaan matematik yang akan dicari turunannya.
\item
  \textbf{x}: Lokasi atau titik yang akan dicari gradiennya
\item
  \textbf{n}: Orde diferensiasi yang digunakan. Orde diferensiasi yang dapat digunakan adalah 1 sampai 8.
\item
  \textbf{method}: Metode estimasi yang digunakan. Metode yang dapat digunakan antara lain:
\item
  ``central'': metode titik pusat
\item
  ``forward'': metode selisih maju
\item
  ``bacward'': metode selisih mundur.
\item
  \textbf{\ldots{}}: argumen tambahan fungsi \textbf{f}.
\end{itemize}
\end{quote}

Berikut adalah contoh penerapan fungsi \texttt{fderiv()} untuk memperoleh turunan pertama dan kedua persamaan matematik pada Contoh \ref{exm:diffexm}:

\begin{Shaded}
\begin{Highlighting}[]
\FunctionTok{library}\NormalTok{(pracma)}
\end{Highlighting}
\end{Shaded}

\begin{verbatim}
## 
## Attaching package: 'pracma'
\end{verbatim}

\begin{verbatim}
## The following objects are masked from 'package:rootSolve':
## 
##     gradient, hessian
\end{verbatim}

\begin{verbatim}
## The following objects are masked from 'package:Matrix':
## 
##     expm, lu, tril, triu
\end{verbatim}

\begin{Shaded}
\begin{Highlighting}[]
\CommentTok{\# turunan 1}
\FunctionTok{fderiv}\NormalTok{(}\ControlFlowTok{function}\NormalTok{(x)\{}\FunctionTok{exp}\NormalTok{(}\SpecialCharTok{{-}}\NormalTok{x)}\SpecialCharTok{*}\FunctionTok{sin}\NormalTok{(}\DecValTok{2}\SpecialCharTok{*}\NormalTok{x)}\SpecialCharTok{+}\DecValTok{1}\NormalTok{\},}
       \AttributeTok{x =} \DecValTok{1}\NormalTok{, }\AttributeTok{n =} \DecValTok{1}\NormalTok{, }\AttributeTok{h =} \DecValTok{1}\SpecialCharTok{*}\NormalTok{.Machine}\SpecialCharTok{$}\NormalTok{double.eps}\SpecialCharTok{\^{}}\NormalTok{(}\DecValTok{1}\SpecialCharTok{/}\DecValTok{3}\NormalTok{), }
       \AttributeTok{method =} \StringTok{"central"}\NormalTok{)}
\end{Highlighting}
\end{Shaded}

\begin{verbatim}
## [1] -0.6407
\end{verbatim}

\begin{Shaded}
\begin{Highlighting}[]
\CommentTok{\# turunan 2}
\FunctionTok{fderiv}\NormalTok{(}\ControlFlowTok{function}\NormalTok{(x)\{}\FunctionTok{exp}\NormalTok{(}\SpecialCharTok{{-}}\NormalTok{x)}\SpecialCharTok{*}\FunctionTok{sin}\NormalTok{(}\DecValTok{2}\SpecialCharTok{*}\NormalTok{x)}\SpecialCharTok{+}\DecValTok{1}\NormalTok{\},}
       \AttributeTok{x =} \DecValTok{1}\NormalTok{, }\AttributeTok{n =} \DecValTok{2}\NormalTok{, }\AttributeTok{h =} \DecValTok{1}\SpecialCharTok{*}\NormalTok{.Machine}\SpecialCharTok{$}\NormalTok{double.eps}\SpecialCharTok{\^{}}\NormalTok{(}\DecValTok{1}\SpecialCharTok{/}\DecValTok{3}\NormalTok{), }
       \AttributeTok{method =} \StringTok{"central"}\NormalTok{)}
\end{Highlighting}
\end{Shaded}

\begin{verbatim}
## [1] -0.3912
\end{verbatim}

Fungsi \texttt{numderiv()} menggunakan ekstrapolasi Richardson untuk melakukan taksiran turunan suatu persamaan matematik. Berbeda dengan fungsi lainnya, fungsi \texttt{numderiv()} tidak hanya menampilkan hasil diferensiasi, fungsi ini juga menampilkan error absolut, error relatif, dan jumlah iterasi yang berlangsung. Berikut adalah format fungsi yang digunakan:

\begin{Shaded}
\begin{Highlighting}[]
\FunctionTok{numderiv}\NormalTok{(f, x0, }\AttributeTok{maxiter =} \DecValTok{16}\NormalTok{, }\AttributeTok{h =} \DecValTok{1}\SpecialCharTok{/}\DecValTok{2}\NormalTok{, ..., }
         \AttributeTok{tol =}\NormalTok{ .Machine}\SpecialCharTok{$}\NormalTok{double.eps)}
\end{Highlighting}
\end{Shaded}

\begin{quote}
\textbf{Catatan:}

\begin{itemize}
\tightlist
\item
  \textbf{f}: Fungsi persamaan matematik yang akan dicari turunannya.
\item
  \textbf{x0}: Lokasi atau titik yang akan dicari gradiennya
\item
  \textbf{maxiter}: Iterasi maksimum yang digunakan.
\item
  \textbf{h}: \emph{step size} yang digunakan
\item
  \textbf{tol}: toleransi error yang digunakan.
\end{itemize}
\end{quote}

Berikut adalah contoh penerapan fungsi \texttt{numderiv()} untuk memperoleh turunan pertama persamaan matematik pada Contoh \ref{exm:diffexm}:

\begin{Shaded}
\begin{Highlighting}[]
\FunctionTok{numderiv}\NormalTok{(}\ControlFlowTok{function}\NormalTok{(x)\{}\FunctionTok{exp}\NormalTok{(}\SpecialCharTok{{-}}\NormalTok{x)}\SpecialCharTok{*}\FunctionTok{sin}\NormalTok{(}\DecValTok{2}\SpecialCharTok{*}\NormalTok{x)}\SpecialCharTok{+}\DecValTok{1}\NormalTok{\},}
       \AttributeTok{x0 =} \DecValTok{1}\NormalTok{, }\AttributeTok{h =} \DecValTok{1}\SpecialCharTok{*}\NormalTok{.Machine}\SpecialCharTok{$}\NormalTok{double.eps}\SpecialCharTok{\^{}}\NormalTok{(}\DecValTok{1}\SpecialCharTok{/}\DecValTok{3}\NormalTok{))}
\end{Highlighting}
\end{Shaded}

\begin{verbatim}
## $df
## [1] -0.6407
## 
## $rel.err
## [1] 7.631e-11
## 
## $niter
## [1] 2
\end{verbatim}

Fungsi \texttt{numderiv()} memiliki keterbatasan dalam penggunaannya. Argumen \texttt{x0} yang digunakan haruslah angka numerik tunggal. Fungsi \texttt{numdiff()} mengatasi keterbatasan tersebut. Fungsi ini dapat menerima input berupa vektor, sehingga dapat digunakan untuk mencari nilai turunan pada sejumlah titik. Selain itu, output fungsi ini lebih sederhana, dimana hanya menampilkan hasil diferensiasinya saja. Berikut adalah format fungsi \texttt{numdiff()}:

\begin{Shaded}
\begin{Highlighting}[]
\FunctionTok{numdiff}\NormalTok{(f, x, }\AttributeTok{maxiter =} \DecValTok{16}\NormalTok{, }\AttributeTok{h =} \DecValTok{1}\SpecialCharTok{/}\DecValTok{2}\NormalTok{, ..., }
         \AttributeTok{tol =}\NormalTok{ .Machine}\SpecialCharTok{$}\NormalTok{double.eps)}
\end{Highlighting}
\end{Shaded}

\begin{quote}
\textbf{Catatan:}

\begin{itemize}
\tightlist
\item
  \textbf{f}: Fungsi persamaan matematik yang akan dicari turunannya.
\item
  \textbf{x}: Vektor titik yang akan dicari gradiennya
\item
  \textbf{maxiter}: Iterasi maksimum yang digunakan.
\item
  \textbf{h}: \emph{step size} yang digunakan
\item
  \textbf{tol}: toleransi error yang digunakan.
\end{itemize}
\end{quote}

Berikut adalah contoh penerapan fungsi \texttt{numdiff()} untuk memperoleh turunan pertama persamaan matematik pada Contoh \ref{exm:diffexm}:

\begin{Shaded}
\begin{Highlighting}[]
\FunctionTok{numdiff}\NormalTok{(}\ControlFlowTok{function}\NormalTok{(x)\{}\FunctionTok{exp}\NormalTok{(}\SpecialCharTok{{-}}\NormalTok{x)}\SpecialCharTok{*}\FunctionTok{sin}\NormalTok{(}\DecValTok{2}\SpecialCharTok{*}\NormalTok{x)}\SpecialCharTok{+}\DecValTok{1}\NormalTok{\},}
       \AttributeTok{x =} \DecValTok{1}\SpecialCharTok{:}\DecValTok{4}\NormalTok{, }\AttributeTok{h =} \DecValTok{1}\SpecialCharTok{*}\NormalTok{.Machine}\SpecialCharTok{$}\NormalTok{double.eps}\SpecialCharTok{\^{}}\NormalTok{(}\DecValTok{1}\SpecialCharTok{/}\DecValTok{3}\NormalTok{))}
\end{Highlighting}
\end{Shaded}

\begin{verbatim}
## [1] -0.64070 -0.07450  0.10952 -0.02345
\end{verbatim}

Fungsi \texttt{grad()} pada Paket \texttt{pracma} berbeda dengan yang digunakan pada Paket \texttt{numDeriv}. Perbedaan utama fungsi pada kedua Paket tersebut adalah metode estimasi yang digunakan untuk menghitung turunan pertama suatu persamaan matematik. Pada Paket \texttt{pracma}, metode yang digunakan adalah metode titik pusat, sedangkan pada Paket \texttt{numDeriv} metode yang digunakan adalah metode selisih maju, ekstrapolasi Richardson, dan \emph{complex}. Format fungsi \texttt{grad()} pada Paket \texttt{pracma} adalah sebagai berikut:

\begin{Shaded}
\begin{Highlighting}[]
\FunctionTok{grad}\NormalTok{(f, x0, }\AttributeTok{heps =}\NormalTok{ .Machine}\SpecialCharTok{$}\NormalTok{double.eps}\SpecialCharTok{\^{}}\NormalTok{(}\DecValTok{1}\SpecialCharTok{/}\DecValTok{3}\NormalTok{), }
\NormalTok{     ...)}
\end{Highlighting}
\end{Shaded}

\begin{quote}
\textbf{Catatan:}

\begin{itemize}
\tightlist
\item
  \textbf{f}: Fungsi persamaan matematik yang akan dicari turunannya.
\item
  \textbf{x0}: Titik yang akan dicari gradiennya
\item
  \textbf{heps}: \emph{step size} yang digunakan
\item
  \textbf{\ldots{}}: Argumen lain yang digunakan pada fungsi \textbf{f}.
\end{itemize}
\end{quote}

Berikut adalah contoh penerapan fungsi \texttt{grad()} untuk memperoleh turunan pertama persamaan matematik pada Contoh \ref{exm:diffexm}:

\begin{Shaded}
\begin{Highlighting}[]
\FunctionTok{grad}\NormalTok{(}\ControlFlowTok{function}\NormalTok{(x)\{}\FunctionTok{exp}\NormalTok{(}\SpecialCharTok{{-}}\NormalTok{x)}\SpecialCharTok{*}\FunctionTok{sin}\NormalTok{(}\DecValTok{2}\SpecialCharTok{*}\NormalTok{x)}\SpecialCharTok{+}\DecValTok{1}\NormalTok{\}, }\AttributeTok{x0 =} \DecValTok{1}\NormalTok{)}
\end{Highlighting}
\end{Shaded}

\begin{verbatim}
## [1] -0.6407
\end{verbatim}

\hypertarget{newtoncotes}{%
\section{Metode Integrasi Newton-Cotes}\label{newtoncotes}}

Metode integrasi Newton-Cotes secara umum merupakan metode integrasi yang dilakukan dengan membagi area di bawah kurva suatu fungsi menjadi beberapa panel dengan terlebih dahulu menetapkan batas atas dan batas bawah interval. Integral atau luas area di bawah kurva ditentukan berdasarkan jumlah luas panel yang digunakan untuk mendekati luas area di bawah kurva.

Terdapat beberapa metode yang akan penulis jelaskan pada sub-Chapter ini. Metode-metode tersebut antara lain:

\begin{itemize}
\tightlist
\item
  Metode integral Riemann
\item
  Metode trapezoida
\item
  Metode Simpson 1/3
\item
  Metode Simpson 3/8
\end{itemize}

\hypertarget{riemann}{%
\subsection{Metode Integral Riemann}\label{riemann}}

Metode integral Riemann dilakukan dengan membagi interval di bawah kurva suatu fungsi matematik sebanyak \(m\) subinterval sama besar. Pada setiap subinterval dibentuk persegi panjang setinggi kurva pada setiap titik tengah persegi panjang tersebut. Area setiap subinterval diperoleh dengan mengalikan panjang dan lebar masing-masing persegi panjang. Jumlah masing-masing area tersebut digunakan untuk menaksir interval integral suatu fungsi dengan interval tertentu. Fungsi proses integrasi menggunakan metode titik tengah dapat dituliskan pada Persamaan \eqref{eq:reimann}.

\begin{equation}
\int_a^bf\left(x\right)dx\ \approx\sum_{i=1}^mf\left(i\ \frac{\left|b-a\right|}{m}-\frac{\left|b-a\right|}{2m}\right)\ \frac{\left|b-a\right|}{m}
  \label{eq:reimann}
\end{equation}

dimana \(b\) dan \(a\) masing-masing merupakan batas atas dan bawah interval kurva yang hendak dihitung integralnya.

Error dari metode ini dapat diestimasi menggunakan Persamaan \eqref{eq:reimann2}.

\begin{equation}
\int_a^bh\left(x\right)dx=-\ \frac{\left(b-a\right)^3}{24m^2}f^{\left(2\right)}\left(\xi\right)
  \label{eq:reimann2}
\end{equation}

dimana \(\xi\) merupakan nilai antara \(a\) dan \(b\).

\begin{example}
\protect\hypertarget{exm:reimannexm}{}\label{exm:reimannexm}Hitunglah intergral fungsi di bawah ini menggunakan metode integral Reimann dengan interval 0 sampai 1 dan jumlah panel 2 dan 4!
\end{example}

\[
\int_0^1 x^2 dx
\]

\textbf{Jawab}:

Fungsi pada Contoh \ref{exm:reimannexm} dapat diselesaikan menggunakan metode analitik. Penyelesaian analitik fungsi tersebut adalah sebagai berikut:

\[
\int_0^1 x^2 dx = \left[\frac{x^3}{3}\right]_{_0}^{^1}=\frac{1^3}{3}-\frac{0^3}{3}=0,333...
\]

Penyelesaian numerik menggunakan metode titik tengah dengan jumlah panel 2 dapat dilakukan dengan menentukan lokasi titik tengah kedua panel. Berdasarkan interval fungsi dapat kita tentukan titik tengah kedua panel berada pada \(x=0,25\) dan \(x=0,75\). Perhitungan dilakukan seperti berikut:

\[
\int_0^1 x^2 dx \approx \left(f\left(0,25\right)+f\left(0,75\right)\right)\ \frac{1-0}{2}=\frac{0,25^2+0,75^2}{2}=0,3125
\]
Untuk meningkatkan akurasi dari nilai yang dihasilkan, jumlah panel dapat ditingkatkan. Untuk jumlah panel 4, titik tengah berada pada \(x=\left\{0,125;0,375;0,625;0,875\right\}\).

\[
\int_0^1 x^2 dx \approx \left(f\left(0,125\right)+f\left(0,375\right)+f\left(0,625\right)+f\left(0,875\right)\right)\ \frac{1-0}{4}=0,328125
\]

Visualisasi proses integrasi dengan metode Riemann dapat dilihat pada Gambar \ref{fig:midpt}.

\begin{figure}

{\centering \includegraphics[width=0.9\linewidth]{./images/midpt} 

}

\caption{Visualisasi integral Riemann dengan 2 panel dan 4 panel (sumber:Howard, 2017).}\label{fig:midpt}
\end{figure}

Berdasarkan Persamaan \eqref{eq:reimann}, kita dapat mengembangkan fungsi \texttt{R} yang dapat digunakan untuk melakukan perhitungan integral Riemann. Sintaks fungsi tersebut adalah sebagai berikut:

\begin{Shaded}
\begin{Highlighting}[]
\NormalTok{riemann }\OtherTok{\textless{}{-}} \ControlFlowTok{function}\NormalTok{(f, a, b, }\AttributeTok{m =} \DecValTok{100}\NormalTok{)\{}
\NormalTok{  n\_width }\OtherTok{\textless{}{-}}\NormalTok{ (b}\SpecialCharTok{{-}}\NormalTok{a)}\SpecialCharTok{/}\NormalTok{m}
\NormalTok{  x }\OtherTok{\textless{}{-}} \FunctionTok{seq}\NormalTok{(a, b}\SpecialCharTok{{-}}\NormalTok{n\_width, }\AttributeTok{length.out =}\NormalTok{ m) }\SpecialCharTok{+}\NormalTok{ n\_width}\SpecialCharTok{/}\DecValTok{2}
\NormalTok{  y }\OtherTok{\textless{}{-}} \FunctionTok{f}\NormalTok{(x)}
  
  \FunctionTok{return}\NormalTok{(}\FunctionTok{sum}\NormalTok{(y)}\SpecialCharTok{*}\FunctionTok{abs}\NormalTok{(b}\SpecialCharTok{{-}}\NormalTok{a)}\SpecialCharTok{/}\NormalTok{m)}
\NormalTok{\}}
\end{Highlighting}
\end{Shaded}

Kita akan menghitung kembali fungsi pada Contoh \ref{exm:reimannexm} dengan menggunakan jumlah panel 2, 4 dan 100. Berikut adalah sintaks yang digunakan:

\begin{Shaded}
\begin{Highlighting}[]
\CommentTok{\# m=2}
\FunctionTok{riemann}\NormalTok{(}\ControlFlowTok{function}\NormalTok{(x) x}\SpecialCharTok{\^{}}\DecValTok{2}\NormalTok{, }\AttributeTok{a=}\DecValTok{0}\NormalTok{, }\AttributeTok{b=}\DecValTok{1}\NormalTok{, }\AttributeTok{m=}\DecValTok{2}\NormalTok{)}
\end{Highlighting}
\end{Shaded}

\begin{verbatim}
## [1] 0.3125
\end{verbatim}

\begin{Shaded}
\begin{Highlighting}[]
\CommentTok{\# m=4}
\FunctionTok{riemann}\NormalTok{(}\ControlFlowTok{function}\NormalTok{(x) x}\SpecialCharTok{\^{}}\DecValTok{2}\NormalTok{, }\AttributeTok{a=}\DecValTok{0}\NormalTok{, }\AttributeTok{b=}\DecValTok{1}\NormalTok{, }\AttributeTok{m=}\DecValTok{4}\NormalTok{)}
\end{Highlighting}
\end{Shaded}

\begin{verbatim}
## [1] 0.3281
\end{verbatim}

\begin{Shaded}
\begin{Highlighting}[]
\CommentTok{\# m=100}
\FunctionTok{riemann}\NormalTok{(}\ControlFlowTok{function}\NormalTok{(x) x}\SpecialCharTok{\^{}}\DecValTok{2}\NormalTok{, }\AttributeTok{a=}\DecValTok{0}\NormalTok{, }\AttributeTok{b=}\DecValTok{1}\NormalTok{)}
\end{Highlighting}
\end{Shaded}

\begin{verbatim}
## [1] 0.3333
\end{verbatim}

Berdasarkan teori yang telah dipaparkan sebelumnya, kita ketahui bahwa untuk memperoleh nilai pendekatan integral yang sebenarnya kita dapat meningkatkan jumlah panel yang digunakan. Untuk mengetahui jumlah panel minimum yang diperlukan untuk memperoleh hasil integrasi yang stabil, kita akan melakukan simulasi menggunakan data yang disajikan pada Contoh \ref{exm:reimannexm} dengan memvariasikan jumlah panel yang akan digunakan. Pada simulasi yang akan dilakukan kita akan coba memvariasikan jumlah panel dari 2 hingga 100. Berikut adalah sintaks yang digunakan:

\begin{figure}

{\centering \includegraphics[width=0.9\linewidth]{Algoritma_Pemrograman_R_files/figure-latex/midpt2-1} 

}

\caption{Visualisasi simulasi pemilihan jumlah panel minimum metode integrasi Riemann.}\label{fig:midpt2}
\end{figure}

Berdasarkan hasil simulasi dapat disimpulkan jumlah panel minimum yang diperlukan untuk memperoleh hasil integrasi yang stabil kira-kira sebesar \(m=40\).

\hypertarget{trapezoida}{%
\subsection{Metode Trapezoida}\label{trapezoida}}

Pendekatan trapezoida dilakukan dengan melakukan pendekatan area dibawah kurva fungsi \(y=f\left(x\right)\) dengan subinterval \(\left[x_i,x_{i+1}\right]\) menggunakan trapesium. Untuk memahami pendekatan yang digunakan pembaca dapat memperhatikan Gambar \ref{fig:trapviz}.

\begin{figure}

{\centering \includegraphics[width=0.9\linewidth]{./images/trapviz} 

}

\caption{Visualisasi integragrasi numerik menggunakan metode tapezoida (sumber: Jones et.al., 2014).}\label{fig:trapviz}
\end{figure}

Fungsi proses integrasi menggunakan metode trapezoida dapat dituliskan pada Persamaan \eqref{eq:trap}.

\begin{equation}
\int_a^bf\left(x\right)dx\ = \lim\limits_{m \to \infty} \sum_{i=1}^m\frac{\left(c_{i+1}-c_i\right)\times\left(f\left(c_{i+1}\right)+f\left(c_i\right)\right)}{m}
  \label{eq:trap}
\end{equation}

dimana

\[
c_i=a+\frac{\left(b-a\right)}{n}i
\]
\(n\) merupakan nilai subinterval dan \(m\) merupakan jumlah panel trapesium yang digunakan.

Error dari metode ini dapat diestimasi menggunakan Persamaan \eqref{eq:trap2}.

\begin{equation}
\int_a^bh\left(x\right)dx=-\ \frac{\left(b-a\right)^3}{12m^2}f^{\left(2\right)}\left(\xi\right)
  \label{eq:trap2}
\end{equation}

dimana \(\xi\) merupakan nilai antara \(a\) dan \(b\).

\begin{example}
\protect\hypertarget{exm:trapnexm}{}\label{exm:trapnexm}Hitung kembali nilai intergrasi persamaan pada Contoh \ref{exm:reimannexm} menggunakan metode trapezoida dengan jumlah panel m=2!
\end{example}

\textbf{Jawab}:

Penyelesaian numerik menggunakan trapezoida dengan jumlah panel 2 dapat dilakukan dengan menentukan lokasi titik evaluasi. Berdasarkan Gambar \ref{fig:trapviz2}, terdapat 3 batas subinterval yaitu pada \(a\),\(\frac{\left(b-a\right)}{2}\), dan \(b\). Perhitungan intergral menggunakan ketiga titik evaluasi tersebut adalah sebagai berikut:

\[
\int_0^1 x^2 dx \approx \frac{\left(0,5\right)\left(0,25+1,25\right)}{2}=0,375
\]

\begin{figure}

{\centering \includegraphics[width=0.7\linewidth]{./images/trapviz2} 

}

\caption{Visualisasi integrasi metode trapezoida dengan 2 panel (sumber:Howard, 2017).}\label{fig:trapviz2}
\end{figure}

Berdasarkan Persamaan \eqref{eq:trap}, kita dapat mengembangkan fungsi \texttt{R} yang dapat digunakan untuk melakukan perhitungan integral metode trapezoida. Sintaks fungsi tersebut adalah sebagai berikut:

\begin{Shaded}
\begin{Highlighting}[]
\NormalTok{trap }\OtherTok{\textless{}{-}} \ControlFlowTok{function}\NormalTok{(f, a, b, }\AttributeTok{m=}\DecValTok{100}\NormalTok{)\{}
\NormalTok{  x }\OtherTok{\textless{}{-}} \FunctionTok{seq}\NormalTok{(a, b, }\AttributeTok{length.out =}\NormalTok{ m}\SpecialCharTok{+}\DecValTok{1}\NormalTok{)}
\NormalTok{  y }\OtherTok{\textless{}{-}} \FunctionTok{f}\NormalTok{(x)}
  
\NormalTok{  p\_area }\OtherTok{\textless{}{-}} \FunctionTok{sum}\NormalTok{((y[}\DecValTok{2}\SpecialCharTok{:}\NormalTok{(m}\SpecialCharTok{+}\DecValTok{1}\NormalTok{)] }\SpecialCharTok{+}\NormalTok{ y[}\DecValTok{1}\SpecialCharTok{:}\NormalTok{m])) }
\NormalTok{  p\_area }\OtherTok{\textless{}{-}}\NormalTok{ p\_area }\SpecialCharTok{*} \FunctionTok{abs}\NormalTok{(b}\SpecialCharTok{{-}}\NormalTok{a)}\SpecialCharTok{/}\NormalTok{(}\DecValTok{2}\SpecialCharTok{*}\NormalTok{m)}
  \FunctionTok{return}\NormalTok{(p\_area)}
\NormalTok{\}}
\end{Highlighting}
\end{Shaded}

Kita dapat menghitung kembali intergral persamaan pada Contoh \ref{exm:reimannexm} menggunakan fungsi \texttt{trap()} yang telah dibuat. Berikut adalah sintaks yang digunakan:

\begin{Shaded}
\begin{Highlighting}[]
\FunctionTok{trap}\NormalTok{(}\ControlFlowTok{function}\NormalTok{(x)x}\SpecialCharTok{\^{}}\DecValTok{2}\NormalTok{, }\AttributeTok{a=}\DecValTok{0}\NormalTok{, }\AttributeTok{b=}\DecValTok{1}\NormalTok{, }\AttributeTok{m=}\DecValTok{2}\NormalTok{)}
\end{Highlighting}
\end{Shaded}

\begin{verbatim}
## [1] 0.375
\end{verbatim}

Untuk mengetahui jumlah panel minimum yang diperlukan untuk memperoleh hasil integrasi yang stabil pada persamaan tersebut, kita akan kembali melakukan simulasi menggunakan variasi jumlah panel yang digunakan. Dalam simulasi variasi jumlah panel yang digunakan adalah 2 sampai 100. Berikut adalah sintaks yang digunakan:

\begin{verbatim}
## [1] 0.3334
\end{verbatim}

\begin{figure}

{\centering \includegraphics[width=0.9\linewidth]{Algoritma_Pemrograman_R_files/figure-latex/trapviz3-1} 

}

\caption{Visualisasi simulasi pemilihan jumlah panel minimum metode integrasi trapezoida.}\label{fig:trapviz3}
\end{figure}

Berdasarkan hasil simulasi diperoleh nilai panel minimum sebesar \(m=20\). Hasil yang diperoleh tersebut menujukkan bahwa metode trapezoida lebih efisien dalam proses komputasi dibandingkan metode Riemann.

\hypertarget{simpson}{%
\subsection{Metode Simpson}\label{simpson}}

Metode Simpson membagi subinterval \(\left[a,b\right]\) menjadi
\(n\) subinterval, dimana \(n\) merupakan bilangan genap. Untuk setiap pasang subinterval, luas area di bawah fungsi \(f\left(x\right)\) ditaksir menggunakan polinomial berderajat 2.

Misalkan \(u<v<w\) merupakan titik sembarang pada suatu fungsi yang akan dicari integralnya yang terpisah sejauh \(h\). Untuk \(x\in\left[u,w\right]\) kita ingin menaksir \(f\left(x\right)\) menggunakan parabola yang melalui titik \(\left(u, f\left(u\right)\right)\), \(\left(v, f\left(v\right)\right)\), dan \(\left(w, f\left(w\right)\right)\). Terdapat tepat 1 parabola \(p\left(x\right)\) yang dapat dibentuk dari ketiga titik koordinat tersebut yang ditunjukkan melalui Persamaan \eqref{eq:simp}.

\begin{equation}
p\left(x\right)=f\left(u\right)\frac{\left(x-v\right)\left(x-w\right)}{\left(u-v\right)\left(u-w\right)}+f\left(v\right)\frac{\left(x-u\right)\left(x-w\right)}{\left(v-u\right)\left(v-w\right)}+f\left(w\right)\frac{\left(x-u\right)\left(x-v\right)}{\left(w-u\right)\left(w-v\right)}
  \label{eq:simp}
\end{equation}

Sebagai taksiran luas di bawah kurva \(y=f\left(x\right)\) digunakan \(\int_{w}^u p\left(x\right)dx\). Hasil integrasi kurva Persamaan \eqref{eq:simp} disajikan pada Persamaan \eqref{eq:simp2}.

\begin{equation}
\int_w^up\left(x\right)dx=\frac{h}{3}\left(f\left(u\right)+4f\left(v\right)+f\left(w\right)\right)
  \label{eq:simp2}
\end{equation}

Sekarang asumsikan \(n\) merupakan bilangan genap, maka kita perlu menambahkan taksiran untuk subinterval \(\left[x_{2i},x_{2i+2}\right]\) untuk memperoleh taksiran \(S\) pada integral \(\int_{a}^b f\left(x\right)dx\) yang disajikan pada Persamaan \eqref{eq:simp3}.

\begin{equation}
S\approx\frac{h}{3}\left(f_0+4\sum_{i=1,3,5.\dots}^{n-1}f_i+2\sum_{i=2,4,6,\dots}^{n-2}f_i+f_n\right)
  \label{eq:simp3}
\end{equation}

Persamaan \eqref{eq:simp3} disebut sebagai kaidah Simpson 1/3 karena terdapat koefisien 1/3 pada bagian depan persamaan tersebut. Persamaan tersebut juga mudah diingat mengingat pola koefisien persamaan tersebut adalah \(1,4,2,4,2,\dots ,2,4,1\). Namun penggunaan kaidah 1/3 Simpson mengharuskan jumlah subinterval \(n\) genap. Kondisi tersebut jelas berbeda dengan metode trapezoida yang tidak mensyaratkan jumlah selang.

Error dari metode Simpson 1/3 dapat dihitung menggunakan Persamaan \eqref{eq:simp3}.

\begin{equation}
\int_a^bh\left(x\right)dx=-\ \frac{\left(b-a\right)^5}{180m^4}f^{\left(4\right)}\left(\xi\right)
  \label{eq:simp4}
\end{equation}

dimana \(\xi\) merupakan nilai antara \(a\) dan \(b\).

\begin{center}\rule{0.5\linewidth}{0.5pt}\end{center}

\textbf{Algoritma Metode Simpson}

\begin{enumerate}
\def\labelenumi{\arabic{enumi}.}
\tightlist
\item
  Tentukan fungsi \(f\left(x\right)\) dan selang integrasinya \(\left[a,b\right]\).
\item
  Tentukan jumlah subinterval \(n\).
\item
  Hitung nilai selang subinterval \(h\), \(h=\frac{b-a}{n}\).
\item
  Tentukan awal integrasi \(x_0=a\) dan akhir \(x_n=b\) dan hitung nilai \(f\left(a\right)\) dan \(f\left(b\right)\).
\item
  Untuk \(x=1,2,\dots,n-1\),
\end{enumerate}

\begin{itemize}
\tightlist
\item
  jika ganjil, hitung: \(4\times f\left(x\right)\)
\item
  jika genap, hitung: \(2\times f\left(x\right)\)
\end{itemize}

\begin{enumerate}
\def\labelenumi{\arabic{enumi}.}
\setcounter{enumi}{5}
\tightlist
\item
  Jumlahkan nilai-nilai taksiran tersebut menggunakan Persamaan \eqref{eq:simp3}.
\end{enumerate}

\begin{center}\rule{0.5\linewidth}{0.5pt}\end{center}

Berdasarkan algoritma tersebut, kita dapat membentuk fungsi \texttt{simpson()} yang dapat digunakan untuk melakukan integrasi menggunakan metode Simpson 1/3. Berikut adalah sintaks yang digunakan:

\begin{Shaded}
\begin{Highlighting}[]
\NormalTok{simpson }\OtherTok{\textless{}{-}} \ControlFlowTok{function}\NormalTok{(f, a, b, }\AttributeTok{m=}\DecValTok{100}\NormalTok{)\{}
\NormalTok{  h }\OtherTok{\textless{}{-}}\NormalTok{ (b}\SpecialCharTok{{-}}\NormalTok{a)}\SpecialCharTok{/}\NormalTok{m }\CommentTok{\# jarak selang}
\NormalTok{  x }\OtherTok{\textless{}{-}}\NormalTok{ a }\CommentTok{\# awal selang}
\NormalTok{  I }\OtherTok{\textless{}{-}} \FunctionTok{f}\NormalTok{(a)}\SpecialCharTok{+}\FunctionTok{f}\NormalTok{(b)}
\NormalTok{  sigma }\OtherTok{\textless{}{-}} \DecValTok{0}
  
  \ControlFlowTok{if}\NormalTok{(m}\SpecialCharTok{\%\%}\DecValTok{2} \SpecialCharTok{!=} \DecValTok{0}\NormalTok{)\{}
    \FunctionTok{stop}\NormalTok{(}\StringTok{"Jumlah panel harus genap"}\NormalTok{)}
\NormalTok{  \}}\ControlFlowTok{else}\NormalTok{\{}
    \ControlFlowTok{for}\NormalTok{(i }\ControlFlowTok{in} \DecValTok{1}\SpecialCharTok{:}\NormalTok{(m}\DecValTok{{-}1}\NormalTok{))\{}
\NormalTok{    x }\OtherTok{\textless{}{-}}\NormalTok{ x}\SpecialCharTok{+}\NormalTok{h}
    \ControlFlowTok{if}\NormalTok{(i}\SpecialCharTok{\%\%}\DecValTok{2}\SpecialCharTok{==}\DecValTok{0}\NormalTok{)\{}
\NormalTok{      sigma }\OtherTok{\textless{}{-}}\NormalTok{ sigma }\SpecialCharTok{+} \DecValTok{2}\SpecialCharTok{*}\FunctionTok{f}\NormalTok{(x)}
\NormalTok{    \}}\ControlFlowTok{else}\NormalTok{\{}
\NormalTok{      sigma }\OtherTok{\textless{}{-}}\NormalTok{ sigma }\SpecialCharTok{+} \DecValTok{4}\SpecialCharTok{*}\FunctionTok{f}\NormalTok{(x)}
\NormalTok{      \}}
\NormalTok{    \}}
\NormalTok{  \}}
  
  \FunctionTok{return}\NormalTok{((h}\SpecialCharTok{/}\DecValTok{3}\NormalTok{)}\SpecialCharTok{*}\NormalTok{(I}\SpecialCharTok{+}\NormalTok{sigma))}
\NormalTok{\}}
\end{Highlighting}
\end{Shaded}

\begin{example}
\protect\hypertarget{exm:simpexm}{}\label{exm:simpexm}Hitung integral persamaan di bawah ini dengan menggunakan jumlah panel m=8!
\end{example}

\[
\int_{0}^1 \frac{1}{1+x}dx
\]

\textbf{Jawab}:

lebar selang \(h\) dapat dihitung seperti berikut:

\[
h = \frac{1-0}{8}=0,125
\]

Integral persamaan tersebut selanjutnya dapat dihitung menggunakan Persamaan \eqref{eq:simp3}:

\[
S\approx\frac{0,125}{3}\left(f\left(0\right)+4f\left(0,125\right)+2f\left(0,25\right)\dots+f\left(1\right)\right)\approx 0,69412
\]

Fungsi \texttt{simpson()} juga menghasilkan nilai yang serupa dengan perhitungan manual yang telah dilakukan. Berikut adalah sintaks yang digunakan:

\begin{Shaded}
\begin{Highlighting}[]
\FunctionTok{simpson}\NormalTok{(}\ControlFlowTok{function}\NormalTok{(x)}\DecValTok{1}\SpecialCharTok{/}\NormalTok{(}\DecValTok{1}\SpecialCharTok{+}\NormalTok{x), }\AttributeTok{a=}\DecValTok{0}\NormalTok{, }\AttributeTok{b=}\DecValTok{1}\NormalTok{, }\AttributeTok{m=}\DecValTok{8}\NormalTok{)}
\end{Highlighting}
\end{Shaded}

\begin{verbatim}
## [1] 0.6932
\end{verbatim}

\hypertarget{simpson38}{%
\subsection{Metode Simpson 3/8}\label{simpson38}}

Jika pada metode Simpson 1/3 digunakan pendekatan polinomial berderajat 2 untuk mencari luas dibawah kurva, pada metode Simpson 3/8 digunakan pendekatan polinomial berderajat 3 untuk memperoleh hasil yang lebih baik. Bentuk umum integrasi yang digunakan disajikan pada Persamaan \eqref{eq:simp38}.

\begin{equation}
\int_a^bf\left(x\right)dx\approx\frac{3h}{8}\left(f_0+3\sum_{i=1;i\ne3,6,9,\dots}^{n-1}f_i+2\sum_{i=3,6,9,\dots}^{n-3}f_i+f_n\right)
  \label{eq:simp38}
\end{equation}

Error dari metode ini dapat diestimasi menggunakan Persamaan \eqref{eq:simp382}.

\begin{equation}
\int_a^bh\left(x\right)dx=-\ \frac{\left(b-a\right)^5}{80m^4}f^{\left(5\right)}\left(\xi\right)
  \label{eq:simp382}
\end{equation}

\begin{center}\rule{0.5\linewidth}{0.5pt}\end{center}

\textbf{Algoritma Metode Simpson 3/8}

\begin{enumerate}
\def\labelenumi{\arabic{enumi}.}
\tightlist
\item
  Tentukan fungsi \(f\left(x\right)\) dan selang integrasinya \(\left[a,b\right]\).
\item
  Tentukan jumlah subinterval \(n\).
\item
  Hitung nilai selang subinterval \(h\), \(h=\frac{b-a}{n}\).
\item
  Tentukan awal integrasi \(x_0=a\) dan akhir \(x_n=b\) dan hitung nilai \(f\left(a\right)\) dan \(f\left(b\right)\).
\item
  Untuk \(x=1,2,\dots,n-1\),
\end{enumerate}

\begin{itemize}
\tightlist
\item
  jika bukan kelipatan 3, hitung: \(3\times f\left(x\right)\)
\item
  jika kelipatan 3, hitung: \(2\times f\left(x\right)\)
\end{itemize}

\begin{enumerate}
\def\labelenumi{\arabic{enumi}.}
\setcounter{enumi}{5}
\tightlist
\item
  Jumlahkan nilai-nilai taksiran tersebut menggunakan Persamaan \eqref{eq:simp38}.
\end{enumerate}

\begin{center}\rule{0.5\linewidth}{0.5pt}\end{center}

Berdasarkan algoritma tersebut, fungsi \texttt{R} dapat disusun untuk melakukan komputasi metode simpson 3/8. Berikut adalah sintaks fungsi tersebut:

\begin{Shaded}
\begin{Highlighting}[]
\NormalTok{simpson38 }\OtherTok{\textless{}{-}} \ControlFlowTok{function}\NormalTok{(f, a, b, }\AttributeTok{m=}\DecValTok{90}\NormalTok{)\{}
\NormalTok{  h }\OtherTok{\textless{}{-}}\NormalTok{ (b}\SpecialCharTok{{-}}\NormalTok{a)}\SpecialCharTok{/}\NormalTok{m }\CommentTok{\# jarak selang}
\NormalTok{  x }\OtherTok{\textless{}{-}}\NormalTok{ a }\CommentTok{\# awal selang}
\NormalTok{  I }\OtherTok{\textless{}{-}} \FunctionTok{f}\NormalTok{(a)}\SpecialCharTok{+}\FunctionTok{f}\NormalTok{(b)}
\NormalTok{  sigma }\OtherTok{\textless{}{-}} \DecValTok{0}
  
  \ControlFlowTok{if}\NormalTok{(m}\SpecialCharTok{\%\%}\DecValTok{3} \SpecialCharTok{!=} \DecValTok{0}\NormalTok{)\{}
    \FunctionTok{stop}\NormalTok{(}\StringTok{"jumlah panel harus kelipatan 3"}\NormalTok{)}
\NormalTok{  \}}\ControlFlowTok{else}\NormalTok{\{}
    \ControlFlowTok{for}\NormalTok{(i }\ControlFlowTok{in} \DecValTok{1}\SpecialCharTok{:}\NormalTok{(m}\DecValTok{{-}1}\NormalTok{))\{}
\NormalTok{      x }\OtherTok{\textless{}{-}}\NormalTok{ x}\SpecialCharTok{+}\NormalTok{h}
      \ControlFlowTok{if}\NormalTok{(i}\SpecialCharTok{\%\%}\DecValTok{3}\SpecialCharTok{==}\DecValTok{0}\NormalTok{)\{}
\NormalTok{      sigma }\OtherTok{\textless{}{-}}\NormalTok{ sigma }\SpecialCharTok{+} \DecValTok{2}\SpecialCharTok{*}\FunctionTok{f}\NormalTok{(x)}
\NormalTok{      \}}\ControlFlowTok{else}\NormalTok{\{}
\NormalTok{      sigma }\OtherTok{\textless{}{-}}\NormalTok{ sigma }\SpecialCharTok{+} \DecValTok{3}\SpecialCharTok{*}\FunctionTok{f}\NormalTok{(x)}
\NormalTok{      \}}
\NormalTok{    \}}
\NormalTok{  \}}
  
  \FunctionTok{return}\NormalTok{((}\DecValTok{3}\SpecialCharTok{*}\NormalTok{h}\SpecialCharTok{/}\DecValTok{8}\NormalTok{)}\SpecialCharTok{*}\NormalTok{(I}\SpecialCharTok{+}\NormalTok{sigma))}
\NormalTok{\}}
\end{Highlighting}
\end{Shaded}

\begin{example}
\protect\hypertarget{exm:simp38exm}{}\label{exm:simp38exm}Hitung kembali integral persamaan yang disajikan pada Contoh \ref{exm:simpexm} menggunakan fungsi simpson38() yang telah dibuat sebelumnya!
\end{example}

\textbf{Jawab}:

Berikut adalah sintaks yang digunakan untuk melakukan proses integrasi Simpson 3/8 menggunakan fungsi \texttt{simpson38()}:

\begin{Shaded}
\begin{Highlighting}[]
\FunctionTok{simpson38}\NormalTok{(}\ControlFlowTok{function}\NormalTok{(x)}\DecValTok{1}\SpecialCharTok{/}\NormalTok{(}\DecValTok{1}\SpecialCharTok{+}\NormalTok{x), }\AttributeTok{a=}\DecValTok{0}\NormalTok{, }\AttributeTok{b=}\DecValTok{1}\NormalTok{, }\AttributeTok{m=}\DecValTok{9}\NormalTok{)}
\end{Highlighting}
\end{Shaded}

\begin{verbatim}
## [1] 0.6932
\end{verbatim}

\hypertarget{metode-integrasi-newton-cotes-mengunakan-fungsi-lainnya}{%
\section{Metode Integrasi Newton-Cotes Mengunakan Fungsi Lainnya}\label{metode-integrasi-newton-cotes-mengunakan-fungsi-lainnya}}

Terdapat sejumlah Paket yang menyediakan fungsi yang dapat digunakan untuk melakukan proses integrasi suatu fungsi pada \texttt{R}. Paket \texttt{pracma} menyediakan fungsi-fungsi seperti \texttt{trapz()} dan \texttt{cotes()}.

Fungsi \texttt{trapz()} merupakan fungsi yang digunakan untuk melakukan integrasi dengan pendekatan trapesium. Penggunaan fungsi tersebut untuk melakukan perhitungan integral tidak dapat dilakukan secara langsung, kita perlu membuat terlebih dahulu seri koordinat \texttt{x} dan koordinat \texttt{y}. Program selanjutnya akan melakukan interpolasi linier terhadap selang yang saling berdekatan. Luas masing-masing panel selang selanjutnya dihitung dan dijumlahkan untuk memperoleh nilai integral. Format fungsi tersebut adalah sebagai berikut:

\begin{Shaded}
\begin{Highlighting}[]
\FunctionTok{trapz}\NormalTok{(x, y)}
\end{Highlighting}
\end{Shaded}

\begin{quote}
\textbf{Catatan:}

\begin{itemize}
\tightlist
\item
  \textbf{x}: vektor sumbu x.
\item
  \textbf{y}: vektor sumbu y.
\end{itemize}
\end{quote}

Untuk lebih memahami penerapannya berikut disajikan contoh untuk mencari integral persamaan pada Contoh \ref{exm:reimannexm}:

\begin{Shaded}
\begin{Highlighting}[]
\FunctionTok{library}\NormalTok{(pracma)}

\NormalTok{f }\OtherTok{\textless{}{-}} \ControlFlowTok{function}\NormalTok{(x)x}\SpecialCharTok{\^{}}\DecValTok{2}
\NormalTok{x }\OtherTok{\textless{}{-}} \FunctionTok{seq}\NormalTok{(}\DecValTok{0}\NormalTok{, }\DecValTok{1}\NormalTok{, }\AttributeTok{length.out =} \DecValTok{101}\NormalTok{) }\CommentTok{\# membuat subinterval panel sebanyak 100}
\NormalTok{y }\OtherTok{\textless{}{-}} \FunctionTok{f}\NormalTok{(x)}

\FunctionTok{trapz}\NormalTok{(x, y)}
\end{Highlighting}
\end{Shaded}

\begin{verbatim}
## [1] 0.3333
\end{verbatim}

Fungsi lain yang dapat digunakan untuk menghitung integral suatu fungsi menggunakan metode Newton-Cotes adalah \texttt{cotes()}. Pada fungsi tersebut kita perlu menyatakan jumlah subinterval yang digunakan dan jumlah nodes interpolasi yang digunakan. Jumlah nodes akan menentukan fungsi polinomial pendekatan yang digunakan untuk menghitung luas di bawah suatu fungsi. Jika jumlah nodes diatur menjadi dua, maka fungsi pendekatannya berupa garis linier (metode trapezoida). Jika nodes diatur menjadi 3 maka pendekatannya berupa fungsi kuadrat (metode Simpson 1/3). Secara sederhana derajat polinomial \(n\) memerlukan \(n+1\) nodes. Format fungsi \texttt{cotes()} disajikan sebagai berikut:

\begin{Shaded}
\begin{Highlighting}[]
\FunctionTok{cotes}\NormalTok{(f, a, b, n, nodes, ...)}
\end{Highlighting}
\end{Shaded}

\begin{quote}
\textbf{Catatan:}

\begin{itemize}
\tightlist
\item
  \textbf{f}: fungsi yang akan dicari integralnya
\item
  \textbf{a}: batas atas.
\item
  \textbf{b}: batas bawah.
\item
  \textbf{n}: jumlah subinterval atau panel.
\item
  \textbf{nodes}: jumlah nodes yang digunakan untuk interpolasi fungsi pada tiap subinterval.
\end{itemize}
\end{quote}

Untuk lebih memahami penerapannya berikut adalah contoh perhitungan intergral menggunakan persamaan pada Contoh \ref{exm:simpexm}.

\begin{Shaded}
\begin{Highlighting}[]
\NormalTok{f }\OtherTok{\textless{}{-}} \ControlFlowTok{function}\NormalTok{(x)}\DecValTok{1}\SpecialCharTok{/}\NormalTok{(}\DecValTok{1}\SpecialCharTok{+}\NormalTok{x)}
\NormalTok{a }\OtherTok{\textless{}{-}} \DecValTok{0}\NormalTok{; b }\OtherTok{\textless{}{-}} \DecValTok{1}

\CommentTok{\# metode trapezoida}
\FunctionTok{cotes}\NormalTok{(f, a, b, }\AttributeTok{n=}\DecValTok{8}\NormalTok{, }\AttributeTok{nodes=}\DecValTok{2}\NormalTok{)}
\end{Highlighting}
\end{Shaded}

\begin{verbatim}
## [1] 0.6941
\end{verbatim}

\begin{Shaded}
\begin{Highlighting}[]
\CommentTok{\# metode Simpson 1/3}
\FunctionTok{cotes}\NormalTok{(f, a, b, }\AttributeTok{n=}\DecValTok{8}\NormalTok{, }\AttributeTok{nodes=}\DecValTok{3}\NormalTok{)}
\end{Highlighting}
\end{Shaded}

\begin{verbatim}
## [1] 0.6932
\end{verbatim}

\begin{Shaded}
\begin{Highlighting}[]
\CommentTok{\# metode Simpson 3/8}
\FunctionTok{cotes}\NormalTok{(f, a, b, }\AttributeTok{n=}\DecValTok{9}\NormalTok{, }\AttributeTok{nodes=}\DecValTok{4}\NormalTok{)}
\end{Highlighting}
\end{Shaded}

\begin{verbatim}
## [1] 0.6932
\end{verbatim}

\hypertarget{kuadgauss}{%
\section{Metode Kuadratur Gauss}\label{kuadgauss}}

Metode Newton-Cotes sangat \emph{powerful}, tetapi metode tersebut memiliki dua fitur yang kurang diinginkan. Pertama, kita harus menggunakan evaluasi fungsi sejumlah \(n +1\) untuk hasil presisi dan pada polinomial berderajat \(n\). Hal tersebut mungkin tampak seperti rasio yang baik, tetapi dalam praktiknya, jumlah titik evaluasi akan sering ditingkatkan untuk memperoleh akurasi yang lebih tinggi, namun hasil yang diperoleh juga tidak presisi. Sebagai contoh pembaca dapat melakukan simulasi dengan memvariasikan jumlah penel yang digunakan untuk memperoleh nilai integral sebuah fungsi. Jika kita plotkan hasil yang kita peroleh, grafik yang muncul berupa garis yang berosilasi khusunya pada penggunaan polinomial berderajat tinggi yang menunjukkan hasil yang diperoleh menjadi kurang presisi.

Kelemahan kedua, metode Newton-Cotes memerlukan fungsi terintegrasi untuk dievaluasi pada node yang berjarak sama. Ini benar terlepas dari fungsi yang digunakan. Setiap panel membutuhkan node dengan jarak yang sama di dalamnya. Ini bisa menjadi masalah dengan fungsi periodik, di mana diskontinuitas periodik dapat secara kebetulan mendarat di titik evaluasi. Jika panel Newton-Cotes dapat menyebabkan masalah, kita dapat menggunakan integrasi Gaussian untuk menyelesaikan integral.

Secara umum integrasi Gauss berusaha memperoleh pendekatan luas dibawah kurva fungsi dengan memecah fungsi tersebut menjadi faktor bobot \(c\) dan \(f\left(x\right)\) yang merupakan polinomial pendekatannya. Integral diperoleh melalui hasil kali dari bobot dan fungsi polinomial. Jumlah bobot dan fungsi yang digunakan bergantung pada orde \(n\) polinomial yang akan digunakan untuk mengestimasi integral suatu fungsi. Bentuk umum dari kaidah Gauss tersebut ditampilkan pada Persamaan \eqref{eq:gauss}.

\begin{equation}
\int_{-1}^1f\left(x\right)dt\approx c_1f\left(x_1\right)+c_2f\left(x_2\right)+\dots+c_nf\left(x_n\right)
  \label{eq:gauss}
\end{equation}

dimana \(c\) merupakan faktor bobot dan \(x\) merupakan titik evaluasi.

Nilai faktor bobot dan nilai masing-masing titik evaluasi disajikan pada Gambar \ref{fig:gaussviz}.

\begin{figure}

{\centering \includegraphics[width=0.9\linewidth]{./images/gaussviz} 

}

\caption{Tabulasi faktor bobot, titik evaluasi dan galat pemotongan.}\label{fig:gaussviz}
\end{figure}

Fungsi yang akan dicari nilai integrannya pada umumnya tidak hanya memiliki daerah batas \(\left[-1,1\right]\), sehingga pendekatan kuadratur Gauss tidak dapat digunakan secara langsung pada fungsi yang tidak memiliki batas tersebut. Agar kuadratur Gauss tetap dapat digunakan, fungsi tersebut perlu dilakukan transformasi. Proses transformasi dituliskan pada Persamaan \eqref{eq:gauss2}.

\begin{equation}
\int_a^bf\left(x\right)dx=\frac{b-a}{2}\int_{-1}^1f\left(\frac{a+b+\left(b-a\right)t}{2}\right)dt
  \label{eq:gauss2}
\end{equation}

\begin{center}\rule{0.5\linewidth}{0.5pt}\end{center}

\textbf{Algoritma Kuadratur Gauss-Legendre}

\begin{enumerate}
\def\labelenumi{\arabic{enumi}.}
\tightlist
\item
  Tentukan fungsi \(f\left(x\right)\) dan selang integrasinya \(\left[a,b\right]\).
\item
  Lakukan transformasi fungsi tersebut hingga diperoleh fungsi dengan selang \(\left[-1,1\right]\) menggunakan Persamaan \eqref{eq:gauss2}.
\item
  Tentukan orde polinomial \(n\) yang akan digunakan.
\item
  Lakukan proses integrasi dengan mengalikan faktor bobot \(c\) dengan \(f\left(x_i\right)\) seperti yang ditujukkan pada Persamaan \eqref{eq:gauss}.
\end{enumerate}

\begin{center}\rule{0.5\linewidth}{0.5pt}\end{center}

Kita dapat membangun suatu fungsi pada \texttt{R} untuk melakukan integrasi Gauss berdasarkan algoritma tersebut. Sintaks fungsi tersebut adalah sebagai berikut:

\begin{Shaded}
\begin{Highlighting}[]
\NormalTok{gauss\_legendre }\OtherTok{\textless{}{-}} \ControlFlowTok{function}\NormalTok{(f, n)\{}
    \ControlFlowTok{if}\NormalTok{(n}\SpecialCharTok{==}\DecValTok{2}\NormalTok{)\{}
\NormalTok{      c }\OtherTok{\textless{}{-}} \FunctionTok{c}\NormalTok{(}\DecValTok{1}\NormalTok{,}\DecValTok{1}\NormalTok{)}
\NormalTok{      x }\OtherTok{\textless{}{-}} \FunctionTok{c}\NormalTok{(}\SpecialCharTok{{-}}\FloatTok{0.577350269}\NormalTok{, }\FloatTok{0.577350269}\NormalTok{)}
\NormalTok{      integral }\OtherTok{\textless{}{-}} \DecValTok{0}
      
      \ControlFlowTok{for}\NormalTok{(i }\ControlFlowTok{in} \DecValTok{1}\SpecialCharTok{:}\NormalTok{n)\{}
\NormalTok{        integral }\OtherTok{\textless{}{-}}\NormalTok{ integral }\SpecialCharTok{+}\NormalTok{ c[i]}\SpecialCharTok{*}\FunctionTok{f}\NormalTok{(x[i])}
\NormalTok{      \}}
\NormalTok{    \}}\ControlFlowTok{else} \ControlFlowTok{if}\NormalTok{(n}\SpecialCharTok{==}\DecValTok{3}\NormalTok{)\{}
\NormalTok{      c }\OtherTok{\textless{}{-}} \FunctionTok{c}\NormalTok{(}\FloatTok{0.555555556}\NormalTok{,}\FloatTok{0.888888889}\NormalTok{,}\FloatTok{0.555555556}\NormalTok{)}
\NormalTok{      x }\OtherTok{\textless{}{-}} \FunctionTok{c}\NormalTok{(}\SpecialCharTok{{-}}\FloatTok{0.774596669}\NormalTok{,}\DecValTok{0}\NormalTok{,}\FloatTok{0.774596669}\NormalTok{)}
\NormalTok{      integral }\OtherTok{\textless{}{-}} \DecValTok{0}
      
      \ControlFlowTok{for}\NormalTok{(i }\ControlFlowTok{in} \DecValTok{1}\SpecialCharTok{:}\NormalTok{n)\{}
\NormalTok{        integral }\OtherTok{\textless{}{-}}\NormalTok{ integral }\SpecialCharTok{+}\NormalTok{ c[i]}\SpecialCharTok{*}\FunctionTok{f}\NormalTok{(x[i])}
\NormalTok{      \}}
\NormalTok{    \}}\ControlFlowTok{else} \ControlFlowTok{if}\NormalTok{(n}\SpecialCharTok{==}\DecValTok{4}\NormalTok{)\{}
\NormalTok{      c }\OtherTok{\textless{}{-}} \FunctionTok{c}\NormalTok{(}\FloatTok{0.347854845}\NormalTok{,}\FloatTok{0.652145155}\NormalTok{,}\FloatTok{0.652145155}\NormalTok{,}
             \FloatTok{0.347854845}\NormalTok{)}
\NormalTok{      x }\OtherTok{\textless{}{-}} \FunctionTok{c}\NormalTok{(}\SpecialCharTok{{-}}\FloatTok{0.861136312}\NormalTok{,}\SpecialCharTok{{-}}\FloatTok{0.339981044}\NormalTok{,}\FloatTok{0.339981044}\NormalTok{,}
             \FloatTok{0.861136312}\NormalTok{)}
\NormalTok{      integral }\OtherTok{\textless{}{-}} \DecValTok{0}
      
      \ControlFlowTok{for}\NormalTok{(i }\ControlFlowTok{in} \DecValTok{1}\SpecialCharTok{:}\NormalTok{n)\{}
\NormalTok{        integral }\OtherTok{\textless{}{-}}\NormalTok{ integral }\SpecialCharTok{+}\NormalTok{ c[i]}\SpecialCharTok{*}\FunctionTok{f}\NormalTok{(x[i])}
\NormalTok{      \}}
\NormalTok{    \}}\ControlFlowTok{else} \ControlFlowTok{if}\NormalTok{(n}\SpecialCharTok{==}\DecValTok{5}\NormalTok{)\{}
\NormalTok{      c }\OtherTok{\textless{}{-}} \FunctionTok{c}\NormalTok{(}\FloatTok{0.236926885}\NormalTok{,}\FloatTok{0.478628670}\NormalTok{,}\FloatTok{0.568888889}\NormalTok{,}
              \FloatTok{0.478628670}\NormalTok{,}\FloatTok{0.236926885}\NormalTok{)}
\NormalTok{      x }\OtherTok{\textless{}{-}} \FunctionTok{c}\NormalTok{(}\SpecialCharTok{{-}}\FloatTok{0.906179846}\NormalTok{,}\SpecialCharTok{{-}}\FloatTok{0.538469310}\NormalTok{,}\DecValTok{0}\NormalTok{,}
             \FloatTok{0.538469310}\NormalTok{, }\FloatTok{0.906179846}\NormalTok{)}
\NormalTok{      integral }\OtherTok{\textless{}{-}} \DecValTok{0}
      
      \ControlFlowTok{for}\NormalTok{(i }\ControlFlowTok{in} \DecValTok{1}\SpecialCharTok{:}\NormalTok{n)\{}
\NormalTok{        integral }\OtherTok{\textless{}{-}}\NormalTok{ integral }\SpecialCharTok{+}\NormalTok{ c[i]}\SpecialCharTok{*}\FunctionTok{f}\NormalTok{(x[i])}
\NormalTok{      \}}
\NormalTok{    \}}\ControlFlowTok{else} \ControlFlowTok{if}\NormalTok{(n}\SpecialCharTok{==}\DecValTok{6}\NormalTok{)\{}
\NormalTok{      c }\OtherTok{\textless{}{-}} \FunctionTok{c}\NormalTok{(}\FloatTok{0.171324492}\NormalTok{,}\FloatTok{0.360761573}\NormalTok{,}\FloatTok{0.467913935}\NormalTok{,}
              \FloatTok{0.467913935}\NormalTok{,}\FloatTok{0.360761573}\NormalTok{,}\FloatTok{0.171324492}\NormalTok{)}
\NormalTok{      x }\OtherTok{\textless{}{-}} \FunctionTok{c}\NormalTok{(}\SpecialCharTok{{-}}\FloatTok{0.932469514}\NormalTok{,}\SpecialCharTok{{-}}\FloatTok{0.661209386}\NormalTok{,}\SpecialCharTok{{-}}\FloatTok{0.238619186}\NormalTok{,}
             \FloatTok{0.238619186}\NormalTok{, }\FloatTok{0.661209386}\NormalTok{,}\FloatTok{0.932469514}\NormalTok{)}
\NormalTok{      integral }\OtherTok{\textless{}{-}} \DecValTok{0}
      
      \ControlFlowTok{for}\NormalTok{(i }\ControlFlowTok{in} \DecValTok{1}\SpecialCharTok{:}\NormalTok{n)\{}
\NormalTok{        integral }\OtherTok{\textless{}{-}}\NormalTok{ integral }\SpecialCharTok{+}\NormalTok{ c[i]}\SpecialCharTok{*}\FunctionTok{f}\NormalTok{(x[i])}
\NormalTok{      \}}
\NormalTok{    \}}\ControlFlowTok{else}\NormalTok{\{}
      \FunctionTok{stop}\NormalTok{(}\StringTok{"n harus ditentukan"}\NormalTok{)}
\NormalTok{    \}}
    \FunctionTok{return}\NormalTok{(integral)}
\NormalTok{\}}
\end{Highlighting}
\end{Shaded}

\begin{example}
\protect\hypertarget{exm:gausslanexm}{}\label{exm:gausslanexm}Hitunglah integral fungsi berikut menggunakan metode Gauss-Legendre 2 titik!
\end{example}

\[
\int_1^2 \left(x^2+1\right)dx
\]

\textbf{Jawab}:

Agar fungsi tersebut dapat dicari nilai integralnya menggunakan metode Gauss-Legendre, fungsi tersebut perlu ditransformasi terlebih dahulu:

\[
\int_1^2 \left(x^2+1\right)dx=0,5\int_{-1}^1 \left[\left(1,5+0,5t\right)^2+1\right]dt
\]

Jadi dalam hal ini

\[
f\left(t\right)=\left(1,5+0,5t\right)^2+1
\]

maka

\[
\int_1^2 \left(x^2+1\right)dx=0,5\left[1\times f\left(\frac{1}{\sqrt{3}}\right)+1\times f\left(\frac{1}{\sqrt{3}}\right)\right]\approx 3,333333333
\]

Kita juga dapat menggunakan fungsi \texttt{gauss\_legendre()} untuk melakukan integrasi Gauss-Legendre pada dua titik. Berikut adalah sintaks yang digunakan:

\begin{Shaded}
\begin{Highlighting}[]
\FloatTok{0.5}\SpecialCharTok{*}\FunctionTok{gauss\_legendre}\NormalTok{(}\ControlFlowTok{function}\NormalTok{(x)((}\FloatTok{1.5+0.5}\SpecialCharTok{*}\NormalTok{x)}\SpecialCharTok{\^{}}\DecValTok{2}\NormalTok{)}\SpecialCharTok{+}\DecValTok{1}\NormalTok{,}
               \AttributeTok{n=}\DecValTok{2}\NormalTok{)}
\end{Highlighting}
\end{Shaded}

\begin{verbatim}
## [1] 3.333
\end{verbatim}

\hypertarget{othergauss}{%
\section{\texorpdfstring{Metode Gauss-Legendre Menggunakan Fungsi \texttt{legendre.quadrature()}}{Metode Gauss-Legendre Menggunakan Fungsi legendre.quadrature()}}\label{othergauss}}

Fungsi \texttt{legendre.quadrature()} dari Paket \texttt{gaussquand} dapat dijadikan alternatif untuk menghitung integral suatu fungsi menggunakan metode Gauss-Lagendre. Fortmat fungsi tersebut adalah sebagai berikut:

\begin{Shaded}
\begin{Highlighting}[]
\FunctionTok{legendre.quadrature}\NormalTok{(functn, rule, }\AttributeTok{lower=}\SpecialCharTok{{-}}\DecValTok{1}\NormalTok{, }\AttributeTok{upper=}\DecValTok{1}\NormalTok{, }
  \AttributeTok{weighted =} \ConstantTok{TRUE}\NormalTok{, ...)}
\end{Highlighting}
\end{Shaded}

\begin{quote}
\textbf{Catatan:}

\begin{itemize}
\tightlist
\item
  \textbf{functn}: fungsi yang akan dicari integralnya
\item
  \textbf{rule}: data frame yang terdiri atas orde n aturan kuadratur legendre
\item
  \textbf{lower}: batas atas.
\item
  \textbf{upper}: batas bawah.
\item
  \textbf{weighted}: nilai boolean yang menyatakan apakah bobot fungsi disertakan dalam integran
\item
  \textbf{\ldots{}}: argumen tambahan functn
\end{itemize}
\end{quote}

Untuk menentukan \texttt{rule} pada fungsi \texttt{legendre.quadrature()}, diperlukan fungsi lain yang tersedia pada Paket \texttt{gaussquad}. Fungsi tersebut adalah \texttt{legendre.quadrature.rules()}. Fungsi tersebut akan menampilkan list data frame berdasarkan orde polinomial yang digunakan sebagai taksiran yang terdiri atas faktor bobot dan titik evaluasi. Format fungsi tersebut adalah sebagai berikut:

\begin{Shaded}
\begin{Highlighting}[]
\FunctionTok{legendre.quadrature.rules}\NormalTok{(n,}\AttributeTok{normalized=}\ConstantTok{FALSE}\NormalTok{)}
\end{Highlighting}
\end{Shaded}

\begin{quote}
\textbf{Catatan:}

\begin{itemize}
\tightlist
\item
  \textbf{n}: orde tertinggi polinomial yang akan ditampilkan
\item
  \textbf{normalized}: nilai boolean. Jika bernilai \texttt{TRUE}, aturan digunakan untuk polinomial ortogonal.
\end{itemize}
\end{quote}

Untuk memahami penerapan kedua fungsi tersebut berikut disajikan contoh penerapannya:

\begin{Shaded}
\begin{Highlighting}[]
\FunctionTok{library}\NormalTok{(gaussquad)}
\CommentTok{\# menampilkan aturan untuk orde 4}
\FunctionTok{legendre.quadrature.rules}\NormalTok{(}\DecValTok{4}\NormalTok{)}
\end{Highlighting}
\end{Shaded}

\begin{verbatim}
## [[1]]
##   x w
## 1 0 2
## 
## [[2]]
##         x w
## 1  0.5774 1
## 2 -0.5774 1
## 
## [[3]]
##            x      w
## 1  7.746e-01 0.5556
## 2  7.772e-16 0.8889
## 3 -7.746e-01 0.5556
## 
## [[4]]
##         x      w
## 1  0.8611 0.3479
## 2  0.3400 0.6521
## 3 -0.3400 0.6521
## 4 -0.8611 0.3479
\end{verbatim}

\begin{Shaded}
\begin{Highlighting}[]
\CommentTok{\# mencari integral suatu fungsi dengan orde}
\CommentTok{\# gauss{-}legendre sebesar 4}
\NormalTok{f }\OtherTok{\textless{}{-}} \ControlFlowTok{function}\NormalTok{(x)x}\SpecialCharTok{\^{}}\DecValTok{6}
\NormalTok{rule }\OtherTok{\textless{}{-}} \FunctionTok{legendre.quadrature.rules}\NormalTok{(}\DecValTok{4}\NormalTok{)[[}\DecValTok{4}\NormalTok{]]}

\FunctionTok{legendre.quadrature}\NormalTok{(f, rule, }\AttributeTok{lower=}\SpecialCharTok{{-}}\DecValTok{1}\NormalTok{, }\AttributeTok{upper=}\DecValTok{1}\NormalTok{)}
\end{Highlighting}
\end{Shaded}

\begin{verbatim}
## [1] 0.2857
\end{verbatim}

\hypertarget{adaptifint}{%
\section{Metode Integrasi Adaptif}\label{adaptifint}}

Integrasi adaptif menyediakan pendekatan yang berbeda untuk memperoleh nilai intergral suatu fungsi. Salah satu prinsip utama dari analisis numerik adalah bahwa kita harus berkomitmen pada semacam analisis manusia terhadap suatu masalah sebelum mencoba menyelesaikannya secara algoritmik. Metode analisis numerik umumnya tidak dapat menyelesaikan semua masalah dengan sangat baik. Jadi pengetahuan terhadap masalah yang hendak diselesaikan dapat memungkinkan kita memilih metode numerik yang lebih baik sesuai dengan masalah. Misalnya, dalam konteks integrasi numerik, diskontinuitas pada titik akhir tidak akan cocok untuk solusi Newton-Cotes yang bersifat tertutup.

Tentu saja, akan lebih baik jika kita bisa memprogram komputer untuk mempelajari sesuatu tentang masalah, daripada kita melakukan itu sendiri. Metode adaptif memberikan pendekatan untuk melakukan hal ini. Metode integrasi adaptif memeriksa integral yang mereka operasikan dan mengubah parameter mereka sendiri untuk meningkatkan kualitas integrasi. Algoritma adaptif yang paling sederhana memberikan pendekatan \emph{brute force} untuk peningkatan kualitas dengan memeriksa \emph{error} integrasi. Di sisi lain, jika kita tahu \emph{error}-nya, secara teoritis kita bisa memperbaiki estimasi. Di situlah batas \emph{error} pada algoritma Newton-Cotes dapat membantu.

Jika kita dapat menemukan sesuatu tentang \emph{error} tersebut, kita dapat menggunakan informasi itu untuk memperbaiki estimasi pada proses integrasi. Bayangkan kita sedang mengintegrasikan suatu fungsi, \(f \left(x\right)\) pada batas \(\left[a, b\right]\). Jika kita menggunakan metode titik tengah (integral Riemann). Kita telah menetahui \emph{error} maksimum yang mungkin terjadi pada metode tersebut melalui Persamaan \eqref{eq:reimann2}. Dua pengamatan segera menjadi jelas. Terlepas dari apa fungsi \(f \left(x\right)\) itu, atau turunan keduanya, dua perubahan dapat dilakukan pada integrasi untuk meningkatkan kualitasnya. Pertama, \emph{error} adalah proporsi terhadap kubik dari panjang domain integrasi. Mengurangi panjang meningkatkan kualitas dan memotong panjang menjadi dua memberikan peningkatan kualitas delapan kali lipat. Kedua, \emph{error} berbanding terbalik dengan jumlah panel m yang digunakan . Meningkatkan panel mengurangi \emph{error} dan menggandakan jumlah panel yang disediakan dapat meeningkatkan kualitas empat kali lipat.

Kita dapat merancang algoritma di sekitar pengamatan ini. Pertama, kita dapat memperkirakan nilai integral \(Q_1\), menggunakan aturan titik tengah 1-point. Kedua, kita bisa memperkirakannya lagi menggunakan aturan titik tengah 2-point \(Q_2\). Karena kita telah menggandakan jumlah titik dalam aturan, kita sekarang tahu bahwa perbedaan maksimum antara \(Q_1\) dan \(Q\), nilai sebenarnya dari integral, tidak lebih dari 4 kali lebih besar dari perbedaan maksimum antara \(Q_2\) dan \(Q\). Jika \(Q_2 −Q\) kurang dari toleransi tertentu, maka perbedaan antara \(Q_1\) dan \(Q_2\) harus kurang dari tiga kali toleransi yang sama.

Kita peeriksa perbedaan antara dua perkiraan. Jika perbedaannya lebih besar dari toleransi, kita mungkin masih berada dalam toleransi, tetapi kita belum pasti. Jadi proses membagi wilayah integrasi menjadi dua, dan menerapkan integrator adaptif untuk kedua bagian, secara terpisah, menjumlahkan hasilnya akan terus dilakukan.

\begin{center}\rule{0.5\linewidth}{0.5pt}\end{center}

\textbf{Algoritma Metode Riemann Adaptif}

\begin{enumerate}
\def\labelenumi{\arabic{enumi}.}
\tightlist
\item
  Tentukan fungsi \(f\left(x\right)\) dan selang integrasinya \(\left[a,b\right]\).
\item
  Tentukan jumlah subinterval \(m\).
\item
  Jika \(m=1\), hitung luas area di bawah kurva dengan pendekatan metode Riemann dengan \(m=2\).
\item
  Jika \(n>1\),
\end{enumerate}

\begin{itemize}
\tightlist
\item
  Hitung \(Q_1\) dengan pendekatan metode Riemann dan \(m=1\)
\item
  Hitung \(Q_2\) dengan pendekatan metode Riemann dan \(m=2\)
\end{itemize}

\begin{enumerate}
\def\labelenumi{\arabic{enumi}.}
\setcounter{enumi}{4}
\tightlist
\item
  Jika \(Q_1-Q_2>3\times\text{nilai toleransi}\),
\end{enumerate}

\begin{itemize}
\tightlist
\item
  Perkecil \(m\) sebanyak 1, \(m-1\)
\item
  Perkecil \(\text{nilai toleransi}\) menjadi setengahnya
\item
  Bagi area integrasi menjadi dua bagian dengan menetapkan \(c\) sebagai batas, sehingga terdapat dua batas yaitu: \(\left[a, c\right]\) dan \(\left[c, b\right]\).
\item
  Lakukan perhitungan kembali integral pada masing-masing batas tersebut menggunakan metode Riemann dan cek apakah \(Q_1-Q_2>3\times\text{nilai toleransi}\).
\end{itemize}

\begin{enumerate}
\def\labelenumi{\arabic{enumi}.}
\setcounter{enumi}{5}
\tightlist
\item
  Jika \(Q_1-Q_2<3\times\text{nilai toleransi}\), luas integral = \(Q_2\).
\end{enumerate}

\begin{center}\rule{0.5\linewidth}{0.5pt}\end{center}

Kita dapat membangun sebuah fungsi integral adaptif menggunakan algoritma tersebut. Sintaks fungsi tersebut adalah sebagai berikut:

\begin{Shaded}
\begin{Highlighting}[]
\NormalTok{riemann\_adaptint }\OtherTok{\textless{}{-}} \ControlFlowTok{function}\NormalTok{(f, a, b, }\AttributeTok{m=}\DecValTok{10}\NormalTok{, }\AttributeTok{tol=}\FloatTok{1e{-}8}\NormalTok{)\{}
  \ControlFlowTok{if}\NormalTok{(m}\SpecialCharTok{\textless{}}\DecValTok{1}\NormalTok{)\{}
    \FunctionTok{stop}\NormalTok{(}\StringTok{"m harus \textgreater{}= 1"}\NormalTok{)}
\NormalTok{  \}}\ControlFlowTok{else} \ControlFlowTok{if}\NormalTok{(m}\SpecialCharTok{==}\DecValTok{1}\NormalTok{)\{}
\NormalTok{    m }\OtherTok{\textless{}{-}} \DecValTok{2}
\NormalTok{    n\_width }\OtherTok{\textless{}{-}}\NormalTok{ (b}\SpecialCharTok{{-}}\NormalTok{a)}\SpecialCharTok{/}\NormalTok{m}
\NormalTok{    x }\OtherTok{\textless{}{-}} \FunctionTok{seq}\NormalTok{(a, b}\SpecialCharTok{{-}}\NormalTok{n\_width, }\AttributeTok{length.out =}\NormalTok{ m) }\SpecialCharTok{+}\NormalTok{ n\_width}\SpecialCharTok{/}\DecValTok{2}
\NormalTok{    y }\OtherTok{\textless{}{-}} \FunctionTok{f}\NormalTok{(x)}
\NormalTok{    area }\OtherTok{\textless{}{-}} \FunctionTok{sum}\NormalTok{(y)}\SpecialCharTok{*}\FunctionTok{abs}\NormalTok{(b}\SpecialCharTok{{-}}\NormalTok{a)}\SpecialCharTok{/}\NormalTok{m}
\NormalTok{  \}}\ControlFlowTok{else}\NormalTok{\{}
    
\NormalTok{    m1 }\OtherTok{\textless{}{-}} \DecValTok{1}
\NormalTok{    n\_width1 }\OtherTok{\textless{}{-}}\NormalTok{ (b}\SpecialCharTok{{-}}\NormalTok{a)}\SpecialCharTok{/}\NormalTok{m1}
\NormalTok{    x1 }\OtherTok{\textless{}{-}} \FunctionTok{seq}\NormalTok{(a, b}\SpecialCharTok{{-}}\NormalTok{n\_width1, }\AttributeTok{length.out =}\NormalTok{ m1) }\SpecialCharTok{+}\NormalTok{ n\_width1}\SpecialCharTok{/}\DecValTok{2}
\NormalTok{    y1 }\OtherTok{\textless{}{-}} \FunctionTok{f}\NormalTok{(x1)}
\NormalTok{    q1 }\OtherTok{\textless{}{-}} \FunctionTok{sum}\NormalTok{(y1)}\SpecialCharTok{*}\FunctionTok{abs}\NormalTok{(b}\SpecialCharTok{{-}}\NormalTok{a)}\SpecialCharTok{/}\NormalTok{m1}
    
\NormalTok{    m2 }\OtherTok{\textless{}{-}} \DecValTok{2}
\NormalTok{    n\_width2 }\OtherTok{\textless{}{-}}\NormalTok{ (b}\SpecialCharTok{{-}}\NormalTok{a)}\SpecialCharTok{/}\NormalTok{m2}
\NormalTok{    x2 }\OtherTok{\textless{}{-}} \FunctionTok{seq}\NormalTok{(a, b}\SpecialCharTok{{-}}\NormalTok{n\_width2, }\AttributeTok{length.out =}\NormalTok{ m2) }\SpecialCharTok{+}\NormalTok{ n\_width2}\SpecialCharTok{/}\DecValTok{2}
\NormalTok{    y2 }\OtherTok{\textless{}{-}} \FunctionTok{f}\NormalTok{(x2)}
\NormalTok{    q2 }\OtherTok{\textless{}{-}} \FunctionTok{sum}\NormalTok{(y2)}\SpecialCharTok{*}\FunctionTok{abs}\NormalTok{(b}\SpecialCharTok{{-}}\NormalTok{a)}\SpecialCharTok{/}\NormalTok{m2}
    
    \ControlFlowTok{if}\NormalTok{(}\FunctionTok{abs}\NormalTok{(q1}\SpecialCharTok{{-}}\NormalTok{q2)}\SpecialCharTok{\textgreater{}}\DecValTok{3}\SpecialCharTok{*}\NormalTok{tol)\{}
\NormalTok{      m }\OtherTok{\textless{}{-}}\NormalTok{ m}\DecValTok{{-}1}
\NormalTok{      tol }\OtherTok{\textless{}{-}}\NormalTok{ tol}\SpecialCharTok{/}\DecValTok{2}
\NormalTok{      c }\OtherTok{\textless{}{-}}\NormalTok{ (a}\SpecialCharTok{+}\NormalTok{b)}\SpecialCharTok{/}\DecValTok{2}
\NormalTok{      lt }\OtherTok{\textless{}{-}} \FunctionTok{riemann\_adaptint}\NormalTok{(f, a, c, }\AttributeTok{m=}\NormalTok{m, }\AttributeTok{tol=}\NormalTok{tol)}
\NormalTok{      rt }\OtherTok{\textless{}{-}} \FunctionTok{riemann\_adaptint}\NormalTok{(f, c, b, }\AttributeTok{m=}\NormalTok{m, }\AttributeTok{tol=}\NormalTok{tol) }
\NormalTok{      area }\OtherTok{\textless{}{-}}\NormalTok{ lt}\SpecialCharTok{+}\NormalTok{rt}
\NormalTok{    \}}\ControlFlowTok{else}\NormalTok{\{}
\NormalTok{      area }\OtherTok{\textless{}{-}}\NormalTok{ q2}
\NormalTok{    \}}
\NormalTok{  \}}
  
  \FunctionTok{return}\NormalTok{(area)}
\NormalTok{\}}
\end{Highlighting}
\end{Shaded}

\begin{example}
\protect\hypertarget{exm:adaptexm}{}\label{exm:adaptexm}Hitung integral fungsi di bawah ini dengan menggunakan integral Riemann adaptif dengan jumlah panel yang digunakan m=100!
\end{example}

\[
\int_{1}^{10} \sin\left(x\right)^2+\log\left(x\right)dx
\]

\textbf{Jawab}:

Kita akan menghitung integral dari fungsi tersebut menggunakan fungsi \texttt{R} yang telah dibuat. Berikut adalah sintaks yang digunakan:

\begin{Shaded}
\begin{Highlighting}[]
\FunctionTok{riemann\_adaptint}\NormalTok{(}\ControlFlowTok{function}\NormalTok{(x)}\FunctionTok{sin}\NormalTok{(x)}\SpecialCharTok{\^{}}\DecValTok{2} \SpecialCharTok{+} \FunctionTok{log}\NormalTok{(x),}
                 \AttributeTok{a=}\DecValTok{1}\NormalTok{, }\AttributeTok{b=}\DecValTok{10}\NormalTok{, }\AttributeTok{m=}\DecValTok{100}\NormalTok{)}
\end{Highlighting}
\end{Shaded}

\begin{verbatim}
## [1] 18.52
\end{verbatim}

\hypertarget{metode-integral-adaptif-menggunakan-fungsi-lainnya-pada-r}{%
\section{\texorpdfstring{Metode Integral Adaptif Menggunakan Fungsi Lainnya Pada \texttt{R}}{Metode Integral Adaptif Menggunakan Fungsi Lainnya Pada R}}\label{metode-integral-adaptif-menggunakan-fungsi-lainnya-pada-r}}

Terdapat dua buah fungsi yang hendak penulis kenalkan pada pembaca yang berfungsi untuk melakukan integrasi adaptif pada \texttt{R}. Fungsi-fungsi tersebut antara lain: \texttt{integrate()} dari Paket \texttt{base} dan \texttt{integral()} dari Paket \texttt{pracma}.

Fungsi \texttt{integrate()} merupakan fungsi yang akan melakukan integrasi numerik menggunakan metode kudratur adaptif untuk sebuah variabel dengan selang terbatas (\emph{finite}) maupun tidak terbatas (\emph{infinite}). Format fungsi tersebut secara umum adalah sebagai berikut:

\begin{Shaded}
\begin{Highlighting}[]
\FunctionTok{integrate}\NormalTok{(f, lower, upper, ..., }\AttributeTok{subdivisions =}\NormalTok{ 100L,}
          \AttributeTok{rel.tol =}\NormalTok{ .Machine}\SpecialCharTok{$}\NormalTok{double.eps}\SpecialCharTok{\^{}}\FloatTok{0.25}\NormalTok{, }
          \AttributeTok{abs.tol =}\NormalTok{ rel.tol)}
\end{Highlighting}
\end{Shaded}

\begin{quote}
\textbf{Catatan:}

\begin{itemize}
\tightlist
\item
  \textbf{f}: fungsi yang akan dicari integralnya
\item
  \textbf{lower}: batas bawah.
\item
  \textbf{upper}: batas atas.
\item
  \textbf{\ldots{}}: argumen tambahan functn
\item
  \textbf{subdivision}: jumlah subinterval atau panel yang akan digunakan.
\item
  \textbf{rel.tol}: nilai akurasi relatif yang hendak dicapai
\item
  \textbf{abs.tol}: nilai akurasi absolut yang hendak dicapai
\end{itemize}
\end{quote}

Contoh penerapan fungsi \texttt{integrate()} adalah sebagai berikut:

\begin{Shaded}
\begin{Highlighting}[]
\FunctionTok{integrate}\NormalTok{(}\ControlFlowTok{function}\NormalTok{(x)}\FunctionTok{sin}\NormalTok{(x)}\SpecialCharTok{\^{}}\DecValTok{2} \SpecialCharTok{+} \FunctionTok{log}\NormalTok{(x),}
          \AttributeTok{lower=}\DecValTok{1}\NormalTok{, }\AttributeTok{upper=}\DecValTok{10}\NormalTok{,}
          \AttributeTok{rel.tol =} \FloatTok{1e{-}8}\NormalTok{)}
\end{Highlighting}
\end{Shaded}

\begin{verbatim}
## 18.52 with absolute error < 4.1e-10
\end{verbatim}

Fungsi lainnya yang dapat digunakan untuk melakukan komputasi integral adaptif adalah fungsi \texttt{integral()} dari Paket \texttt{pracma}. Terdapat dua buah metode integrasi adaptif yang dapat digunakan pada fungsi tersebut yaitu: Gauss-Konrod dan Simpson. Metode Clenshaw-Curtis yang tersedia masih belum dapat melakukan integrasi adaptif melalui fungsi tersebut. Format umum fungsi \texttt{integral()} adalah sebagai berikut:

\begin{Shaded}
\begin{Highlighting}[]
\FunctionTok{integral}\NormalTok{(fun, xmin, xmax,}
         \AttributeTok{method =} \FunctionTok{c}\NormalTok{(}\StringTok{"Kronrod"}\NormalTok{, }\StringTok{"Clenshaw"}\NormalTok{,}\StringTok{"Simpson"}\NormalTok{),}
         \AttributeTok{no\_intervals =} \DecValTok{8}\NormalTok{, }\AttributeTok{reltol =} \FloatTok{1e{-}8}\NormalTok{, }
         \AttributeTok{abstol =} \DecValTok{0}\NormalTok{, ...)}
\end{Highlighting}
\end{Shaded}

\begin{quote}
\textbf{Catatan:}

\begin{itemize}
\tightlist
\item
  \textbf{fun}: fungsi yang akan dicari integralnya
\item
  \textbf{xmin}: batas bawah.
\item
  \textbf{xmax}: batas atas.
\item
  \textbf{method}: metode imtegrasi yang digunakan.
\item
  \textbf{\ldots{}}: argumen tambahan fun.
\item
  \textbf{no\_intervals}: jumlah subinterval atau panel yang akan digunakan.
\item
  \textbf{reltol}: nilai akurasi relatif yang hendak dicapai
\item
  \textbf{abstol}: nilai akurasi absolut yang hendak dicapai
\end{itemize}
\end{quote}

Berikut merupakan contoh penerapan fungsi \texttt{integral()}:

\begin{Shaded}
\begin{Highlighting}[]
\NormalTok{pracma}\SpecialCharTok{::}\FunctionTok{integral}\NormalTok{(}\ControlFlowTok{function}\NormalTok{(x)}\FunctionTok{sin}\NormalTok{(x)}\SpecialCharTok{\^{}}\DecValTok{2} \SpecialCharTok{+} \FunctionTok{log}\NormalTok{(x),}
          \AttributeTok{xmin=}\DecValTok{1}\NormalTok{, }\AttributeTok{xmax=}\DecValTok{10}\NormalTok{, }\AttributeTok{method=}\StringTok{"Simpson"}\NormalTok{,}
          \AttributeTok{reltol =} \FloatTok{1e{-}8}\NormalTok{)}
\end{Highlighting}
\end{Shaded}

\begin{verbatim}
## [1] 18.52
\end{verbatim}

\hypertarget{romberg}{%
\section{Metode Integrasi Romberg}\label{romberg}}

Seperti halnya algoritma integrasi adaptif, integrasi Romberg adalah perluasan yang relatif mudah dari keluarga algoritma Newton-Cotes. Keduanya bekerja dengan menggunakan iterasi yang disempurnakan dari beberapa metode Newton-Cotes yang mendasarinya untuk memberikan perkiraan nilai integral yang lebih akurat. Tidak seperti proses komputasi fungsi \texttt{riemann\_adapint()}, integrasi Romberg bukanlah pendekatan adaptif terhadap integrasi. Hal tersebut berarti metode Romberg tidak mengubah perilakunya sendiri berdasarkan perilaku fungsi yang akan diintegrasikan. Sebaliknya, kita mengeksploitasi perilaku fungsi trapesium pada batas untuk menghasilkan estimasi integral.

Untuk memahami integrasi Romberg, kita harus mulai dengan implementasi rekursif dari aturan trapesium. Jika kita mulai dengan suatu fungsi, \(T\left(f, m\right)\) di mana \(T\) adalah fungsi trapesium, \(f\) adalah fungsi yang akan diintegrasikan, dan \(m\) adalah jumlah panel untuk diintegrasikan, maka,

\begin{equation}
S\left(f, m\right)=\frac{4T\left(f, m\right)-T\left(f, m/2\right)}{3}
  \label{eq:romberg}
\end{equation}

di mana \(S\) adalah aturan Simpson. Kemudian, jika kita mendefinisikan \(T \left(f, 0\right) = \left(b − a\right) \left(f \left(b\right) + f \left(a\right)\right) = 2\), maka fungsi rekursif kita selesai, karena berdasarkan hubungan ini, fraksi yang diberikan dalam Persamaan \eqref{eq:romberg} juga merupakan perkiraan untuk integral.

Secara umum,

\begin{equation}
I_{j,k}=\frac{4^k I_{j,k-1}-I_{j-1,k-1}}{4^k-1}
  \label{eq:romberg2}
\end{equation}

di mana \(I_{0, 0}\) adalah aturan trapesium satu panel dan \(I_{j, 0}\) adalah aturan trapesium dengan panel \(2^j\). Dengan menggunakan fungsi-fungsi dasar ini, \(I_{j, k}\) dapat ditemukan secara iteratif sebagai matriks segitiga-bawah di mana masing-masing nilai di kolom yang bukan paling kiri adalah fungsi dari nilai di sebelah kiri dan entri di atasnya.

Definisi rekursif ini muncul dari ekstrapolasi Richardson. Ketika diterapkan pada algoritma trapesium, yang konvergen menuju nilai sebenarnya dari integral sebagai \(m\) (jumlah panel) meningkat, hubungan dalam Persamaan \eqref{eq:romberg2} muncul. Penting untuk dipahami bahwa pada batas ketika \(k\) mendekati tak terhingga, nilai \(I_{j, k}\) adalah nilai sejati integral. Untuk nilai yang lebih kecil dari \(k\), integral Romberg masih hanya perkiraan, meskipun hasil yang diperoleh sangat bagus.

\begin{center}\rule{0.5\linewidth}{0.5pt}\end{center}

\textbf{Algoritma Metode Integrasi Romberg}

\begin{enumerate}
\def\labelenumi{\arabic{enumi}.}
\tightlist
\item
  Tentukan fungsi \(f\left(x\right)\) dan selang integrasinya \(\left[a,b\right]\).
\item
  Tentukan jumlah subinterval \(m\).
\item
  Bentuk matrik \(R\) dengan ukuran \(m\times m\) yang akan menampung hasil perhitungan.
\item
  Untuk \(R_{1,1}\) hitung integral fungsi menggunakan metode trapezoida dengan \(m=1\).
\item
  Untuk \(j=2,\dots,m\) dan \(k=1\), hitung integral dengan jumlah panel \(m=2^{j-1}\)
\item
  Untuk \(j=2,\dots,m\) dan \(k=2,\dots,m\) hitung nilai perbaikan nilai integrasi menggunakan Persamaan \eqref{eq:romberg2}.
\item
  Solusi integrasi diperoleh pada \(R_{m,m}\).
\end{enumerate}

\begin{center}\rule{0.5\linewidth}{0.5pt}\end{center}

Berdasarkan algoritma tersebut, kita akan menyusun suatu fungsi pada \texttt{R} untuk melakukan proses komputasi integrasi dengan metode Romberg. Berikut adalah sintaks fungsi yang dibuat:

\begin{Shaded}
\begin{Highlighting}[]
\NormalTok{romberg }\OtherTok{\textless{}{-}} \ControlFlowTok{function}\NormalTok{(f, a, b, m, }\AttributeTok{tab=}\ConstantTok{FALSE}\NormalTok{)\{}
\NormalTok{  R }\OtherTok{\textless{}{-}} \FunctionTok{matrix}\NormalTok{(}\ConstantTok{NA}\NormalTok{, }\AttributeTok{nrow=}\NormalTok{m, }\AttributeTok{ncol=}\NormalTok{m)}
  
\NormalTok{  R[}\DecValTok{1}\NormalTok{,}\DecValTok{1}\NormalTok{] }\OtherTok{\textless{}{-}} \FunctionTok{trap}\NormalTok{(f, a, b, }\AttributeTok{m=}\DecValTok{1}\NormalTok{)}
  \ControlFlowTok{for}\NormalTok{(j }\ControlFlowTok{in} \DecValTok{2}\SpecialCharTok{:}\NormalTok{m)\{}
\NormalTok{    R[j,}\DecValTok{1}\NormalTok{] }\OtherTok{\textless{}{-}} \FunctionTok{trap}\NormalTok{(f, a, b, }\AttributeTok{m=}\DecValTok{2}\SpecialCharTok{\^{}}\NormalTok{(j}\DecValTok{{-}1}\NormalTok{))}
    \ControlFlowTok{for}\NormalTok{(k }\ControlFlowTok{in} \DecValTok{2}\SpecialCharTok{:}\NormalTok{j)\{}
\NormalTok{      k4 }\OtherTok{\textless{}{-}} \DecValTok{4}\SpecialCharTok{\^{}}\NormalTok{(k}\DecValTok{{-}1}\NormalTok{)}
\NormalTok{      R[j,k] }\OtherTok{\textless{}{-}}\NormalTok{ (k4}\SpecialCharTok{*}\NormalTok{R[j,k}\DecValTok{{-}1}\NormalTok{]}\SpecialCharTok{{-}}\NormalTok{R[j}\DecValTok{{-}1}\NormalTok{,k}\DecValTok{{-}1}\NormalTok{])}\SpecialCharTok{/}\NormalTok{(k4}\DecValTok{{-}1}\NormalTok{)}
\NormalTok{    \}}
\NormalTok{  \}}
  
  \ControlFlowTok{if}\NormalTok{(tab}\SpecialCharTok{==}\ConstantTok{TRUE}\NormalTok{)\{}
    \FunctionTok{return}\NormalTok{(R)}
\NormalTok{  \}}\ControlFlowTok{else}\NormalTok{\{}
    \FunctionTok{return}\NormalTok{(R[m,m])}
\NormalTok{  \}}
\NormalTok{\}}
\end{Highlighting}
\end{Shaded}

\begin{example}
\protect\hypertarget{exm:rombergexm}{}\label{exm:rombergexm}Hitung integral fungsi yang ditampilkan pada Contoh \ref{exm:adaptexm} dengan m=10!
\end{example}

\textbf{Jawab}:

Kita dapat menggunakan fungsi \texttt{romberg()} untuk melakukan proses integrasi menggunakan metode Romberg. Berikut adalah sintaks yang digunakan:

\begin{Shaded}
\begin{Highlighting}[]
\CommentTok{\# menampilkan matriks proses perhitungan}
\FunctionTok{romberg}\NormalTok{(}\ControlFlowTok{function}\NormalTok{(x)}\FunctionTok{sin}\NormalTok{(x)}\SpecialCharTok{\^{}}\DecValTok{2} \SpecialCharTok{+} \FunctionTok{log}\NormalTok{(x),}
        \AttributeTok{a=}\DecValTok{1}\NormalTok{, }\AttributeTok{b=}\DecValTok{10}\NormalTok{, }\AttributeTok{m=}\DecValTok{10}\NormalTok{, }\AttributeTok{tab=}\ConstantTok{TRUE}\NormalTok{)}
\end{Highlighting}
\end{Shaded}

\begin{verbatim}
##        [,1]  [,2]  [,3]  [,4]  [,5]  [,6]  [,7]  [,8]
##  [1,] 14.88    NA    NA    NA    NA    NA    NA    NA
##  [2,] 17.35 18.18    NA    NA    NA    NA    NA    NA
##  [3,] 18.19 18.47 18.48    NA    NA    NA    NA    NA
##  [4,] 18.43 18.52 18.52 18.52    NA    NA    NA    NA
##  [5,] 18.50 18.52 18.52 18.52 18.52    NA    NA    NA
##  [6,] 18.52 18.52 18.52 18.52 18.52 18.52    NA    NA
##  [7,] 18.52 18.52 18.52 18.52 18.52 18.52 18.52    NA
##  [8,] 18.52 18.52 18.52 18.52 18.52 18.52 18.52 18.52
##  [9,] 18.52 18.52 18.52 18.52 18.52 18.52 18.52 18.52
## [10,] 18.52 18.52 18.52 18.52 18.52 18.52 18.52 18.52
##        [,9] [,10]
##  [1,]    NA    NA
##  [2,]    NA    NA
##  [3,]    NA    NA
##  [4,]    NA    NA
##  [5,]    NA    NA
##  [6,]    NA    NA
##  [7,]    NA    NA
##  [8,]    NA    NA
##  [9,] 18.52    NA
## [10,] 18.52 18.52
\end{verbatim}

Berdasarkan hasil perhitungan nilai integral fungsi tersebut adalah 18.5249.

\hypertarget{metode-integrasi-romberg-menggunakan-fungsi-lainnya}{%
\section{Metode Integrasi Romberg Menggunakan Fungsi Lainnya}\label{metode-integrasi-romberg-menggunakan-fungsi-lainnya}}

Fungsi \texttt{romberg()} pada Paket \texttt{pracma} dapat digunakan untuk melakukan integrasi metode Romberg. Format umum fungsi tersebut adalah sebagai berikut:

\begin{Shaded}
\begin{Highlighting}[]
\FunctionTok{romberg}\NormalTok{(f, a, b, }\AttributeTok{maxit =} \DecValTok{25}\NormalTok{, }\AttributeTok{tol =} \FloatTok{1e{-}12}\NormalTok{, ...)}
\end{Highlighting}
\end{Shaded}

\begin{quote}
\textbf{Catatan:}

\begin{itemize}
\tightlist
\item
  \textbf{f}: fungsi yang akan dicari integralnya
\item
  \textbf{a}: batas bawah.
\item
  \textbf{b}: batas atas.
\item
  \textbf{\ldots{}}: argumen tambahan functn
\item
  \textbf{maxit}: jumlah iterasi maksimum.
\item
  \textbf{tol}: nilai akurasi yang hendak dicapai
\end{itemize}
\end{quote}

Berikut adalah contoh penerapan fungsi \texttt{romberg()}:

\begin{Shaded}
\begin{Highlighting}[]
\NormalTok{pracma}\SpecialCharTok{::}\FunctionTok{romberg}\NormalTok{(}\ControlFlowTok{function}\NormalTok{(x)}\FunctionTok{sin}\NormalTok{(x)}\SpecialCharTok{\^{}}\DecValTok{2} \SpecialCharTok{+} \FunctionTok{log}\NormalTok{(x),}
        \AttributeTok{a=}\DecValTok{1}\NormalTok{, }\AttributeTok{b=}\DecValTok{10}\NormalTok{)}
\end{Highlighting}
\end{Shaded}

\begin{verbatim}
## $value
## [1] 18.52
## 
## $iter
## [1] 8
## 
## $rel.error
## [1] 7.105e-15
\end{verbatim}

\hypertarget{montecarlo}{%
\section{Metode Integrasi Monte Carlo}\label{montecarlo}}

Nama Monte Carlo berasal dari daerah di Monako, yang terkenal karena aktikvitas kasino dan perjudiannya. Jelas, permainan kasino yang baik tergantung pada keacakan, seperti juga metode Monte Carlo. Nama ini menggambarkan pentingnya keacakan dalam proses karena algoritma Monte Carlo menggunakan generator angka acak untuk membedakan input ke suatu fungsi.

Angka acak harus berasal dari domain fungsi yang diharapkan. Selanjutnya, fungsi itu sendiri bersifat deterministik karena untuk diberikan dua input dari domain fungsi \(x_1\) dan \(x_2\). Jika \(x_1 = x_2\), maka \(f \left(x_1\right) = f \left(x_2\right)\). Generator angka acak digunakan untuk menghasilkan sejumlah besar input dan fungsinya dijalankan pada setiap input. Akhirnya, hasil yang diperoleh dikumpulkan sesuai dengan model logika yang sesuai dengan analisis yang dilakukan.

Metode Monte Carlo dapat digunakan untuk integrasi numerik dalam jumlah dimensi apa pun yang diberikan. Pendekatan mendasar metode ini adalah menempatkan beberapa titik \(m\) secara acak di atas domain untuk diintegrasikan. Jika titik terletak ``di bawah'' garis fungsi, maka titik tersebut dianggap dalam area integrasi. Jika titiknya ``di atas'' garis fungsi, maka titik tersebut bukan berada diluar garis integrasi. Area di bawah perkiraan kurva adalah persentase titik di bawah garis.

Beberapa algoritma Monte Carlo yang paling awal digunakan untuk menemukan area di bawah kurva atau untuk memperkirakan nilai \(\pi\) sebuah hobi favorit matematikawan sejak dahulu. Satu pendekatan menciptakan seperempat lingkaran, menggunakan fungsi \(f \left(x\right) = \sqrt{1-x^2}\). Melalui domain \(\left[0, 1\right]\), ini adalah fungsi dan merupakan hasil dari penyelesaian persamaan standar untuk lingkaran, \(x^2 + y^2 = r\) untuk \(y\) di mana \(r = 1\).

Gambar \ref{fig:monteviz} menunjukkan plot fungsi ini. Selain itu, 20 titik acak dipilih. Jika titik di bawah kurva dilambangkan dengan lingkaran hitam terisi dan titik-titik kurva dilambangkan dengan titik bulat kosong. Dalam contoh ini, terdapat 15 titik berada di bawah kurva, mengarah ke estimasi area luas area 0,75. Karena kurva mewakili seperempat lingkaran, estimasi untuk \(\pi\) adalah 3. Meningkatkan jumlah tes titik acak meningkatkan ketepatan estimasi dan akurasi.

\begin{figure}

{\centering \includegraphics[width=0.7\linewidth]{./images/monte} 

}

\caption{Visualisasi metode Monte-Carlo untuk fungsi setengah lingkaran dengan jumlah bilangan acak 20 (sumber: Jones et.al., 2014).}\label{fig:monteviz}
\end{figure}

Bentuk umum metode Monte-Carlo disajikan pada Persamaan \eqref{eq:monte}.

\begin{equation}
\int_a^bf\left(x\right)dx\approx\left(b-a\right)\frac{1}{N}\sum_{i=1}^Nf\left(x_i\right)
  \label{eq:monte}
\end{equation}

dimana \(N\) merupakan jumlah titik yang akan dievaluasi.

\begin{center}\rule{0.5\linewidth}{0.5pt}\end{center}

\textbf{Algoritma Metode Monte Carlo}

\begin{enumerate}
\def\labelenumi{\arabic{enumi}.}
\tightlist
\item
  Tentukan fungsi \(f\left(x\right)\) dan selang integrasinya \(\left[a,b\right]\).
\item
  Tentukan jumlah titik acak yang akan digunakan \(N\).
\item
  Lakukan produksi titik acak \(x\) dengan selang \(\left[a,b\right]\) sejumlah \(N\)
\item
  Hitung \(f\left(x_i\right)\)
\item
  Hitung estimasi area menggunakan Persamaan \eqref{eq:monte}
\end{enumerate}

\begin{center}\rule{0.5\linewidth}{0.5pt}\end{center}

Berdasarkan algoritma tersebut, fungsi \texttt{R} dapat dibangun untuk melakukan integrasi numerik menggunakan metode Monte Carlo. Berikut adalah sintaks yang digunakan:

\begin{Shaded}
\begin{Highlighting}[]
\NormalTok{monte\_int }\OtherTok{\textless{}{-}} \ControlFlowTok{function}\NormalTok{(f, a, b, }\AttributeTok{m=}\FloatTok{1e6}\NormalTok{)\{}
\NormalTok{  x }\OtherTok{\textless{}{-}} \FunctionTok{runif}\NormalTok{(m, }\AttributeTok{min=}\NormalTok{a, }\AttributeTok{max=}\NormalTok{b)}
  
  \FunctionTok{return}\NormalTok{((b}\SpecialCharTok{{-}}\NormalTok{a)}\SpecialCharTok{*}\FunctionTok{sum}\NormalTok{(}\FunctionTok{f}\NormalTok{(x))}\SpecialCharTok{/}\NormalTok{m)}
\NormalTok{\}}
\end{Highlighting}
\end{Shaded}

\begin{example}
\protect\hypertarget{exm:monteexm}{}\label{exm:monteexm}Hitung integral fungsi yang ditampilkan pada Contoh \ref{exm:adaptexm} menggunakan metode Monte-Carlo dengan m=1e6!
\end{example}

\textbf{Jawab}:

Integrasi Monte Carlo menggunakan menggunakan fungsi \texttt{monte\_int()} disajikan pada sintaks berikut:

\begin{Shaded}
\begin{Highlighting}[]
\FunctionTok{monte\_int}\NormalTok{(}\ControlFlowTok{function}\NormalTok{(x)}\FunctionTok{sin}\NormalTok{(x)}\SpecialCharTok{\^{}}\DecValTok{2} \SpecialCharTok{+} \FunctionTok{log}\NormalTok{(x), }\AttributeTok{a=}\DecValTok{1}\NormalTok{, }\AttributeTok{b=}\DecValTok{10}\NormalTok{)}
\end{Highlighting}
\end{Shaded}

\begin{verbatim}
## [1] 18.52
\end{verbatim}

Hasil yang diperoleh sedikit berbeda dengan yang dihasilkan oleh metode lainnya. Hal ini disebabkan oleh penggunaan bilangan acak pada proses integrasi. Selain itu, metode ini juga menghasilkan kualitas hasil yang rendah dengan tingkat komputasi yang tinggi dibandingkan dengan metode Newton-Cotes.

Keunggulan metode Monte Carlo dibandingkan metode sebelumnya adalah kemampuan untuk menangani proses integrasi berganda. Berikut adalah bentuk umum proses integrasi bivariat menggunakan metode Monte Carlo:

\begin{equation}
\int \int f\left(x,y\right)dx\approx V\frac{1}{N}\sum_{i=1}^Nf\left(x_i,y_i\right)
  \label{eq:monte2}
\end{equation}

dimana \(V\) merupakan area perpotongan \(x-y\) dimana fungsi \(f\left(x,y\right)\) diintegrasikan.

\begin{center}\rule{0.5\linewidth}{0.5pt}\end{center}

\textbf{Algoritma Metode Monte Carlo Bivariat}

\begin{enumerate}
\def\labelenumi{\arabic{enumi}.}
\tightlist
\item
  Tentukan fungsi \(f\left(x\right)\) dan domain integrasinya pada masing-masing sumbu \(x\) dan \(y\)
\item
  Tentukan jumlah titik acak yang akan digunakan \(N\).
\item
  Lakukan produksi titik acak \(x\) dan \(y\) masing-masing domain sumbunya sejumlah \(N\)
\item
  Hitung \(V=\left( x_{max}-x_{min}\right) \times \left(y_{max}-y_{max}\right)\)
\item
  Hitung estimasi volume menggunakan Persamaan \eqref{eq:monte2}
\end{enumerate}

\begin{center}\rule{0.5\linewidth}{0.5pt}\end{center}

\begin{Shaded}
\begin{Highlighting}[]
\NormalTok{monte\_int2 }\OtherTok{\textless{}{-}} \ControlFlowTok{function}\NormalTok{(f, xdom, ydom, }\AttributeTok{m=}\DecValTok{1000}\NormalTok{)\{}
\NormalTok{  xmin }\OtherTok{\textless{}{-}} \FunctionTok{min}\NormalTok{(xdom)}
\NormalTok{  xmax }\OtherTok{\textless{}{-}} \FunctionTok{max}\NormalTok{(xdom)}
\NormalTok{  ymin }\OtherTok{\textless{}{-}} \FunctionTok{min}\NormalTok{(ydom)}
\NormalTok{  ymax }\OtherTok{\textless{}{-}} \FunctionTok{max}\NormalTok{(ydom)}
  
\NormalTok{  x }\OtherTok{\textless{}{-}} \FunctionTok{runif}\NormalTok{(m, }\AttributeTok{min=}\NormalTok{xmin, }\AttributeTok{max=}\NormalTok{xmax)}
\NormalTok{  y }\OtherTok{\textless{}{-}} \FunctionTok{runif}\NormalTok{(m, }\AttributeTok{min=}\NormalTok{ymin, }\AttributeTok{max=}\NormalTok{ymax)}
  
\NormalTok{  V }\OtherTok{\textless{}{-}}\NormalTok{ (xmax}\SpecialCharTok{{-}}\NormalTok{xmin)}\SpecialCharTok{*}\NormalTok{(ymax}\SpecialCharTok{{-}}\NormalTok{ymin)}
  
  \FunctionTok{return}\NormalTok{(V}\SpecialCharTok{*}\FunctionTok{sum}\NormalTok{(}\FunctionTok{f}\NormalTok{(x,y))}\SpecialCharTok{/}\NormalTok{m)}
\NormalTok{\}}
\end{Highlighting}
\end{Shaded}

\begin{example}
\protect\hypertarget{exm:monteexm2}{}\label{exm:monteexm2}Hitung volume melalui integrasi persamaan berikut menggunakan metode Monte Carlo dengan domain x = {[}0,1{]} dan y={[}0,1{]}!
\end{example}

\[
\int_{0}^1 \int_{0}^1 x^2y \ dy \ dx
\]

\textbf{Jawab}:

Berikut adalah sintaks yang digunakan untuk melakukan integrasi persamaan tersebut menggunakan metode Monte Carlo bivariat:

\begin{Shaded}
\begin{Highlighting}[]
\FunctionTok{monte\_int2}\NormalTok{(}\ControlFlowTok{function}\NormalTok{(x,y)x}\SpecialCharTok{\^{}}\DecValTok{2}\SpecialCharTok{*}\NormalTok{y,}
           \AttributeTok{xdom=}\FunctionTok{c}\NormalTok{(}\DecValTok{0}\NormalTok{,}\DecValTok{1}\NormalTok{), }\AttributeTok{ydom=}\FunctionTok{c}\NormalTok{(}\DecValTok{0}\NormalTok{,}\DecValTok{1}\NormalTok{))}
\end{Highlighting}
\end{Shaded}

\begin{verbatim}
## [1] 0.168
\end{verbatim}

Karena metode Monte Carlo tidak deterministik, \emph{error} integrasi Monte Carlo tidak dibatasi dalam pengertian yang telah kita lihat sejauh ini. Namun, kita dapat memperkirakan varians dari estimasi yang dihasilkan, yang berkurang dengan meningkatnya jumlah poin:

\begin{equation}
\text{Var}\frac{1}{N}\sum f\left(\right)=\frac{\sigma^2}{N}
  \label{eq:monte3}
\end{equation}

dimana \(\sigma^2=\text{Var}\ f\left(\right)\). Definisi ini juga digunakan untuk proses integrasi dengan dimensi yang lebih tinggi.

Perlu diketahui pula bahwa metode Monte Carlo hanya dapat digunakan jika nilai terendah dari variabel bebas yang digunakan tidak negatif. Hal ini dilakukan untuk mencegah adanya pengurangan dengan nilai negatif sehingga hasil integrasi jauh lebih besar dari yang seharusnya.

\hypertarget{studi-kasus-2}{%
\section{Studi Kasus}\label{studi-kasus-2}}

\hypertarget{penerjung-payung}{%
\subsection{Penerjung Payung}\label{penerjung-payung}}

Pada studi kasus kali ini, penulis akan memberikan contoh penerapan integrasi numerik dalam menganalisa jarak jatuh seorang penerjung payung yang melompat dari sebuah pesawat. Kecepatan penerjun payung dapat dituliskan ke dalam sebuah fungsi dari waktu,

\begin{equation}
v\left(t\right)=\frac{gm}{c}\left(1-e^{-\left(c/m\right)t}\right)
  \label{eq:terjun}
\end{equation}

dimana \(v\) adalah kecepatan penerjun payung dalam \(m/dt\), \(g\) adalah percepatan gravitasi sebesar \(9,8 \ m/dt^2\), \(m\) adalah massa penerjun payung sebesar \(68,1 \ kg\), dan \(c\) adalah koefisien tahanan udara sebesar \(12,5\ kg/dt\).

Misalkan kita ingin mengetahui seberapa jauh penerjun telah jatuh setelah waktu tertentu \(t\). Karena kecepatan merupakan turunan pertama dari fungsi jarak, maka jarak penerjun dari titik terjun \(\left(t=0\right)\) adalah:

\begin{equation}
d=\int_{0}^{t}v\left(t\right)dt=\int_{0}^{t}\frac{gm}{c}\left(1-e^{-\left(c/m\right)t}\right)dt
  \label{eq:terjun2}
\end{equation}

Jika kita ingin mengetahui jarak yang telah ditempuh saat \(t=10\), kita dapat melakukan integrasi pada persamaan tersebut dengan domain \(t=\left[0,10\right]\). Persamaan \eqref{eq:terjun2} dapat dinyatakan menjadi Persamaan \eqref{eq:terjun3} dengan memasukkan semua komponen yang telah diketahui sebelumnya.

\begin{equation}
d=\int_{0}^{10}\frac{9,8\times 68,1}{12,5}\left(1-e^{-\left(12,5/68,1\right)^t}\right)dt
  \label{eq:terjun3}
\end{equation}

Kita dapat menyelesaikan Persamaan \eqref{eq:terjun3} dengan menggunakan metode-metode integrasi numerik yang telah dijabarkan sebelumnya. Berikut adalah sintaks untuk masing-masing metode tersebut:

\begin{Shaded}
\begin{Highlighting}[]
\NormalTok{f }\OtherTok{\textless{}{-}} \ControlFlowTok{function}\NormalTok{(x)((}\FloatTok{9.8}\SpecialCharTok{*}\FloatTok{68.1}\NormalTok{)}\SpecialCharTok{/}\FloatTok{12.5}\NormalTok{)}\SpecialCharTok{*}\NormalTok{(}\DecValTok{1}\SpecialCharTok{{-}}\FunctionTok{exp}\NormalTok{(}\SpecialCharTok{{-}}\NormalTok{(}\FloatTok{12.5}\SpecialCharTok{/}\FloatTok{68.1}\NormalTok{)}\SpecialCharTok{*}\NormalTok{x))}
\NormalTok{a }\OtherTok{\textless{}{-}} \DecValTok{0}\NormalTok{; b }\OtherTok{\textless{}{-}} \DecValTok{10}
\end{Highlighting}
\end{Shaded}

\textbf{Newton-Cotes}

\begin{Shaded}
\begin{Highlighting}[]
\CommentTok{\# Metode Riemann}
\FunctionTok{riemann}\NormalTok{(f, a, b)}
\end{Highlighting}
\end{Shaded}

\begin{verbatim}
## [1] 289.4
\end{verbatim}

\begin{Shaded}
\begin{Highlighting}[]
\CommentTok{\# Metode Trapezoida}
\FunctionTok{trap}\NormalTok{(f, a, b)}
\end{Highlighting}
\end{Shaded}

\begin{verbatim}
## [1] 289.4
\end{verbatim}

\begin{Shaded}
\begin{Highlighting}[]
\CommentTok{\# Metode Simpson 1/3}
\FunctionTok{simpson}\NormalTok{(f, a, b)}
\end{Highlighting}
\end{Shaded}

\begin{verbatim}
## [1] 289.4
\end{verbatim}

\begin{Shaded}
\begin{Highlighting}[]
\CommentTok{\# Metode Simpson 3/8}
\FunctionTok{simpson38}\NormalTok{(f, a, b)}
\end{Highlighting}
\end{Shaded}

\begin{verbatim}
## [1] 289.4
\end{verbatim}

\textbf{Metode Adaptif}

\begin{Shaded}
\begin{Highlighting}[]
\FunctionTok{riemann\_adaptint}\NormalTok{(f, a, b, }\AttributeTok{m=}\DecValTok{100}\NormalTok{)}
\end{Highlighting}
\end{Shaded}

\begin{verbatim}
## [1] 289.4
\end{verbatim}

\textbf{Metode Romberg}

\begin{Shaded}
\begin{Highlighting}[]
\FunctionTok{romberg}\NormalTok{(f, a, b, }\AttributeTok{m=}\DecValTok{10}\NormalTok{)}
\end{Highlighting}
\end{Shaded}

\begin{verbatim}
## [1] 289.4
\end{verbatim}

\textbf{Metode Monte Carlo}

\begin{Shaded}
\begin{Highlighting}[]
\FunctionTok{monte\_int}\NormalTok{(f, a, b)}
\end{Highlighting}
\end{Shaded}

\begin{verbatim}
## [1] 289.6
\end{verbatim}

\hypertarget{referensi-4}{%
\section{Referensi}\label{referensi-4}}

\begin{enumerate}
\def\labelenumi{\arabic{enumi}.}
\tightlist
\item
  Bloomfield, V.A. 2014. \textbf{Using R for Numerical Analysis in Science and Engineering}. CRC Press.
\item
  Chapra, S.C. Canale, R.P. 2015. \textbf{Numerical Methods For Engineers, Seventh Edition}. Mc Graw Hill.
\item
  Howard, J.P. 2017. \textbf{Computational Methods for Numerical Analysis with R}. CRC Press.
\item
  Kreyszig, E. 2011. \textbf{Advanced Engineering Mathematics, 10th Edition}. John Wiley \& Sons.
\item
  Sanjaya, M. 2015. \textbf{Metode Numerik Berbasis Phython}. Penerbit Gava Media: Yogyakarta.
\item
  Suparno, S. 2008. \textbf{Komputasi untuk Sains dan Teknik Edisi II}. Departemen Fisika-FMIPA Universitas Indonesia.
\end{enumerate}

\hypertarget{latihan-3}{%
\section{Latihan}\label{latihan-3}}

\begin{enumerate}
\def\labelenumi{\arabic{enumi}.}
\tightlist
\item
  Hitung integral fungsi \(f\left(x\right)=\sin^2\left(x\right)\) pada domain \(x \in \left[0,\pi\right]\) !
\item
  Tuliskan fungsi \texttt{R} yang dapat melakukan integrasi Riemann dengan aturan titik kiri !
\item
  Buatlah sebuah fungsi \texttt{R} yang dapat melakukan integrasi adaptif menggunakan metode Simpson 1/3 !
\item
  Kerjakan kembali soal 3 dengan menggunakan metode Simpson 3/8! (\textbf{Note}: pembaca dapat melakukan pencarian algoritma di internet dan mentrasformasikannya menjadi sintaks \texttt{R})
\item
  Fungsi \texttt{monte\_int()} hanya mampu melakukan integrasi pada domain positif. Buatlah algoritma baru sehingga metode ini dapat melakukan integrasi pada domain negatif !
\end{enumerate}

\hypertarget{diffeq}{%
\chapter{Persamaan Diferensial}\label{diffeq}}

Persamaan diferensial merupakan persoalan matematis yang sering dijumpai dalam bidang teknik lingkungan. Sering kali suatu persamaan diferensial tidak dapat diselesaikan secara analitik sehingga diperlukan metode numerik untuk menyelesaikannya. Pada Chapter \ref{diffeq}, kita akan membahas masalah-masalah dalam persamaan diferensial dan metode penyelesaiannya. Adapun yang akan dibahas pada Chapter \ref{diffeq} kali ini antara lain:

\begin{itemize}
\tightlist
\item
  \emph{Initial value problems}
\item
  Sistem persamaan diferensial
\item
  Persamaan diferensial parsial
\end{itemize}

\hypertarget{ivp}{%
\section{\texorpdfstring{\emph{Initial value problems}}{Initial value problems}}\label{ivp}}

\emph{Initial value problems} merupakan permasalahan yang sering ditemukan pada proses dekomposisi zat kimia atau polutan dalam reaktor. Penyelesaiaan persamaan diferensial biasanya dipersulit dengan tidak tersedianya informasi yang cukup untuk menyelesaikannya.Sebuah persamaan diferensial \(f'\left(x,\dots\right)\) merupakan hasil diferensiasi beberapa fungsi \(f\left(x,\dots\right)\). Proses penyelesaian persamaan diferensial, dan menemukan nilai \(f \left(x,\dots\right)\) untuk beberapa nilai x,\ldots{} tidak dimungkinkan karena integral dari \(f'\left(x,\dots\right)\) hanya digambarkan bentuk umum. Pergeseran vertikal, atas atau bawah, tidak diketahui. Pergeseran vertikal ini menghasilkan konstanta integrasi.

Selama proses diferensiasi, nilai apapun dari proses pergeseran vertikal (integrasi) akan hilang sebagai akibat dari eliminasi konstanta yang memiliki turunan 0. Kita biasa melakukannya ketika mengintegrasikan fungsi dengan menambahkan konstanta \(+ C\) pada proses integrasi ke integral yang tidak terbatas. Hal ini terkadang bukan menjadi permasalahan sebab jika menemukan nilai integrasi pada suatu batas tertentu syarat \(+ C\) dibatalkan dan konstanta integrasi tidak diperlukan.

Untuk persamaan diferensial biasa, tidak ada pembatalan yang nyaman, yang mengarah ke \emph{initial value problems}. \emph{Initial value problems} memberikan nilai \(f\left(x_0,\dots\right)\), di mana \(x_0\) biasanya bernilai 0, meski tidak diharuskan. Nilai awal ini memberikan informasi yang cukup untuk menyelesaikan persamaan dan menemukan nilai aktual dari \(f\left(x,\dots\right)\) untuk sejumlah nilai \(x\). Terdapat beberapa metode yang akan dibahas pada Chapter \ref{ivp}, antara lain:

\begin{itemize}
\tightlist
\item
  Metode Euler
\item
  Metode Heun
\item
  Metode Titik Tengah
\item
  Metode Runge-Kutta Orde 4
\item
  Metode multistep linier
\end{itemize}

\hypertarget{eulermethod}{%
\subsection{Metode Euler}\label{eulermethod}}

Metode Euler merupakan metode paling sederhana yang diturunkan dari deret Taylor. Penyelesaian \emph{initial value problems} menggunakan metode Euler dilakukan melalui Persamaan \eqref{eq:euler}.

\begin{equation}
y_{i+1}=y_i+f\left(x_i,y_i\right)h
  \label{eq:euler}
\end{equation}

dimana \(i\) merupakan tahapan iterasi.

\begin{center}\rule{0.5\linewidth}{0.5pt}\end{center}

\textbf{Algoritma Metode Euler}

\begin{enumerate}
\def\labelenumi{\arabic{enumi}.}
\tightlist
\item
  Tentukan titik awal integrasi \(x_0\) dan \(y_0\).
\item
  Tentukan jumlah iterasi \(n\) dan \emph{step size} \(h\) yang digunakan.
\item
  Lakukan integrasi menggunakan Persamaan \eqref{eq:euler}.
\end{enumerate}

\begin{center}\rule{0.5\linewidth}{0.5pt}\end{center}

Algoritma tersebut, selanjutnya dapat disusun ke dalam sebuah fungsi \texttt{R}. Fungsi tersebut adalah sebagai berikut:

\begin{Shaded}
\begin{Highlighting}[]
\NormalTok{euler }\OtherTok{\textless{}{-}} \ControlFlowTok{function}\NormalTok{(f, x0, y0, h, n)\{}
\NormalTok{  x }\OtherTok{\textless{}{-}}\NormalTok{ x0}
\NormalTok{  y }\OtherTok{\textless{}{-}}\NormalTok{ y0}
  
  \ControlFlowTok{for}\NormalTok{(i }\ControlFlowTok{in} \DecValTok{1}\SpecialCharTok{:}\NormalTok{n)\{}
\NormalTok{    y0 }\OtherTok{\textless{}{-}}\NormalTok{ y0 }\SpecialCharTok{+}\NormalTok{ h}\SpecialCharTok{*}\FunctionTok{f}\NormalTok{(x0, y0)}
\NormalTok{    x0 }\OtherTok{\textless{}{-}}\NormalTok{ x0 }\SpecialCharTok{+}\NormalTok{ h}
\NormalTok{    x }\OtherTok{\textless{}{-}} \FunctionTok{c}\NormalTok{(x,x0)}
\NormalTok{    y }\OtherTok{\textless{}{-}} \FunctionTok{c}\NormalTok{(y, y0)}
\NormalTok{  \}}
  
  \FunctionTok{return}\NormalTok{(}\FunctionTok{data.frame}\NormalTok{(}\AttributeTok{x=}\NormalTok{x, }\AttributeTok{y=}\NormalTok{y))}
\NormalTok{\}}
\end{Highlighting}
\end{Shaded}

\begin{example}
\protect\hypertarget{exm:eulerexmp}{}\label{exm:eulerexmp}Selesaikan persamaan diferensial di bawah ini, jika diketahui f(0)=1 menggunakan h=0,05 dan n=100!
\end{example}

\[
f'\left(x,y\right)=\frac{y}{2x+1}
\]

\textbf{Jawab}:

Penyelesaian secara analitik persamaan tersebut untuk nilai \(f\left(0\right)=1\) sebagai berikut:

\[
f\left(x\right)=\sqrt{2x+1}
\]

Secara numerik persamaan tersebut dapat diselesaikan sebagai berikut:

\textbf{iterasi 1}

\[
y\left(0,5\right)=1+0,05\times\frac{1}{\left(2\cdot 0\right)+1}=1,05
\]

\textbf{iterasi 2}

\[
y\left(0,1\right)=1,05+0,05\times\frac{1}{\left(2\cdot0,05\right)+1}=1.097727
\]

Kita dapat juga menggunakan fungsi \texttt{euler()} untuk menyelesaikan persamaan tersebut secara numerik. Hasil yang diperoleh selanjutnya diplotkan dengan hasil yang diperoleh menggunakan metode analitik. Berikut adalah sintaks yang digunakan:

\begin{Shaded}
\begin{Highlighting}[]
\CommentTok{\# metode numerik}
\NormalTok{f1 }\OtherTok{\textless{}{-}} \ControlFlowTok{function}\NormalTok{(x,y)\{y}\SpecialCharTok{/}\NormalTok{(}\DecValTok{2}\SpecialCharTok{*}\NormalTok{x}\SpecialCharTok{+}\DecValTok{1}\NormalTok{)\}}
\NormalTok{num }\OtherTok{\textless{}{-}} \FunctionTok{euler}\NormalTok{(f1, }\AttributeTok{x0=}\DecValTok{0}\NormalTok{, }\AttributeTok{y0=}\DecValTok{1}\NormalTok{, }\AttributeTok{h=}\FloatTok{0.05}\NormalTok{, }\AttributeTok{n=}\DecValTok{100}\NormalTok{)}

\CommentTok{\# metode analitik}
\NormalTok{f2 }\OtherTok{\textless{}{-}} \ControlFlowTok{function}\NormalTok{(x)\{}\FunctionTok{sqrt}\NormalTok{(}\DecValTok{2}\SpecialCharTok{*}\NormalTok{x}\SpecialCharTok{+}\DecValTok{1}\NormalTok{)\}}
\NormalTok{x0 }\OtherTok{\textless{}{-}} \DecValTok{0}
\NormalTok{y0 }\OtherTok{\textless{}{-}} \DecValTok{1}
\NormalTok{x }\OtherTok{\textless{}{-}}\NormalTok{ x0}
\NormalTok{y }\OtherTok{\textless{}{-}}\NormalTok{ y0}

\ControlFlowTok{for}\NormalTok{(i }\ControlFlowTok{in} \DecValTok{1}\SpecialCharTok{:}\DecValTok{100}\NormalTok{)\{}
\NormalTok{  y0 }\OtherTok{\textless{}{-}} \FunctionTok{f2}\NormalTok{(x0}\FloatTok{+0.05}\NormalTok{)}
\NormalTok{  x0 }\OtherTok{\textless{}{-}}\NormalTok{ x0}\FloatTok{+0.05}
\NormalTok{  x }\OtherTok{\textless{}{-}} \FunctionTok{c}\NormalTok{(x, x0)}
\NormalTok{  y }\OtherTok{\textless{}{-}} \FunctionTok{c}\NormalTok{(y, y0)}
\NormalTok{\}}
\NormalTok{true }\OtherTok{\textless{}{-}} \FunctionTok{data.frame}\NormalTok{(}\AttributeTok{x=}\NormalTok{x, }\AttributeTok{y=}\NormalTok{y)}
\end{Highlighting}
\end{Shaded}

\begin{figure}

{\centering \includegraphics[width=0.8\linewidth]{Algoritma_Pemrograman_R_files/figure-latex/eulerviz-1} 

}

\caption{Visualisasi integrasi numerik dengan metode Euler dan metode analitik}\label{fig:eulerviz}
\end{figure}

Berdasarkan hasil visualisasi dapat dilihat bahwa metode Euler dapat dengan baik memberikan pendekatan nilai integrasi persamaan. Pembaca dapat mencoba untuk melakukan simulasi kembali dengan nilai \(h\) yang lebih kecil.

\hypertarget{heun}{%
\subsection{Metode Heun}\label{heun}}

Metode Heun merupakan salah satu peningkatan dari metode Euler. Metode ini melibatkan 2 buah persamaan. Persamaan pertama disebut sebagai persamaan prediktor yang digunakan untuk memprediksi nilai integrasi awal (Persamaan \eqref{eq:heun1}). Persamaan kedua disebut sebagai persamaan korektor yang mengoreksi hasil integrasi awal (Persamaan \eqref{eq:heun2}). Metode Heun pada \emph{Chapter} ini merupakan metode prediktor-korektor satu tahapan. Akurasi integrasi dapat ditingkatkan dengan melakukan koreksi ulang terhadap nilai koreksi semula menggunakan persamaan kedua.

\begin{equation}
y_{i+1}^{0}=y_i+f\left(x_i,y_i\right)h
  \label{eq:heun1}
\end{equation}

\begin{equation}
y_{i+1}=y_i+\frac{f\left(x_i,y_i\right)+f\left(x_{i+1},y_{i+1}^0\right)}{2}h
  \label{eq:heun2}
\end{equation}

\begin{center}\rule{0.5\linewidth}{0.5pt}\end{center}

\textbf{Algoritma Metode Heun}

\begin{enumerate}
\def\labelenumi{\arabic{enumi}.}
\tightlist
\item
  Tentukan titik awal integrasi \(x_0\) dan \(y_0\).
\item
  Tentukan jumlah iterasi \(n\) dan \emph{step size} \(h\) yang digunakan.
\item
  Lakukan prediksi nilai awal dengan Persamaan \eqref{eq:heun1}.
\item
  Lakukan koreksi nilai awal menggunakan Persamaan \eqref{eq:heun2}.
\item
  Lakukan koreksi terhadap nilai koreksi yang dihasilkan sebelumnya menggunakan Persamaan \eqref{eq:heun2}.
\end{enumerate}

\begin{center}\rule{0.5\linewidth}{0.5pt}\end{center}

Kita dapat membangun sebuah fungsi yang dapat melakukan proses integrasi menggunakan metode Heun. Berikut adalah sintaks yang digunakan:

\begin{Shaded}
\begin{Highlighting}[]
\NormalTok{heun }\OtherTok{\textless{}{-}} \ControlFlowTok{function}\NormalTok{(f, x0, y0, h, n, }\AttributeTok{iter=}\DecValTok{1}\NormalTok{)\{}
\NormalTok{  x }\OtherTok{\textless{}{-}}\NormalTok{ x0}
\NormalTok{  y }\OtherTok{\textless{}{-}}\NormalTok{ y0}
  
  \ControlFlowTok{for}\NormalTok{(i }\ControlFlowTok{in} \DecValTok{1}\SpecialCharTok{:}\NormalTok{n)\{}
\NormalTok{    ypred0 }\OtherTok{\textless{}{-}} \FunctionTok{f}\NormalTok{(x0,y0)}
\NormalTok{    ypred1 }\OtherTok{\textless{}{-}}\NormalTok{ y0 }\SpecialCharTok{+}\NormalTok{ h}\SpecialCharTok{*}\NormalTok{ypred0}
\NormalTok{    ypred2 }\OtherTok{\textless{}{-}} \FunctionTok{f}\NormalTok{(x0}\SpecialCharTok{+}\NormalTok{h,ypred1)}
\NormalTok{    ykor }\OtherTok{\textless{}{-}}\NormalTok{ y0 }\SpecialCharTok{+}\NormalTok{ h}\SpecialCharTok{*}\NormalTok{(ypred0}\SpecialCharTok{+}\NormalTok{ypred2)}\SpecialCharTok{/}\DecValTok{2}
    \ControlFlowTok{if}\NormalTok{(iter}\SpecialCharTok{!=}\DecValTok{1}\NormalTok{)\{}
      \ControlFlowTok{for}\NormalTok{(i }\ControlFlowTok{in} \DecValTok{1}\SpecialCharTok{:}\NormalTok{iter)\{}
\NormalTok{        ykor }\OtherTok{\textless{}{-}}\NormalTok{ y0 }\SpecialCharTok{+}\NormalTok{ h}\SpecialCharTok{*}\NormalTok{(ypred0}\SpecialCharTok{+}\FunctionTok{f}\NormalTok{(x0}\SpecialCharTok{+}\NormalTok{h,ykor))}\SpecialCharTok{/}\DecValTok{2}
\NormalTok{      \}}
\NormalTok{    \}}
\NormalTok{    y0 }\OtherTok{\textless{}{-}}\NormalTok{ ykor}
\NormalTok{    x0 }\OtherTok{\textless{}{-}}\NormalTok{ x0 }\SpecialCharTok{+}\NormalTok{ h}
\NormalTok{    x }\OtherTok{\textless{}{-}} \FunctionTok{c}\NormalTok{(x, x0)}
\NormalTok{    y }\OtherTok{\textless{}{-}} \FunctionTok{c}\NormalTok{(y, y0)}
\NormalTok{  \}}
  
  \FunctionTok{return}\NormalTok{(}\FunctionTok{data.frame}\NormalTok{(}\AttributeTok{x=}\NormalTok{x,}\AttributeTok{y=}\NormalTok{y))}
\NormalTok{\}}
\end{Highlighting}
\end{Shaded}

\begin{example}
\protect\hypertarget{exm:heunexmp}{}\label{exm:heunexmp}Selesaikan kembali persamaan yang ditampilkan pada Contoh \ref{exm:eulerexmp} menggunakan metode Heun!
\end{example}

\textbf{Jawab}:

Contoh perhitungan secara manual menggunakan metode Heun untuk sekali iterasi adalah sebagai berikut:

\[
f'\left(0;1\right)=\frac{1}{\left(2\cdot 0\right)+1}=1
\]

\[
y_{1}^0=1+0,05\cdot1=1,05
\]

\[
y'_1=f'\left(0,05;1,05\right)=\frac{1,05}{\left(2\cdot 0,05\right)+1}=0,9545455
\]

\[
y_1=1+0,05\cdot\frac{1+0,9545455}{2}=1,047727
\]

Penyelesaian persamaan tersebut menggunakan fungsi \texttt{heun()} dengan iterasi pada nilai koreksi sebanyak 1 kali disajikan pada sintaks berikut:

\begin{Shaded}
\begin{Highlighting}[]
\NormalTok{num }\OtherTok{\textless{}{-}} \FunctionTok{heun}\NormalTok{(f1, }\AttributeTok{x0=}\DecValTok{0}\NormalTok{, }\AttributeTok{y0=}\DecValTok{1}\NormalTok{, }\AttributeTok{h=}\FloatTok{0.05}\NormalTok{, }\AttributeTok{n=}\DecValTok{100}\NormalTok{)}
\end{Highlighting}
\end{Shaded}

\begin{figure}

{\centering \includegraphics[width=0.8\linewidth]{Algoritma_Pemrograman_R_files/figure-latex/heunviz-1} 

}

\caption{Visualisasi integrasi numerik dengan metode Heun dan metode analitik}\label{fig:heunviz}
\end{figure}

\hypertarget{midpt}{%
\subsection{Metode Titik Tengah}\label{midpt}}

Metode titik tengah menggunakan setengah \emph{step size} pada metode Euler untuk melakukan estimasi terhadap integral suatu persamaan diferensial. Metode ini melakukan perhitungan melalui dua tahapan yaitu: menghitung nilai estimasi integral pada setengah \emph{step size}(Persamaan \eqref{eq:mid1}) dan menghitung nilai integral menggunkan hasil perhitungan setengah \emph{step size} sebelumnya (Persamaan \eqref{eq:mid2}).

\begin{equation}
y_{i+\frac{1}{2}}=y_i+f\left(x_i,y_i\right)\frac{h}{2}
  \label{eq:mid1}
\end{equation}

\begin{equation}
y_{i+1}=y_i+f\left(x_{i+\frac{1}{2}},y_{i,\frac{1}{2}}\right)h
  \label{eq:mid2}
\end{equation}

\begin{center}\rule{0.5\linewidth}{0.5pt}\end{center}

\textbf{Algoritma Metode Tengah}

\begin{enumerate}
\def\labelenumi{\arabic{enumi}.}
\tightlist
\item
  Tentukan titik awal integrasi \(x_0\) dan \(y_0\).
\item
  Tentukan jumlah iterasi \(n\) dan \emph{step size} \(h\) yang digunakan.
\item
  Lakukan integrasi pada setengah tahapan iterasi menggunakan Persamaan \eqref{eq:mid1}.
\item
  Lakukan iterasi pada setengah tahapan selanjutnya menggunakan Persamaan \eqref{eq:mid2}.
\end{enumerate}

\begin{center}\rule{0.5\linewidth}{0.5pt}\end{center}

Berdasarkan algoritma tersebut, kita dapat membangun sebuah fungsi pada \texttt{R} yang adapat digunakan untuk menyelesaikan persamaan diferensial menggunakan metode titik tengah. Berikut adalah sintaks yang digunakan:

\begin{Shaded}
\begin{Highlighting}[]
\NormalTok{midpt }\OtherTok{\textless{}{-}} \ControlFlowTok{function}\NormalTok{(f, x0, y0, h, n)\{}
\NormalTok{  x }\OtherTok{\textless{}{-}}\NormalTok{ x0}
\NormalTok{  y }\OtherTok{\textless{}{-}}\NormalTok{ y0}
  
  \ControlFlowTok{for}\NormalTok{(i }\ControlFlowTok{in} \DecValTok{1}\SpecialCharTok{:}\NormalTok{n)\{}
\NormalTok{    s1 }\OtherTok{\textless{}{-}}\NormalTok{ y0 }\SpecialCharTok{+} \FunctionTok{f}\NormalTok{(x0,y0) }\SpecialCharTok{*}\NormalTok{ h}\SpecialCharTok{/}\DecValTok{2}
\NormalTok{    s2 }\OtherTok{\textless{}{-}}\NormalTok{ h }\SpecialCharTok{*} \FunctionTok{f}\NormalTok{(x0}\SpecialCharTok{+}\NormalTok{h}\SpecialCharTok{/}\DecValTok{2}\NormalTok{,s1)}
\NormalTok{    y0 }\OtherTok{\textless{}{-}}\NormalTok{ y0 }\SpecialCharTok{+}\NormalTok{ s2}
\NormalTok{    x0 }\OtherTok{\textless{}{-}}\NormalTok{ x0 }\SpecialCharTok{+}\NormalTok{ h}
\NormalTok{    x }\OtherTok{\textless{}{-}} \FunctionTok{c}\NormalTok{(x, x0)}
\NormalTok{    y }\OtherTok{\textless{}{-}} \FunctionTok{c}\NormalTok{(y, y0)}
\NormalTok{  \}}
  
  \FunctionTok{return}\NormalTok{(}\FunctionTok{data.frame}\NormalTok{(}\AttributeTok{x=}\NormalTok{x,}\AttributeTok{y=}\NormalTok{y))}
\NormalTok{\}}
\end{Highlighting}
\end{Shaded}

\begin{example}
\protect\hypertarget{exm:midexmp}{}\label{exm:midexmp}Selesaikan kembali persamaan yang ditampilkan pada Contoh \ref{exm:eulerexmp} menggunakan metode titik tengah!
\end{example}

\textbf{Jawab}:

Contoh perhitungan secara manual menggunakan metode titik tengah untuk sekali iterasi adalah sebagai berikut:

\[
y_{\frac{1}{2}}=1+\frac{1}{\left(2\cdot 0\right)+1}\cdot\frac{0,05}{2}=1,025
\]

\[
y_{1}=1+\frac{1,025}{\left(2\cdot 0,025\right)+1}\cdot0,05=1,0488
\]

Penyelesaian persamaan tersebut menggunakan fungsi \texttt{midpt()} dengan iterasi pada nilai koreksi sebanyak 1 kali disajikan pada sintaks berikut:

\begin{Shaded}
\begin{Highlighting}[]
\NormalTok{num }\OtherTok{\textless{}{-}} \FunctionTok{midpt}\NormalTok{(f1, }\AttributeTok{x0=}\DecValTok{0}\NormalTok{, }\AttributeTok{y0=}\DecValTok{1}\NormalTok{, }\AttributeTok{h=}\FloatTok{0.05}\NormalTok{, }\AttributeTok{n=}\DecValTok{100}\NormalTok{)}
\end{Highlighting}
\end{Shaded}

\begin{figure}

{\centering \includegraphics[width=0.8\linewidth]{Algoritma_Pemrograman_R_files/figure-latex/midviz-1} 

}

\caption{Visualisasi integrasi numerik dengan metode titik tengah dan metode analitik}\label{fig:midviz}
\end{figure}

\hypertarget{rungekutta}{%
\subsection{Metode Runge-Kutta Orde 4}\label{rungekutta}}

Runge-Kutta orde 4 merupakan metode yang paling populer dalam penyelesaian persamaan diferensial. Metode ini dapat memperoleh akurasi deret Taylor tanpa memerlukan diferensiasi orde yang lebih tinggi. Metode Runge-Kutta orde 4 dituliskan ke dalam Persamaan \eqref{eq:rk}.

\begin{equation}
y_{i+1}=y_i+\frac{1}{6}\left(k_1+2k_2+2k_3+k_4\right)h
  \label{eq:rk}
\end{equation}

dimana

\begin{equation}
k_1=f\left(x_i,y_i\right)
  \label{eq:rk1}
\end{equation}

\begin{equation}
k_2=f\left(x_i+\frac{1}{2}h,y_i+\frac{1}{2}k_1h\right)
  \label{eq:rk2}
\end{equation}

\begin{equation}
k_3=f\left(x_i+\frac{1}{2}h,y_i+\frac{1}{2}k_2h\right)
  \label{eq:rk3}
\end{equation}

\begin{equation}
k_4=f\left(x_i+h,y_i+k_3h\right)
  \label{eq:rk4}
\end{equation}

\begin{center}\rule{0.5\linewidth}{0.5pt}\end{center}

\textbf{Algoritma Metode Tengah}

\begin{enumerate}
\def\labelenumi{\arabic{enumi}.}
\tightlist
\item
  Tentukan titik awal integrasi \(x_0\) dan \(y_0\).
\item
  Tentukan jumlah iterasi \(n\) dan \emph{step size} \(h\) yang digunakan.
\item
  Lakukan integrasi menggunakan Persamaan \eqref{eq:rk}.
\end{enumerate}

\begin{center}\rule{0.5\linewidth}{0.5pt}\end{center}

Berdasarkan algoritma tersebut, kita dapat membangun sebuah fungsi pada \texttt{R} yang adapat digunakan untuk menyelesaikan persamaan diferensial menggunakan metode Runge-Kutta orde 4. Berikut adalah sintaks yang digunakan:

\begin{Shaded}
\begin{Highlighting}[]
\NormalTok{rk4 }\OtherTok{\textless{}{-}} \ControlFlowTok{function}\NormalTok{(f, x0, y0, h, n)\{}
\NormalTok{  x }\OtherTok{\textless{}{-}}\NormalTok{ x0}
\NormalTok{  y }\OtherTok{\textless{}{-}}\NormalTok{ y0}
  
  \ControlFlowTok{for}\NormalTok{(i }\ControlFlowTok{in} \DecValTok{1}\SpecialCharTok{:}\NormalTok{n)\{}
\NormalTok{    k1 }\OtherTok{\textless{}{-}} \FunctionTok{f}\NormalTok{(x0,y0)}
\NormalTok{    k2 }\OtherTok{\textless{}{-}} \FunctionTok{f}\NormalTok{(x0}\FloatTok{+0.5}\SpecialCharTok{*}\NormalTok{h,y0}\FloatTok{+0.5}\SpecialCharTok{*}\NormalTok{k1}\SpecialCharTok{*}\NormalTok{h)}
\NormalTok{    k3 }\OtherTok{\textless{}{-}} \FunctionTok{f}\NormalTok{(x0}\FloatTok{+0.5}\SpecialCharTok{*}\NormalTok{h,y0}\FloatTok{+0.5}\SpecialCharTok{*}\NormalTok{k2}\SpecialCharTok{*}\NormalTok{h)}
\NormalTok{    k4 }\OtherTok{\textless{}{-}} \FunctionTok{f}\NormalTok{(x0}\SpecialCharTok{+}\NormalTok{h,y0}\SpecialCharTok{+}\NormalTok{k3}\SpecialCharTok{*}\NormalTok{h)}
\NormalTok{    y0 }\OtherTok{\textless{}{-}}\NormalTok{ y0 }\SpecialCharTok{+}\NormalTok{ (}\DecValTok{1}\SpecialCharTok{/}\DecValTok{6}\NormalTok{)}\SpecialCharTok{*}\NormalTok{(k1}\SpecialCharTok{+}\DecValTok{2}\SpecialCharTok{*}\NormalTok{k2}\SpecialCharTok{+}\DecValTok{2}\SpecialCharTok{*}\NormalTok{k3}\SpecialCharTok{+}\NormalTok{k4)}\SpecialCharTok{*}\NormalTok{h}
\NormalTok{    x0 }\OtherTok{\textless{}{-}}\NormalTok{ x0 }\SpecialCharTok{+}\NormalTok{ h}
\NormalTok{    x }\OtherTok{\textless{}{-}} \FunctionTok{c}\NormalTok{(x, x0)}
\NormalTok{    y }\OtherTok{\textless{}{-}} \FunctionTok{c}\NormalTok{(y, y0)}
\NormalTok{  \}}
  
  \FunctionTok{return}\NormalTok{(}\FunctionTok{data.frame}\NormalTok{(}\AttributeTok{x=}\NormalTok{x,}\AttributeTok{y=}\NormalTok{y))}
\NormalTok{\}}
\end{Highlighting}
\end{Shaded}

\begin{example}
\protect\hypertarget{exm:rkexmp}{}\label{exm:rkexmp}Selesaikan kembali persamaan yang ditampilkan pada Contoh \ref{exm:eulerexmp} menggunakan metode Runge-Kutta orde 4!
\end{example}

\textbf{Jawab}:

Contoh perhitungan secara manual menggunakan metode titik tengah untuk sekali iterasi adalah sebagai berikut:

\[
k_1=\frac{1}{\left(2\cdot 0\right)+1}=1
\]

\[
k_2=\frac{1+\frac{1}{2}\cdot1\cdot0,05}{\left(2\cdot 0\cdot\frac{1}{2}\cdot0,05\right)+1}=1,025
\]

\[
k_2=\frac{1+\frac{1}{2}\cdot1,025\cdot0,05}{\left(2\cdot 0\cdot\frac{1}{2}\cdot0,05\right)+1}=1,025625
\]

\[
k_2=\frac{1+1,025625\cdot0,05}{\left(2\cdot 0\cdot0,05\right)+1}=1,051281
\]

\[
y_1=1+\frac{1}{6}\left(1+2\cdot1,025+2\cdot1,025625+1,051281\right)0,05=1.051271
\]

\begin{Shaded}
\begin{Highlighting}[]
\DecValTok{1}\SpecialCharTok{+}\NormalTok{(}\DecValTok{1}\SpecialCharTok{/}\DecValTok{6}\NormalTok{)}\SpecialCharTok{*}\NormalTok{(}\DecValTok{1}\SpecialCharTok{+}\DecValTok{2}\SpecialCharTok{*}\FloatTok{1.025}\SpecialCharTok{+}\DecValTok{2}\SpecialCharTok{*}\FloatTok{1.025625+1.051281}\NormalTok{)}\SpecialCharTok{*}\FloatTok{0.05}
\end{Highlighting}
\end{Shaded}

\begin{verbatim}
## [1] 1.051
\end{verbatim}

Iterasi dapat pula dilakukan dengan menggunakan fungsi \texttt{rk4()}. Berikut adalah sintaks yang digunakan:

\begin{Shaded}
\begin{Highlighting}[]
\NormalTok{num }\OtherTok{\textless{}{-}} \FunctionTok{rk4}\NormalTok{(f1, }\AttributeTok{x0=}\DecValTok{0}\NormalTok{, }\AttributeTok{y0=}\DecValTok{1}\NormalTok{, }\AttributeTok{h=}\FloatTok{0.05}\NormalTok{, }\AttributeTok{n=}\DecValTok{100}\NormalTok{)}
\end{Highlighting}
\end{Shaded}

\begin{figure}

{\centering \includegraphics[width=0.8\linewidth]{Algoritma_Pemrograman_R_files/figure-latex/rkviz-1} 

}

\caption{Visualisasi integrasi numerik dengan metode Runge-Kutta orde 4 dan metode analitik}\label{fig:rkviz}
\end{figure}

\hypertarget{mml}{%
\subsection{Metode Multistep Linier}\label{mml}}

Jika metode Runge-Kutta mengalami kesulitan karena terlalu banyak evaluasi fungsi yang digunakan, masuk akal untuk bertanya apakah kita dapat menggunakan kembali beberapa evaluasi fungsi sebelumnya, yang sudah kita buat. Sebagai contoh, kita ingin tahu apakah kita dapat menggunakan kembali estimasi \(f\left(0,1\right)\) dan \(f\left(0,2\right)\) untuk memperkirakan nilai \(f\left(0,3\right)\). Jika kita dapat menggunakan kembali perkiraan sebelumnya, kita dapat memperoleh akurasi tambahan tanpa menimbulkan penalti kinerja yang terkait dengan evaluasi fungsi tambahan. Metode multistep linier dikembangkan untuk mengatasi masalah ini.

Di satu sisi, metode multistep linier dasar untuk persamaan diferensial hanya mencakup satu titik \(x_i\), dalam perhitungan \(x_i + 1\). Ini persis bagaimana fungsi metode Euler dan metode Euler merupakan metode multistep linier dasar. Metode selanjutnya menggunakan \(x_i − 1\) dan \(x_i\) untuk menghitung \(x_i + 1\). Metode Adams-Bashforth menggunakan tambahan berbobot, termasuk bobot negatif, dari langkah dan poin untuk sampai pada langkah berikutnya. Seperti metode numerik lainnya, bobot muncul dari interpolasi polinomial titik yang tersedia.

Metode Adam-Bashforth orde 2 didasarkan pada Persamaan \eqref{eq:adam}.

\begin{equation}
y_{i+2}=y_{i+1}+\frac{h}{2}\left(3f\left(x_{i+1},y_{i+1}\right)-f\left(x_i,y_i\right)\right)
  \label{eq:adam}
\end{equation}

Pendekatan ini melakukan interpolasi antara titik sebelumnya untuk memperkirakan titik ketiga dalam grup. Titik ketiga ini menjadi titik tengah dari iterasi berikutnya saat seluruh proses berlanjut. Karena nilai sebelumnya disimpan dan digunakan kembali, evaluasi fungsi tambahan tidak diperlukan. Jika metode Runge-Kutta dapat dibandingkan dengan tip-toeing melalui bidang vektor, maka metode Adams-Bashforth dapat sama dibandingkan dengan menjalankan melalui bidang vektor. Namun, ini tidak berarti metode Adams-Bashforth lebih unggul.

\begin{center}\rule{0.5\linewidth}{0.5pt}\end{center}

\textbf{Algoritma Metode Tengah}

\begin{enumerate}
\def\labelenumi{\arabic{enumi}.}
\tightlist
\item
  Tentukan titik awal integrasi \(x_0\) dan \(y_0\).
\item
  Tentukan jumlah iterasi \(n\) dan \emph{step size} \(h\) yang digunakan.
\item
  Lakukan pendekatan pada iterasi ke-1 menggunakan metode Euler.
\item
  Lakukan integrasi ke-2 sampai n menggunakan Persamaan \eqref{eq:adam}.
\end{enumerate}

\begin{center}\rule{0.5\linewidth}{0.5pt}\end{center}

Berdasarkan algoritma tersebut, kita dapat membangun sebuah fungsi pada \texttt{R} yang adapat digunakan untuk menyelesaikan persamaan diferensial menggunakan metode multistep linier. Berikut adalah sintaks yang digunakan:

\begin{Shaded}
\begin{Highlighting}[]
\NormalTok{adambashforth }\OtherTok{\textless{}{-}} \ControlFlowTok{function}\NormalTok{(f, x0, y0, h, n)\{}
  \CommentTok{\# pendekatan Euler untuk x1 dan y1}
\NormalTok{  y1 }\OtherTok{\textless{}{-}}\NormalTok{ y0 }\SpecialCharTok{+}\NormalTok{ h}\SpecialCharTok{*}\FunctionTok{f}\NormalTok{(x0,y0)}
\NormalTok{  x1 }\OtherTok{\textless{}{-}}\NormalTok{ x0 }\SpecialCharTok{+}\NormalTok{ h}
  
\NormalTok{  x }\OtherTok{\textless{}{-}} \FunctionTok{c}\NormalTok{(x0,x1)}
\NormalTok{  y }\OtherTok{\textless{}{-}} \FunctionTok{c}\NormalTok{(y0,y1)}
\NormalTok{  n }\OtherTok{\textless{}{-}}\NormalTok{ n}\DecValTok{{-}1}
  
  \ControlFlowTok{for}\NormalTok{(i }\ControlFlowTok{in} \DecValTok{1}\SpecialCharTok{:}\NormalTok{n)\{}
\NormalTok{    yn }\OtherTok{\textless{}{-}}\NormalTok{ y1 }\SpecialCharTok{+} \FloatTok{1.5}\SpecialCharTok{*}\NormalTok{h}\SpecialCharTok{*}\FunctionTok{f}\NormalTok{(x1,y1) }\SpecialCharTok{{-}} \FloatTok{0.5}\SpecialCharTok{*}\NormalTok{h}\SpecialCharTok{*}\FunctionTok{f}\NormalTok{(x0,y0)}
\NormalTok{    xn }\OtherTok{\textless{}{-}}\NormalTok{ x1 }\SpecialCharTok{+}\NormalTok{ h}
    
\NormalTok{    y0 }\OtherTok{\textless{}{-}}\NormalTok{ y1}
\NormalTok{    x0 }\OtherTok{\textless{}{-}}\NormalTok{ x1}
\NormalTok{    y1 }\OtherTok{\textless{}{-}}\NormalTok{ yn}
\NormalTok{    x1 }\OtherTok{\textless{}{-}}\NormalTok{ xn}
    
\NormalTok{    y }\OtherTok{\textless{}{-}} \FunctionTok{c}\NormalTok{(y,y1)}
\NormalTok{    x }\OtherTok{\textless{}{-}} \FunctionTok{c}\NormalTok{(x,x1)}
\NormalTok{  \}}
  
  \FunctionTok{return}\NormalTok{(}\FunctionTok{data.frame}\NormalTok{(}\AttributeTok{x=}\NormalTok{x,}\AttributeTok{y=}\NormalTok{y))}
\NormalTok{\}}
\end{Highlighting}
\end{Shaded}

\begin{example}
\protect\hypertarget{exm:adamexmp}{}\label{exm:adamexmp}Selesaikan kembali persamaan yang ditampilkan pada Contoh \ref{exm:eulerexmp} menggunakan metode Runge-Kutta orde 4!
\end{example}

\textbf{Jawab}:

Untuk melakukan iterasi, kita dapat menggunakan fungsi \texttt{adambashforth()}. Berikut adalah sintaks yang digunakan:

\begin{Shaded}
\begin{Highlighting}[]
\NormalTok{num }\OtherTok{\textless{}{-}} \FunctionTok{rk4}\NormalTok{(f1, }\AttributeTok{x0=}\DecValTok{0}\NormalTok{, }\AttributeTok{y0=}\DecValTok{1}\NormalTok{, }\AttributeTok{h=}\FloatTok{0.05}\NormalTok{, }\AttributeTok{n=}\DecValTok{100}\NormalTok{)}
\end{Highlighting}
\end{Shaded}

\begin{figure}

{\centering \includegraphics[width=0.8\linewidth]{Algoritma_Pemrograman_R_files/figure-latex/adamviz-1} 

}

\caption{Visualisasi integrasi numerik dengan metode Adam-Bashfoth orde 2 dan metode analitik}\label{fig:adamviz}
\end{figure}

\hypertarget{sysdiff}{%
\section{Sistem Persamaan Diferensial}\label{sysdiff}}

Sistem persamaan diferensial akan sering pembaca temui dalam pemodelan sistem dinamik. Pada proses pemodelan sistem tersebut akan ditemukan aksi-interasi antar komponennya yang dinyatakan ke dalam suatu sistem persamaan diferensial.

Pada fungsi \texttt{rk4sys()} disusun sebuah fungsi pada \texttt{R} untuk melakukan iterasi terhadap sistem persamaan diferensial menggunakan metode Runge-Kutta orde 4. Berikut adalah sintaks yang digunakan:

\begin{Shaded}
\begin{Highlighting}[]
\NormalTok{rk4sys }\OtherTok{\textless{}{-}} \ControlFlowTok{function}\NormalTok{(f, x0, y0, h, n)\{}
\NormalTok{  x }\OtherTok{\textless{}{-}}\NormalTok{ x0}
\NormalTok{  y }\OtherTok{\textless{}{-}}\NormalTok{ y0}
  
\NormalTok{  values }\OtherTok{\textless{}{-}} \FunctionTok{data.frame}\NormalTok{(}\AttributeTok{x=}\NormalTok{x,}\FunctionTok{t}\NormalTok{(y0))}
  \ControlFlowTok{for}\NormalTok{(i }\ControlFlowTok{in} \DecValTok{1}\SpecialCharTok{:}\NormalTok{n)\{}
\NormalTok{    k1 }\OtherTok{\textless{}{-}} \FunctionTok{f}\NormalTok{(x0,y0)}
\NormalTok{    k2 }\OtherTok{\textless{}{-}} \FunctionTok{f}\NormalTok{(x0}\FloatTok{+0.5}\SpecialCharTok{*}\NormalTok{h,y0}\FloatTok{+0.5}\SpecialCharTok{*}\NormalTok{k1}\SpecialCharTok{*}\NormalTok{h)}
\NormalTok{    k3 }\OtherTok{\textless{}{-}} \FunctionTok{f}\NormalTok{(x0}\FloatTok{+0.5}\SpecialCharTok{*}\NormalTok{h,y0}\FloatTok{+0.5}\SpecialCharTok{*}\NormalTok{k2}\SpecialCharTok{*}\NormalTok{h)}
\NormalTok{    k4 }\OtherTok{\textless{}{-}} \FunctionTok{f}\NormalTok{(x0}\SpecialCharTok{+}\NormalTok{h,y0}\SpecialCharTok{+}\NormalTok{k3}\SpecialCharTok{*}\NormalTok{h)}
\NormalTok{    y0 }\OtherTok{\textless{}{-}}\NormalTok{ y0 }\SpecialCharTok{+}\NormalTok{ (}\DecValTok{1}\SpecialCharTok{/}\DecValTok{6}\NormalTok{)}\SpecialCharTok{*}\NormalTok{(k1}\SpecialCharTok{+}\DecValTok{2}\SpecialCharTok{*}\NormalTok{k2}\SpecialCharTok{+}\DecValTok{2}\SpecialCharTok{*}\NormalTok{k3}\SpecialCharTok{+}\NormalTok{k4)}\SpecialCharTok{*}\NormalTok{h}
\NormalTok{    x0 }\OtherTok{\textless{}{-}}\NormalTok{ x0 }\SpecialCharTok{+}\NormalTok{ h}
\NormalTok{    values }\OtherTok{\textless{}{-}} \FunctionTok{rbind}\NormalTok{(values, }\FunctionTok{data.frame}\NormalTok{(}\AttributeTok{x=}\NormalTok{x0,}\FunctionTok{t}\NormalTok{(y0)))}
\NormalTok{  \}}
  
  \FunctionTok{return}\NormalTok{(values)}
\NormalTok{\}}
\end{Highlighting}
\end{Shaded}

\begin{example}
\protect\hypertarget{exm:rksysexmp}{}\label{exm:rksysexmp}Sebuah model populasi yang dibagi menjadi tiga kelompok umur: anak (0-12 tahun), melahirkan anak (13-40 tahun), dan usia (41 tahun atau lebih). Kelompok 0-12 meningkat berdasarkan kelahiran di kelompok 13-40, dan menurun dengan kematian dan lewat bagian ke kelompok 13-40. Kelompok 13--40 meningkat dengan perolehan dari kelompok 0-12, dan berkurang dengan kematian dan lewat bagian ke dalam kelompok\textgreater{} = 41. Kelompok\textgreater{} = 41 meningkat dengan keuntungan dari kelompok 13-40, dan menurun dengan kematian. Parameter dipilih untuk mewakili tingkat kelahiran dan kematian yang cukup tinggi seperti yang ditemukan di banyak masyarakat berkembang. Model interaksi tersebut dinyatakan ke dalam persamaan interkasi di bawah ini! Simulasikan dinamika populasi pada model tersebut pada 10 tahun ke depan, jika diketahui populasi semula masing-masing kelompok usia secara berurutan adalah 200,400, dan 400, serta nilai masing-masing koefisien kelahiran, kematian kelompok 1, kematian kelompok2, dan kematian kelompok 3 secara berurutan adalah 0,5;0,1;0,1;0,25!
\end{example}

\[
pop1' = b\cdot pop2'+\frac{11}{12}\cdot pop1'\left(1-d1\right)  
\]

\[
pop2' = \frac{1}{12}\cdot pop1'\cdot\left(1-d1\right)+\frac{26}{27}\cdot pop2'\cdot \left(1-d2\right)  
\]

\[
pop3, = \frac{1}{27}\cdot pop2,\cdot\left(1-d2\right)+ pop3'\cdot \left(1-d3\right)  
\]

\textbf{Jawab}:

Sistem persamaan diferensial perlu ditransformasi ke dalam bentuk sebuah fungsi pada \texttt{R}.

\begin{Shaded}
\begin{Highlighting}[]
\NormalTok{Population }\OtherTok{=} \ControlFlowTok{function}\NormalTok{(x,y) \{}
   \CommentTok{\# parameter}
\NormalTok{   b }\OtherTok{=} \FloatTok{0.5} \CommentTok{\# Birth rate in 13{-}40 group}
\NormalTok{   d1 }\OtherTok{=} \FloatTok{0.1} \CommentTok{\# Death rate of 0{-}12 group}
\NormalTok{   d2 }\OtherTok{=} \FloatTok{0.1} \CommentTok{\# Death rate of 13{-}40 group}
\NormalTok{   d3 }\OtherTok{=} \FloatTok{0.25} \CommentTok{\# Death rate of 41 and older}
   
\NormalTok{   y1 }\OtherTok{=}\NormalTok{ y[}\DecValTok{1}\NormalTok{] }\CommentTok{\# 0{-}12 group population}
\NormalTok{   y2 }\OtherTok{=}\NormalTok{ y[}\DecValTok{2}\NormalTok{] }\CommentTok{\# 13{-}40 group population}
\NormalTok{   y3 }\OtherTok{=}\NormalTok{ y[}\DecValTok{3}\NormalTok{] }\CommentTok{\# 41 and older population}
\NormalTok{   y1.new }\OtherTok{=}\NormalTok{ b}\SpecialCharTok{*}\NormalTok{y2 }\SpecialCharTok{+}\NormalTok{ (}\DecValTok{11}\SpecialCharTok{/}\DecValTok{12}\NormalTok{)}\SpecialCharTok{*}\NormalTok{y1}\SpecialCharTok{*}\NormalTok{(}\DecValTok{1} \SpecialCharTok{{-}}\NormalTok{ d1)}
\NormalTok{   y2.new }\OtherTok{=}\NormalTok{ (}\DecValTok{1}\SpecialCharTok{/}\DecValTok{12}\NormalTok{)}\SpecialCharTok{*}\NormalTok{y1}\SpecialCharTok{*}\NormalTok{(}\DecValTok{1} \SpecialCharTok{{-}}\NormalTok{ d1) }\SpecialCharTok{+}\NormalTok{ (}\DecValTok{26}\SpecialCharTok{/}\DecValTok{27}\NormalTok{)}\SpecialCharTok{*}\NormalTok{y2}\SpecialCharTok{*}\NormalTok{(}\DecValTok{1} \SpecialCharTok{{-}}\NormalTok{ d2)}
\NormalTok{   y3.new }\OtherTok{=}\NormalTok{ (}\DecValTok{1}\SpecialCharTok{/}\DecValTok{27}\NormalTok{)}\SpecialCharTok{*}\NormalTok{y2}\SpecialCharTok{*}\NormalTok{(}\DecValTok{1} \SpecialCharTok{{-}}\NormalTok{ d2) }\SpecialCharTok{+}\NormalTok{ y3}\SpecialCharTok{*}\NormalTok{(}\DecValTok{1} \SpecialCharTok{{-}}\NormalTok{ d3)}
   \FunctionTok{return}\NormalTok{(}\FunctionTok{c}\NormalTok{(y1.new, y2.new, y3.new))}
\NormalTok{\}}
\end{Highlighting}
\end{Shaded}

Nilai awal dituliskan seperti berikut:

\begin{Shaded}
\begin{Highlighting}[]
\CommentTok{\# Nilai awal}
\NormalTok{y }\OtherTok{=} \FunctionTok{c}\NormalTok{(}\AttributeTok{pop1=}\DecValTok{200}\NormalTok{,}\AttributeTok{pop2=}\DecValTok{400}\NormalTok{,}\AttributeTok{pop3=}\DecValTok{400}\NormalTok{)}
\end{Highlighting}
\end{Shaded}

Simulasi model ditampilkan pada sintaks berikut:

\begin{Shaded}
\begin{Highlighting}[]
\NormalTok{pop }\OtherTok{\textless{}{-}} \FunctionTok{rk4sys}\NormalTok{(Population, }\AttributeTok{x0=}\DecValTok{0}\NormalTok{, }\AttributeTok{y0=}\NormalTok{y, }\AttributeTok{h=}\FloatTok{0.1}\NormalTok{, }\AttributeTok{n=}\DecValTok{100}\NormalTok{)}
\FunctionTok{head}\NormalTok{(pop)}
\end{Highlighting}
\end{Shaded}

\begin{verbatim}
##     x  pop1  pop2  pop3
## 1 0.0 200.0 400.0 400.0
## 2 0.1 239.0 437.9 432.6
## 3 0.2 283.4 479.6 467.9
## 4 0.3 334.0 525.4 506.1
## 5 0.4 391.4 575.9 547.4
## 6 0.5 456.4 631.3 592.1
\end{verbatim}

\begin{figure}

{\centering \includegraphics[width=0.8\linewidth]{Algoritma_Pemrograman_R_files/figure-latex/sysdiffviz-1} 

}

\caption{Visualisasi hasil simulasi model dinamika populasi}\label{fig:sysdiffviz}
\end{figure}

\hypertarget{penyelesaian-persamaan-diferensial-dan-sistem-persamaan-diferensial-menggunakan-fungsi-ode}{%
\section{\texorpdfstring{Penyelesaian Persamaan Diferensial dan Sistem Persamaan Diferensial Menggunakan Fungsi \texttt{ode()}}{Penyelesaian Persamaan Diferensial dan Sistem Persamaan Diferensial Menggunakan Fungsi ode()}}\label{penyelesaian-persamaan-diferensial-dan-sistem-persamaan-diferensial-menggunakan-fungsi-ode}}

Fungsi \texttt{ode()} pada paket \texttt{deSolve} merupakan salah satu fungsi yang dapat digunakan untuk menyelesaikan persamaan dan sistem persamaan diferensial. Fungsi ini mudah digunakan serta menyediakan berbagai macam metode iterasi numerik untuk menyelesaikan persamaan diferensial. Format umum fungsi ini antara lain:

\begin{Shaded}
\begin{Highlighting}[]
\FunctionTok{ode}\NormalTok{(y, times, func, parms,method , ...)}
\end{Highlighting}
\end{Shaded}

\begin{quote}
\textbf{Catatan:}

\begin{itemize}
\item
  \texttt{y} : nilai awal (kondisi awal) suatu persamaan diferensial.
\item
  \texttt{times} : deret waktu yang terdiri dari kapan simulasi dimulai, kapan simulasi berkahir dan berapa \emph{step size yang digunakan}.
\item
  \texttt{func} : fungsi yang berisi persamaan diferensial. \emph{Return value} pada fungsi haruslah berupa list.
\item
  \texttt{parms} : list parameter yang diinputan kedalam \texttt{func}.
\item
  \texttt{method} : sebuah string yang berupa metode intgrasi yang digunakan. Metode yang tersedia dan kegunaan metode tersebut antara lain:

  \begin{itemize}
  \tightlist
  \item
    ``bdf'': menangani persamaan diferensial menggunakan formula diferensiasi mundur (\emph{bacward differensiation}) dan cocok untuk menangani kondisi \emph{stiff}. Metode ini setara dengan \texttt{method="lsode"}.
  \item
    ``bdf\_d'': menggunakan formula diferensiasi mundur yang memanfaatkan iterasi Jacobi-Newton (mengabaikan elemen diagonal Jacobian). Cocok digunakan persamaan atau sistem persamaan diferensial kondisi \emph{stiff}. Metode ini setara dengan \texttt{method="lsode",mf=23}.
  \item
    ``adams'': metode Adams yang menggunakan iterasi fungsional (tanpa menggunakan Jacobian). Cocok digunakan untuk persamaan atau sistem persamaan \emph{non stiff}. Metode ini setara dengan \texttt{method="lsode",\ mf=10}.
  \item
    ``impAdams'': metode Adams implisit yang menggunakan iterasi Newton-Raphson. Metode ini setara dengan \texttt{method="lsode",mf=12}.
  \item
    ``impAdams\_d'': metode Adams implisit yang menggunakan iterasi Jacobi-Newton. Metode ini setara dengan \texttt{method="lsode,\ mf=13"}.
  \item
    ``euler'': metode iterasi Euler
  \item
    ``rk4'': metode iterasi Runge-Kutta orde 4.
  \item
    metode lain: ``lsoda'', ``lsode'', ``lsodes'', ``lsodar'', ``vode'', ``daspk'', ``ode23'', ``ode45'', ``radau''.
  \end{itemize}
\item
  \texttt{...}: argumen tambahan untuk integrator atau method.
\end{itemize}
\end{quote}

Contoh penerapan fungsi tersebut menggunakan Contoh \ref{exm:rksysexmp}, sebagai berikut:

\begin{Shaded}
\begin{Highlighting}[]
\FunctionTok{library}\NormalTok{(deSolve)}

\CommentTok{\# sistem persamaan diferensial}
\NormalTok{Population }\OtherTok{=} \ControlFlowTok{function}\NormalTok{(t,y,param) \{}
\NormalTok{  y1 }\OtherTok{=}\NormalTok{ y[}\DecValTok{1}\NormalTok{] }\CommentTok{\# 0{-}12 group population}
\NormalTok{  y2 }\OtherTok{=}\NormalTok{ y[}\DecValTok{2}\NormalTok{] }\CommentTok{\# 13{-}40 group population}
\NormalTok{  y3 }\OtherTok{=}\NormalTok{ y[}\DecValTok{3}\NormalTok{] }\CommentTok{\# 41 and older population}
\NormalTok{  y1.new }\OtherTok{=}\NormalTok{ b}\SpecialCharTok{*}\NormalTok{y2 }\SpecialCharTok{+} \DecValTok{11}\SpecialCharTok{/}\DecValTok{12}\SpecialCharTok{*}\NormalTok{y1}\SpecialCharTok{*}\NormalTok{(}\DecValTok{1} \SpecialCharTok{{-}}\NormalTok{ d1)}
\NormalTok{  y2.new }\OtherTok{=} \DecValTok{1}\SpecialCharTok{/}\DecValTok{12}\SpecialCharTok{*}\NormalTok{y1}\SpecialCharTok{*}\NormalTok{(}\DecValTok{1} \SpecialCharTok{{-}}\NormalTok{ d1) }\SpecialCharTok{+} \DecValTok{26}\SpecialCharTok{/}\DecValTok{27}\SpecialCharTok{*}\NormalTok{y2}\SpecialCharTok{*}\NormalTok{(}\DecValTok{1} \SpecialCharTok{{-}}\NormalTok{ d2)}
\NormalTok{  y3.new }\OtherTok{=} \DecValTok{1}\SpecialCharTok{/}\DecValTok{27}\SpecialCharTok{*}\NormalTok{y2}\SpecialCharTok{*}\NormalTok{(}\DecValTok{1} \SpecialCharTok{{-}}\NormalTok{ d2) }\SpecialCharTok{+}\NormalTok{ y3}\SpecialCharTok{*}\NormalTok{(}\DecValTok{1} \SpecialCharTok{{-}}\NormalTok{ d3)}
  \FunctionTok{return}\NormalTok{(}\FunctionTok{list}\NormalTok{(}\FunctionTok{c}\NormalTok{(y1.new, y2.new, y3.new)))}
\NormalTok{\}}

\CommentTok{\# parameter}
\NormalTok{b }\OtherTok{=} \FloatTok{0.5} \CommentTok{\# Birth rate in 13{-}40 group}
\NormalTok{d1 }\OtherTok{=} \FloatTok{0.1} \CommentTok{\# Death rate of 0{-}12 group}
\NormalTok{d2 }\OtherTok{=} \FloatTok{0.1} \CommentTok{\# Death rate of 13{-}40 group}
\NormalTok{d3 }\OtherTok{=} \FloatTok{0.25} \CommentTok{\# Death rate of 41 and older}

\CommentTok{\# nilai awal}
\NormalTok{y }\OtherTok{=} \FunctionTok{c}\NormalTok{(}\AttributeTok{pop1=}\DecValTok{200}\NormalTok{, }\AttributeTok{pop2=}\DecValTok{400}\NormalTok{, }\AttributeTok{pop3=}\DecValTok{400}\NormalTok{)}

\CommentTok{\# waktu}
\NormalTok{Tahun }\OtherTok{=} \FunctionTok{seq}\NormalTok{(}\DecValTok{0}\NormalTok{,}\DecValTok{10}\NormalTok{,}\FloatTok{0.1}\NormalTok{)}

\CommentTok{\# simulasi}
\NormalTok{out }\OtherTok{=} \FunctionTok{ode}\NormalTok{(}\AttributeTok{func =}\NormalTok{ Population, }\AttributeTok{y =}\NormalTok{ y, }\AttributeTok{times =}\NormalTok{ Tahun, }
          \AttributeTok{parms =} \FunctionTok{c}\NormalTok{(b,d1,d2,d3), }\AttributeTok{method =} \StringTok{"rk4"}\NormalTok{)}
\end{Highlighting}
\end{Shaded}

\begin{figure}

{\centering \includegraphics[width=0.8\linewidth]{Algoritma_Pemrograman_R_files/figure-latex/sysdiffviz2-1} 

}

\caption{Visualisasi hasil simulasi model dinamika populasi menggunakan paket desolve}\label{fig:sysdiffviz2}
\end{figure}

\hypertarget{pasialdiff}{%
\section{Persamaan Diferensial Parsial}\label{pasialdiff}}

Persamaan diferensial parsial (PDE) banyak dijumpai pada pemodelan transport polutan dalam bidang teknik lingkungan. Persamaan diferensial parsial merupakan persamaan diferensial yang melibatkan lebih dari satu variabel independen, biasanya variabel waktu dan satu atau lebih variabel posisi atau beberapa variabel spasial. PDE diklasifikasikan menjadi 3 jenis: parabolik (\emph{time-dependent} dan difusif), hiperbolik (\emph{time-dependent} dan gelombang), dan eliptik (\emph{time-independent}). Dalam penyelesaian PDE pada umumnya kita menggunakan metode FTCS (\emph{forward in time, centered in space}). Untuk memahami definisi tersebut, pembaca dapat membaca kembali Chapter \ref{finitediff}.

\hypertarget{persamaan-difusi}{%
\subsection{Persamaan Difusi}\label{persamaan-difusi}}

Persamaan difusi merupakan contoh PDE parabolik. Persamaan difusi satu dimensi spasial ditampilkan pada Persamaan \eqref{eq:difusi}.

\begin{equation}
\frac{\partial C}{\partial t}=D\frac{\partial^2C}{\partial x^2}
  \label{eq:difusi}
\end{equation}

Persamaan tersebut mirip dengan persamaan konduksi panas. Pada persamaan konduksi panas, variabel konsentrasi \(C\) diganti dengan variabel temperatur \(T\), dan koefisien difusi D diganti dengan koefsien difusi termal K.

Untuk menyelesaikan Persamaan \eqref{eq:difusi}, persamaan tersebut diubah ke dalam bentuk metode beda hingga, dimana turunan waktu menggunakan pendekatan Euler (metode beda hingga maju) dan turunan variabel spasial diubah ke dalam bentuk pendekatan titik pusat. Proses diskretisasi Persamaan \eqref{eq:difusi}, ditampilkan pada Persamaan \eqref{eq:difusi2}.

\begin{equation}
\frac{C\left(i+1,j\right)-C\left(i,j\right)}{\Delta t}=D\frac{C\left(i,j+1\right)+C\left(i,j-1\right)-2C\left(i,j\right)}{\Delta x^2}
  \label{eq:difusi2}
\end{equation}

dimana \(i\) merupakan \emph{step} untuk variabel waktu \(t\) dan \(j\) merupakan \emph{step} untuk variabel spasial \(x\).

Persamaan \eqref{eq:difusi2} dapat disusun kembali sehingga menjadi Persamaan \eqref{eq:difusi3} yang menyatakan persamaan konsentrasi \(C\) pada saat \(i+1\) pada posisi \(j\).

\begin{equation}
C\left(i+1,j\right)=C\left(i,j\right)+A\left[C\left(i,j+1\right)+C\left(i,j-1\right)-2C\left(i,j\right)\right]
  \label{eq:difusi3}
\end{equation}

dimana

\begin{equation}
A=D\frac{\Delta t}{\Delta x^2}
  \label{eq:difusi4}
\end{equation}

Untuk stabilitas komputasi, pemilihan peningkatan waktu terhadap jarak yang dinyatakan pada nilai \(A\) harus \(\le\frac{1}{2}\).

Pada contoh berikut, kita akan melakukan simulasi menggunakan Persamaan \eqref{eq:difusi3}. Parameter yang digunakan dan nilai awal yang digunakan dinyatakan pada sintaks berikut:

\begin{Shaded}
\begin{Highlighting}[]
\NormalTok{dt    }\OtherTok{\textless{}{-}} \DecValTok{3}                \CommentTok{\# Timestep, s}
\NormalTok{dx    }\OtherTok{\textless{}{-}} \FloatTok{0.1}              \CommentTok{\# Distance step, cm}
\NormalTok{D     }\OtherTok{\textless{}{-}} \FloatTok{1e{-}4}             \CommentTok{\# Diffusion coeff, cm\^{}2/s}

\CommentTok{\# Cek apakah syarat stabilitas terpenuhi}
\NormalTok{D}\SpecialCharTok{*}\NormalTok{dt}\SpecialCharTok{/}\NormalTok{dx}\SpecialCharTok{\^{}}\DecValTok{2} \SpecialCharTok{\textless{}=} \FloatTok{0.5}
\end{Highlighting}
\end{Shaded}

\begin{verbatim}
## [1] TRUE
\end{verbatim}

\begin{Shaded}
\begin{Highlighting}[]
\CommentTok{\# Desain grid points}
\NormalTok{L     }\OtherTok{\textless{}{-}} \DecValTok{1}                \CommentTok{\# Length from {-}L/2 to L/2}
\NormalTok{n     }\OtherTok{\textless{}{-}}\NormalTok{ L}\SpecialCharTok{/}\NormalTok{dx }\SpecialCharTok{+} \DecValTok{1}         \CommentTok{\# Number of grid points}
\NormalTok{x     }\OtherTok{\textless{}{-}} \FunctionTok{seq}\NormalTok{(}\SpecialCharTok{{-}}\NormalTok{L}\SpecialCharTok{/}\DecValTok{2}\NormalTok{,L}\SpecialCharTok{/}\DecValTok{2}\NormalTok{,dx) }\CommentTok{\# Location of grid points}
\NormalTok{steps }\OtherTok{\textless{}{-}} \DecValTok{30}               \CommentTok{\# Number of iterations}
\NormalTok{time  }\OtherTok{\textless{}{-}} \DecValTok{0}\SpecialCharTok{:}\NormalTok{steps}
\end{Highlighting}
\end{Shaded}

Langkah selanjutnya adalah inisiasi konsentrasi awal, dimana seluruh konsentrasi awal pada tiap grid adalah nol kecuali pada grid pusat.

\begin{Shaded}
\begin{Highlighting}[]
\NormalTok{C }\OtherTok{\textless{}{-}} \FunctionTok{matrix}\NormalTok{(}\FunctionTok{rep}\NormalTok{(}\DecValTok{0}\NormalTok{, (steps}\SpecialCharTok{+}\DecValTok{1}\NormalTok{)}\SpecialCharTok{*}\NormalTok{n), }\AttributeTok{nrow =}\NormalTok{ steps}\SpecialCharTok{+}\DecValTok{1}\NormalTok{, }\AttributeTok{ncol =}\NormalTok{ n)}
\NormalTok{C[}\DecValTok{1}\NormalTok{, }\FunctionTok{round}\NormalTok{(n}\SpecialCharTok{/}\DecValTok{2}\NormalTok{)] }\OtherTok{\textless{}{-}} \DecValTok{1}\SpecialCharTok{/}\NormalTok{dx  }\CommentTok{\# initial spike at central point}
\end{Highlighting}
\end{Shaded}

Simulasi selanjutnya dilakukan dengan melakukan iterasi pada \emph{grid points} konsentrasi yang telah dibuat.

\begin{Shaded}
\begin{Highlighting}[]
\CommentTok{\# Loop over desired number of time steps}
\ControlFlowTok{for}\NormalTok{(i }\ControlFlowTok{in} \DecValTok{1}\SpecialCharTok{:}\NormalTok{(steps}\DecValTok{{-}1}\NormalTok{))\{}
  \ControlFlowTok{for}\NormalTok{(j }\ControlFlowTok{in} \DecValTok{2}\SpecialCharTok{:}\NormalTok{(n}\DecValTok{{-}1}\NormalTok{))\{}
\NormalTok{    C[i}\SpecialCharTok{+}\DecValTok{1}\NormalTok{,j] }\OtherTok{\textless{}{-}}\NormalTok{ C[i,j] }\SpecialCharTok{+}\NormalTok{ (D}\SpecialCharTok{*}\NormalTok{dt}\SpecialCharTok{/}\NormalTok{dx}\SpecialCharTok{\^{}}\DecValTok{2}\NormalTok{)}\SpecialCharTok{*}\NormalTok{(C[i,j}\SpecialCharTok{+}\DecValTok{1}\NormalTok{] }\SpecialCharTok{+}\NormalTok{ C[i,j}\DecValTok{{-}1}\NormalTok{] }\SpecialCharTok{{-}} \DecValTok{2}\SpecialCharTok{*}\NormalTok{C[i,j])}
\NormalTok{  \}}
\NormalTok{\}}
\end{Highlighting}
\end{Shaded}

Visualisasi dari hasil simulasi tersebut ditampilkan pada Gambar \ref{fig:difusiviz}.

\begin{figure}

{\centering \includegraphics[width=0.8\linewidth]{Algoritma_Pemrograman_R_files/figure-latex/difusiviz-1} 

}

\caption{Visualisasi 3D simulasi difusi partikulat}\label{fig:difusiviz}
\end{figure}

\hypertarget{persamaan-gelombang}{%
\subsection{Persamaan Gelombang}\label{persamaan-gelombang}}

Persamaan gelombang 1 dimensi ditampilkan pada Persamaan \eqref{eq:wave1}.

\begin{equation}
\frac{\partial^2W}{\partial t^2}=c^2\frac{\partial^2W}{\partial x^2}
  \label{eq:wave1}
\end{equation}

dimana \(W\) merupakan pemindahan dan \(c\) merupakan kecepatan gelombang. Persamaan \eqref{eq:wave1} merupakan bentuk PDE hiperbolik. Versi lebih sederhana dari Persamaan \eqref{eq:wave1} adalah persamaan adveksi yang ditampilkan pada Persamaan \eqref{eq:wave2}.

\begin{equation}
\frac{\partial y}{\partial t}=-c\frac{\partial y}{\partial x}
  \label{eq:wave2}
\end{equation}

Persamaan \eqref{eq:wave2} menggambarkan evolusi pada bidang skalar \(y\left(x,y\right)\) dibawa oleh gelombang dengan kecepatan konstan \(c\) dan bergerak dari kiri ke kanan jika \(c>0\). Persamaan adveksi merupakan contoh palin sederhana dari persamaan konservasi flux.

Diskretisasi Persamaan \eqref{eq:wave2} ditampilkan pada Persamaan \eqref{eq:wave3}.

\begin{equation}
y\left(i+1,j\right)=y\left(i,j\right)-\frac{c\Delta t}{2\Delta x}\left[y\left(i,j+1\right)-y\left(i,j-1\right)\right]
  \label{eq:wave3}
\end{equation}

Pada contoh berikut, kita akan melakukan simulasi menggunakan Persamaan \eqref{eq:wave2}. Parameter yang digunakan dan nilai awal yang digunakan dinyatakan pada sintaks berikut:

\begin{Shaded}
\begin{Highlighting}[]
\NormalTok{dt    }\OtherTok{\textless{}{-}} \FloatTok{0.002}              \CommentTok{\# Timestep, s}
\NormalTok{L     }\OtherTok{\textless{}{-}} \DecValTok{1}                  \CommentTok{\# Length from {-}L/2 to L/2}
\NormalTok{n     }\OtherTok{\textless{}{-}} \DecValTok{50}                 \CommentTok{\# Number of grid points}
\NormalTok{v     }\OtherTok{\textless{}{-}} \DecValTok{1}                  \CommentTok{\# Wavespeed, cm/s}
\NormalTok{dx    }\OtherTok{\textless{}{-}}\NormalTok{ L}\SpecialCharTok{/}\NormalTok{n                }\CommentTok{\# Distance step, cm}
\NormalTok{x     }\OtherTok{\textless{}{-}}\NormalTok{ (}\DecValTok{1}\SpecialCharTok{:}\NormalTok{n}\FloatTok{{-}0.5}\NormalTok{)}\SpecialCharTok{*}\NormalTok{dx}\SpecialCharTok{{-}}\NormalTok{L}\SpecialCharTok{/}\DecValTok{2}   \CommentTok{\# Location of grid points}
\NormalTok{steps }\OtherTok{\textless{}{-}}\NormalTok{ L}\SpecialCharTok{/}\NormalTok{(v}\SpecialCharTok{*}\NormalTok{dt)           }\CommentTok{\# Number of iterations}
\NormalTok{time  }\OtherTok{\textless{}{-}} \DecValTok{0}\SpecialCharTok{:}\NormalTok{steps  }
\NormalTok{sig   }\OtherTok{\textless{}{-}} \FloatTok{0.1}                \CommentTok{\# Standard deviation of initial Gaussian wave}
\NormalTok{amp0  }\OtherTok{\textless{}{-}} \FunctionTok{exp}\NormalTok{(}\SpecialCharTok{{-}}\NormalTok{x}\SpecialCharTok{\^{}}\DecValTok{2}\SpecialCharTok{/}\NormalTok{(}\DecValTok{2}\SpecialCharTok{*}\NormalTok{sig}\SpecialCharTok{\^{}}\DecValTok{2}\NormalTok{))}\CommentTok{\# Initial Gaussian amplitude}
\end{Highlighting}
\end{Shaded}

Kondisi awal dan kondisi batas periodik dinyatakan sebagai berikut:

\begin{Shaded}
\begin{Highlighting}[]
\NormalTok{C }\OtherTok{\textless{}{-}} \FunctionTok{matrix}\NormalTok{(}\FunctionTok{rep}\NormalTok{(}\DecValTok{0}\NormalTok{, (steps}\SpecialCharTok{+}\DecValTok{1}\NormalTok{)}\SpecialCharTok{*}\NormalTok{n), }\AttributeTok{nrow =}\NormalTok{ steps}\SpecialCharTok{+}\DecValTok{1}\NormalTok{, }\AttributeTok{ncol =}\NormalTok{ n)}
\NormalTok{C[}\DecValTok{1}\NormalTok{, ] }\OtherTok{\textless{}{-}}\NormalTok{ amp0}
\NormalTok{jplus1 }\OtherTok{\textless{}{-}} \FunctionTok{c}\NormalTok{(}\DecValTok{2}\SpecialCharTok{:}\NormalTok{n,}\DecValTok{1}\NormalTok{)}
\NormalTok{jminus1 }\OtherTok{\textless{}{-}} \FunctionTok{c}\NormalTok{(n,}\DecValTok{1}\SpecialCharTok{:}\NormalTok{(n}\DecValTok{{-}1}\NormalTok{))}
\end{Highlighting}
\end{Shaded}

Langkah selanjutnya melakukan interasi untuk melihat perubahan nilai pada \emph{grid points}.

\begin{Shaded}
\begin{Highlighting}[]
\ControlFlowTok{for}\NormalTok{(i }\ControlFlowTok{in} \DecValTok{1}\SpecialCharTok{:}\NormalTok{steps)\{}
  \ControlFlowTok{for}\NormalTok{(j }\ControlFlowTok{in} \DecValTok{1}\SpecialCharTok{:}\NormalTok{n)\{}
\NormalTok{    C[i}\SpecialCharTok{+}\DecValTok{1}\NormalTok{,j] }\OtherTok{\textless{}{-}}\NormalTok{ C[i,j] }\SpecialCharTok{+}\NormalTok{ (v}\SpecialCharTok{*}\NormalTok{dt}\SpecialCharTok{/}\NormalTok{(}\DecValTok{2}\SpecialCharTok{*}\NormalTok{dx))}\SpecialCharTok{*}\NormalTok{(C[i,jplus1[j]] }\SpecialCharTok{{-}}\NormalTok{ C[i,jminus1[j]])}
\NormalTok{  \}}
\NormalTok{\}}
\end{Highlighting}
\end{Shaded}

Visualisasi dari hasil simulasi tersebut ditampilkan pada Gambar \ref{fig:waveviz}.

\begin{figure}

{\centering \includegraphics[width=0.8\linewidth]{Algoritma_Pemrograman_R_files/figure-latex/waveviz-1} 

}

\caption{Visualisasi simulasi adveksi}\label{fig:waveviz}
\end{figure}

\hypertarget{laplace}{%
\subsection{Persamaan Laplace}\label{laplace}}

Persamaan Laplace dalam dua dimensi disajikan pada Persamaan \eqref{eq:laplace1}.

\begin{equation}
\frac{\partial V}{\partial x^2}+\frac{\partial V}{\partial y^2}=0
  \label{eq:laplace1}
\end{equation}

Persamaan \eqref{eq:laplace1} merupakan contoh tipe ketiga PDE, persamaan elips. Persoalan ini sering muncul dalam bidang elektrostatik, gravitasi, dan bidang lain di mana potensi \(V\) harus dihitung sebagai fungsi posisi. Jika ada muatan atau massa dalam ruang, dan jika kita menggeneralisasi ke tiga dimensi, persamaannya menjadi persamaan Poisson

\begin{equation}
\frac{\partial V}{\partial x^2}+\frac{\partial V}{\partial y^2}+\frac{\partial V}{\partial z^2}=0
  \label{eq:laplace2}
\end{equation}

Bergantung pada geometri masalahnya, Persamaan \eqref{eq:laplace2} juga dapat ditulis dalam koordinat bola, silindris, atau lainnya.

Untuk menyelesaikan persamaan eliptik jenis ini, persamaan harus diberi syarat batas. Biasanya dengan menentukan bahwa titik, garis, atau permukaan tertentu dipertahankan pada nilai konstan potensial. Kemudian potensi di titik lain disesuaikan sampai mencapai perkiraan yang diinginkan. (Dalam kasus yang jarang terjadi, persamaan dengan kondisi batas dapat diselesaikan dengan tepat secara analitis; tetapi biasanya solusi perkiraan harus cukup.)

Ada banyak pendekatan untuk solusi numerik dari persamaan Laplace. Mungkin yang paling sederhana adalah Jacobi, di mana titik-titik interior secara berturut-turut didekati dengan rata-rata titik-titik di sekitarnya, sedangkan titik-batas dibatasi pada nilai-nilai tetap dan yang ditentukan. Kita asumsikan sebagai contoh bidang persegi, dibatasi oleh \(\left(0,1\right)\) dalam arah \(x\) dan \(y\), di mana ujung pada \(y = 1\) dipertahankan pada \(V = 1\) dan tiga tepi lainnya dipertahankan pada \(V = 0\). Kita buat tebakan awal yang paling sederhana untuk potensi di titik interior, tetapi ini akan disamakan saat solusinya menyatu.

Pada contoh berikut, kita akan melakukan simulasi persamaan Laplace menggunakan metode Jacobi. Parameter yang digunakan dalam simulasi adalah sebagai berikut:

\begin{Shaded}
\begin{Highlighting}[]
\NormalTok{n   }\OtherTok{\textless{}{-}} \DecValTok{30}                 \CommentTok{\# Number of grid points/side}
\NormalTok{L   }\OtherTok{\textless{}{-}} \DecValTok{1}                  \CommentTok{\# Length of a side}
\NormalTok{dx  }\OtherTok{\textless{}{-}}\NormalTok{ L}\SpecialCharTok{/}\NormalTok{(n}\DecValTok{{-}1}\NormalTok{)            }\CommentTok{\# Grid spacing}
\NormalTok{x   }\OtherTok{\textless{}{-}}\NormalTok{ y    }\OtherTok{\textless{}{-}} \DecValTok{0}\SpecialCharTok{:}\NormalTok{(n}\DecValTok{{-}1}\NormalTok{)}\SpecialCharTok{*}\NormalTok{dx }\CommentTok{\# x and y coorfinates}
\end{Highlighting}
\end{Shaded}

Tebakan awal untuk profil voltase adalah sebagai berikut:

\begin{Shaded}
\begin{Highlighting}[]
\NormalTok{V0  }\OtherTok{\textless{}{-}} \DecValTok{1}
\NormalTok{V   }\OtherTok{\textless{}{-}} \FunctionTok{matrix}\NormalTok{(V0}\SpecialCharTok{/}\DecValTok{2}\SpecialCharTok{*}\FunctionTok{sin}\NormalTok{(}\DecValTok{2}\SpecialCharTok{*}\NormalTok{pi}\SpecialCharTok{*}\NormalTok{x}\SpecialCharTok{/}\NormalTok{L)}\SpecialCharTok{*}\FunctionTok{sin}\NormalTok{(}\DecValTok{2}\SpecialCharTok{*}\NormalTok{pi}\SpecialCharTok{*}\NormalTok{y}\SpecialCharTok{/}\NormalTok{L),}
              \AttributeTok{nrow =}\NormalTok{ n, }\AttributeTok{ncol =}\NormalTok{ n, }\AttributeTok{byrow =} \ConstantTok{TRUE}\NormalTok{)}
\end{Highlighting}
\end{Shaded}

Kondisi batas di tetapkan seperti berikut:

\begin{Shaded}
\begin{Highlighting}[]
\NormalTok{V[}\DecValTok{1}\NormalTok{,] }\OtherTok{\textless{}{-}} \DecValTok{0}
\NormalTok{V[n,] }\OtherTok{\textless{}{-}} \DecValTok{0}
\NormalTok{V[,}\DecValTok{1}\NormalTok{] }\OtherTok{\textless{}{-}} \DecValTok{0}
\NormalTok{V[,n] }\OtherTok{\textless{}{-}}\NormalTok{ V0}\SpecialCharTok{*}\FunctionTok{rep}\NormalTok{(}\DecValTok{1}\NormalTok{,n)}
\end{Highlighting}
\end{Shaded}

Selanjutnya visualisasikan tebakan awal.

\begin{figure}

{\centering \includegraphics[width=0.8\linewidth]{Algoritma_Pemrograman_R_files/figure-latex/laplaceviz-1} 

}

\caption{Visualisasi tebakan awal solusi persaman Laplace}\label{fig:laplaceviz}
\end{figure}

Proses iterasi menggunakan metode Jacobi ditampilkan pada sintaks berikut:

\begin{Shaded}
\begin{Highlighting}[]
\NormalTok{newV  }\OtherTok{\textless{}{-}}\NormalTok{ V}
\NormalTok{itmax }\OtherTok{\textless{}{-}}\NormalTok{ n}\SpecialCharTok{\^{}}\DecValTok{2}
\NormalTok{tol   }\OtherTok{\textless{}{-}} \FloatTok{1e{-}4}
\ControlFlowTok{for}\NormalTok{(it }\ControlFlowTok{in} \DecValTok{1}\SpecialCharTok{:}\NormalTok{itmax)\{}
\NormalTok{  dVsum }\OtherTok{\textless{}{-}} \DecValTok{0}
  \ControlFlowTok{for}\NormalTok{(i }\ControlFlowTok{in} \DecValTok{2}\SpecialCharTok{:}\NormalTok{(n}\DecValTok{{-}1}\NormalTok{))\{}
    \ControlFlowTok{for}\NormalTok{(j }\ControlFlowTok{in} \DecValTok{2}\SpecialCharTok{:}\NormalTok{(n}\DecValTok{{-}1}\NormalTok{))\{}
\NormalTok{      newV[i,j] }\OtherTok{\textless{}{-}} \FloatTok{0.25}\SpecialCharTok{*}\NormalTok{(V[i}\DecValTok{{-}1}\NormalTok{,j]}\SpecialCharTok{+}\NormalTok{V[i}\SpecialCharTok{+}\DecValTok{1}\NormalTok{,j]}\SpecialCharTok{+}\NormalTok{V[i,j}\DecValTok{{-}1}\NormalTok{]}\SpecialCharTok{+}\NormalTok{V[i,j}\SpecialCharTok{+}\DecValTok{1}\NormalTok{])}
\NormalTok{      dVsum }\OtherTok{\textless{}{-}}\NormalTok{ dVsum }\SpecialCharTok{+} \FunctionTok{abs}\NormalTok{(}\DecValTok{1}\SpecialCharTok{{-}}\NormalTok{V[i,j]}\SpecialCharTok{/}\NormalTok{newV[i,j])}
\NormalTok{    \}}
\NormalTok{  \}}
\NormalTok{  V }\OtherTok{\textless{}{-}}\NormalTok{ newV}
\NormalTok{  dV }\OtherTok{=}\NormalTok{ dVsum}\SpecialCharTok{/}\NormalTok{(n}\DecValTok{{-}2}\NormalTok{)}\SpecialCharTok{\^{}}\DecValTok{2} \CommentTok{\# Average deviation from prev. value}
  \ControlFlowTok{if}\NormalTok{(dV }\SpecialCharTok{\textless{}}\NormalTok{ tol) }\ControlFlowTok{break} \CommentTok{\# Desired tolerance achived}
\NormalTok{\}}

\NormalTok{it }\CommentTok{\# Iterations to achieve convergence to tol}
\end{Highlighting}
\end{Shaded}

\begin{verbatim}
## [1] 419
\end{verbatim}

\begin{Shaded}
\begin{Highlighting}[]
\NormalTok{dV}
\end{Highlighting}
\end{Shaded}

\begin{verbatim}
## [1] 9.908e-05
\end{verbatim}

Hasil simulasi selanjutnya di visualisasikan kembali.

\begin{figure}

{\centering \includegraphics[width=0.8\linewidth]{Algoritma_Pemrograman_R_files/figure-latex/laplaceviz2-1} 

}

\caption{Visualisasi hasil simulasi solusi persaman Lplace}\label{fig:laplaceviz2}
\end{figure}

\texttt{\#\#\ Penyelesaian\ Persamaan\ Diferensial\ Parsial\ Menggunakan\ Paket}ReacTran`

Package \texttt{ReacTran} memfasilitasi pemodelan transport reaktif dalam 1, 2, dan 3 dimensi. Paket ini ``berisi rutinitas yang memungkinkan pengembangan model transportasi reaktif dalam sistem perairan (sungai, danau, lautan), media berpori (agregat flok, sedimen, \ldots) dan bahkan organisme yang diidealkan.

Pada paket \texttt{ReacTran} terdapat sejumlah fasilitas fungsi, antara lain:

\begin{itemize}
\tightlist
\item
  Fungsi untuk menyiapkan kisi beda hingga (1D atau 2D)
\item
  Fungsi untuk melampirkan parameter dan properti ke kisi ini (1D atau 2D)
\item
  Fungsi untuk menghitung jangka waktu transport adveksi-difusi di atas grid (1D, 2D, 3D)
\item
  Berbagai fungsi lainnya.
\end{itemize}

Saat paket \texttt{ReacTran} dimuat, paket ini juga memuat dua paket pendukung, yaitu: \texttt{rootSolve} dan \texttt{deSolve}. Paket \texttt{rootSolve} berguna untuk memecahkan persamaan diferensial untuk kondisi tunak baik persamaan diferensial uniform atau multikomponen (1D, 2D, dan 3D). Sedangkan \texttt{deSolve} berguna untuk menyediakan fungsi yang digunakan untuk memperoleh penyelesaian numerik persamaan diferensial biasa (ODE), persamaan diferensial parsial (PDE), persamaan aljabar diferensial (DAE) dan persamaan \emph{delay differential}.

\hypertarget{setup.grid.1d}{%
\subsection{\texorpdfstring{\texttt{setup.grid.1D()}}{setup.grid.1D()}}\label{setup.grid.1d}}

Fungsi \texttt{setup.grid.1D()} digunakan untuk membentuk kisi satu dimensi. Secara sederhana fungsi ini membagi ruang satu dimensi \(L\) antara \(x.up\) dan \(x.down\) menjadi sejumlah \(N\) kisi sebesar \texttt{dx.1}. Format umum fungsi tersebut, adalah sebagai berikut:

\begin{Shaded}
\begin{Highlighting}[]
\FunctionTok{setup.grid.1D}\NormalTok{(}\AttributeTok{x.up=}\DecValTok{0}\NormalTok{, }\AttributeTok{x.down=}\ConstantTok{NULL}\NormalTok{,}\AttributeTok{L=}\ConstantTok{NULL}\NormalTok{, }
              \AttributeTok{N=}\ConstantTok{NULL}\NormalTok{, }\AttributeTok{dx.1=}\ConstantTok{NULL}\NormalTok{, }
              \AttributeTok{p.dx.1=} \FunctionTok{rep}\NormalTok{(}\DecValTok{1}\NormalTok{,}\FunctionTok{length}\NormalTok{(L)), }
              \AttributeTok{max.dx.1=}\NormalTok{L, }\AttributeTok{dx.N=}\ConstantTok{NULL}\NormalTok{, }
              \AttributeTok{p.dx.N=}\FunctionTok{rep}\NormalTok{(}\DecValTok{1}\NormalTok{,}\FunctionTok{length}\NormalTok{(L)), }
              \AttributeTok{max.dx.N=}\NormalTok{L)}
\end{Highlighting}
\end{Shaded}

\begin{quote}
\textbf{Catatan:}

\begin{itemize}
\tightlist
\item
  \texttt{x.up} : posisi hilir.
\item
  \texttt{x.down} : posisi hulu.
\item
  \texttt{L} : \texttt{x.down}-\texttt{x.up}.
\item
  \texttt{N} : jumlah kisi = L/dx.1.
\end{itemize}
\end{quote}

Pada situasi yang lebih kompleks, ukuran sel atau kisi dapat bervariasi, atau dapat pula lebih dari satu zona. Kondisi ini dijelaskan lebih jauh pada laman bantuan fungsi.

Nilai yang dihasilkan dari fungsi \texttt{setup.grid.1D()} termasuk \texttt{x.mid} (vektor sepanjang \(N\) yang menyatakan posisi titik tengah) dan \texttt{x.int} (vektor sepanjang \(N+1\) yang menyatakan posisi antar muka antara kisi sel dimana flux diukur).

Fungsi plot untuk \texttt{grid.1D()} memvisualissikan baik posisi sel dan ketebalan kotak, menampilkan \texttt{x.mid} dan \texttt{x.int}. Contoh di halaman bantuan menunjukkan perilaku ini.

\texttt{setup.grid.1D()} berfungsi sebagai titik awal untuk \texttt{setup.grid.2D}, yang membuat kisi di atas domain persegi panjang yang ditentukan oleh dua kisi 1D ortogonal.

\hypertarget{setup.prop.1d}{%
\subsection{\texorpdfstring{\texttt{setup.prop.1D()}}{setup.prop.1D()}}\label{setup.prop.1d}}

Banyak model transportasi akan melibatkan kisi-kisi dengan properti konstan. Tetapi jika beberapa properti yang mempengaruhi difusi atau adveksi bervariasi dengan posisi di grid, variasi dapat digabungkan dengan fungsi \texttt{setup.prop.1D()} (atau \texttt{setup.prop.2D()} dalam dua dimensi).

Diberikan fungsi matematis atau matriks data, fungsi \texttt{setup.prop.1D()} menghitung nilai properti yang diminati di tengah sel kisi dan pada antarmuka antar sel. Format fungsi tersebut adalah sebagai berikut:

\begin{Shaded}
\begin{Highlighting}[]
\FunctionTok{setup.prop.1D}\NormalTok{(}\AttributeTok{func=}\ConstantTok{NULL}\NormalTok{, }\AttributeTok{value=}\ConstantTok{NULL}\NormalTok{, }\AttributeTok{xy=}\ConstantTok{NULL}\NormalTok{,}
              \AttributeTok{interpolate=}\StringTok{"spline"}\NormalTok{,grid, ...)}
\end{Highlighting}
\end{Shaded}

\begin{quote}
\textbf{Catatan:}

\begin{itemize}
\tightlist
\item
  \texttt{func} : fungsi yang mengatur ketergantungan spasial properti.
\item
  \texttt{value} : nilai konstan yang diberikan ke properti jika tidak ada ketergantungan spasial.
\item
  \texttt{xy} : matriks data di mana kolom pertama memberikan posisi, dan kolom kedua memberikan nilai-nilai yang diinterpolasi di atas grid.
\item
  \texttt{iterpolate} : metode interpolasi yang digunakan (spline atau linier).
\item
  \texttt{grid}: objek yang dibentuk melalui fungsi \texttt{setup.grid.1D()}.
\item
  \texttt{...}: argumen tambahan \texttt{func}.
\end{itemize}
\end{quote}

\hypertarget{tran.1d}{%
\subsection{\texorpdfstring{\texttt{tran.1D()}}{tran.1D()}}\label{tran.1d}}

Fungsi ini menghitung istilah transportasi --- laju perubahan konsentrasi akibat difusi dan adveksi --- dalam model cairan 1D (fraksi volume = 1) atau padatan berpori (fraksi volume dapat bervariasi dan \textless1).\texttt{tran.1D()} juga digunakan untuk masalah dalam geometri bola atau silinder, meskipun dalam kasus ini antarmuka sel jaringan akan memiliki area variabel. Format fungsi ini adalah sebagai berikut:

\begin{Shaded}
\begin{Highlighting}[]
\FunctionTok{tran.1D}\NormalTok{(C, }\AttributeTok{C.up =}\NormalTok{ C[}\DecValTok{1}\NormalTok{], }\AttributeTok{C.down =}\NormalTok{ C[}\FunctionTok{length}\NormalTok{(C)], }
        \AttributeTok{flux.up =} \ConstantTok{NULL}\NormalTok{, }\AttributeTok{flux.down =} \ConstantTok{NULL}\NormalTok{, }
        \AttributeTok{a.bl.up =} \ConstantTok{NULL}\NormalTok{, }\AttributeTok{a.bl.down =} \ConstantTok{NULL}\NormalTok{, }\AttributeTok{D =} \DecValTok{0}\NormalTok{, }
        \AttributeTok{v =} \DecValTok{0}\NormalTok{, }\AttributeTok{AFDW =} \DecValTok{1}\NormalTok{, }\AttributeTok{VF =} \DecValTok{1}\NormalTok{, }\AttributeTok{A =} \DecValTok{1}\NormalTok{, dx, }
        \AttributeTok{full.check =} \ConstantTok{FALSE}\NormalTok{, }\AttributeTok{full.output =} \ConstantTok{FALSE}\NormalTok{)}
\end{Highlighting}
\end{Shaded}

\begin{quote}
\textbf{Catatan:}

\begin{itemize}
\tightlist
\item
  \texttt{C} : vektor konsentrasi pada titik tengah kisi sel.
\item
  \texttt{C.up}, \texttt{C.down} : konsentrasi pada hulu dan hilir batas.
\item
  \texttt{flux.up},\texttt{flux.down} : flux dari dan keluar sistem pada hulu dan hilir batas.
\item
  Jika terdapat tranport konvektif sepanjang hulu dan hilir batas, \texttt{a.bl.up} dan \texttt{a.bl.down} merupakan koefisiennya.
\item
  \texttt{D}: Koefisien difusi.
\item
  \texttt{v}: koefisien adveksi.
\item
  \texttt{VF}: fraksi volume.
\item
  \texttt{A}: fraksi area.
\item
  \texttt{dx}: ketebalan kisi, baik nilai konstan atau vektor.
\item
  \texttt{full.check}, \texttt{full.output}: nilai logik untuk memeriksa konsistensi dan mengatur output dari perhitungan. Keduanya \texttt{FALSE} secara default.
\end{itemize}
\end{quote}

Ketika \texttt{full.output\ =\ FALSE}, nilai-nilai yang dikembalikan oleh \texttt{trans.1D()} adalah \texttt{dC}, yaitu: laju perubahan \texttt{C} di pusat setiap sel kisi karena transportasi, dan \texttt{flux.up} dan \texttt{flux.down}, yatu: fluks masuk dan keluar dari model di batas hulu dan hilir.

\texttt{ReacTran} juga memiliki fungsi untuk memperkirakan istilah difusi dan adveksi dalam model dua dan tiga dimensi, dan dalam koordinat silinder dan kutub. Jumlah input tumbuh dengan dimensi, tetapi input pada dasarnya sama seperti pada kasus 1D. Lihat halaman bantuan untuk \texttt{tran.2D()}, \texttt{tran.3D()}, \texttt{tran.cylindrical()}, dan \texttt{tran.polar()}.

Namun penyempurnaan lain adalah fungsi \texttt{tran.volume.1D()}, yang memperkirakan istilah transportasi volumetrik dalam model 1D. Berbeda dengan \texttt{tran.1D()}, yang menggunakan fluks (massa per satuan luas per satuan waktu), \texttt{tran.volume.1D()} menggunakan aliran (massa per satuan waktu). Ini berguna untuk memodelkan saluran yang area lintas bagiannya berubah, ketika perubahan area tidak perlu dimodelkan secara eksplisit. Ini juga memungkinkan input lateral dari saluran samping.

\hypertarget{ode.1d-atau-steady.1d}{%
\subsection{\texorpdfstring{\texttt{ode.1D()} atau \texttt{steady.1D()}}{ode.1D() atau steady.1D()}}\label{ode.1d-atau-steady.1d}}

Setelah kisi telah diatur dan properti ditetapkan dan model transport telah diformulasikan dengan \texttt{tran.1D()} (atau analog 2D atau 3D-nya), maka \texttt{ReacTran} memanggil \texttt{ode.1D()} dari paket \texttt{deSolve} jika solusi \emph{time-dependent} diperlukan, atau \texttt{stable.1D()} dari paket \texttt{rootSolve} jika solusi kondisi-stabil diinginkan. Sistem ODE yang dihasilkan dari metode pendekatan garis biasanya \emph{sparse} dan \emph{non-stiff}. Integrator dalam \texttt{deSolve}, seperti ``lsoda'' (metode default 1D) sangat cocok untuk menangani sistem persamaan tersebut. Jika sistem ODE \emph{non-stiff}, maka ``adams'' umumnya merupakan metode yang baik.

\hypertarget{contoh-penerapan-paket-reactran}{%
\section{\texorpdfstring{Contoh Penerapan Paket \texttt{ReacTran}}{Contoh Penerapan Paket ReacTran}}\label{contoh-penerapan-paket-reactran}}

\hypertarget{persamaan-adveksi-difuksi-1d}{%
\subsection{Persamaan Adveksi-Difuksi 1D}\label{persamaan-adveksi-difuksi-1d}}

Pada contoh kali ini, kita akan memodifikasi persamaan difusi satu dimensi yang telah dilakukan sebelumnya dengan menambahkan bagian adveksi serta diselesaikan menggunakan paket \texttt{Reactran}.

Siapkan kisi menggunakan fungsi \texttt{setup.grid.1D()} dan persiapkan nilai parameter persamaan.

\begin{Shaded}
\begin{Highlighting}[]
\FunctionTok{library}\NormalTok{(ReacTran)}
\NormalTok{N     }\OtherTok{\textless{}{-}} \DecValTok{100}          \CommentTok{\# Number of grid cells}
\NormalTok{xgrid }\OtherTok{\textless{}{-}} \FunctionTok{setup.grid.1D}\NormalTok{(}\AttributeTok{x.up =} \DecValTok{0}\NormalTok{, }\AttributeTok{x.down =} \DecValTok{1}\NormalTok{, }\AttributeTok{N =}\NormalTok{ N)}
\NormalTok{x     }\OtherTok{\textless{}{-}}\NormalTok{ xgrid}\SpecialCharTok{$}\NormalTok{x.mid  }\CommentTok{\# Midpoints of grid cells}
\NormalTok{D     }\OtherTok{\textless{}{-}} \FloatTok{1e{-}4}         \CommentTok{\# Diffusion coefficient}
\NormalTok{v     }\OtherTok{\textless{}{-}} \FloatTok{0.1}          \CommentTok{\# Advection velocity}
\end{Highlighting}
\end{Shaded}

Bentuk fungsi yang menyatakan persamaan adveksi-difusi.

\begin{Shaded}
\begin{Highlighting}[]
\NormalTok{Diffusion }\OtherTok{\textless{}{-}} \ControlFlowTok{function}\NormalTok{(t, Y, parms) \{}
\NormalTok{    tran}\OtherTok{\textless{}{-}}\FunctionTok{tran.1D}\NormalTok{(}\AttributeTok{C=}\NormalTok{Y,}\AttributeTok{C.up=}\DecValTok{0}\NormalTok{,}\AttributeTok{C.down=}\DecValTok{0}\NormalTok{,}\AttributeTok{D=}\NormalTok{D,}\AttributeTok{v=}\NormalTok{v,}\AttributeTok{dx=}\NormalTok{ xgrid)}
    \FunctionTok{list}\NormalTok{(}\AttributeTok{dY =}\NormalTok{ tran}\SpecialCharTok{$}\NormalTok{dC, }\AttributeTok{flux.up =}\NormalTok{ tran}\SpecialCharTok{$}\NormalTok{flux.up,}
        \AttributeTok{flux.down =}\NormalTok{ tran }\SpecialCharTok{$}\NormalTok{flux.down)}
\NormalTok{\}}
\end{Highlighting}
\end{Shaded}

Inisiasi konsentrasi awal pada kisi sel.

\begin{Shaded}
\begin{Highlighting}[]
\NormalTok{Yini }\OtherTok{\textless{}{-}} \FunctionTok{rep}\NormalTok{(}\DecValTok{0}\NormalTok{,N) }\CommentTok{\# Initial concentration = 0}
\NormalTok{Yini[}\DecValTok{2}\NormalTok{] }\OtherTok{\textless{}{-}} \DecValTok{100}   \CommentTok{\# Except in the second cell}
\end{Highlighting}
\end{Shaded}

Lakukan perhitungandengn menggunakan \emph{time step} 0,01.

\begin{Shaded}
\begin{Highlighting}[]
\CommentTok{\# Calculate for 5 time units}
\NormalTok{times }\OtherTok{\textless{}{-}} \FunctionTok{seq}\NormalTok{(}\AttributeTok{from =} \DecValTok{0}\NormalTok{, }\AttributeTok{to =} \DecValTok{5}\NormalTok{, }\AttributeTok{by =} \FloatTok{0.01}\NormalTok{)}
\NormalTok{out }\OtherTok{\textless{}{-}} \FunctionTok{ode.1D}\NormalTok{(}\AttributeTok{y =}\NormalTok{ Yini, }\AttributeTok{times =}\NormalTok{ times, }\AttributeTok{func =}\NormalTok{ Diffusion,}
             \AttributeTok{parms =} \ConstantTok{NULL}\NormalTok{,}\AttributeTok{dimens =}\NormalTok{ N)}
\end{Highlighting}
\end{Shaded}

Visualisasikan hasil perhitungan.

\begin{figure}

{\centering \includegraphics[width=0.8\linewidth]{Algoritma_Pemrograman_R_files/figure-latex/advdifviz-1} 

}

\caption{Visualisasi hasil simulasi persamaan adveksi-difusi menggunakan paket ReacTran}\label{fig:advdifviz}
\end{figure}

\hypertarget{persamaan-gelombang-1d}{%
\subsection{Persamaan Gelombang 1D}\label{persamaan-gelombang-1d}}

Persamaan \eqref{eq:wave1} dapat diselesaikan dengan cara yang sama dengan persamaan difusi dengan mengatur nilai \(c^2=D\), memisalkan \(W=u\) dan \(\frac{\partial u}{\partial t}=v\), dan menyelesaikannya dengan cara yang familiar untuk variabel berpasangan \(\left(u,v\right)\). Di sini kita misalkan persamaan gelombang 1D untuk senar yang dipetik, yang awalnya ditahan pada \(0\) amplitudo untuk \(x <−25\) dan \(x> 25\), dan direntangkan secara linear hingga maksimum pada \(x = 0\). \texttt{ode.1D()} digunakan untuk menyelesaikan set himpunan ODE simultan dengan \(c = 1\).

Langkah pertama yang harus dilakukan adalah melakukan penetapan parameter.

\begin{Shaded}
\begin{Highlighting}[]
\NormalTok{dx    }\OtherTok{\textless{}{-}} \FloatTok{0.2}          \CommentTok{\# Spacing of grid cells}
\CommentTok{\# String extends from {-}100 to +100}
\NormalTok{xgrid }\OtherTok{\textless{}{-}} \FunctionTok{setup.grid.1D}\NormalTok{(}\AttributeTok{x.up =} \SpecialCharTok{{-}}\DecValTok{100}\NormalTok{, }
                       \AttributeTok{x.down =} \DecValTok{100}\NormalTok{, }\AttributeTok{dx.1 =}\NormalTok{ dx)}
\NormalTok{x     }\OtherTok{\textless{}{-}}\NormalTok{ xgrid}\SpecialCharTok{$}\NormalTok{x.mid  }\CommentTok{\# midpoints of grid cells}
\NormalTok{N     }\OtherTok{\textless{}{-}}\NormalTok{ xgrid}\SpecialCharTok{$}\NormalTok{N      }\CommentTok{\# number of grid cells}
\end{Highlighting}
\end{Shaded}

Menetapkan kondisi awal seperti ketinggian senar dan kecepatan.

\begin{Shaded}
\begin{Highlighting}[]
\NormalTok{uini  }\OtherTok{\textless{}{-}} \FunctionTok{rep}\NormalTok{(}\DecValTok{0}\NormalTok{,N) }\CommentTok{\# String height vector before stretching}
\NormalTok{vini  }\OtherTok{\textless{}{-}} \FunctionTok{rep}\NormalTok{(}\DecValTok{0}\NormalTok{,N) }\CommentTok{\# Initial string velocity vector}
\NormalTok{displ }\OtherTok{\textless{}{-}} \DecValTok{10}       \CommentTok{\# Initial displacement at center of string}
\CommentTok{\# Impose initial triangular height profile on string between {-} 25}
\ControlFlowTok{for}\NormalTok{(i }\ControlFlowTok{in} \DecValTok{1}\SpecialCharTok{:}\NormalTok{N) \{}
  \ControlFlowTok{if}\NormalTok{ (x[i] }\SpecialCharTok{\textgreater{}} \SpecialCharTok{{-}}\DecValTok{25} \SpecialCharTok{\&}\NormalTok{ x[i] }\SpecialCharTok{\textless{}=} \DecValTok{0}\NormalTok{) uini[i] }\OtherTok{=}\NormalTok{ displ}\SpecialCharTok{/}\DecValTok{25}\SpecialCharTok{*}\NormalTok{(}\DecValTok{25} \SpecialCharTok{+}\NormalTok{ x[i])}\ControlFlowTok{else}
  \ControlFlowTok{if}\NormalTok{ (x[i] }\SpecialCharTok{\textgreater{}} \DecValTok{0} \SpecialCharTok{\&}\NormalTok{ x[i] }\SpecialCharTok{\textless{}} \DecValTok{25}\NormalTok{) uini[i] }\OtherTok{=}\NormalTok{ displ}\SpecialCharTok{/}\DecValTok{25}\SpecialCharTok{*}\NormalTok{(}\DecValTok{25} \SpecialCharTok{{-}}\NormalTok{ x[i])}
\NormalTok{\}}
\NormalTok{yini  }\OtherTok{\textless{}{-}} \FunctionTok{c}\NormalTok{(uini, vini)}
\end{Highlighting}
\end{Shaded}

Menetapkan deret waktu yang digunakan dalam simulasi.

\begin{Shaded}
\begin{Highlighting}[]
\NormalTok{times }\OtherTok{\textless{}{-}} \FunctionTok{seq}\NormalTok{(}\AttributeTok{from =} \DecValTok{0}\NormalTok{, }\AttributeTok{to =} \DecValTok{50}\NormalTok{, }\AttributeTok{by =} \DecValTok{1}\NormalTok{)}
\end{Highlighting}
\end{Shaded}

Membangun fungsi yang akan diselesaikan secara numerik.

\begin{Shaded}
\begin{Highlighting}[]
\NormalTok{ wave }\OtherTok{\textless{}{-}} \ControlFlowTok{function}\NormalTok{(t,y,parms) \{}
\NormalTok{    u }\OtherTok{\textless{}{-}}\NormalTok{ y[}\DecValTok{1}\SpecialCharTok{:}\NormalTok{N] }\CommentTok{\# Separate displacement and velocity vectors}
\NormalTok{    v }\OtherTok{\textless{}{-}}\NormalTok{ y[(N}\SpecialCharTok{+}\DecValTok{1}\NormalTok{)}\SpecialCharTok{:}\NormalTok{(}\DecValTok{2}\SpecialCharTok{*}\NormalTok{N)]}
\NormalTok{    du}\OtherTok{\textless{}{-}}\NormalTok{v}
\NormalTok{    dv}\OtherTok{\textless{}{-}}\FunctionTok{tran.1D}\NormalTok{(}\AttributeTok{C=}\NormalTok{u,}\AttributeTok{C.up=}\DecValTok{0}\NormalTok{,}\AttributeTok{C.down=}\DecValTok{0}\NormalTok{,}\AttributeTok{D=}\DecValTok{1}\NormalTok{,}\AttributeTok{dx=}\NormalTok{xgrid)}\SpecialCharTok{$}\NormalTok{dC}
    \FunctionTok{return}\NormalTok{(}\FunctionTok{list}\NormalTok{(}\FunctionTok{c}\NormalTok{(du, dv))) }
\NormalTok{\}}
\end{Highlighting}
\end{Shaded}

Selesaikan persamaan menggunakan fungsi \texttt{ode.1D()} dengan metode \texttt{adams}.

\begin{Shaded}
\begin{Highlighting}[]
\NormalTok{out }\OtherTok{\textless{}{-}} \FunctionTok{ode.1D}\NormalTok{(}\AttributeTok{func =}\NormalTok{ wave, }\AttributeTok{y =}\NormalTok{ yini, }\AttributeTok{times =}\NormalTok{ times,}
             \AttributeTok{parms =} \ConstantTok{NULL}\NormalTok{, }\AttributeTok{method =} \StringTok{"adams"}\NormalTok{,}
             \AttributeTok{dimens =}\NormalTok{ N, }\AttributeTok{names =} \FunctionTok{c}\NormalTok{(}\StringTok{"u"}\NormalTok{, }\StringTok{"v"}\NormalTok{))}
\NormalTok{u }\OtherTok{\textless{}{-}} \FunctionTok{subset}\NormalTok{(out, }\AttributeTok{which =} \StringTok{"u"}\NormalTok{) }\CommentTok{\# Extract displacement vector}
\end{Highlighting}
\end{Shaded}

Visualisasikan hasil perhitungan.

\begin{figure}

{\centering \includegraphics[width=0.8\linewidth]{Algoritma_Pemrograman_R_files/figure-latex/wave2viz-1} 

}

\caption{Visualisasi hasil simulasi persamaan gelombang menggunakan paket ReacTran}\label{fig:wave2viz}
\end{figure}

\hypertarget{persamaan-laplace}{%
\subsection{Persamaan Laplace}\label{persamaan-laplace}}

Kita akan kembali melakukan simulasi terhadap Persamaan \eqref{eq:laplace1} menggunakan metode lainnya. Pada contoh kali ini, gradien pada sumbu \(y\) bernilai -1. Gradien hanyalah fluks, \(D \left(\partial C = \partial x\right)\), dengan \(D\) set sama dengan 1. Fungsi penyelesaian yang digunakan adalah \texttt{steady.2D()}, karena tidak ada ketergantungan waktu dalam persamaan. Sebagai kondisi awal yang arbitrer, kami menggunakan nomor acak \(N_x \times N_y\) yang terdistribusi secara seragam. Kita juga harus menentukan \(nspec\), jumlah spesies dalam model (hanya satu, potensi, dalam hal ini), \(dimens\), vektor dengan 2 nilai dengan jumlah sel dalam arah \(x\) dan \(y\), dan \texttt{lrw}, panjang array. Lihat halaman bantuan untuk \texttt{steady.2D()} untuk informasi lebih detail.

Langkah pertama dalam melakukan simulasi adalah menetapkan parameter model.

\begin{Shaded}
\begin{Highlighting}[]
\NormalTok{Nx    }\OtherTok{\textless{}{-}} \DecValTok{100}
\NormalTok{Ny    }\OtherTok{\textless{}{-}} \DecValTok{100}
\NormalTok{xgrid }\OtherTok{\textless{}{-}} \FunctionTok{setup.grid.1D}\NormalTok{(}\AttributeTok{x.up =} \DecValTok{0}\NormalTok{, }\AttributeTok{x.down =} \DecValTok{1}\NormalTok{, }\AttributeTok{N =}\NormalTok{ Nx)}
\NormalTok{ygrid }\OtherTok{\textless{}{-}} \FunctionTok{setup.grid.1D}\NormalTok{(}\AttributeTok{x.up =} \DecValTok{0}\NormalTok{, }\AttributeTok{x.down =} \DecValTok{1}\NormalTok{, }\AttributeTok{N =}\NormalTok{ Ny)}
\NormalTok{x     }\OtherTok{\textless{}{-}}\NormalTok{ xgrid}\SpecialCharTok{$}\NormalTok{x.mid}
\NormalTok{y     }\OtherTok{\textless{}{-}}\NormalTok{ ygrid}\SpecialCharTok{$}\NormalTok{x.mid}
\end{Highlighting}
\end{Shaded}

Bentuk fungsi yang akan diselesaikan secara numerik.

\begin{Shaded}
\begin{Highlighting}[]
\NormalTok{laplace }\OtherTok{\textless{}{-}} \ControlFlowTok{function}\NormalTok{(t, U, parms) \{}
\NormalTok{    w }\OtherTok{=} \FunctionTok{matrix}\NormalTok{(}\AttributeTok{nrow =}\NormalTok{ Nx, }\AttributeTok{ncol =}\NormalTok{ Ny, }\AttributeTok{data =}\NormalTok{ U)}
\NormalTok{    dw }\OtherTok{=} \FunctionTok{tran.2D}\NormalTok{(}\AttributeTok{C =}\NormalTok{ w, }\AttributeTok{C.x.up =} \DecValTok{0}\NormalTok{, }\AttributeTok{C.y.down =} \DecValTok{0}\NormalTok{,}
    \AttributeTok{flux.y.up =} \DecValTok{0}\NormalTok{,}
    \AttributeTok{flux.y.down =} \SpecialCharTok{{-}}\DecValTok{1}\NormalTok{,}
    \AttributeTok{D.x =} \DecValTok{1}\NormalTok{, }\AttributeTok{D.y =} \DecValTok{1}\NormalTok{,}
    \AttributeTok{dx =}\NormalTok{ xgrid, }\AttributeTok{dy =}\NormalTok{ ygrid)}\SpecialCharTok{$}\NormalTok{dC}
    \FunctionTok{list}\NormalTok{(dw) }
\NormalTok{\}}
\end{Highlighting}
\end{Shaded}

Mulia dengan bilangan acak uniform sebagai kondisi awal, kemudian selesaikan untuk memperoleh nilai pada kondisi tunak dan buat plot kontur.

\begin{Shaded}
\begin{Highlighting}[]
\NormalTok{out }\OtherTok{=} \FunctionTok{steady.2D}\NormalTok{(}\AttributeTok{y =} \FunctionTok{runif}\NormalTok{(Nx}\SpecialCharTok{*}\NormalTok{Ny), }\AttributeTok{func =}\NormalTok{ laplace, }
                  \AttributeTok{parms =} \ConstantTok{NULL}\NormalTok{,}\AttributeTok{nspec =} \DecValTok{1}\NormalTok{, }
                  \AttributeTok{dimens =} \FunctionTok{c}\NormalTok{(Nx, Ny), }\AttributeTok{lrw =} \FloatTok{1e7}\NormalTok{)}
\end{Highlighting}
\end{Shaded}

\begin{figure}

{\centering \includegraphics[width=0.8\linewidth]{Algoritma_Pemrograman_R_files/figure-latex/lapviz-1} 

}

\caption{Visualisasi hasil simulasi persamaan laplace pada kondisi tunak menggunakan paket ReacTran}\label{fig:lapviz}
\end{figure}

\hypertarget{persamaan-poisson-untuk-dipol}{%
\subsection{Persamaan Poisson untuk Dipol}\label{persamaan-poisson-untuk-dipol}}

Pada contoh kali ini, kita akan menyelesaikan persamaan Poisson untuk dipol yang ditampilkan pada Persamaan \eqref{eq:dipol}.

\begin{equation}
\frac{\partial^2w}{\partial x^2}+\frac{\partial^2w}{\partial y^2}=-\frac{\rho}{\varepsilon_0}
  \label{eq:dipol}
\end{equation}

untuk dipol yang terletak di tengah selembar persegi jika tidak pada 0 potensial. Untuk mempermudah, kita dapat mengatur semua faktor skala sama dengan satu. Dalam definisi fungsi poisson, nilai-nilai dalam matriks \(N_x \times N_y\) w adalah input melalui vektor data \(U\). Seperti dalam persamaan Laplace di atas, kita menetapkan nilai awal \(w\) pada sel-sel grid sama dengan angka acak uniform.

Langkah pertama dalam melakukan simulasi adalah menetapkan parameter model.

\begin{Shaded}
\begin{Highlighting}[]
\NormalTok{Nx    }\OtherTok{\textless{}{-}} \DecValTok{100}
\NormalTok{Ny    }\OtherTok{\textless{}{-}} \DecValTok{100}
\NormalTok{xgrid }\OtherTok{\textless{}{-}} \FunctionTok{setup.grid.1D}\NormalTok{(}\AttributeTok{x.up =} \DecValTok{0}\NormalTok{, }\AttributeTok{x.down =} \DecValTok{1}\NormalTok{, }\AttributeTok{N =}\NormalTok{ Nx)}
\NormalTok{ygrid }\OtherTok{\textless{}{-}} \FunctionTok{setup.grid.1D}\NormalTok{(}\AttributeTok{x.up =} \DecValTok{0}\NormalTok{, }\AttributeTok{x.down =} \DecValTok{1}\NormalTok{, }\AttributeTok{N =}\NormalTok{ Ny)}
\NormalTok{x     }\OtherTok{\textless{}{-}}\NormalTok{ xgrid}\SpecialCharTok{$}\NormalTok{x.mid}
\NormalTok{y     }\OtherTok{\textless{}{-}}\NormalTok{ ygrid}\SpecialCharTok{$}\NormalTok{x.mid}
\end{Highlighting}
\end{Shaded}

Cari nilai \(x\) dan \(y\) pada titik kisi yang mendekati \(\left(0,4;0,5\right)\) untuk muatan positif dan \(\left(0,6;0,5\right)\) untuk muatan negatif.

\begin{Shaded}
\begin{Highlighting}[]
\CommentTok{\# x and y coordinates of positive and negative charges}
\NormalTok{ipos }\OtherTok{\textless{}{-}} \FunctionTok{which.min}\NormalTok{(}\FunctionTok{abs}\NormalTok{(x }\SpecialCharTok{{-}} \FloatTok{0.4}\NormalTok{))}
\NormalTok{jpos }\OtherTok{\textless{}{-}} \FunctionTok{which.min}\NormalTok{(}\FunctionTok{abs}\NormalTok{(y }\SpecialCharTok{{-}} \FloatTok{0.50}\NormalTok{))}
\NormalTok{ineg }\OtherTok{\textless{}{-}} \FunctionTok{which.min}\NormalTok{(}\FunctionTok{abs}\NormalTok{(x }\SpecialCharTok{{-}} \FloatTok{0.6}\NormalTok{))}
\NormalTok{jneg }\OtherTok{\textless{}{-}} \FunctionTok{which.min}\NormalTok{(}\FunctionTok{abs}\NormalTok{(y }\SpecialCharTok{{-}} \FloatTok{0.50}\NormalTok{))}
\end{Highlighting}
\end{Shaded}

Bentuk fungsi Poisson yang akan diselesaikan secara numerik.

\begin{Shaded}
\begin{Highlighting}[]
\NormalTok{poisson }\OtherTok{\textless{}{-}} \ControlFlowTok{function}\NormalTok{(t, U, parms) \{}
\NormalTok{    w }\OtherTok{=} \FunctionTok{matrix}\NormalTok{(}\AttributeTok{nrow =}\NormalTok{ Nx, }\AttributeTok{ncol =}\NormalTok{ Ny, }\AttributeTok{data =}\NormalTok{ U)}
\NormalTok{    dw }\OtherTok{=} \FunctionTok{tran.2D}\NormalTok{(}\AttributeTok{C =}\NormalTok{ w, }\AttributeTok{C.x.up =} \DecValTok{0}\NormalTok{, }\AttributeTok{C.y.down =} \DecValTok{0}\NormalTok{,}
    \AttributeTok{flux.y.up =} \DecValTok{0}\NormalTok{,}
    \AttributeTok{flux.y.down =} \DecValTok{0}\NormalTok{,}
    \AttributeTok{D.x =} \DecValTok{1}\NormalTok{, }\AttributeTok{D.y =} \DecValTok{1}\NormalTok{,}
    \AttributeTok{dx =}\NormalTok{ xgrid, }\AttributeTok{dy =}\NormalTok{ ygrid)}\SpecialCharTok{$}\NormalTok{dC}
\NormalTok{    dw[ipos,jpos] }\OtherTok{=}\NormalTok{ dw[ipos,jpos] }\SpecialCharTok{+} \DecValTok{1}
\NormalTok{    dw[ineg,jneg] }\OtherTok{=}\NormalTok{ dw[ineg,jneg] }\SpecialCharTok{{-}} \DecValTok{1}
    \FunctionTok{list}\NormalTok{(dw) }
\NormalTok{\}}
\end{Highlighting}
\end{Shaded}

Selesaikan untuk kondisi tunak dan buat visualisasinya.

\begin{Shaded}
\begin{Highlighting}[]
\NormalTok{out }\OtherTok{\textless{}{-}} \FunctionTok{steady.2D}\NormalTok{(}\AttributeTok{y =} \FunctionTok{runif}\NormalTok{(Nx}\SpecialCharTok{*}\NormalTok{Ny), }
                 \AttributeTok{func =}\NormalTok{ poisson, }
                 \AttributeTok{parms =} \ConstantTok{NULL}\NormalTok{, }
                 \AttributeTok{nspec =} \DecValTok{1}\NormalTok{, }
                 \AttributeTok{dimens =} \FunctionTok{c}\NormalTok{(Nx, Ny), }
                 \AttributeTok{lrw =} \FloatTok{1e7}\NormalTok{)}
\end{Highlighting}
\end{Shaded}

\begin{figure}

{\centering \includegraphics[width=0.8\linewidth]{Algoritma_Pemrograman_R_files/figure-latex/poviz-1} 

}

\caption{Visualisasi hasil simulasi persamaan Poisson pada kondisi tunak menggunakan paket ReacTran}\label{fig:poviz}
\end{figure}

\hypertarget{studi-kasus-3}{%
\section{Studi Kasus}\label{studi-kasus-3}}

\hypertarget{model-lotka-voltera}{%
\subsection{Model Lotka-Voltera}\label{model-lotka-voltera}}

Model Lotka-Voltera merupakan model yang menggambarkan interaksi antara predator dan mangsa. Pada studi kasus kali ini model yang akan digunakan merupakan model dengan 3 bentuk interaksi yaitu: tanaman \(u\), herbivora \(v\), dan karnivora \(w\). Sistem persamaan diferensial sistem tersebut ditampilkan pada kumpulan persamaan berikut:

\begin{equation}
\frac{du}{dt}=au-\alpha_1f_1\left(u,v\right)
  \label{eq:lv}
\end{equation}

\begin{equation}
\frac{dv}{dt}=-bv+\alpha_1f_1\left(u,v\right)-\alpha_2f_2\left(v,w\right)
  \label{eq:lv2}
\end{equation}

\begin{equation}
\frac{dw}{dt}=-c\left(w-w\ast\right)+\alpha_2f_2\left(v,w\right)
  \label{eq:lv3}
\end{equation}

dimana \(w\ast\) merupakan tingkat predasi minimum untuk menstabilkan populasi ketika populasi mangsa rendah. Interaksi antar komponen digambarkan ke dalam bentuk persamaan logistik.

\begin{equation}
f_i\left(x,y\right)=\frac{xy}{1+k_ix}
  \label{eq:lv4}
\end{equation}

Langkah pertama untuk menyelesaikan model adalah melakukan penentuan parameter model dan pembentukan fungsi model.

\begin{Shaded}
\begin{Highlighting}[]
\FunctionTok{library}\NormalTok{(deSolve)}
\NormalTok{f }\OtherTok{\textless{}{-}} \ControlFlowTok{function}\NormalTok{(x,y,k)\{x}\SpecialCharTok{*}\NormalTok{y}\SpecialCharTok{/}\NormalTok{(}\DecValTok{1}\SpecialCharTok{+}\NormalTok{k}\SpecialCharTok{*}\NormalTok{x)\}}
\NormalTok{model }\OtherTok{\textless{}{-}} \ControlFlowTok{function}\NormalTok{(t, xx, parms) \{}
\NormalTok{    u }\OtherTok{=}\NormalTok{ xx[}\DecValTok{1}\NormalTok{] }\CommentTok{\# plant resource}
\NormalTok{    v }\OtherTok{=}\NormalTok{ xx[}\DecValTok{2}\NormalTok{] }\CommentTok{\# herbivore}
\NormalTok{    w }\OtherTok{=}\NormalTok{ xx[}\DecValTok{3}\NormalTok{] }\CommentTok{\# carnivore}
    \FunctionTok{with}\NormalTok{(}\FunctionTok{as.list}\NormalTok{(parms),\{}
\NormalTok{    du }\OtherTok{=}\NormalTok{ a}\SpecialCharTok{*}\NormalTok{u }\SpecialCharTok{{-}}\NormalTok{ alpha1}\SpecialCharTok{*}\FunctionTok{f}\NormalTok{(u, v, k1)}
\NormalTok{    dv }\OtherTok{=} \SpecialCharTok{{-}}\NormalTok{b}\SpecialCharTok{*}\NormalTok{v }\SpecialCharTok{+}\NormalTok{ alpha1}\SpecialCharTok{*}\FunctionTok{f}\NormalTok{(u, v, k1) }\SpecialCharTok{{-}}\NormalTok{ alpha2}\SpecialCharTok{*}\FunctionTok{f}\NormalTok{(v, w, k2)}
\NormalTok{    dw }\OtherTok{=} \SpecialCharTok{{-}}\NormalTok{c}\SpecialCharTok{*}\NormalTok{(w }\SpecialCharTok{{-}}\NormalTok{ wstar) }\SpecialCharTok{+}\NormalTok{ alpha2}\SpecialCharTok{*}\FunctionTok{f}\NormalTok{(v, w, k2)}
    \FunctionTok{list}\NormalTok{(}\FunctionTok{c}\NormalTok{(du, dv, dw))}
\NormalTok{\})\}}

\NormalTok{times }\OtherTok{\textless{}{-}} \FunctionTok{seq}\NormalTok{(}\DecValTok{0}\NormalTok{, }\DecValTok{200}\NormalTok{, }\FloatTok{0.1}\NormalTok{)}
\NormalTok{parms }\OtherTok{\textless{}{-}} \FunctionTok{c}\NormalTok{(}\AttributeTok{a=}\DecValTok{1}\NormalTok{, }\AttributeTok{b=}\DecValTok{1}\NormalTok{, }\AttributeTok{c=}\DecValTok{10}\NormalTok{, }\AttributeTok{alpha1=}\FloatTok{0.2}\NormalTok{, }\AttributeTok{alpha2=}\DecValTok{1}\NormalTok{,}
\NormalTok{k1}\OtherTok{\textless{}{-}}\FloatTok{0.05}\NormalTok{, }\AttributeTok{k2=}\DecValTok{0}\NormalTok{, }\AttributeTok{wstar=}\FloatTok{0.006}\NormalTok{)}
\NormalTok{xstart }\OtherTok{\textless{}{-}} \FunctionTok{c}\NormalTok{(}\AttributeTok{u=}\DecValTok{10}\NormalTok{, }\AttributeTok{v=}\DecValTok{5}\NormalTok{, }\AttributeTok{w=}\FloatTok{0.1}\NormalTok{)}
\end{Highlighting}
\end{Shaded}

Proses iterasi selanjutnya akan menggunakan metode \texttt{lsoda}, dimana metode iterasi ini dapat berganti secara otomatis dalam menangani sistem persamaan diferensial \emph{stiff} dan \emph{non-stiff}. Persamaan diferensial dikatakan \emph{stiff} apabila variabel dependen berubah berdasarkan 2 atau lebih variabel independen yang sangat berbeda skalanya.

\begin{Shaded}
\begin{Highlighting}[]
\NormalTok{out }\OtherTok{=} \FunctionTok{ode}\NormalTok{(xstart, times, model, parms, }
          \AttributeTok{method =}\StringTok{"lsoda"}\NormalTok{)}
\end{Highlighting}
\end{Shaded}

Ketiga variabel selanjutnya divisualisasikan.

\begin{figure}

{\centering \includegraphics[width=0.8\linewidth]{Algoritma_Pemrograman_R_files/figure-latex/lvviz-1} 

}

\caption{Visualisasi simulasi model Lotka-Voltera}\label{fig:lvviz}
\end{figure}

\hypertarget{referensi-5}{%
\section{Referensi}\label{referensi-5}}

\begin{enumerate}
\def\labelenumi{\arabic{enumi}.}
\tightlist
\item
  Bloomfield, V.A. 2014. \textbf{Using R for Numerical Analysis in Science and Engineering}. CRC Press.
\item
  Chapra, S.C. Canale, R.P. 2015. \textbf{Numerical Methods For Engineers, Seventh Edition}. Mc Graw Hill.
\item
  Griffiths, G.W. 2016. \textbf{Numerical analysis using R : solutions to ODEs and PDEs}. Cambridge University Press.
\item
  Howard, J.P. 2017. \textbf{Computational Methods for Numerical Analysis with R}. CRC Press.
\item
  Kreyszig, E. 2011. \textbf{Advanced Engineering Mathematics, 10th Edition}. John Wiley \& Sons.
\item
  Soetaert,K., Cash J., Mazzia F. 2012. \textbf{Solving Differential Equations in R}. Springer.
\item
  Suparno, S. 2008. \textbf{Komputasi untuk Sains dan Teknik Edisi II}. Departemen Fisika-FMIPA Universitas Indonesia.
\end{enumerate}

\hypertarget{latihan-4}{%
\section{Latihan}\label{latihan-4}}

\begin{enumerate}
\def\labelenumi{\arabic{enumi}.}
\tightlist
\item
  Tunjukkan 10 hasil iterasi metode Euler untuk persamaan diferensial \(f'\left(x,y\right)=y\) dimana \(x_0=0\), \(y_0=2\) dan \emph{step size} \(h=0,1\)!
\item
  Kerjakan lagi soal no.1 dengan menggunakan metode lainnya dan bandingkan seluruh metode tersebut menggunakan penyelesaian analitiknya!
\item
  Bentuk kembali fungsi \texttt{rk4sys()} menggunakan algoritma metode Adams-Bashforth dan namai fungsi tersebut \texttt{adamsys()}!
\end{enumerate}

\hypertarget{dataanalysis}{%
\chapter{Analisis Data}\label{dataanalysis}}

Pada Chapter \ref{dataanalysis}, kita akan membahas mengenai cara melakukan analisis data pada \texttt{R}. Pada \emph{Chapter} ini penulis akan memperkenalkan fungsi-fungsi yang ada pada \texttt{R} yang dapat membantu kita menganalisis data dan melakukan sejumlah uji statistik.

Pada \emph{Chapter} ini kita tidak akan berfokus pada persamaan-persamaan matematika yang menjadi dasar suatu uji statistik. \emph{Chapter} ini menitik beratkan pada bagaimana pembaca dapat melakukan sejumlah uji statistik pada \texttt{R} dan gambaran metode yang digunakan.

\hypertarget{importdata}{%
\section{Import Data}\label{importdata}}

Kita dapat melakukan import data dalam berbagai format pada \texttt{R}. Namun, pada \emph{sub-chapter} ini hanya akan dibahas bagaimana cara mengimport data dari file dengan format \texttt{.csv}dan \texttt{.txt}. Secara umum fungsi-fungsi yang digunakan untuk membaca data pada file dengan format tersebut adalah sebagai berikut:

\begin{Shaded}
\begin{Highlighting}[]
\FunctionTok{read.table}\NormalTok{(file, }\AttributeTok{header =} \ConstantTok{FALSE}\NormalTok{, }\AttributeTok{sep =} \StringTok{""}\NormalTok{, }\AttributeTok{dec =} \StringTok{"."}\NormalTok{,}
           \AttributeTok{stringsAsFactors =} \FunctionTok{default.stringsAsFactors}\NormalTok{())}

\FunctionTok{read.csv}\NormalTok{(file, }\AttributeTok{header =} \ConstantTok{TRUE}\NormalTok{, }\AttributeTok{sep =} \StringTok{","}\NormalTok{, }\AttributeTok{dec =} \StringTok{"."}\NormalTok{)}

\FunctionTok{read.csv2}\NormalTok{(file, }\AttributeTok{header =} \ConstantTok{TRUE}\NormalTok{, }\AttributeTok{sep =} \StringTok{";"}\NormalTok{, }\AttributeTok{dec =} \StringTok{","}\NormalTok{)}

\FunctionTok{read.delim}\NormalTok{(file, }\AttributeTok{header =} \ConstantTok{TRUE}\NormalTok{, }\AttributeTok{sep =} \StringTok{"}\SpecialCharTok{\textbackslash{}t}\StringTok{"}\NormalTok{, }\AttributeTok{dec =} \StringTok{"."}\NormalTok{)}

\FunctionTok{read.delim2}\NormalTok{(file, }\AttributeTok{header =} \ConstantTok{TRUE}\NormalTok{, }\AttributeTok{sep =} \StringTok{"}\SpecialCharTok{\textbackslash{}t}\StringTok{"}\NormalTok{, }\AttributeTok{dec =} \StringTok{","}\NormalTok{)}
\end{Highlighting}
\end{Shaded}

\begin{quote}
\textbf{Catatan:}

\begin{itemize}
\tightlist
\item
  \texttt{file} : lokasi dan nama file yang akan dibaca diakhiri dengan format file. Secara \emph{default} fungsi akan membaca file yang ada pada \emph{working directory}. Untuk mengetahui lokasi \emph{working directory}, jalankan fungsi \texttt{getwd()}. Salin file yang akan dibaca pada lokasi \emph{working directory}.
\item
  \texttt{header} : nilai logik yang menunjukkan apakah baris pertama pada file yang dibaca akan dibaca sebagai nama kolom.
\item
  \texttt{sep} : simbol yang menujukkan pemisah antar data. Pemisah antar data dapat berupa ``",'';``,''.'', dll.
\item
  \texttt{dec} : simbol yang menujukkan desimal. Pemisah desimal dapat berupa ``.'' atau ``,''.
\item
  \texttt{stringsAsFactors} : nilai logik yang menunjukkan apakah jenis data \texttt{string} akan dikonversi menjadi \texttt{factor}.
\end{itemize}
\end{quote}

Kelima fungsi tersebut digunakan untuk membaca data tabular atau data yang disusun kedalam format tabel. Fungsi \texttt{read.table()} merupakan bentuk umum dari keempat fungsi lainnya. Fungsi tersebut dapat digunakan untuk membaca data dalam kedua format yang telah disebutkan sebelumnya. Fungsi lainnya lebih spesifi, dimana fungsi \texttt{read.csv()} dan \texttt{read.csv2()} digunakan untuk membaca data dengan ekstensi \texttt{.csv}, sedangkan \texttt{read.delim()} dan \texttt{read.delim2()} untuk membaca data dengan ekstensi \texttt{.txt}. Berikut adalah contoh bagaimana cara membaca data dengan nama \texttt{data.csv} yang ada pada \emph{working directory} dengan pemisah antar data berupa \texttt{;} dan tanda koma berupa \texttt{,}:

\begin{Shaded}
\begin{Highlighting}[]
\NormalTok{data }\OtherTok{\textless{}{-}} \FunctionTok{read.table}\NormalTok{(}\AttributeTok{file=}\StringTok{"data.csv"}\NormalTok{, }\AttributeTok{sep=}\StringTok{";"}\NormalTok{, }\AttributeTok{dec=}\StringTok{","}\NormalTok{)}
\end{Highlighting}
\end{Shaded}

\hypertarget{librarydata}{%
\section{\texorpdfstring{Membaca Data Dari \emph{Library}}{Membaca Data Dari Library}}\label{librarydata}}

Untuk keperluan pendidikan atau pengujian sebuah fungsi biasanya dalam sebuah \emph{library} disediakan dataset yang siap digunakan. \texttt{R} melalui \emph{library} \texttt{datasets} menyediakan sejumlah data yang dapat digunakan untuk berlatih menggunakan \texttt{R}. Berikut adalah fungsi yang digunakan untuk mengecek dataset apa saja yang tersedia pada sebuah \emph{library}:

\begin{Shaded}
\begin{Highlighting}[]
\FunctionTok{data}\NormalTok{(}\AttributeTok{package=}\FunctionTok{.packages}\NormalTok{(}\AttributeTok{all.available =} \ConstantTok{TRUE}\NormalTok{))}
\end{Highlighting}
\end{Shaded}

\begin{quote}
\textbf{Catatan:}

\begin{itemize}
\tightlist
\item
  \texttt{package}: nama \emph{library} yang hendak dicek dataset yang tersedia.
\end{itemize}
\end{quote}

Berikut adalah contoh cara melakukan pengecekan pada dataset yang tersedia pada \emph{library} \texttt{datasets}:

\begin{Shaded}
\begin{Highlighting}[]
\FunctionTok{data}\NormalTok{(}\AttributeTok{package=}\StringTok{"datasets"}\NormalTok{)}

\CommentTok{\# cek seluruh dataset dari seluruh library yg telah dimuat}
\FunctionTok{data}\NormalTok{()}
\end{Highlighting}
\end{Shaded}

\hypertarget{summarystats}{%
\section{Ringkasan Data}\label{summarystats}}

Terdapat sejumlah fungsi yang akan pembaca sering gunakan untuk mengecek dataset yang akan pembaca analisa. Fungsi-fungsi tersebut antara lain:

\begin{itemize}
\tightlist
\item
  \texttt{head()}: mengecek \(n\) (\emph{default} 6) observasi teratas.
\item
  \texttt{tail()}: mengecek \(n\) (\emph{default} 6) observasi terbawah.
\item
  \texttt{str()}: mengecek struktur data atau jenis data pada masing-masing kolom. Jenis data yang ada pada \texttt{R} dapat berupa \texttt{num} (numerik), \texttt{int} (integer), \texttt{Factor}(factor), \texttt{date} (tanggal), dan \texttt{chr} (karakter atau string).
\item
  \texttt{summary()}: ringkasan data.
\end{itemize}

Berikut adalah contoh penerapan fungsi-fungsi tersebut pada dataset \texttt{iris}:

\begin{Shaded}
\begin{Highlighting}[]
\CommentTok{\# cek 10 observasi teratas}
\FunctionTok{head}\NormalTok{(iris, }\DecValTok{10}\NormalTok{)}
\end{Highlighting}
\end{Shaded}

\begin{verbatim}
##    Sepal.Length Sepal.Width Petal.Length Petal.Width
## 1           5.1         3.5          1.4         0.2
## 2           4.9         3.0          1.4         0.2
## 3           4.7         3.2          1.3         0.2
## 4           4.6         3.1          1.5         0.2
## 5           5.0         3.6          1.4         0.2
## 6           5.4         3.9          1.7         0.4
## 7           4.6         3.4          1.4         0.3
## 8           5.0         3.4          1.5         0.2
## 9           4.4         2.9          1.4         0.2
## 10          4.9         3.1          1.5         0.1
##    Species
## 1   setosa
## 2   setosa
## 3   setosa
## 4   setosa
## 5   setosa
## 6   setosa
## 7   setosa
## 8   setosa
## 9   setosa
## 10  setosa
\end{verbatim}

\begin{Shaded}
\begin{Highlighting}[]
\CommentTok{\# cek 10 observasi terbawah}
\FunctionTok{tail}\NormalTok{(iris, }\DecValTok{10}\NormalTok{)}
\end{Highlighting}
\end{Shaded}

\begin{verbatim}
##     Sepal.Length Sepal.Width Petal.Length Petal.Width
## 141          6.7         3.1          5.6         2.4
## 142          6.9         3.1          5.1         2.3
## 143          5.8         2.7          5.1         1.9
## 144          6.8         3.2          5.9         2.3
## 145          6.7         3.3          5.7         2.5
## 146          6.7         3.0          5.2         2.3
## 147          6.3         2.5          5.0         1.9
## 148          6.5         3.0          5.2         2.0
## 149          6.2         3.4          5.4         2.3
## 150          5.9         3.0          5.1         1.8
##       Species
## 141 virginica
## 142 virginica
## 143 virginica
## 144 virginica
## 145 virginica
## 146 virginica
## 147 virginica
## 148 virginica
## 149 virginica
## 150 virginica
\end{verbatim}

\begin{Shaded}
\begin{Highlighting}[]
\CommentTok{\# cek struktur data}
\FunctionTok{str}\NormalTok{(iris)}
\end{Highlighting}
\end{Shaded}

\begin{verbatim}
## 'data.frame':    150 obs. of  5 variables:
##  $ Sepal.Length: num  5.1 4.9 4.7 4.6 5 5.4 4.6 5 4.4 4.9 ...
##  $ Sepal.Width : num  3.5 3 3.2 3.1 3.6 3.9 3.4 3.4 2.9 3.1 ...
##  $ Petal.Length: num  1.4 1.4 1.3 1.5 1.4 1.7 1.4 1.5 1.4 1.5 ...
##  $ Petal.Width : num  0.2 0.2 0.2 0.2 0.2 0.4 0.3 0.2 0.2 0.1 ...
##  $ Species     : Factor w/ 3 levels "setosa","versicolor",..: 1 1 1 1 1 1 1 1 1 1 ...
\end{verbatim}

\begin{Shaded}
\begin{Highlighting}[]
\CommentTok{\# ringkasan data}
\FunctionTok{summary}\NormalTok{(iris)}
\end{Highlighting}
\end{Shaded}

\begin{verbatim}
##   Sepal.Length   Sepal.Width    Petal.Length 
##  Min.   :4.30   Min.   :2.00   Min.   :1.00  
##  1st Qu.:5.10   1st Qu.:2.80   1st Qu.:1.60  
##  Median :5.80   Median :3.00   Median :4.35  
##  Mean   :5.84   Mean   :3.06   Mean   :3.76  
##  3rd Qu.:6.40   3rd Qu.:3.30   3rd Qu.:5.10  
##  Max.   :7.90   Max.   :4.40   Max.   :6.90  
##   Petal.Width        Species  
##  Min.   :0.1   setosa    :50  
##  1st Qu.:0.3   versicolor:50  
##  Median :1.3   virginica :50  
##  Mean   :1.2                  
##  3rd Qu.:1.8                  
##  Max.   :2.5
\end{verbatim}

Fungsi-fungsi lainnya yang dapat digunakan untuk melakukan analisis statistika deskriptif adalah sebagai berikut:

\begin{itemize}
\tightlist
\item
  \texttt{mean()} : menghitung nilai rata-rata variabel numerik.
\item
  \texttt{sd()} : menghitung simpangan baku variabel numerik.
\item
  \texttt{var()} : menghitung varians variabel numerik.
\item
  \texttt{median()} : menghitung median suatu variabel numerik.
\item
  \texttt{range()} : memperoleh nilai minimum dan maksimum suatu variabel numerik.
\item
  \texttt{IQR()} : memperoleh nilai jarak antar kuartil.
\item
  \texttt{quantile()} : memperoleh kuantil variabel numerik.
\end{itemize}

Berikut adalah contoh penerapan fungsi-fungsi tersebut:

\begin{Shaded}
\begin{Highlighting}[]
\FunctionTok{attach}\NormalTok{(airquality)}

\CommentTok{\# rata{-}rata konsentrasi ozon}
\FunctionTok{mean}\NormalTok{(Ozone, }\AttributeTok{na.rm =} \ConstantTok{TRUE}\NormalTok{)}
\end{Highlighting}
\end{Shaded}

\begin{verbatim}
## [1] 42.13
\end{verbatim}

\begin{Shaded}
\begin{Highlighting}[]
\CommentTok{\# median konsentrasi ozon}
\FunctionTok{median}\NormalTok{(Ozone, }\AttributeTok{na.rm =} \ConstantTok{TRUE}\NormalTok{)}
\end{Highlighting}
\end{Shaded}

\begin{verbatim}
## [1] 31.5
\end{verbatim}

\begin{Shaded}
\begin{Highlighting}[]
\CommentTok{\# simpangan baku konsentrasi ozon}
\FunctionTok{sd}\NormalTok{(Ozone, }\AttributeTok{na.rm =} \ConstantTok{TRUE}\NormalTok{)}
\end{Highlighting}
\end{Shaded}

\begin{verbatim}
## [1] 32.99
\end{verbatim}

\begin{Shaded}
\begin{Highlighting}[]
\CommentTok{\# varians konsentrasi ozon}
\FunctionTok{var}\NormalTok{(Ozone, }\AttributeTok{na.rm =} \ConstantTok{TRUE}\NormalTok{)}
\end{Highlighting}
\end{Shaded}

\begin{verbatim}
## [1] 1088
\end{verbatim}

\begin{Shaded}
\begin{Highlighting}[]
\CommentTok{\# range konsentrasi ozon}
\FunctionTok{range}\NormalTok{(Ozone, }\AttributeTok{na.rm =} \ConstantTok{TRUE}\NormalTok{)}
\end{Highlighting}
\end{Shaded}

\begin{verbatim}
## [1]   1 168
\end{verbatim}

\begin{Shaded}
\begin{Highlighting}[]
\CommentTok{\# IQR konsentrasi ozon}
\FunctionTok{IQR}\NormalTok{(Ozone, }\AttributeTok{na.rm =} \ConstantTok{TRUE}\NormalTok{)}
\end{Highlighting}
\end{Shaded}

\begin{verbatim}
## [1] 45.25
\end{verbatim}

\begin{Shaded}
\begin{Highlighting}[]
\CommentTok{\# kuartil 1, 2 dan 3 konsentrasi ozon}
\FunctionTok{quantile}\NormalTok{(Ozone, }\AttributeTok{probs =} \FunctionTok{c}\NormalTok{(}\FloatTok{0.25}\NormalTok{, }\FloatTok{0.5}\NormalTok{, }\FloatTok{0.75}\NormalTok{), }\AttributeTok{na.rm =} \ConstantTok{TRUE}\NormalTok{)}
\end{Highlighting}
\end{Shaded}

\begin{verbatim}
##   25%   50%   75% 
## 18.00 31.50 63.25
\end{verbatim}

\begin{Shaded}
\begin{Highlighting}[]
\FunctionTok{detach}\NormalTok{(airquality)}
\end{Highlighting}
\end{Shaded}

\hypertarget{norm}{%
\section{Uji Normalitas Data Tunggal}\label{norm}}

Pada analisis statistik inferensial khususnya pada pengujian hipotesis, asumsi normalitas merupakan sesuatu yang harus terpenuhi jika prosedur uji yang digunakan merupakan prosedur uji parametrik. Terdapat dua buah cara untuk melakukan uji tersebut, antara lain:

\begin{enumerate}
\def\labelenumi{\arabic{enumi}.}
\tightlist
\item
  Metode grafis (qq-plot, ECDF, plot densitas, histogram, dan boxplot).
\item
  Metode matematis (Shapiro-Wilk, Cramer-von Mises, Shapiro-Francia, Anderson-Darling, Liliefors, Pearson Chi-square, dll).
\end{enumerate}

Pada \emph{Chapter} ini, kita akan berfokus pada uji matematis karena cara pengujian dengan menggunakan metode grafis telah penulis jabarkan pada \emph{Chapter} visualisasi data.

Metode uji normalitas yang sering digunakan pada \texttt{R} adalah metode Shapiro-Wilk. Metode ini merupakan metode uji yang memiliki power yang besar khusunya untuk ukuran sampel yang relatif kecil. Versi awal metode ini terbatas dengan jumlah sampel 3 sampai 50 sampel. Versi selanjutnya mengalami modifikasi sehingga dapat menangani sampel sampai dengan 5000 sampel bahkan lebih.

Untuk melakukan uji SHapiro-Wilk pada \texttt{R}, pembaca dapat menggunakan fungsi \texttt{shapiro.test()}. Format fungsi tersebut adalah sebagai beriku:

\begin{Shaded}
\begin{Highlighting}[]
\FunctionTok{shapiro.test}\NormalTok{(x)}
\end{Highlighting}
\end{Shaded}

\begin{quote}
\textbf{Catatan:}

\begin{itemize}
\tightlist
\item
  \texttt{x} : vektor numerik.
\end{itemize}
\end{quote}

Untuk lebih memahami impelementasi fungsi tersebut pada data, berikut adalah contoh penerapan fungsi tersebut untuk menguji normalitas distribusi konsentrasi ozon pada dataset \texttt{airquality}:

\begin{Shaded}
\begin{Highlighting}[]
\FunctionTok{shapiro.test}\NormalTok{(airquality}\SpecialCharTok{$}\NormalTok{Ozone)}
\end{Highlighting}
\end{Shaded}

\begin{verbatim}
## 
##  Shapiro-Wilk normality test
## 
## data:  airquality$Ozone
## W = 0.88, p-value = 3e-08
\end{verbatim}

Berdasarkan hasil uji diperoleh nilai p-value \textless{} 0,05, sehingga \(H_0\) ditolak dan dapat disimpulkan bahwa distribusi konsentrasi ozon tidak mengikuti distribusi normal. Untuk lebih memahami prosedur pengujian normalitas distribusi suatu data pembaca dapat membaca lebih lanjut pada tautan \href{https://environmental-data-modeling.netlify.com/tutorial/11_uji_hipotesis/\#11-4-uji-asumsi-normalitas-distribusi-data}{Environmental Data Modelin}.

\hypertarget{uji-rata-rata-satu-dan-dua-sampel}{%
\section{Uji Rata-Rata Satu dan Dua Sampel}\label{uji-rata-rata-satu-dan-dua-sampel}}

Uji rata-rata satu sampel merupakan uji statistik untuk menguji apakah rata-rata suatu sampel berasal dari suatu populasi yang telah diketahui nilai rata-ratanya. Sedangkan uji rata-rata untuk dua populasi dilakukan untuk menguji apakah kedua selisis rata-rata populasi tersebut bernilai nol yang menujukkan bahwa kedua populasi tersebut memiliki nilai rata-rata yang sama. Uji rata-rata dua populasi dapat dilakukan untuk sampel independen (contoh: uji rata-rata performa dua buah IPAL) dan berpasangan (contoh: uji rata-rata input dan output IPAL).

Untuk melakukan uji rata-rata pada \texttt{R} dapat digunakan fungsi \texttt{t.test()} untuk uji parametrik dan \texttt{wilcox.test()} untuk melakukan uji non-parametrik \emph{sign rank test}. Format fungsi-fungsi tersebut adalah sebagai berikut:

\begin{Shaded}
\begin{Highlighting}[]
\FunctionTok{t.test}\NormalTok{(x, }\AttributeTok{y =} \ConstantTok{NULL}\NormalTok{,}
       \AttributeTok{alternative =} \FunctionTok{c}\NormalTok{(}\StringTok{"two.sided"}\NormalTok{, }\StringTok{"less"}\NormalTok{, }\StringTok{"greater"}\NormalTok{),}
       \AttributeTok{mu =} \DecValTok{0}\NormalTok{, }\AttributeTok{paired =} \ConstantTok{FALSE}\NormalTok{, }\AttributeTok{var.equal =} \ConstantTok{FALSE}\NormalTok{,}
       \AttributeTok{conf.level =} \FloatTok{0.95}\NormalTok{, ...)}

\FunctionTok{wilcox.test}\NormalTok{(x, }\AttributeTok{y =} \ConstantTok{NULL}\NormalTok{,}
            \AttributeTok{alternative =} \FunctionTok{c}\NormalTok{(}\StringTok{"two.sided"}\NormalTok{, }\StringTok{"less"}\NormalTok{, }\StringTok{"greater"}\NormalTok{),}
            \AttributeTok{mu =} \DecValTok{0}\NormalTok{, }\AttributeTok{paired =} \ConstantTok{FALSE}\NormalTok{, }\AttributeTok{conf.level =} \FloatTok{0.95}\NormalTok{, ...)}
\end{Highlighting}
\end{Shaded}

\begin{quote}
\textbf{Catatan:}

\begin{itemize}
\tightlist
\item
  \texttt{x,y} : vektor numerik. Jika argumen \texttt{x} dan \texttt{y} diisikan maka uji hipotesis dilakukan untuk dua buah populasi.
\item
  \texttt{alternative}: digunakan untuk menentukan jenis uji hipotesis apakah satu sisi(``less'' dan ``greater''), atau dua sisi (``two.sided'').
\item
  \texttt{mu} : nilai rata-rata populasi atau nilai rata-rata selisih antar populasi jika dilakukan uji hipotesis terhadap dua populasi. Secara default nilainya 0.
\item
  \texttt{paired} : nilai logikal yang menentukan apakah uji dua populasi digunakan untuk sampel berpasangan (TRUE) atau tidak (FALSE).
\item
  \texttt{var.equal} : nilai logikal yang menunjukkan apakah varians kedua populasi diasumsikan sama atau berbeda.
\item
  \texttt{conf.level} : tingkat kepercayaan. Secara default tingkat kepercayaan yang digunakan adalah 95\%.
\end{itemize}
\end{quote}

Berikut adalah contoh penerapan fungsi tersebut untuk uji hipotesis satu dan dua populasi:

\begin{Shaded}
\begin{Highlighting}[]
\CommentTok{\# Uji hipotesis konsentrasi ozon = 40 ppm}
\CommentTok{\# parametrik}
\FunctionTok{t.test}\NormalTok{(}\AttributeTok{x=}\NormalTok{airquality}\SpecialCharTok{$}\NormalTok{Ozone, }\AttributeTok{alternative =} \StringTok{"two.sided"}\NormalTok{,}
       \AttributeTok{mu =} \DecValTok{40}\NormalTok{)}
\end{Highlighting}
\end{Shaded}

\begin{verbatim}
## 
##  One Sample t-test
## 
## data:  airquality$Ozone
## t = 0.7, df = 115, p-value = 0.5
## alternative hypothesis: true mean is not equal to 40
## 95 percent confidence interval:
##  36.06 48.20
## sample estimates:
## mean of x 
##     42.13
\end{verbatim}

\begin{Shaded}
\begin{Highlighting}[]
\CommentTok{\# nonparametrik}
\FunctionTok{wilcox.test}\NormalTok{(}\AttributeTok{x=}\NormalTok{airquality}\SpecialCharTok{$}\NormalTok{Ozone, }\AttributeTok{alternative =} \StringTok{"two.sided"}\NormalTok{,}
       \AttributeTok{mu =} \DecValTok{40}\NormalTok{)}
\end{Highlighting}
\end{Shaded}

\begin{verbatim}
## 
##  Wilcoxon signed rank test with continuity
##  correction
## 
## data:  airquality$Ozone
## V = 3188, p-value = 0.7
## alternative hypothesis: true location is not equal to 40
\end{verbatim}

\begin{Shaded}
\begin{Highlighting}[]
\CommentTok{\# Uji hipotesis dua populasi}
\NormalTok{dni3 }\OtherTok{\textless{}{-}} \FunctionTok{dimnames}\NormalTok{(iris3)}
\NormalTok{ii }\OtherTok{\textless{}{-}} \FunctionTok{data.frame}\NormalTok{(}\FunctionTok{matrix}\NormalTok{(}\FunctionTok{aperm}\NormalTok{(iris3, }\FunctionTok{c}\NormalTok{(}\DecValTok{1}\NormalTok{,}\DecValTok{3}\NormalTok{,}\DecValTok{2}\NormalTok{)), }\AttributeTok{ncol =} \DecValTok{4}\NormalTok{,}
                        \AttributeTok{dimnames =} \FunctionTok{list}\NormalTok{(}\ConstantTok{NULL}\NormalTok{, }\FunctionTok{sub}\NormalTok{(}\StringTok{" L."}\NormalTok{,}\StringTok{".Length"}\NormalTok{,}
                                        \FunctionTok{sub}\NormalTok{(}\StringTok{" W."}\NormalTok{,}\StringTok{".Width"}\NormalTok{, dni3[[}\DecValTok{2}\NormalTok{]])))),}
    \AttributeTok{Species =} \FunctionTok{gl}\NormalTok{(}\DecValTok{3}\NormalTok{, }\DecValTok{50}\NormalTok{, }\AttributeTok{labels =} \FunctionTok{sub}\NormalTok{(}\StringTok{"S"}\NormalTok{, }\StringTok{"s"}\NormalTok{, }\FunctionTok{sub}\NormalTok{(}\StringTok{"V"}\NormalTok{, }\StringTok{"v"}\NormalTok{, dni3[[}\DecValTok{3}\NormalTok{]]))))}
\CommentTok{\# parametrik}
\FunctionTok{t.test}\NormalTok{(}\AttributeTok{x=}\NormalTok{iris}\SpecialCharTok{$}\NormalTok{Sepal.Length[iris}\SpecialCharTok{$}\NormalTok{Species}\SpecialCharTok{==}\StringTok{"setosa"}\NormalTok{], }
       \AttributeTok{y=}\NormalTok{ii}\SpecialCharTok{$}\NormalTok{Sepal.Length[iris}\SpecialCharTok{$}\NormalTok{Species}\SpecialCharTok{==}\StringTok{"versicolor"}\NormalTok{])}
\end{Highlighting}
\end{Shaded}

\begin{verbatim}
## 
##  Welch Two Sample t-test
## 
## data:  iris$Sepal.Length[iris$Species == "setosa"] and ii$Sepal.Length[iris$Species == "versicolor"]
## t = -11, df = 87, p-value <2e-16
## alternative hypothesis: true difference in means is not equal to 0
## 95 percent confidence interval:
##  -1.1057 -0.7543
## sample estimates:
## mean of x mean of y 
##     5.006     5.936
\end{verbatim}

\begin{Shaded}
\begin{Highlighting}[]
\CommentTok{\# nonparametrik}
\FunctionTok{wilcox.test}\NormalTok{(}\AttributeTok{x=}\NormalTok{iris}\SpecialCharTok{$}\NormalTok{Sepal.Length[iris}\SpecialCharTok{$}\NormalTok{Species}\SpecialCharTok{==}\StringTok{"setosa"}\NormalTok{], }
       \AttributeTok{y=}\NormalTok{ii}\SpecialCharTok{$}\NormalTok{Sepal.Length[iris}\SpecialCharTok{$}\NormalTok{Species}\SpecialCharTok{==}\StringTok{"versicolor"}\NormalTok{])}
\end{Highlighting}
\end{Shaded}

\begin{verbatim}
## 
##  Wilcoxon rank sum test with continuity
##  correction
## 
## data:  iris$Sepal.Length[iris$Species == "setosa"] and ii$Sepal.Length[iris$Species == "versicolor"]
## W = 168, p-value = 8e-14
## alternative hypothesis: true location shift is not equal to 0
\end{verbatim}

Fungsi \texttt{t.test()} akan menghasilkan output berupa nilai t uji, derajat kebebasan (df), nilai p-value, rentang estimasi nilai rata-rata berdasarkan tingkat kepercayaan yang digunakan, serta estimasi nilai rata-rata sampel. Fungsi \texttt{wilcox.test()} akan menghasilkan dua buah output yaitu nilai W dan p-value berdasarkan nilai W yang dihasilkan.

\hypertarget{korelasi-antar-variabel}{%
\section{Korelasi Antar Variabel}\label{korelasi-antar-variabel}}

Pada sebuah analisa, kita sering kali tertarik untuk menganalisa hubungan atau korelasi antara satu variabel terhadap variabel lainnya. Pengamatan adanya korelasi antar variabel dapat dilakukan melalui visualisasi menggunakan \emph{scatterplot} dan perhitungan matematis menggunakan metode Pearson untuk metode parametrik dan metode rangking Spearman dan Kendall untuk metode non-parametrik. Pada \emph{Chapter} ini kita akan berfokus untuk melakukan uji korelasi menggunakan \texttt{R} menggunakan metode matematis.

Pada \texttt{R} uji korelasi dapat dilakukan dengan menggunakan fungsi \texttt{cor.test()}. Format fungsi tersebut adalah sebagai berikut:

\begin{Shaded}
\begin{Highlighting}[]
\FunctionTok{cor.test}\NormalTok{(x, y,}
         \AttributeTok{alternative =} \FunctionTok{c}\NormalTok{(}\StringTok{"two.sided"}\NormalTok{, }\StringTok{"less"}\NormalTok{, }\StringTok{"greater"}\NormalTok{),}
         \AttributeTok{method =} \FunctionTok{c}\NormalTok{(}\StringTok{"pearson"}\NormalTok{, }\StringTok{"kendall"}\NormalTok{, }\StringTok{"spearman"}\NormalTok{),}
         \AttributeTok{conf.level =} \FloatTok{0.95}\NormalTok{)}
\end{Highlighting}
\end{Shaded}

\begin{quote}
\textbf{Catatan:}

\begin{itemize}
\tightlist
\item
  \texttt{x,y} : vektor numerik.
\item
  \texttt{alternative}: digunakan untuk menentukan jenis uji hipotesis apakah satu sisi(``less'' dan ``greater''), atau dua sisi (``two.sided'').
\item
  \texttt{method} : metode perhitungan korelasi yang digunakan.
\item
  \texttt{conf.level} : tingkat kepercayaan. Secara default tingkat kepercayaan yang digunakan adalah 95\%.
\end{itemize}
\end{quote}

Berikut adalah penerapan fungsi \texttt{cor.test()} berdasarkan metode-metode yang telah disediakan pada fungsi tersebut:

\begin{Shaded}
\begin{Highlighting}[]
\CommentTok{\# Pearson}
\FunctionTok{cor.test}\NormalTok{(}\AttributeTok{x =}\NormalTok{ airquality}\SpecialCharTok{$}\NormalTok{Ozone, }\AttributeTok{y =}\NormalTok{ airquality}\SpecialCharTok{$}\NormalTok{Solar.R,}
         \AttributeTok{alternative =} \StringTok{"two.sided"}\NormalTok{,}
         \AttributeTok{method =} \StringTok{"pearson"}\NormalTok{)}
\end{Highlighting}
\end{Shaded}

\begin{verbatim}
## 
##  Pearson's product-moment correlation
## 
## data:  airquality$Ozone and airquality$Solar.R
## t = 3.9, df = 109, p-value = 2e-04
## alternative hypothesis: true correlation is not equal to 0
## 95 percent confidence interval:
##  0.1732 0.5021
## sample estimates:
##    cor 
## 0.3483
\end{verbatim}

\begin{Shaded}
\begin{Highlighting}[]
\CommentTok{\# Kendall}
\FunctionTok{cor.test}\NormalTok{(}\AttributeTok{x =}\NormalTok{ airquality}\SpecialCharTok{$}\NormalTok{Ozone, }\AttributeTok{y =}\NormalTok{ airquality}\SpecialCharTok{$}\NormalTok{Solar.R,}
         \AttributeTok{alternative =} \StringTok{"two.sided"}\NormalTok{,}
         \AttributeTok{method =} \StringTok{"kendall"}\NormalTok{)}
\end{Highlighting}
\end{Shaded}

\begin{verbatim}
## 
##  Kendall's rank correlation tau
## 
## data:  airquality$Ozone and airquality$Solar.R
## z = 3.7, p-value = 2e-04
## alternative hypothesis: true tau is not equal to 0
## sample estimates:
##    tau 
## 0.2403
\end{verbatim}

\begin{Shaded}
\begin{Highlighting}[]
\CommentTok{\# Spearman}
\FunctionTok{cor.test}\NormalTok{(}\AttributeTok{x =}\NormalTok{ airquality}\SpecialCharTok{$}\NormalTok{Ozone, }\AttributeTok{y =}\NormalTok{ airquality}\SpecialCharTok{$}\NormalTok{Solar.R,}
         \AttributeTok{alternative =} \StringTok{"two.sided"}\NormalTok{,}
         \AttributeTok{method =} \StringTok{"spearman"}\NormalTok{)}
\end{Highlighting}
\end{Shaded}

\begin{verbatim}
## Warning in cor.test.default(x = airquality$Ozone, y =
## airquality$Solar.R, : Cannot compute exact p-value
## with ties
\end{verbatim}

\begin{verbatim}
## 
##  Spearman's rank correlation rho
## 
## data:  airquality$Ozone and airquality$Solar.R
## S = 148561, p-value = 2e-04
## alternative hypothesis: true rho is not equal to 0
## sample estimates:
##    rho 
## 0.3482
\end{verbatim}

Berdasarkan output yang dihasilkan, metode Pearson menghasilkan output berupa nilai t uji, derajat kebebasan, nilai p-value, rentang estimasi nilai korelasi berdasarkan tingkat kepercayaan, dan estimasi nilai korelasi. Metode Kendall dan Spearman disisi lai menghasilkan output berupa nilai z uji dan S untuk masing-masing metode serta nilai p-value berdasarkan nilai statistika uji dan estimasi koefisien korelasi.

\hypertarget{analisis-varians}{%
\section{Analisis Varians}\label{analisis-varians}}

Pada \emph{sub-Chapter} sebelumnya penulis telah menjelaskan uji rata-rata untuk satu sampel dan dua sampel. Pada kenyataannya dalam sebuah percobaan laboratorium, kita tidak hanya membandingkan dua buah grup sampel saja, namun beberapa grup dan sejumlah faktor. Untuk menganalisa apakah variasi perlakuan pada kelompok sampel akan memberikan hasil yang berbeda-beda pada rata-rata tiap grup atau tidak diperlukan analisis varians untuk menganilisa variasi perlakuan atau faktor pada masing-masing grup. Analisis varians dapat dilakukan baik untuk satu faktor maupun dua faktor atau lebih. Untuk melakukannya pada \texttt{R}, kita dapat menggunakan fungsi \texttt{aov()} untuk analisis varians dengan metode parametrik dan \texttt{kruskal.test()} untuk analisis varians dengan menggunakan metode nonparametrik. Berikut adalah format kedua fungsi tersebut:

\begin{Shaded}
\begin{Highlighting}[]
\FunctionTok{aov}\NormalTok{(formula, }\AttributeTok{data =} \ConstantTok{NULL}\NormalTok{)}

\FunctionTok{kruskal.test}\NormalTok{(formula, data)}
\end{Highlighting}
\end{Shaded}

\begin{quote}
\textbf{Catatan:}

\begin{itemize}
\tightlist
\item
  \texttt{formula} : formula model yang digunakan.
\item
  \texttt{data}: dataset yang akan digunakan.
\end{itemize}
\end{quote}

Berikut adalah contoh penerapan kedua fungsi tersebut untuk melihat apakah terdapat beda pada rata-rata konsentrasi bulanan ozon menggunakan dataset \texttt{airquality}:

\begin{Shaded}
\begin{Highlighting}[]
\FunctionTok{summary}\NormalTok{(}\FunctionTok{aov}\NormalTok{(Ozone}\SpecialCharTok{\textasciitilde{}}\NormalTok{Month, airquality))}
\end{Highlighting}
\end{Shaded}

\begin{verbatim}
##              Df Sum Sq Mean Sq F value Pr(>F)  
## Month         1   3387    3387    3.17  0.078 .
## Residuals   114 121756    1068                 
## ---
## Signif. codes:  
## 0 '***' 0.001 '**' 0.01 '*' 0.05 '.' 0.1 ' ' 1
## 37 observations deleted due to missingness
\end{verbatim}

\begin{Shaded}
\begin{Highlighting}[]
\FunctionTok{kruskal.test}\NormalTok{(Ozone}\SpecialCharTok{\textasciitilde{}}\NormalTok{Month, airquality)}
\end{Highlighting}
\end{Shaded}

\begin{verbatim}
## 
##  Kruskal-Wallis rank sum test
## 
## data:  Ozone by Month
## Kruskal-Wallis chi-squared = 29, df = 4, p-value
## = 7e-06
\end{verbatim}

Berdasarkan hasil yang diperoleh diketahui bahwa rata-rata konsentrasi bulanan ozon tidak sama tiap bulannya atau minimal terdapat satu bulan dimana konsentrasi ozonnya berbeda secara signifikan dengan konsentrasi ozon pada bulan-bulan lainnya. Untuk lebih memahami terkait analisis varians pada \texttt{R} dan cara membaca output kedua fungsi tersebut, pembaca dapat membaca tulisan pada halaman situs \href{http://www.sthda.com/english/wiki/comparing-means-in-r}{sthda}.

\hypertarget{analisis-komponen-utama}{%
\section{Analisis Komponen Utama}\label{analisis-komponen-utama}}

Analisis komponen utama menggunakan transformasi ortogonal (umumnya nilai singular atau dekomposisi nilai eigen) untuk mengubah seperangkat variabel pengamatan yang mungkin berkorelasi menjadi seperangkat variabel tidak berkorelasi (ortogonal) yang disebut komponen utama. Transformasi didefinisikan sedemikian rupa sehingga komponen utama pertama memiliki varians setinggi mungkin (menyumbang variabilitas pada data sebanyak mungkin), dan masing-masing komponen berikutnya pada gilirannya memiliki varians tertinggi yang mungkin di bawah kendala, dimana komponen tersebut menjadi ortogonal ke komponen sebelumnya.

Dalam R, analisis komponen utama umumnya dilakukan dengan fungsi \texttt{prcomp\ ()}. Format fungsi tersebut adalah sebagai berikut:

\begin{Shaded}
\begin{Highlighting}[]
\FunctionTok{prcomp}\NormalTok{(x, }\AttributeTok{retx =} \ConstantTok{TRUE}\NormalTok{, }\AttributeTok{center =} \ConstantTok{TRUE}\NormalTok{, }\AttributeTok{scale. =} \ConstantTok{FALSE}\NormalTok{,}
       \AttributeTok{tol =} \ConstantTok{NULL}\NormalTok{)}
\end{Highlighting}
\end{Shaded}

\begin{quote}
\textbf{Catatan:}

\begin{itemize}
\tightlist
\item
  \texttt{x} : data frame atau matriks kompleks numerik.
\item
  \texttt{retx} : nilai logik yang mengindikasikan apakah variabel hasil rotasi perlu ditampilkan.
\item
  \texttt{center} : nilai logik yang mengidikasikan apakah variabel perlu dilakukan pergeseran sehingga nilai rata-ratanya berpusat pada nilai nol.
\item
  \texttt{scale} : nilai logik yang mengidikasikan apakah variabel perlu dilakukan penskalaan sebelum dilakukan analisis.
\item
  \texttt{tol} : nilai toleransi yang menunjukkan batas nilai bagi komponen yang akan dipertahankan. Komponen yang dihilangkan jika simpangan bakunya kurang dari atau sama dengan \texttt{tol\ x\ simpangan\ baku\ pc1}.
\end{itemize}
\end{quote}

Untuk memahami penerapan fungsi tersebut, kita akan melakukan simulasi menggunakan dataset \texttt{iris}. Output yang dihasilkan di bawah ini menunjukkan bagaimana empat variabel numerik ditransformasikan menjadi empat komponen utama. Penskalaan data mungkin tidak diperlukan dalam kasus ini, karena keempat pengukuran memiliki unit yang sama dan besarnya sama. Namun, penskalaan umumnya merupakan praktik yang baik.

\begin{Shaded}
\begin{Highlighting}[]
\NormalTok{iris\_use }\OtherTok{\textless{}{-}}\NormalTok{ iris[,}\SpecialCharTok{{-}}\DecValTok{5}\NormalTok{] }\CommentTok{\# menghilangkan variabel non{-}numerik}
\NormalTok{iris\_pca }\OtherTok{\textless{}{-}} \FunctionTok{prcomp}\NormalTok{(iris\_use, }\AttributeTok{scale. =} \ConstantTok{TRUE}\NormalTok{)}
\NormalTok{iris\_pca}
\end{Highlighting}
\end{Shaded}

\begin{verbatim}
## Standard deviations (1, .., p=4):
## [1] 1.7084 0.9560 0.3831 0.1439
## 
## Rotation (n x k) = (4 x 4):
##                  PC1      PC2     PC3     PC4
## Sepal.Length  0.5211 -0.37742  0.7196  0.2613
## Sepal.Width  -0.2693 -0.92330 -0.2444 -0.1235
## Petal.Length  0.5804 -0.02449 -0.1421 -0.8014
## Petal.Width   0.5649 -0.06694 -0.6343  0.5236
\end{verbatim}

Fungsi \texttt{summary} memberikan proporsi varians total yang dikaitkan dengan masing-masing komponen utama, dan proporsi kumulatif ketika masing-masing komponen ditambahkan. Kita melihat bahwa dua komponen pertama berperan lebih dari 95\% dari total varians.

\begin{Shaded}
\begin{Highlighting}[]
\FunctionTok{summary}\NormalTok{(iris\_pca)}
\end{Highlighting}
\end{Shaded}

\begin{verbatim}
## Importance of components:
##                         PC1   PC2    PC3     PC4
## Standard deviation     1.71 0.956 0.3831 0.14393
## Proportion of Variance 0.73 0.229 0.0367 0.00518
## Cumulative Proportion  0.73 0.958 0.9948 1.00000
\end{verbatim}

Histogram (hasil plot dalam analisis \texttt{prcomp}) secara grafis merekapitulasi proporsi varian yang disumbangkan oleh masing-masing komponen utama, sementara biplot menunjukkan bagaimana variabel awal diproyeksikan pada dua komponen utama pertama (Gambar \ref{fig:prcomp}). Ini juga menunjukkan koordinat dari masing-masing sampel dalam ruang (PC1, PC2). Satu spesies iris (yang berubahmenjadi setosa dari analisis kluster di bawah ini) secara jelas dipisahkan dari dua spesies lain dalam ruang koordinat ini.

\begin{figure}

{\centering \includegraphics[width=0.8\linewidth]{Algoritma_Pemrograman_R_files/figure-latex/prcomp-1} 

}

\caption{Analisis komponen utama data iris.}\label{fig:prcomp}
\end{figure}

Untuk informasi lebih lanjut terkait metode analisis komponen utama, pembaca dapat membacanya pada laman \href{https://little-book-of-r-for-multivariate-analysis.readthedocs.io/en/latest/}{Little Book of R for Multivariate Analysis}.

\hypertarget{analisis-cluster}{%
\section{Analisis Cluster}\label{analisis-cluster}}

Analisis cluster mencoba untuk mengurutkan satu set objek ke dalam kelompok (cluster) sedemikian rupa sehingga objek dalam cluster yang sama lebih mirip satu sama lain dibandingkan objek pada cluster lainnya. Ini digunakan untuk analisis eksplorasi melalui proses penambangan data (\emph{data mining}) di banyak bidang, seperti bioinformatika, biologi evolusi, analisis gambar, lingkungan, dan pembelajaran mesin.

Menurut Wikipedia: ``Analisis Cluster itu sendiri bukanlah salah satu algoritma spesifik, tetapi tugas umum yang harus dipecahkan. Ini dapat dicapai dengan berbagai algoritma yang berbeda secara signifikan dalam pengertian mereka tentang apa yang merupakan sebuah cluster dan bagaimana cara menemukannya secara efisien. Gagasan populer mengenai cluster termasuk kelompok dengan jarak rendah di antara anggota cluster, area padat ruang data, interval atau distribusi statistik tertentu. Algoritma pengelompokan dan pengaturan parameter yang sesuai (termasuk nilai-nilai seperti fungsi jarak yang akan digunakan, ambang kepadatan atau jumlah cluster yang diharapkan) tergantung pada dataset individual dan tujuan penggunaan hasil. Analisis cluster seperti itu bukan tugas otomatis, tetapi proses berulang penemuan pengetahuan yang melibatkan \emph{trial and error}. Seringkali diperlukan untuk memodifikasi preprocessing dan parameter sampai hasilnya mencapai properti yang diinginkan.

\hypertarget{analisis-cluster-menggunakan-algoritma-pengelompokan-hierarkis-aglomeratif}{%
\subsection{Analisis Cluster Menggunakan Algoritma Pengelompokan Hierarkis Aglomeratif}\label{analisis-cluster-menggunakan-algoritma-pengelompokan-hierarkis-aglomeratif}}

Hierarchical clustering membangun hierarki cluster, di mana metrik hierarki adalah suatu ukuran ketidaksamaan antar cluster. Menurut halaman bantuan untuk \texttt{hclust()}, metode pengelompokan hierarkis aglomeratif, ``Fungsi ini melakukan analisis hierarki cluster menggunakan seperangkat ketidaksamaan untuk n objek yang dikelompokkan. Awalnya, masing-masing objek ditugaskan ke cluster sendiri dan kemudian algoritma melanjutkan secara iteratif, pada setiap tahap bergabung dengan dua cluster yang paling mirip, terus sampai hanya ada satu cluster. Pada setiap tahap, jarak antar kluster dihitung ulang dengan formula pembaruan ketidaksamaan Lance Williams sesuai dengan metode pengelompokan tertentu yang digunakan.''Ada tujuh metode aglomerasi yang tersedia, dengan lengkap --- yang mencari kluster kompak, bola --- sebagai default. Format umum fungsi \texttt{hclust()} adalah sebagai berikut:

\begin{Shaded}
\begin{Highlighting}[]
\FunctionTok{hclust}\NormalTok{(d, }\AttributeTok{method =} \StringTok{"complete"}\NormalTok{, }\AttributeTok{members =} \ConstantTok{NULL}\NormalTok{)}
\end{Highlighting}
\end{Shaded}

\begin{quote}
\textbf{Catatan:}

\begin{itemize}
\tightlist
\item
  \texttt{d} : struktur ketidaksamaan yang dihasilkan dengan fungsi \texttt{dist}.
\item
  \texttt{method} : metode alglomerasi yang digunakan. Metode yang dapat digunakan antara lain: ``ward.D'', ``ward.D2'', ``single'', ``complete'', ``average'' (= UPGMA), ``mcquitty'' (= WPGMA), ``median'' (= WPGMC) atau ``centroid'' (= UPGMC)
\item
  \texttt{members}: NULL atau vektor dengan ukuran sama dengan \texttt{d}. Untuk info lebih lanjut jalankan sintaks \texttt{?hclust}.
\end{itemize}
\end{quote}

Untuk memahami penerapan fungsi \texttt{hclust()}, kita akan kembali menggunakan data \texttt{iris\_use}. Berikut adalah sintaks yang digunakan:

\begin{Shaded}
\begin{Highlighting}[]
\CommentTok{\# menghitung jarak antar observasi}
\NormalTok{iris\_hclust }\OtherTok{\textless{}{-}} \FunctionTok{dist}\NormalTok{(iris\_use)}

\CommentTok{\# Pembentukan Cluster}
\NormalTok{hc }\OtherTok{\textless{}{-}} \FunctionTok{hclust}\NormalTok{(iris\_hclust)}
\NormalTok{hc}
\end{Highlighting}
\end{Shaded}

\begin{verbatim}
## 
## Call:
## hclust(d = iris_hclust)
## 
## Cluster method   : complete 
## Distance         : euclidean 
## Number of objects: 150
\end{verbatim}

Visualisasi cluster dibentuk melalui dendogram yang ditampilkan pada Gambar \ref{fig:hclust}.

\begin{figure}

{\centering \includegraphics[width=0.8\linewidth]{Algoritma_Pemrograman_R_files/figure-latex/hclust-1} 

}

\caption{Analisis pengelompokan hierarkis alglomeratif data iris.}\label{fig:hclust}
\end{figure}

\hypertarget{pengelompokan-hierarkis-divisif}{%
\subsection{Pengelompokan Hierarkis Divisif}\label{pengelompokan-hierarkis-divisif}}

Menurut halaman bantuan \texttt{diana()} (\emph{DIvisive ANAlysis Clustering}) dalam \emph{library} \texttt{cluster}, ``Algoritma diana membangun hierarki pengelompokan, dimulai dengan satu kluster besar yang berisi semua n pengamatan. Cluster dibagi sampai masing-masing cluster hanya berisi satu pengamatan. Pada setiap tahap, cluster dengan diameter terbesar dipilih. Format fungsi \texttt{diana()} adalah sebagai berikut:

\begin{Shaded}
\begin{Highlighting}[]
\FunctionTok{diana}\NormalTok{(x, }\AttributeTok{metric =} \StringTok{"euclidean"}\NormalTok{, }\AttributeTok{stand =} \ConstantTok{FALSE}\NormalTok{)}
\end{Highlighting}
\end{Shaded}

\begin{quote}
\textbf{Catatan:}

\begin{itemize}
\tightlist
\item
  \texttt{x} : struktur ketidaksamaan yang dihasilkan dengan fungsi \texttt{dist} atau data frame.
\item
  \texttt{metric} : karakter string yang menyatakan metode pengukuran jarak yang digunakan. Metode pengukuran jarak dapat berupa ``euclidean'' dan ``manhattan''.
\item
  \texttt{stand}: vektor logik yang menyatakan apakah data akan dilakukan standardisasi terlebih dahulu sebelum dilakukan analisis.
\end{itemize}
\end{quote}

\begin{figure}

{\centering \includegraphics[width=0.8\linewidth]{Algoritma_Pemrograman_R_files/figure-latex/diana-1} 

}

\caption{Analisis pengelompokan divisif data iris.}\label{fig:diana}
\end{figure}

\hypertarget{pengelompokan-menggunakan-algoritma-k-mean}{%
\subsection{Pengelompokan Menggunakan Algoritma K-Mean}\label{pengelompokan-menggunakan-algoritma-k-mean}}

k-means melakukan pengelompokan n pengamatan ke dalam k cluster di mana setiap pengamatan akan tergabung dengan pusat cluster terdekat. Pengguna harus menentukan jumlah pusat (cluster) yang diinginkan sebagai output. Untuk melakukan pengelompokan dengan algoritma k-means pada \texttt{R} dapat menggunakan fungsi \texttt{kmeans()}. Format fungsi tersebut secara umum adalah sebagai berikut:

\begin{Shaded}
\begin{Highlighting}[]
\FunctionTok{kmeans}\NormalTok{(x, centers, }\AttributeTok{iter.max =} \DecValTok{10}\NormalTok{,}
       \AttributeTok{algorithm =} \FunctionTok{c}\NormalTok{(}\StringTok{"Hartigan{-}Wong"}\NormalTok{, }\StringTok{"Lloyd"}\NormalTok{, }\StringTok{"Forgy"}\NormalTok{,}
                     \StringTok{"MacQueen"}\NormalTok{))}
\end{Highlighting}
\end{Shaded}

\begin{quote}
\textbf{Catatan:}

\begin{itemize}
\tightlist
\item
  \texttt{x} : data frame.
\item
  \texttt{centers} : jumlah cluster yang ingin di buat.
\item
  \texttt{iter.max}: jumlah iterasi maksimum yang diijinkan.
\item
  \texttt{algorithm} : algoritma pengelompokan yang digunakan. Untuk informasi lebih lanjut jalankan sintaks \texttt{?kmeans}.
\end{itemize}
\end{quote}

\begin{Shaded}
\begin{Highlighting}[]
\NormalTok{iris\_kmeans }\OtherTok{\textless{}{-}} \FunctionTok{kmeans}\NormalTok{(iris\_use, }\AttributeTok{centers =} \DecValTok{3}\NormalTok{)}
\NormalTok{iris\_kmeans}
\end{Highlighting}
\end{Shaded}

\begin{verbatim}
## K-means clustering with 3 clusters of sizes 50, 38, 62
## 
## Cluster means:
##   Sepal.Length Sepal.Width Petal.Length Petal.Width
## 1        5.006       3.428        1.462       0.246
## 2        6.850       3.074        5.742       2.071
## 3        5.902       2.748        4.394       1.434
## 
## Clustering vector:
##   [1] 1 1 1 1 1 1 1 1 1 1 1 1 1 1 1 1 1 1 1 1 1 1 1 1 1
##  [26] 1 1 1 1 1 1 1 1 1 1 1 1 1 1 1 1 1 1 1 1 1 1 1 1 1
##  [51] 3 3 2 3 3 3 3 3 3 3 3 3 3 3 3 3 3 3 3 3 3 3 3 3 3
##  [76] 3 3 2 3 3 3 3 3 3 3 3 3 3 3 3 3 3 3 3 3 3 3 3 3 3
## [101] 2 3 2 2 2 2 3 2 2 2 2 2 2 3 3 2 2 2 2 3 2 3 2 3 2
## [126] 2 3 3 2 2 2 2 2 3 2 2 2 2 3 2 2 2 3 2 2 2 3 2 2 3
## 
## Within cluster sum of squares by cluster:
## [1] 15.15 23.88 39.82
##  (between_SS / total_SS =  88.4 %)
## 
## Available components:
## 
## [1] "cluster"      "centers"      "totss"       
## [4] "withinss"     "tot.withinss" "betweenss"   
## [7] "size"         "iter"         "ifault"
\end{verbatim}

\begin{Shaded}
\begin{Highlighting}[]
\CommentTok{\# menghitung lokasi pusat cluster}
\NormalTok{ccent }\OtherTok{=} \ControlFlowTok{function}\NormalTok{(cl) \{}
\NormalTok{ f }\OtherTok{=} \ControlFlowTok{function}\NormalTok{(i) }\FunctionTok{colMeans}\NormalTok{(iris\_use[cl}\SpecialCharTok{==}\NormalTok{i,])}
\NormalTok{ x }\OtherTok{=} \FunctionTok{sapply}\NormalTok{(}\FunctionTok{sort}\NormalTok{(}\FunctionTok{unique}\NormalTok{(cl)), f)}
 \FunctionTok{colnames}\NormalTok{(x) }\OtherTok{=} \FunctionTok{sort}\NormalTok{(}\FunctionTok{unique}\NormalTok{(cl))}
 \FunctionTok{return}\NormalTok{(x)}
\NormalTok{\}}

\FunctionTok{ccent}\NormalTok{(iris\_kmeans}\SpecialCharTok{$}\NormalTok{cluster)}
\end{Highlighting}
\end{Shaded}

\begin{verbatim}
##                  1     2     3
## Sepal.Length 5.006 6.850 5.902
## Sepal.Width  3.428 3.074 2.748
## Petal.Length 1.462 5.742 4.394
## Petal.Width  0.246 2.071 1.434
\end{verbatim}

\hypertarget{pengelompokan-menggunakan-algoritma-pam}{%
\subsection{Pengelompokan Menggunakan Algoritma PAM}\label{pengelompokan-menggunakan-algoritma-pam}}

pam mem-partisi data menjadi k cluster di sekitar medoid. Medoid dari set data yang terbatas merupakan titik data dengan nilai ketidaksamaan rata-rata untuk semua titik data adalah minimum. Hal tersebut menujukkan bahwa medoid merupakan pusat dari set cluster. Menurut halaman bantuan \texttt{pam()}, pendekatan k-medoid lebih kuat daripada pendekatan k-means ``karena meminimalkan jumlah ketidaksamaan daripada jumlah jarak euclidean kuadrat''. Format umum fungsi \texttt{pam()} adalah sebagai berikut:

\begin{Shaded}
\begin{Highlighting}[]
\FunctionTok{pam}\NormalTok{(x, k, }\AttributeTok{diss =} \FunctionTok{inherits}\NormalTok{(x, }\StringTok{"dist"}\NormalTok{),}
    \AttributeTok{metric =} \FunctionTok{c}\NormalTok{(}\StringTok{"euclidean"}\NormalTok{, }\StringTok{"manhattan"}\NormalTok{))}
\end{Highlighting}
\end{Shaded}

\begin{quote}
\textbf{Catatan:}

\begin{itemize}
\tightlist
\item
  \texttt{x} : data frame.
\item
  \texttt{k} : jumlah cluster yang ingin di buat.
\item
  \texttt{method}: metode perhitungan jarak yang digunakan. Untuk informasi lebih lanjut jalankan sintaks \texttt{?pam}.
\end{itemize}
\end{quote}

\begin{Shaded}
\begin{Highlighting}[]
\FunctionTok{library}\NormalTok{(cluster)}
\NormalTok{iris\_pam }\OtherTok{\textless{}{-}} \FunctionTok{pam}\NormalTok{(iris\_use, }\AttributeTok{k=}\DecValTok{3}\NormalTok{)}
\NormalTok{iris\_pam}
\end{Highlighting}
\end{Shaded}

\begin{verbatim}
## Medoids:
##       ID Sepal.Length Sepal.Width Petal.Length
## [1,]   8          5.0         3.4          1.5
## [2,]  79          6.0         2.9          4.5
## [3,] 113          6.8         3.0          5.5
##      Petal.Width
## [1,]         0.2
## [2,]         1.5
## [3,]         2.1
## Clustering vector:
##   [1] 1 1 1 1 1 1 1 1 1 1 1 1 1 1 1 1 1 1 1 1 1 1 1 1 1
##  [26] 1 1 1 1 1 1 1 1 1 1 1 1 1 1 1 1 1 1 1 1 1 1 1 1 1
##  [51] 2 2 3 2 2 2 2 2 2 2 2 2 2 2 2 2 2 2 2 2 2 2 2 2 2
##  [76] 2 2 3 2 2 2 2 2 2 2 2 2 2 2 2 2 2 2 2 2 2 2 2 2 2
## [101] 3 2 3 3 3 3 2 3 3 3 3 3 3 2 2 3 3 3 3 2 3 2 3 2 3
## [126] 3 2 2 3 3 3 3 3 2 3 3 3 3 2 3 3 3 2 3 3 3 2 3 3 2
## Objective function:
##  build   swap 
## 0.6709 0.6542 
## 
## Available components:
##  [1] "medoids"    "id.med"     "clustering"
##  [4] "objective"  "isolation"  "clusinfo"  
##  [7] "silinfo"    "diss"       "call"      
## [10] "data"
\end{verbatim}

\begin{figure}

{\centering \includegraphics[width=0.8\linewidth]{Algoritma_Pemrograman_R_files/figure-latex/pam-1} 

}

\caption{Analisis pengelompokan metode pam data iris.}\label{fig:pam}
\end{figure}

\hypertarget{referensi-6}{%
\section{Referensi}\label{referensi-6}}

\begin{enumerate}
\def\labelenumi{\arabic{enumi}.}
\tightlist
\item
  Bloomfield, V.A. 2014. \textbf{Using R for Numerical Analysis in Science and Engineering}. CRC Press.
\item
  Coqhlan, A. Tanpa Tahun. \textbf{Using R for Multivariate Analysis}. \url{https://little-book-of-r-for-multivariate-analysis.readthedocs.io/en/latest/src/multivariateanalysis.html\#principal-component-analysis}.
\item
  Primartha, R. 2018. \textbf{Belajar Machine Learning Teori dan Praktik}. Penerbit Informatika : Bandung
\item
  Rosadi,D. 2016. \textbf{Analisis Statistika dengan R}. Gadjah Mada University Press: Yogyakarta.
\item
  Rosidi, M. 2019. \textbf{Uji Hipotesis}. \url{https://environmental-data-modeling.netlify.com/tutorial/11_uji_hipotesis/}.
\item
  STHDA. Tanpa Tahun. \textbf{Comparing Means in R}. \url{http://www.sthda.com/english/wiki/comparing-means-in-r}.
\end{enumerate}

\hypertarget{datamod}{%
\chapter{Pemodelan Data}\label{datamod}}

Pada Chapter \ref{datamod}, kita akan membahas cara membentuk model statistik menggunakan \texttt{R}. Terdapat 2 buah jenis model yang akan dibahas pada \emph{Chapter} ini, yaitu: regresi dan klasifikasi. Untuk informasi terkait cara untuk melakukan inferensi berdasarkan hasil yang diperoleh dan cara untuk melakukan prediksi menggunakan model yang terbentuk tidak akan dijelaskan dalam buku ini. Pembaca dapat membaca lebih lanjut pada referensi berikut:

\begin{itemize}
\tightlist
\item
  \href{http://ipsur.r-forge.r-project.org/book/download/IPSUR.pdf}{Introduction to Probability and Statistics Using R}
\item
  \href{http://www.sthda.com/english/}{STHDA}
\item
  \href{https://faculty.marshall.usc.edu/gareth-james/ISL/ISLR\%20Seventh\%20Printing.pdf}{An Introduction to Statistical Learning}
\end{itemize}

\hypertarget{reglin}{%
\section{Regresi Linier}\label{reglin}}

Regresi linier merupakan model sederhana yang paling sering dibahas dalam buku-buku statistika. Modelnya cukup sederhana dimana kita berusaha membentuk model dengan pendekatan garis linier dengan prinsip meminimalkan jumlah kuadrat residual pada data. Model yang tebentuk akan menghasilkan dua buah nilai yaitu nilai konstanta (titik potong sumbu y) dan nilai slope kurva. Model yang terbentuk secara umum haruslah memenuhi asumsi dasar model linier berikut:

\begin{enumerate}
\def\labelenumi{\arabic{enumi}.}
\tightlist
\item
  \textbf{Asumsi liniearitas}: kurva relasi yang terbentuk antara variabel independen terhadap variabel dependen harus linier. Asumsi ini dapat dipelajari melalui plot residual terhadap nilai \emph{fitted value}. Jika asumsi liniearitas terpenuhi, maka titik-titik residual yang di plotkan akan membentuk pola acak. Jika pada plot yang dihasilkan terbentuk pola tidak linear maka transformasi data pada variabel prediktor atau independen diperlukan.
\item
  \textbf{Error atau residu berdristribusi normal}: normalitas error di cek menggunakan qq-plot atau uji normalitas yang telah dibahas pada Chapter \ref{norm}.
\item
  \textbf{\emph{Outlier} dan \emph{high influence point}}: kedua pengamatan tersebut dideteksi melalui qq-plot, plot residual terhadap nilai \emph{fitted value}, dan plot \emph{residuals vs leverage}. Jika \emph{outlier} terjadi akibat adanya error selama pengukuran maka \emph{outlier} dapat dihilangkan.
\item
  \textbf{Error bersifat independen}: independensi residual dapat dideteksi melaui plot korelasi serial dengan mengeplotkan \(r_i\) vs \(r_{i-1}\).
\item
  \textbf{Varians bersifat konstan}: Varians bersifat konstan dicek melalui plot \textbf{square root standardize residual vs fitted value}. Pada kasus dimana varians tidak bersifat konstan, kita dapat memberikan bobot pada model yang akan kita bentuk (\emph{weighted least square}), dimana bobot yang diberikan proporsional dengan invers varians.
\item
  \textbf{multikolinearitas}: tidak ada variabel dependen yang saling berfkorelasi. Multikolinearitas dapat dideteksi melalui plot matriks korelasi. Pada model adanya kolinearitas ditunjukkan dari nilai \emph{variance inflation factor} (VIF) yang tinggi. Secara umum nilai VIF terkecil sebesar 1 dan jika kolinearitas terjadi nilainya dapat lebih besar dari 5 atau 10. Untuk mengatasi kolinearitas pada model dapat dilakukan dengan dua cara, yaitu: mengeluarkan variabel dengan nilai VIF yang tinggi pada model atau menggabungkan dua variabel prediktor yang saling berkorelasi menjadi satu variabel baru.
\end{enumerate}

Pembentukan model linier pada \texttt{R} dilakukan dengan menggunakan fungsi \texttt{lm()}. Format umum fungsi tersebut adalah sebagai berikut:

\begin{Shaded}
\begin{Highlighting}[]
\FunctionTok{lm}\NormalTok{(formula, data, subset, weights)}
\end{Highlighting}
\end{Shaded}

\begin{quote}
\textbf{Catatan:}

\begin{itemize}
\tightlist
\item
  \texttt{formula} : formula model yang hendak dibentuk.
\item
  \texttt{data}: data yang digunakan untuk membentuk model.
\item
  \texttt{subset} : subset data yang akan digunakan dalam pembentukan model.
\item
  \texttt{weight} : nilai pembobotan dalam pembentukan model.
\end{itemize}
\end{quote}

\hypertarget{SLR}{%
\subsection{\texorpdfstring{Regrasi Linier Sederhana (\emph{Simple Linear Regression})}{Regrasi Linier Sederhana (Simple Linear Regression)}}\label{SLR}}

Pada Chapter \ref{SLR} akan diberikan contoh pembentukan model linier sederhana menggunakan dataset \texttt{Boston} dari \emph{library} \texttt{MASS} dengan jumlah observasi sebesar 506 observasi. Pada contoh kali ini kita akan mencoba membentuk model dengan variabel dependen berupa \texttt{medv} (median harga rumah) dan variabel independen berupa \texttt{lstat} (persen rumah tangga dengan status ekonomi menengah ke bawah). Berikut adalh sintaks untuk membentuk model tersebut:

\begin{Shaded}
\begin{Highlighting}[]
\FunctionTok{library}\NormalTok{(MASS)}
\end{Highlighting}
\end{Shaded}

\begin{Shaded}
\begin{Highlighting}[]
\NormalTok{lm.fit }\OtherTok{\textless{}{-}} \FunctionTok{lm}\NormalTok{(medv}\SpecialCharTok{\textasciitilde{}}\NormalTok{lstat, }\AttributeTok{data=}\NormalTok{Boston)}
\FunctionTok{anova}\NormalTok{(lm.fit)}
\end{Highlighting}
\end{Shaded}

\begin{verbatim}
## Analysis of Variance Table
## 
## Response: medv
##            Df Sum Sq Mean Sq F value Pr(>F)    
## lstat       1  23244   23244     602 <2e-16 ***
## Residuals 504  19472      39                   
## ---
## Signif. codes:  
## 0 '***' 0.001 '**' 0.01 '*' 0.05 '.' 0.1 ' ' 1
\end{verbatim}

\begin{Shaded}
\begin{Highlighting}[]
\FunctionTok{summary}\NormalTok{(lm.fit)}
\end{Highlighting}
\end{Shaded}

\begin{verbatim}
## 
## Call:
## lm(formula = medv ~ lstat, data = Boston)
## 
## Residuals:
##    Min     1Q Median     3Q    Max 
## -15.17  -3.99  -1.32   2.03  24.50 
## 
## Coefficients:
##             Estimate Std. Error t value Pr(>|t|)    
## (Intercept)  34.5538     0.5626    61.4   <2e-16 ***
## lstat        -0.9500     0.0387   -24.5   <2e-16 ***
## ---
## Signif. codes:  
## 0 '***' 0.001 '**' 0.01 '*' 0.05 '.' 0.1 ' ' 1
## 
## Residual standard error: 6.22 on 504 degrees of freedom
## Multiple R-squared:  0.544,  Adjusted R-squared:  0.543 
## F-statistic:  602 on 1 and 504 DF,  p-value: <2e-16
\end{verbatim}

Plot residual disajikan pada Gambar \ref{fig:slr}.

\begin{figure}

{\centering \includegraphics[width=0.8\linewidth]{Algoritma_Pemrograman_R_files/figure-latex/slr-1} 

}

\caption{Analisis residual model linier medv vs lstat pada dataset Boston.}\label{fig:slr}
\end{figure}

Berdasarkan hasil plot dapat dilihat bahwa seluruh asumsi model linier tidak terpenuhi. Selain melalui plot residual, uji asumsi model linier dapat juga dilakukan secara matematis. Berikut adalah sintaks yang digunakan:

\begin{Shaded}
\begin{Highlighting}[]
\CommentTok{\# error berdistribusi normal }
\CommentTok{\# (data tidak berdistribusi normal)}
\FunctionTok{shapiro.test}\NormalTok{(}\FunctionTok{residuals}\NormalTok{(lm.fit))}
\end{Highlighting}
\end{Shaded}

\begin{verbatim}
## 
##  Shapiro-Wilk normality test
## 
## data:  residuals(lm.fit)
## W = 0.88, p-value <2e-16
\end{verbatim}

\begin{Shaded}
\begin{Highlighting}[]
\CommentTok{\# varians bersifat konstan }
\CommentTok{\# (varians tidak konstan)}
\FunctionTok{library}\NormalTok{(lmtest)}
\FunctionTok{bptest}\NormalTok{(lm.fit)}
\end{Highlighting}
\end{Shaded}

\begin{verbatim}
## 
##  studentized Breusch-Pagan test
## 
## data:  lm.fit
## BP = 15, df = 1, p-value = 8e-05
\end{verbatim}

\begin{Shaded}
\begin{Highlighting}[]
\CommentTok{\# error bersifat independen}
\CommentTok{\# (error tidak bersifat independen)}
\FunctionTok{dwtest}\NormalTok{(lm.fit, }\AttributeTok{alternative =} \StringTok{"two.sided"}\NormalTok{)}
\end{Highlighting}
\end{Shaded}

\begin{verbatim}
## 
##  Durbin-Watson test
## 
## data:  lm.fit
## DW = 0.89, p-value <2e-16
## alternative hypothesis: true autocorrelation is not 0
\end{verbatim}

\begin{Shaded}
\begin{Highlighting}[]
\CommentTok{\# deteksi outlier (stdres \textgreater{} 2)}
\NormalTok{sres }\OtherTok{\textless{}{-}} \FunctionTok{rstandard}\NormalTok{(lm.fit)}
\NormalTok{sres[}\FunctionTok{which}\NormalTok{(}\FunctionTok{abs}\NormalTok{(sres)}\SpecialCharTok{\textgreater{}}\DecValTok{2}\NormalTok{)] }\CommentTok{\# nomor observasi outlier}
\end{Highlighting}
\end{Shaded}

\begin{verbatim}
##     99    142    162    163    164    167    181 
##  2.038  2.038  2.759  2.787  3.000  3.058  2.003 
##    187    196    204    205    215    225    226 
##  3.172  2.947  2.833  2.934  2.789  2.287  3.200 
##    229    234    257    258    262    263    268 
##  2.560  2.822  2.001  3.274  2.488  3.201  3.628 
##    281    283    284    369    370    371    372 
##  2.326  2.308  2.976  2.991  3.063  2.946  3.946 
##    373    375    413    506 
##  3.847  2.498  2.601 -2.444
\end{verbatim}

\begin{Shaded}
\begin{Highlighting}[]
\CommentTok{\# influential observation}
\CommentTok{\# observasi \textgreater{} percentil 50}
\CommentTok{\# tidak ada observasi dengan jarak cook yang extrim}
\NormalTok{cooksD }\OtherTok{\textless{}{-}} \FunctionTok{cooks.distance}\NormalTok{(lm.fit)}
\NormalTok{p50 }\OtherTok{\textless{}{-}} \FunctionTok{qf}\NormalTok{(}\FloatTok{0.5}\NormalTok{, }\AttributeTok{df1=}\DecValTok{2}\NormalTok{, }\AttributeTok{df2=}\DecValTok{560{-}2}\NormalTok{)}
\FunctionTok{any}\NormalTok{(cooksD}\SpecialCharTok{\textgreater{}}\NormalTok{p50)}
\end{Highlighting}
\end{Shaded}

\begin{verbatim}
## [1] FALSE
\end{verbatim}

\hypertarget{MLR}{%
\subsection{\texorpdfstring{Regresi Linier Berganda (\emph{Multiple Linier Regression})}{Regresi Linier Berganda (Multiple Linier Regression)}}\label{MLR}}

Pada Chapter \ref{MLR}, kita akan membuat tiga buah model regresi linier. Model pertama akan menambahkan variabel \texttt{age} (usia bangunan) pada model sebelumnya, model kedua akan menggunakan seluruh ariabel yang ada, dan model ketiga akan melakukan pembaharuan dengan mengeluarkan variabel dengan VIF paling tinggi dari model kedua. Berikut adalah sintaks untuk membentuk ketiag model tersebut:

\begin{Shaded}
\begin{Highlighting}[]
\FunctionTok{library}\NormalTok{(car)}
\CommentTok{\# Model pertama}
\NormalTok{lm.fit1 }\OtherTok{\textless{}{-}} \FunctionTok{lm}\NormalTok{(medv }\SpecialCharTok{\textasciitilde{}}\NormalTok{ lstat}\SpecialCharTok{+}\NormalTok{age, }\AttributeTok{data=}\NormalTok{Boston)}
\FunctionTok{anova}\NormalTok{(lm.fit1)}
\end{Highlighting}
\end{Shaded}

\begin{verbatim}
## Analysis of Variance Table
## 
## Response: medv
##            Df Sum Sq Mean Sq F value Pr(>F)    
## lstat       1  23244   23244  609.95 <2e-16 ***
## age         1    304     304    7.98 0.0049 ** 
## Residuals 503  19168      38                   
## ---
## Signif. codes:  
## 0 '***' 0.001 '**' 0.01 '*' 0.05 '.' 0.1 ' ' 1
\end{verbatim}

\begin{Shaded}
\begin{Highlighting}[]
\FunctionTok{summary}\NormalTok{(lm.fit1)}
\end{Highlighting}
\end{Shaded}

\begin{verbatim}
## 
## Call:
## lm(formula = medv ~ lstat + age, data = Boston)
## 
## Residuals:
##    Min     1Q Median     3Q    Max 
## -15.98  -3.98  -1.28   1.97  23.16 
## 
## Coefficients:
##             Estimate Std. Error t value Pr(>|t|)    
## (Intercept)  33.2228     0.7308   45.46   <2e-16 ***
## lstat        -1.0321     0.0482  -21.42   <2e-16 ***
## age           0.0345     0.0122    2.83   0.0049 ** 
## ---
## Signif. codes:  
## 0 '***' 0.001 '**' 0.01 '*' 0.05 '.' 0.1 ' ' 1
## 
## Residual standard error: 6.17 on 503 degrees of freedom
## Multiple R-squared:  0.551,  Adjusted R-squared:  0.549 
## F-statistic:  309 on 2 and 503 DF,  p-value: <2e-16
\end{verbatim}

\begin{Shaded}
\begin{Highlighting}[]
\FunctionTok{vif}\NormalTok{(lm.fit1)}
\end{Highlighting}
\end{Shaded}

\begin{verbatim}
## lstat   age 
## 1.569 1.569
\end{verbatim}

Berdasarkan hasil perhitungan diketahui nilai VIF dari model \textless{} 10, sehingga asumsi multikolinearitas terpenuhi. Untuk asumsi lainnya dapat dicek pada plot residual yang ditampilkan pada Gambar \ref{fig:mlr1}.

\begin{figure}

{\centering \includegraphics[width=0.8\linewidth]{Algoritma_Pemrograman_R_files/figure-latex/mlr1-1} 

}

\caption{Analisis residual model linier 1 pada dataset Boston.}\label{fig:mlr1}
\end{figure}

\begin{Shaded}
\begin{Highlighting}[]
\CommentTok{\# Model 2}
\NormalTok{lm.fit2 }\OtherTok{\textless{}{-}} \FunctionTok{lm}\NormalTok{(medv}\SpecialCharTok{\textasciitilde{}}\NormalTok{., }\AttributeTok{data=}\NormalTok{Boston)}
\FunctionTok{anova}\NormalTok{(lm.fit2)}
\end{Highlighting}
\end{Shaded}

\begin{verbatim}
## Analysis of Variance Table
## 
## Response: medv
##            Df Sum Sq Mean Sq F value  Pr(>F)    
## crim        1   6441    6441  286.03 < 2e-16 ***
## zn          1   3554    3554  157.85 < 2e-16 ***
## indus       1   2551    2551  113.30 < 2e-16 ***
## chas        1   1530    1530   67.94 1.5e-15 ***
## nox         1     76      76    3.39 0.06635 .  
## rm          1  10938   10938  485.75 < 2e-16 ***
## age         1     90      90    4.01 0.04581 *  
## dis         1   1780    1780   79.03 < 2e-16 ***
## rad         1     34      34    1.52 0.21883    
## tax         1    330     330   14.64 0.00015 ***
## ptratio     1   1309    1309   58.15 1.3e-13 ***
## black       1    593     593   26.35 4.1e-07 ***
## lstat       1   2411    2411  107.06 < 2e-16 ***
## Residuals 492  11079      23                    
## ---
## Signif. codes:  
## 0 '***' 0.001 '**' 0.01 '*' 0.05 '.' 0.1 ' ' 1
\end{verbatim}

\begin{Shaded}
\begin{Highlighting}[]
\FunctionTok{summary}\NormalTok{(lm.fit2)}
\end{Highlighting}
\end{Shaded}

\begin{verbatim}
## 
## Call:
## lm(formula = medv ~ ., data = Boston)
## 
## Residuals:
##     Min      1Q  Median      3Q     Max 
## -15.594  -2.730  -0.518   1.777  26.199 
## 
## Coefficients:
##              Estimate Std. Error t value Pr(>|t|)    
## (Intercept)  3.65e+01   5.10e+00    7.14  3.3e-12 ***
## crim        -1.08e-01   3.29e-02   -3.29  0.00109 ** 
## zn           4.64e-02   1.37e-02    3.38  0.00078 ***
## indus        2.06e-02   6.15e-02    0.33  0.73829    
## chas         2.69e+00   8.62e-01    3.12  0.00193 ** 
## nox         -1.78e+01   3.82e+00   -4.65  4.2e-06 ***
## rm           3.81e+00   4.18e-01    9.12  < 2e-16 ***
## age          6.92e-04   1.32e-02    0.05  0.95823    
## dis         -1.48e+00   1.99e-01   -7.40  6.0e-13 ***
## rad          3.06e-01   6.63e-02    4.61  5.1e-06 ***
## tax         -1.23e-02   3.76e-03   -3.28  0.00111 ** 
## ptratio     -9.53e-01   1.31e-01   -7.28  1.3e-12 ***
## black        9.31e-03   2.69e-03    3.47  0.00057 ***
## lstat       -5.25e-01   5.07e-02  -10.35  < 2e-16 ***
## ---
## Signif. codes:  
## 0 '***' 0.001 '**' 0.01 '*' 0.05 '.' 0.1 ' ' 1
## 
## Residual standard error: 4.75 on 492 degrees of freedom
## Multiple R-squared:  0.741,  Adjusted R-squared:  0.734 
## F-statistic:  108 on 13 and 492 DF,  p-value: <2e-16
\end{verbatim}

\begin{Shaded}
\begin{Highlighting}[]
\FunctionTok{vif}\NormalTok{(lm.fit2)}
\end{Highlighting}
\end{Shaded}

\begin{verbatim}
##    crim      zn   indus    chas     nox      rm 
##   1.792   2.299   3.992   1.074   4.394   1.934 
##     age     dis     rad     tax ptratio   black 
##   3.101   3.956   7.484   9.009   1.799   1.349 
##   lstat 
##   2.941
\end{verbatim}

Berdasarkan hasil perhitungan diperoleh nilai VIF untuk seluruh varaibel prediktor dalam model \textless{} 10, sehingga asumsi multikolinearitas terpenuhi. Untuk asumsi lainnya dapat dicek pada plot residual yang ditampilkan pada Gambar \ref{fig:mlr2}.

\begin{figure}

{\centering \includegraphics[width=0.8\linewidth]{Algoritma_Pemrograman_R_files/figure-latex/mlr2-1} 

}

\caption{Analisis residual model linier 2 pada dataset Boston.}\label{fig:mlr2}
\end{figure}

Pada model ketiga, kita akan mencoba untuk melakukan pembaharuan pada model kedua dengan melakukan drop variabel dengan vif yang paling tinggi. Pada hasil perhitungan sebelumnya, variabel \texttt{tax} (pajak) memiliki nilai VIF yang paling tinggi, sehingga pada model ketiga variabel tersebut tidak disertakan. Terdapat dua cara untuk melakukannya berikut adalah sintaks yang digunakan:

\begin{Shaded}
\begin{Highlighting}[]
\CommentTok{\# Model 3 (cara 1)}
\NormalTok{lm.fit3 }\OtherTok{\textless{}{-}} \FunctionTok{lm}\NormalTok{(medv}\SpecialCharTok{\textasciitilde{}}\NormalTok{.}\SpecialCharTok{{-}}\NormalTok{tax, }\AttributeTok{data=}\NormalTok{Boston)}

\CommentTok{\# Model 3 (cara 2)}
\NormalTok{lm.fit3 }\OtherTok{\textless{}{-}} \FunctionTok{update}\NormalTok{(lm.fit2, }\SpecialCharTok{\textasciitilde{}}\NormalTok{.}\SpecialCharTok{{-}}\NormalTok{tax)}

\FunctionTok{anova}\NormalTok{(lm.fit3)}
\end{Highlighting}
\end{Shaded}

\begin{verbatim}
## Analysis of Variance Table
## 
## Response: medv
##            Df Sum Sq Mean Sq F value  Pr(>F)    
## crim        1   6441    6441  280.48 < 2e-16 ***
## zn          1   3554    3554  154.78 < 2e-16 ***
## indus       1   2551    2551  111.10 < 2e-16 ***
## chas        1   1530    1530   66.62 2.8e-15 ***
## nox         1     76      76    3.32   0.069 .  
## rm          1  10938   10938  476.32 < 2e-16 ***
## age         1     90      90    3.93   0.048 *  
## dis         1   1780    1780   77.49 < 2e-16 ***
## rad         1     34      34    1.49   0.223    
## ptratio     1   1401    1401   61.02 3.4e-14 ***
## black       1    612     612   26.63 3.6e-07 ***
## lstat       1   2388    2388  103.99 < 2e-16 ***
## Residuals 493  11321      23                    
## ---
## Signif. codes:  
## 0 '***' 0.001 '**' 0.01 '*' 0.05 '.' 0.1 ' ' 1
\end{verbatim}

\begin{Shaded}
\begin{Highlighting}[]
\FunctionTok{summary}\NormalTok{(lm.fit3)}
\end{Highlighting}
\end{Shaded}

\begin{verbatim}
## 
## Call:
## lm(formula = medv ~ crim + zn + indus + chas + nox + rm + age + 
##     dis + rad + ptratio + black + lstat, data = Boston)
## 
## Residuals:
##     Min      1Q  Median      3Q     Max 
## -16.145  -2.914  -0.566   1.744  26.311 
## 
## Coefficients:
##              Estimate Std. Error t value Pr(>|t|)    
## (Intercept)  3.46e+01   5.12e+00    6.76  3.9e-11 ***
## crim        -1.07e-01   3.32e-02   -3.22  0.00138 ** 
## zn           3.64e-02   1.35e-02    2.69  0.00735 ** 
## indus       -6.78e-02   5.58e-02   -1.21  0.22532    
## chas         3.03e+00   8.64e-01    3.51  0.00049 ***
## nox         -1.87e+01   3.85e+00   -4.86  1.6e-06 ***
## rm           3.91e+00   4.21e-01    9.29  < 2e-16 ***
## age         -6.05e-04   1.33e-02   -0.05  0.96380    
## dis         -1.49e+00   2.01e-01   -7.39  6.3e-13 ***
## rad          1.35e-01   4.13e-02    3.26  0.00118 ** 
## ptratio     -9.85e-01   1.32e-01   -7.48  3.5e-13 ***
## black        9.55e-03   2.71e-03    3.52  0.00047 ***
## lstat       -5.22e-01   5.12e-02  -10.20  < 2e-16 ***
## ---
## Signif. codes:  
## 0 '***' 0.001 '**' 0.01 '*' 0.05 '.' 0.1 ' ' 1
## 
## Residual standard error: 4.79 on 493 degrees of freedom
## Multiple R-squared:  0.735,  Adjusted R-squared:  0.729 
## F-statistic:  114 on 12 and 493 DF,  p-value: <2e-16
\end{verbatim}

\begin{Shaded}
\begin{Highlighting}[]
\FunctionTok{vif}\NormalTok{(lm.fit3)}
\end{Highlighting}
\end{Shaded}

\begin{verbatim}
##    crim      zn   indus    chas     nox      rm 
##   1.792   2.184   3.226   1.058   4.369   1.923 
##     age     dis     rad ptratio   black   lstat 
##   3.098   3.954   2.837   1.789   1.348   2.941
\end{verbatim}

Plot residual ditampilkan pada Gambar \ref{fig:mlr3}.

\begin{figure}

{\centering \includegraphics[width=0.8\linewidth]{Algoritma_Pemrograman_R_files/figure-latex/mlr3-1} 

}

\caption{Analisis residual model linier 3 pada dataset Boston.}\label{fig:mlr3}
\end{figure}

\hypertarget{lminter}{%
\subsection{Model Linier dengan Interaksi Antar Variabel Prediktor}\label{lminter}}

Interaksi antar variabel pada model linier dapat dengan mudah dimasukkan kedalam fungsi \texttt{lm()}. Terdapat dua buah cara untuk melakukannya. Cara pertama dengan menggunakan tanda \texttt{:} pada formula (contoh: \(y1 ~ x1+x2+x1:x2\)). Tanda \texttt{:} menyatakan formula persamaan linier memasukkan interaksi antar variabel prediktor di dalamnya. Cara kedua adalah dengan menggunakan tanda \texttt{*}. Cara ini lebih sederhana, dimana fungsi \texttt{lm()} akan secara otomatis menerjemahkannya sebagai serangkaian variabel tunggal dan interaksinya. Berikut adalah contoh penerapannya menggunakan kedua cara tersebut:

\begin{Shaded}
\begin{Highlighting}[]
\CommentTok{\# cara 1}
\NormalTok{lm.inter }\OtherTok{\textless{}{-}} \FunctionTok{lm}\NormalTok{(medv}\SpecialCharTok{\textasciitilde{}}\NormalTok{lstat}\SpecialCharTok{+}\NormalTok{age}\SpecialCharTok{+}\NormalTok{lstat}\SpecialCharTok{:}\NormalTok{age, }\AttributeTok{data=}\NormalTok{Boston)}
\FunctionTok{anova}\NormalTok{(lm.inter)}
\end{Highlighting}
\end{Shaded}

\begin{verbatim}
## Analysis of Variance Table
## 
## Response: medv
##            Df Sum Sq Mean Sq F value Pr(>F)    
## lstat       1  23244   23244  614.85 <2e-16 ***
## age         1    304     304    8.05 0.0047 ** 
## lstat:age   1    190     190    5.04 0.0252 *  
## Residuals 502  18978      38                   
## ---
## Signif. codes:  
## 0 '***' 0.001 '**' 0.01 '*' 0.05 '.' 0.1 ' ' 1
\end{verbatim}

\begin{Shaded}
\begin{Highlighting}[]
\FunctionTok{summary}\NormalTok{(lm.inter)}
\end{Highlighting}
\end{Shaded}

\begin{verbatim}
## 
## Call:
## lm(formula = medv ~ lstat + age + lstat:age, data = Boston)
## 
## Residuals:
##    Min     1Q Median     3Q    Max 
## -15.81  -4.04  -1.33   2.08  27.55 
## 
## Coefficients:
##              Estimate Std. Error t value Pr(>|t|)    
## (Intercept) 36.088536   1.469835   24.55  < 2e-16 ***
## lstat       -1.392117   0.167456   -8.31  8.8e-16 ***
## age         -0.000721   0.019879   -0.04    0.971    
## lstat:age    0.004156   0.001852    2.24    0.025 *  
## ---
## Signif. codes:  
## 0 '***' 0.001 '**' 0.01 '*' 0.05 '.' 0.1 ' ' 1
## 
## Residual standard error: 6.15 on 502 degrees of freedom
## Multiple R-squared:  0.556,  Adjusted R-squared:  0.553 
## F-statistic:  209 on 3 and 502 DF,  p-value: <2e-16
\end{verbatim}

\begin{Shaded}
\begin{Highlighting}[]
\CommentTok{\# Cara 2}
\NormalTok{lm.inter }\OtherTok{\textless{}{-}} \FunctionTok{lm}\NormalTok{(medv}\SpecialCharTok{\textasciitilde{}}\NormalTok{lstat}\SpecialCharTok{*}\NormalTok{age, }\AttributeTok{data=}\NormalTok{Boston)}
\FunctionTok{anova}\NormalTok{(lm.inter)}
\end{Highlighting}
\end{Shaded}

\begin{verbatim}
## Analysis of Variance Table
## 
## Response: medv
##            Df Sum Sq Mean Sq F value Pr(>F)    
## lstat       1  23244   23244  614.85 <2e-16 ***
## age         1    304     304    8.05 0.0047 ** 
## lstat:age   1    190     190    5.04 0.0252 *  
## Residuals 502  18978      38                   
## ---
## Signif. codes:  
## 0 '***' 0.001 '**' 0.01 '*' 0.05 '.' 0.1 ' ' 1
\end{verbatim}

\begin{Shaded}
\begin{Highlighting}[]
\FunctionTok{summary}\NormalTok{(lm.inter)}
\end{Highlighting}
\end{Shaded}

\begin{verbatim}
## 
## Call:
## lm(formula = medv ~ lstat * age, data = Boston)
## 
## Residuals:
##    Min     1Q Median     3Q    Max 
## -15.81  -4.04  -1.33   2.08  27.55 
## 
## Coefficients:
##              Estimate Std. Error t value Pr(>|t|)    
## (Intercept) 36.088536   1.469835   24.55  < 2e-16 ***
## lstat       -1.392117   0.167456   -8.31  8.8e-16 ***
## age         -0.000721   0.019879   -0.04    0.971    
## lstat:age    0.004156   0.001852    2.24    0.025 *  
## ---
## Signif. codes:  
## 0 '***' 0.001 '**' 0.01 '*' 0.05 '.' 0.1 ' ' 1
## 
## Residual standard error: 6.15 on 502 degrees of freedom
## Multiple R-squared:  0.556,  Adjusted R-squared:  0.553 
## F-statistic:  209 on 3 and 502 DF,  p-value: <2e-16
\end{verbatim}

Plot residual ditampilkan pada Gambar \ref{fig:lminter}.

\begin{figure}

{\centering \includegraphics[width=0.8\linewidth]{Algoritma_Pemrograman_R_files/figure-latex/lminter-1} 

}

\caption{Analisis residual model dengan melibatkan interaksi antar variabel pada dataset Boston.}\label{fig:lminter}
\end{figure}

\hypertarget{nlinpred}{%
\subsection{Transformasi Non-linier Pada Prediktor}\label{nlinpred}}

Fungsi \texttt{lm()} juga dapat melibatkan transformasi non-linier prediktor pada argumen \texttt{formula}-nya. Transformasi non-linier dilakukan dengan menambahkan fungsi identitas \texttt{I()}. Sebagai contoh model berikut melibatkan transformasi kuadrat pada variabel \texttt{lstat}:

\begin{Shaded}
\begin{Highlighting}[]
\NormalTok{lm.trans }\OtherTok{\textless{}{-}} \FunctionTok{lm}\NormalTok{(medv}\SpecialCharTok{\textasciitilde{}}\NormalTok{lstat}\SpecialCharTok{+}\FunctionTok{I}\NormalTok{(lstat}\SpecialCharTok{\^{}}\DecValTok{2}\NormalTok{), }\AttributeTok{data=}\NormalTok{Boston)}
\FunctionTok{anova}\NormalTok{(lm.trans)}
\end{Highlighting}
\end{Shaded}

\begin{verbatim}
## Analysis of Variance Table
## 
## Response: medv
##             Df Sum Sq Mean Sq F value Pr(>F)    
## lstat        1  23244   23244     762 <2e-16 ***
## I(lstat^2)   1   4125    4125     135 <2e-16 ***
## Residuals  503  15347      31                   
## ---
## Signif. codes:  
## 0 '***' 0.001 '**' 0.01 '*' 0.05 '.' 0.1 ' ' 1
\end{verbatim}

\begin{Shaded}
\begin{Highlighting}[]
\FunctionTok{summary}\NormalTok{(lm.trans)}
\end{Highlighting}
\end{Shaded}

\begin{verbatim}
## 
## Call:
## lm(formula = medv ~ lstat + I(lstat^2), data = Boston)
## 
## Residuals:
##    Min     1Q Median     3Q    Max 
## -15.28  -3.83  -0.53   2.31  25.41 
## 
## Coefficients:
##             Estimate Std. Error t value Pr(>|t|)    
## (Intercept) 42.86201    0.87208    49.1   <2e-16 ***
## lstat       -2.33282    0.12380   -18.8   <2e-16 ***
## I(lstat^2)   0.04355    0.00375    11.6   <2e-16 ***
## ---
## Signif. codes:  
## 0 '***' 0.001 '**' 0.01 '*' 0.05 '.' 0.1 ' ' 1
## 
## Residual standard error: 5.52 on 503 degrees of freedom
## Multiple R-squared:  0.641,  Adjusted R-squared:  0.639 
## F-statistic:  449 on 2 and 503 DF,  p-value: <2e-16
\end{verbatim}

Cara yang lebih sederhana untuk melibatkan tranformasi polinomial kedalam model linier adalah dengan menggunakan fungsi \texttt{poly()}. Berikut adalah contoh penerapannya:

\begin{Shaded}
\begin{Highlighting}[]
\NormalTok{lm.trans }\OtherTok{\textless{}{-}} \FunctionTok{lm}\NormalTok{(medv}\SpecialCharTok{\textasciitilde{}}\FunctionTok{poly}\NormalTok{(lstat,}\DecValTok{2}\NormalTok{), }\AttributeTok{data=}\NormalTok{Boston)}
\FunctionTok{anova}\NormalTok{(lm.trans)}
\end{Highlighting}
\end{Shaded}

\begin{verbatim}
## Analysis of Variance Table
## 
## Response: medv
##                 Df Sum Sq Mean Sq F value Pr(>F)    
## poly(lstat, 2)   2  27369   13685     449 <2e-16 ***
## Residuals      503  15347      31                   
## ---
## Signif. codes:  
## 0 '***' 0.001 '**' 0.01 '*' 0.05 '.' 0.1 ' ' 1
\end{verbatim}

\begin{Shaded}
\begin{Highlighting}[]
\FunctionTok{summary}\NormalTok{(lm.trans)}
\end{Highlighting}
\end{Shaded}

\begin{verbatim}
## 
## Call:
## lm(formula = medv ~ poly(lstat, 2), data = Boston)
## 
## Residuals:
##    Min     1Q Median     3Q    Max 
## -15.28  -3.83  -0.53   2.31  25.41 
## 
## Coefficients:
##                 Estimate Std. Error t value Pr(>|t|)
## (Intercept)       22.533      0.246    91.8   <2e-16
## poly(lstat, 2)1 -152.460      5.524   -27.6   <2e-16
## poly(lstat, 2)2   64.227      5.524    11.6   <2e-16
##                    
## (Intercept)     ***
## poly(lstat, 2)1 ***
## poly(lstat, 2)2 ***
## ---
## Signif. codes:  
## 0 '***' 0.001 '**' 0.01 '*' 0.05 '.' 0.1 ' ' 1
## 
## Residual standard error: 5.52 on 503 degrees of freedom
## Multiple R-squared:  0.641,  Adjusted R-squared:  0.639 
## F-statistic:  449 on 2 and 503 DF,  p-value: <2e-16
\end{verbatim}

Plot residual ditampilkan pada Gambar \ref{fig:lmtrans}.

\begin{figure}

{\centering \includegraphics[width=0.8\linewidth]{Algoritma_Pemrograman_R_files/figure-latex/lmtrans-1} 

}

\caption{Analisis residual model dengan melibatkan transformasi non-linier variabel prediktor pada dataset Boston.}\label{fig:lmtrans}
\end{figure}

FUngsi trasnformasi lainnya juga dapat digunakan pada pembentukan model linier. Berikut adalah contoh penerapan transformasi logaritmik dan eksponensial pada model linier:

\begin{Shaded}
\begin{Highlighting}[]
\FunctionTok{summary}\NormalTok{(}\FunctionTok{lm}\NormalTok{(medv}\SpecialCharTok{\textasciitilde{}}\FunctionTok{log}\NormalTok{(lstat), }\AttributeTok{data=}\NormalTok{Boston))}
\end{Highlighting}
\end{Shaded}

\begin{verbatim}
## 
## Call:
## lm(formula = medv ~ log(lstat), data = Boston)
## 
## Residuals:
##     Min      1Q  Median      3Q     Max 
## -14.460  -3.501  -0.669   2.169  26.013 
## 
## Coefficients:
##             Estimate Std. Error t value Pr(>|t|)    
## (Intercept)   52.125      0.965    54.0   <2e-16 ***
## log(lstat)   -12.481      0.395   -31.6   <2e-16 ***
## ---
## Signif. codes:  
## 0 '***' 0.001 '**' 0.01 '*' 0.05 '.' 0.1 ' ' 1
## 
## Residual standard error: 5.33 on 504 degrees of freedom
## Multiple R-squared:  0.665,  Adjusted R-squared:  0.664 
## F-statistic: 1e+03 on 1 and 504 DF,  p-value: <2e-16
\end{verbatim}

\begin{Shaded}
\begin{Highlighting}[]
\FunctionTok{summary}\NormalTok{(}\FunctionTok{lm}\NormalTok{(medv}\SpecialCharTok{\textasciitilde{}}\FunctionTok{exp}\NormalTok{(lstat), }\AttributeTok{data=}\NormalTok{Boston))}
\end{Highlighting}
\end{Shaded}

\begin{verbatim}
## 
## Call:
## lm(formula = medv ~ exp(lstat), data = Boston)
## 
## Residuals:
##    Min     1Q Median     3Q    Max 
## -17.57  -5.47  -1.37   2.43  27.43 
## 
## Coefficients:
##              Estimate Std. Error t value Pr(>|t|)    
## (Intercept)  2.26e+01   4.09e-01   55.19   <2e-16 ***
## exp(lstat)  -4.44e-16   2.79e-16   -1.59     0.11    
## ---
## Signif. codes:  
## 0 '***' 0.001 '**' 0.01 '*' 0.05 '.' 0.1 ' ' 1
## 
## Residual standard error: 9.18 on 504 degrees of freedom
## Multiple R-squared:  0.00501,    Adjusted R-squared:  0.00303 
## F-statistic: 2.54 on 1 and 504 DF,  p-value: 0.112
\end{verbatim}

\hypertarget{model-linier-dengan-prediktor-kategorikal}{%
\subsection{Model Linier dengan Prediktor Kategorikal}\label{model-linier-dengan-prediktor-kategorikal}}

Regresi linier dapat pula dilakukan dengan jenis variabel prediktor berupa variabel kategorikal. Untuk dapat melakukannya jenis data variabel kategorikal terlebih dahulu dirubah kedalam factor.

BELUM ADA contoh

\hypertarget{wlr}{%
\subsection{Regresi linier dengan Pembobotan}\label{wlr}}

Pada pembahasan sebelumnya kita telah menyaksikan bahwa sebagian model yang telah terbentuk tidak memenuhi asumsi varians yang konstan. Untuk mengatasi hal tersebut, kita dapat membentuk regresi dengan memberikan bobot sebesar invers variansnya. Untuk melakukannya diperlukan beberapa tahapan, antara lain:

\begin{enumerate}
\def\labelenumi{\arabic{enumi}.}
\tightlist
\item
  Membentuk model linier dari variabel dataset: \texttt{fit\ \textless{}-\ lm(y\textasciitilde{}(variabel\ prediktor))}.
\item
  Menghitung error absolut (\texttt{abse\ \textless{}-\ abs(resid(fit))}) dan nilai *fitted value dari model (\texttt{yhat\ \textless{}-\ fitted(fit}).
\item
  Membentuk kembali model menggunakan data residual absolut (\texttt{efit\ \textless{}-\ lm(abse\textasciitilde{}poly(yhat,2))}) dan menghitung \emph{residual fitted value} (\texttt{shat\ \textless{}-\ fitted(efit)}).
\item
  Gunakan nilai bobot \texttt{w\ \textless{}-\ 1/shat\^{}2} untuk membentuk model regresi dengan pembobotan (\texttt{fitw\ \textless{}-lm(y\textasciitilde{}(variabel\ prediktor),\ weights=w)}).
\end{enumerate}

Kita akan membentuk kembali model menggunakan dataset \texttt{Boston} dengan menggunakan seluruh variabel, namun pada model kali ini kita akan memberikan bobot pada model yang terbentuk. Berikut adalah sintaks yang digunakan:

\begin{Shaded}
\begin{Highlighting}[]
\CommentTok{\# langkah 1}
\NormalTok{fit }\OtherTok{\textless{}{-}} \FunctionTok{lm}\NormalTok{(medv}\SpecialCharTok{\textasciitilde{}}\NormalTok{., }\AttributeTok{data=}\NormalTok{Boston)}

\CommentTok{\# langkah 2}
\NormalTok{abse }\OtherTok{\textless{}{-}} \FunctionTok{abs}\NormalTok{(}\FunctionTok{resid}\NormalTok{(fit))}
\NormalTok{yhat }\OtherTok{\textless{}{-}} \FunctionTok{fitted}\NormalTok{(fit)}

\CommentTok{\# langkah 3}
\NormalTok{efit }\OtherTok{\textless{}{-}} \FunctionTok{lm}\NormalTok{(abse}\SpecialCharTok{\textasciitilde{}}\FunctionTok{poly}\NormalTok{(yhat,}\DecValTok{2}\NormalTok{))}
\NormalTok{shat }\OtherTok{\textless{}{-}} \FunctionTok{fitted}\NormalTok{(efit)}

\CommentTok{\# langkah 4}
\NormalTok{fitw }\OtherTok{\textless{}{-}} \FunctionTok{lm}\NormalTok{(medv}\SpecialCharTok{\textasciitilde{}}\NormalTok{., }\AttributeTok{data =}\NormalTok{ Boston, }\AttributeTok{weights =} \DecValTok{1}\SpecialCharTok{/}\NormalTok{(shat}\SpecialCharTok{\^{}}\DecValTok{2}\NormalTok{))}
\FunctionTok{anova}\NormalTok{(fitw)}
\end{Highlighting}
\end{Shaded}

\begin{verbatim}
## Analysis of Variance Table
## 
## Response: medv
##            Df Sum Sq Mean Sq F value  Pr(>F)    
## crim        1    367     367   188.5 < 2e-16 ***
## zn          1    148     148    76.1 < 2e-16 ***
## indus       1    125     125    63.9 9.3e-15 ***
## chas        1     41      41    21.2 5.2e-06 ***
## nox         1     38      38    19.7 1.1e-05 ***
## rm          1    363     363   186.6 < 2e-16 ***
## age         1     30      30    15.6 9.0e-05 ***
## dis         1    106     106    54.3 7.4e-13 ***
## rad         1      0       0     0.0    0.99    
## tax         1     30      30    15.4 9.9e-05 ***
## ptratio     1     74      74    37.9 1.6e-09 ***
## black       1     64      64    33.1 1.6e-08 ***
## lstat       1    334     334   171.7 < 2e-16 ***
## Residuals 492    958       2                    
## ---
## Signif. codes:  
## 0 '***' 0.001 '**' 0.01 '*' 0.05 '.' 0.1 ' ' 1
\end{verbatim}

\begin{Shaded}
\begin{Highlighting}[]
\FunctionTok{summary}\NormalTok{(fitw)}
\end{Highlighting}
\end{Shaded}

\begin{verbatim}
## 
## Call:
## lm(formula = medv ~ ., data = Boston, weights = 1/(shat^2))
## 
## Weighted Residuals:
##    Min     1Q Median     3Q    Max 
## -4.252 -0.795 -0.037  0.698  9.299 
## 
## Coefficients:
##              Estimate Std. Error t value Pr(>|t|)    
## (Intercept)  39.15845    4.47418    8.75  < 2e-16 ***
## crim         -0.13738    0.03254   -4.22  2.9e-05 ***
## zn            0.03975    0.01367    2.91  0.00380 ** 
## indus         0.03107    0.05211    0.60  0.55129    
## chas          2.88679    0.81633    3.54  0.00044 ***
## nox         -15.84421    3.22177   -4.92  1.2e-06 ***
## rm            2.32247    0.41424    5.61  3.4e-08 ***
## age           0.00587    0.01147    0.51  0.60930    
## dis          -1.13778    0.17774   -6.40  3.6e-10 ***
## rad           0.30255    0.05805    5.21  2.8e-07 ***
## tax          -0.01138    0.00328   -3.47  0.00056 ***
## ptratio      -0.72178    0.11598   -6.22  1.0e-09 ***
## black         0.00878    0.00234    3.75  0.00020 ***
## lstat        -0.62185    0.04746  -13.10  < 2e-16 ***
## ---
## Signif. codes:  
## 0 '***' 0.001 '**' 0.01 '*' 0.05 '.' 0.1 ' ' 1
## 
## Residual standard error: 1.4 on 492 degrees of freedom
## Multiple R-squared:  0.642,  Adjusted R-squared:  0.633 
## F-statistic:   68 on 13 and 492 DF,  p-value: <2e-16
\end{verbatim}

Plot residual ditampilkan pada Gambar \ref{fig:lmw}.

\begin{figure}

{\centering \includegraphics[width=0.8\linewidth]{Algoritma_Pemrograman_R_files/figure-latex/lmw-1} 

}

\caption{Analisis residual model regresi linier dengan pembobotan pada dataset Boston.}\label{fig:lmw}
\end{figure}

\hypertarget{logreg}{%
\section{Regresi Logistik}\label{logreg}}

Pada Chapter \ref{logreg}, kita telah membahas cara untuk membangun model dengan output berupa variabel dengan nilai numerik. Pada Chapter \ref{logreg}, kita akan belajar cara membentuk model regresi dengan 2 respons (0 dan 1). Pada regresi ini pembentukan model didasarkan oleh kurva logistik, dimana melalui kurva tersebut nilai yang dihasilkan akan memiliki rentang dari 0 sampai 1. Karena model yang dibuat bertujuan untuk memprediksi dua buah kemungkinan (0 atau 1), maka diperlukan suatu nilai ambang (y\textless0,5 = 0 dan y \textgreater= 0,5 = 1).

Fungsi \texttt{glm()} dapat digunakan untuk membentuk model regresi logistik. Format umum fungsi tersebut adalah sebagai berikut:

\begin{Shaded}
\begin{Highlighting}[]
\FunctionTok{glm}\NormalTok{(formula, }\AttributeTok{family =}\NormalTok{ gaussian, data, weights, subset,}
\NormalTok{    )}
\end{Highlighting}
\end{Shaded}

\begin{quote}
\textbf{Catatan:}

\begin{itemize}
\tightlist
\item
  \texttt{formula} : formula model yang hendak dibentuk.
\item
  \texttt{family} : distribusi yang digunakan. Untuk regresi logistik digunakan argumen \texttt{family=binomial}
\item
  \texttt{data}: data yang digunakan untuk membentuk model.
\item
  \texttt{subset} : subset data yang akan digunakan dalam pembentukan model.
\item
  \texttt{weight} : nilai pembobotan dalam pembentukan model.
\end{itemize}
\end{quote}

Belum ADA Contoh

\hypertarget{referensi-7}{%
\section{Referensi}\label{referensi-7}}

\begin{enumerate}
\def\labelenumi{\arabic{enumi}.}
\tightlist
\item
  Akritas, M. 2016. \textbf{PROBABILITY \& STATISTICS WITH R FOR ENGINEERS AND SCIENTISTS}. Pearson.
\item
  Bloomfield, V.A. 2014. \textbf{Using R for Numerical Analysis in Science and Engineering}. CRC Press.
\item
  James, G., Witten, D., Hastie, T., Tibshirani, R. 2013. \textbf{An Introduction to Statistical Learning}. Springer.
\item
  Kerns, G.J., 2018. \textbf{Introduction to Probability and Statistics Using R}. Course notes for University of Auckland Paper STATS 330. \url{http://ipsur.r-forge.r-project.org/book/download/IPSUR.pdf}.
\item
  Lee, A., Ihaka, R., Triggs, C. 2012. \textbf{ADVANCED STATISTICAL MODELLING}.
\item
  Primartha, R. 2018. \textbf{Belajar Machine Learning Teori dan Praktik}. Penerbit Informatika : Bandung.
\item
  Rosadi,D. 2016. \textbf{Analisis Statistika dengan R}. Gadjah Mada University Press: Yogyakarta.
\item
  STHDA. \textless(\url{http://www.sthda.com/english/}\textgreater{}
\end{enumerate}

  \bibliography{book.bib,packages.bib}

\end{document}
