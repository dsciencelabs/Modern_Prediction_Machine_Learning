% Options for packages loaded elsewhere
\PassOptionsToPackage{unicode}{hyperref}
\PassOptionsToPackage{hyphens}{url}
%
\documentclass[
]{book}
\usepackage{amsmath,amssymb}
\usepackage{iftex}
\ifPDFTeX
  \usepackage[T1]{fontenc}
  \usepackage[utf8]{inputenc}
  \usepackage{textcomp} % provide euro and other symbols
\else % if luatex or xetex
  \usepackage{unicode-math} % this also loads fontspec
  \defaultfontfeatures{Scale=MatchLowercase}
  \defaultfontfeatures[\rmfamily]{Ligatures=TeX,Scale=1}
\fi
\usepackage{lmodern}
\ifPDFTeX\else
  % xetex/luatex font selection
\fi
% Use upquote if available, for straight quotes in verbatim environments
\IfFileExists{upquote.sty}{\usepackage{upquote}}{}
\IfFileExists{microtype.sty}{% use microtype if available
  \usepackage[]{microtype}
  \UseMicrotypeSet[protrusion]{basicmath} % disable protrusion for tt fonts
}{}
\makeatletter
\@ifundefined{KOMAClassName}{% if non-KOMA class
  \IfFileExists{parskip.sty}{%
    \usepackage{parskip}
  }{% else
    \setlength{\parindent}{0pt}
    \setlength{\parskip}{6pt plus 2pt minus 1pt}}
}{% if KOMA class
  \KOMAoptions{parskip=half}}
\makeatother
\usepackage{xcolor}
\usepackage{longtable,booktabs,array}
\usepackage{calc} % for calculating minipage widths
% Correct order of tables after \paragraph or \subparagraph
\usepackage{etoolbox}
\makeatletter
\patchcmd\longtable{\par}{\if@noskipsec\mbox{}\fi\par}{}{}
\makeatother
% Allow footnotes in longtable head/foot
\IfFileExists{footnotehyper.sty}{\usepackage{footnotehyper}}{\usepackage{footnote}}
\makesavenoteenv{longtable}
\usepackage{graphicx}
\makeatletter
\def\maxwidth{\ifdim\Gin@nat@width>\linewidth\linewidth\else\Gin@nat@width\fi}
\def\maxheight{\ifdim\Gin@nat@height>\textheight\textheight\else\Gin@nat@height\fi}
\makeatother
% Scale images if necessary, so that they will not overflow the page
% margins by default, and it is still possible to overwrite the defaults
% using explicit options in \includegraphics[width, height, ...]{}
\setkeys{Gin}{width=\maxwidth,height=\maxheight,keepaspectratio}
% Set default figure placement to htbp
\makeatletter
\def\fps@figure{htbp}
\makeatother
\setlength{\emergencystretch}{3em} % prevent overfull lines
\providecommand{\tightlist}{%
  \setlength{\itemsep}{0pt}\setlength{\parskip}{0pt}}
\setcounter{secnumdepth}{5}
\usepackage{booktabs}
\usepackage{lscape}

\ifLuaTeX
  \usepackage{selnolig}  % disable illegal ligatures
\fi
\usepackage[]{natbib}
\bibliographystyle{apalike}
\IfFileExists{bookmark.sty}{\usepackage{bookmark}}{\usepackage{hyperref}}
\IfFileExists{xurl.sty}{\usepackage{xurl}}{} % add URL line breaks if available
\urlstyle{same}
\hypersetup{
  pdftitle={Basis Data dan Penelusuran Data},
  pdfauthor={Bakti Siregar, M.Sc},
  hidelinks,
  pdfcreator={LaTeX via pandoc}}

\title{Basis Data dan Penelusuran Data}
\author{Bakti Siregar, M.Sc}
\date{2023-08-26}

\begin{document}
\maketitle

{
\setcounter{tocdepth}{1}
\tableofcontents
}
\hypertarget{kata-pengantar}{%
\chapter*{Kata Pengantar}\label{kata-pengantar}}
\addcontentsline{toc}{chapter}{Kata Pengantar}

Selamat datang dalam modul praktikum mengenai basis data dan penelusuran data. Dalam era digital yang semakin maju, pengelolaan informasi dan akses terhadap data sangatlah penting. Basis data merupakan fondasi utama dalam pengelolaan data yang efisien dan terstruktur, sedangkan penelusuran data memungkinkan kita untuk menggali wawasan berharga dari kumpulan informasi yang tersedia. Dalam modul ini, kita akan menjelajahi konsep-konsep dasar dalam basis data, termasuk jenis-jenis basis data, model data, bahasa kueri, dan praktik terbaik dalam merancang basis data yang optimal. Secara khusus, mudul ini

Selain itu, penelusuran basis data yang menjadi fokus penting adalah menggunakan R Programing dan SQL dalam membuat data analytics system. Penelusuran data melibatkan teknik-teknik dan alat-alat untuk menggali informasi yang berharga dari kumpulan data yang besar dan kompleks. Dengan adanya kemajuan dalam analisis data dan kecerdasan buatan, penelusuran data telah menjadi aspek penting dalam pengambilan keputusan dan inovasi. Penulis berharap bimbingan ini akan memberikan pemahaman yang kokoh tentang basis data dan penelusuran data, serta memberi Anda wawasan yang berguna dalam mengelola data dan mengambil informasi berharga dari sumber daya yang ada. Selamat belajar!

\hypertarget{ringkasan-materi}{%
\section*{Ringkasan Materi}\label{ringkasan-materi}}
\addcontentsline{toc}{section}{Ringkasan Materi}

Adapun isi pembelajaran dalam modul ini adalah sebagai berikut:

\begin{itemize}
\tightlist
\item
  Bab 1
\item
  Bab 2
\item
  Bab 3
\item
  Dst
\end{itemize}

\hypertarget{penulis}{%
\section*{Penulis}\label{penulis}}
\addcontentsline{toc}{section}{Penulis}

\begin{itemize}
\tightlist
\item
  \textbf{Bakti Siregar, M.Sc} adalah Ketua Program Studi di Jurusan Statistika Universitas Matana. Lulusan Magister Matematika Terapan dari National Sun Yat Sen University, Taiwan. Beliau juga merupakan dosen dan konsultan Data Scientist di perusahaan-perusahaan ternama seperti \href{https://www.jne.co.id/id/beranda}{JNE}, \href{https://www.samoragroup.co.id/home/en}{Samora Group}, \href{https://www.pertamina.com/}{Pertamina}, dan lainnya. Beliau memiliki antusiasme khusus dalam mengajar Big Data Analytics, Machine Learning, Optimisasi, dan Analisis Time Series di bidang keuangan dan investasi. Keahliannya juga terlihat dalam penggunaan bahasa pemrograman Statistik seperti R Studio dan Python. Beliau mengaplikasikan sistem basis data MySQL/NoSQL dalam pembelajaran manajemen data, serta mahir dalam menggunakan tools Big Data seperti Spark dan Hadoop. Beberapa project beliau dapat dilihat di link berikut: \href{https://rpubs.com/dsciencelabs}{Rpubs}, \href{https://github.com/dsciencelabs}{Github}, \href{https://dsciencelabs.github.io/web/index.html}{Website}, dan \href{https://www.kaggle.com/baktisiregar/code}{Kaggle}.
\end{itemize}

\hypertarget{asisten-lab}{%
\section*{Asisten Lab}\label{asisten-lab}}
\addcontentsline{toc}{section}{Asisten Lab}

\begin{itemize}
\tightlist
\item
  \textbf{Yonathan Anggraiwan, S.Stat} adalah seorang alumni Statistika yang bersemangat dalam dunia pemrograman dan analisis data. Lahir di Tangerang, minatnya terhadap teknologi dan komputer muncul sejak usia dini. Ia tumbuh dengan rasa ingin tahu yang kuat terhadap bahasa pemrograman, dan ini membawanya menuju dunia analisis data menggunakan bahasa pemrograman R dan Python. Selama menjalankan tugas sebagai asisten lab, Yonathan Anggraiwan berperan dalam membantu mahasiswa dalam memahami konsep-konsep dasar dan kompleks dalam pemrograman R dan Python. Ia memberikan penjelasan yang jelas dan dukungan kepada mahasiswa yang mengalami kesulitan. Selain itu, ia juga terlibat dalam merancang tugas dan ujian praktikum, serta memberikan umpan balik konstruktif kepada para mahasiswa. Dalam perjalanan waktu, Yonathan Anggraiwan mulai mengambil tanggung jawab lebih besar dalam laboratorium. Ia membantu mengembangkan materi pembelajaran tambahan, seperti tutorial online tentang analisis data menggunakan R dan Python. Ia juga aktif dalam berbagai proyek penelitian di bawah bimbingan dosen, yang melibatkan pengolahan data besar untuk analisis statistik dan visualisasi. Dengan semangat yang tinggi, dedikasi, dan keterampilan yang dimilikinya, Yonathan Anggraiwan adalah contoh nyata dari seorang mahasiswa yang berhasil menggabungkan minatnya dalam pemrograman R dan Python dengan peran yang produktif sebagai asisten laboratorium dan kontributor dalam dunia analisis data.
\end{itemize}

\hypertarget{ucapan-terima-kasih}{%
\section*{Ucapan Terima Kasih}\label{ucapan-terima-kasih}}
\addcontentsline{toc}{section}{Ucapan Terima Kasih}

Saya ingin mengucapkan terima kasih yang tulus kepada semua yang telah mendukung dan berkontribusi dalam perjalanan pembuatan modul ``Basis Data dan Penelusuran Data''. Modul ini tidak akan mungkin menjadi kenyataan tanpa kerja keras, semangat, dan dukungan yang luar biasa dari berbagai pihak. Terima kasih juga kepada rekan-rekan dan kolega yang telah memberikan masukan, saran, dan diskusi berharga sepanjang perjalanan penulisan modul ini. Kontribusi kalian telah membantu memperkaya isi modul dan menghadirkan sudut pandang yang beragam. Tentu saja,modul ini tidak akan lengkap tanpa rasa terima kasih kepada para peneliti dan praktisi di bidang basis data dan penelusuran data yang telah menciptakan landasan pengetahuan yang menjadi dasar dari modul ini. Pengalaman dan pengetahuan yang kalian bagikan sangat berharga. Saya juga ingin mengucapkan terima kasih kepada keluarga dan teman-teman saya atas dukungan, pengertian, dan dorongan yang tak henti-hentinya. Tanpa dukungan kalian, perjalanan menulis modul ini pastinya tidak akan semudah ini.

Akhir kata, semoga modul ini dapat memberikan manfaat dan wawasan baru kepada para pembaca yang ingin mendalami dunia basis data dan penelusuran data. Ucapan terima kasih terakhir saya tujukan untuk semua yang telah berkontribusi, baik secara langsung maupun tidak langsung, dalam menghadirkan modul ini kepada para pembaca.

\hypertarget{masukan-saran}{%
\section*{Masukan \& Saran}\label{masukan-saran}}
\addcontentsline{toc}{section}{Masukan \& Saran}

Semua masukan dan tanggapan Anda sangat berarti bagi kami untuk memperbaiki template ini kedepannya. Bagi para pembaca/pengguna yang ingin menyampaikan masukan dan tanggapan, dipersilahkan melalui kontak dibawak ini!

\textbf{Email:} \href{mailto:dsciencelabs@outlook.com}{\nolinkurl{dsciencelabs@outlook.com}}

\hypertarget{pendahuluan}{%
\chapter{Pendahuluan}\label{pendahuluan}}

Sejak tahun 1970, \textbf{Structured Query Language (SQL)} telah digunakan oleh para programmer untuk membangun dan mengakses \textbf{Sistem Basis Data (SBD)}. Banyak sekali perdebatan mengenai cara penyebutan SQL ini, namun pada kenyataannya, kita dapat melafalkannya sebagai ``sequel'' ataupun ``S.Q.L''. Mempelajari bahasa pemrograman umum seperti R adalah penting dan akan lebih baik jika memiliki kemampuan SQL dalam bidang pengolahan data.

\begin{figure}

{\centering \includegraphics[width=1\linewidth]{./images/Bab1/SQL} 

}

\caption{Beberapa Perusahaan Besar Pengguna SQL}\label{fig:user-SQL}
\end{figure}

Banyak perusahaan besar di bidang teknologi menggunakan SQL seperti Uber, Netflix, dan Airbnb. Bahkan dalam perusahaan seperti Facebook, Google dan Amazon, yang telah membuat sendiri \textbf{SBD} berkemampuan tinggi, tetap menggunakan SQL untuk melakukan query dan analisis data.

\hypertarget{apa-itu-sbd}{%
\section{Apa itu SBD?}\label{apa-itu-sbd}}

Secara umum \textbf{SBD} dapat didefinisikan sebagai berikut:

\begin{figure}

{\centering \includegraphics[width=1\linewidth]{./images/Bab1/definisi_DB} 

}

\caption{Definisi Sistem Basis Data}\label{fig:SBD}
\end{figure}

\hypertarget{komponen-sbd}{%
\subsection{Komponen SBD}\label{komponen-sbd}}

Adapun beberapa komponen dasar yang diperlukan dalam SBD adalah:

\begin{figure}

{\centering \includegraphics[width=1\linewidth]{./images/Bab1/komponen_DB} 

}

\caption{Komponen SBD}\label{fig:komponen}
\end{figure}

\begin{center}\rule{0.5\linewidth}{0.5pt}\end{center}

\hypertarget{manfaat-sbd}{%
\subsection{Manfaat SBD}\label{manfaat-sbd}}

Manfaat atau kegunaan penerapan SBD cukup banyak dan cakupannya pun luas dalam mendukung keberadaan lembaga atau organisasi maupun perusahaan, diantaranya:

\begin{figure}

{\centering \includegraphics[width=1\linewidth]{./images/Bab1/manfaat_DB} 

}

\caption{Manfaat SBD}\label{fig:manfaat}
\end{figure}

\begin{center}\rule{0.5\linewidth}{0.5pt}\end{center}

\hypertarget{definisi-sql-vs-nosql}{%
\subsection{Definisi SQL vs NoSQL}\label{definisi-sql-vs-nosql}}

Sebenarnya perbedaan antara \href{https://www.dewaweb.com/blog/sql-pengertian-fungsi-beserta-perintah-dasarnya/}{SQL} dan \href{https://aws.amazon.com/id/nosql/}{NoSQL} secara mendasar sudah dapat dijelaskan dari akronimnya.

\begin{figure}

{\centering \includegraphics[width=1\linewidth]{./images/Bab1/SQLvsNoSQL_DB} 

}

\caption{SQL vs NoSQL}\label{fig:SQLvsNoSQL}
\end{figure}

\emph{SQL} basis data relasional yang menggunakan `relasi' (yang biasanya disebut tabel) untuk menyimpan data dan mencocokkan data tersebut dengan memakai karakteristik umum di setiap dataset. Sedangkan, \emph{NoSQL} adalah database yang menggunakan format JSON untuk setiap dokumennya sehingga mudah dibaca dan dimengerti. NoSQL banyak diminati karena memiliki performa yang tinggi dan bersifat non-relasional sehingga dapat memakai berbagai model data.

\begin{center}\rule{0.5\linewidth}{0.5pt}\end{center}

\hypertarget{perbedaan-sql-vs-nosql}{%
\subsection{Perbedaan SQL vs NoSQL}\label{perbedaan-sql-vs-nosql}}

Sebenarnya banyak perbedaan yang dimiliki di antara dua database tersebut tapi inilah perbedaan yang paling mencolok antara SQL dan NoSQL:

\begin{figure}

{\centering \includegraphics[width=1\linewidth]{./images/Bab1/Perbedaan_DB} 

}

\caption{Perbedaan SQL vs NoSQL}\label{fig:Perbedaan}
\end{figure}

\begin{center}\rule{0.5\linewidth}{0.5pt}\end{center}

\hypertarget{top-7-sql}{%
\subsection{Top 7 SQL}\label{top-7-sql}}

Tercatat sampai bulan Februari 2020 ada 334 jenis database menurut db-engines.com. Berikut ini saya merangkum daftar 7 database terpopuler yang menggunakan \href{https://qwords.com/blog/database-terpopuler/}{SQL} (Relasional):

\begin{figure}

{\centering \includegraphics[width=1\linewidth]{./images/Bab1/7SQL_DB} 

}

\caption{Top 7 Perangkat Lunak SQL}\label{fig:top7SQL}
\end{figure}

\begin{center}\rule{0.5\linewidth}{0.5pt}\end{center}

\hypertarget{top-8-nosql}{%
\subsection{Top 8 NoSQL}\label{top-8-nosql}}

Kebanyakan basis data NoSQL digunakan dalam dunia aplikasi web waktu nyata (real-time web app). Berikut ini adalah ulasan 8 jenis basis data \href{https://www.codepolitan.com/7-basis-data-nosql-populer}{NoSQL} yang paling populer digunakan diseluruh dunia:

\begin{figure}

{\centering \includegraphics[width=1\linewidth]{./images/Bab1/8NoSQL_DB} 

}

\caption{Top 8 Perangkat Lunak NoSQL}\label{fig:top8noSQL}
\end{figure}

\begin{center}\rule{0.5\linewidth}{0.5pt}\end{center}

\hypertarget{mengapa-r-sql}{%
\section{Mengapa R \& SQL?}\label{mengapa-r-sql}}

Menggunakan R dan SQL merupakan kombinasi yang kuat untuk analisis data dan pengelolaan basis data. Keduanya memiliki peran yang berbeda dalam proses analisis dan pengelolaan data. Berikut adalah beberapa alasan mengapa menggunakan R dan SQL bersama:

\begin{itemize}
\tightlist
\item
  \textbf{Kekuatan Analisis R}
\end{itemize}

R adalah bahasa pemrograman yang khusus dirancang untuk analisis statistik dan visualisasi data.
R memiliki berbagai paket (packages) yang menawarkan fungsi statistik dan analisis yang kuat, termasuk regresi, pengelompokan, analisis deret waktu, dan banyak lagi.
Visualisasi yang dapat dihasilkan dengan R sangat bervariasi, dari grafik sederhana hingga visualisasi interaktif yang kompleks.

\begin{itemize}
\tightlist
\item
  \textbf{Manipulasi dan Pengelolaan Data dengan SQL}
\end{itemize}

SQL digunakan untuk mengelola dan mengambil data dari basis data terstruktur.
SQL menyediakan cara efisien untuk membuat, mengubah, menghapus, dan memanipulasi data dalam basis data.
SQL memiliki fitur untuk menggabungkan data dari berbagai tabel, melakukan agregasi, dan menyaring data.

\begin{itemize}
\tightlist
\item
  \textbf{Integrasi Antara R dan SQL}
\end{itemize}

Banyak perpustakaan R yang mendukung koneksi ke basis data menggunakan SQL.
Anda dapat menggunakan perintah SQL dalam skrip R untuk mengambil data dari basis data, memanipulasi data di dalam R, dan kemudian menerapkan analisis statistik menggunakan paket R. Integrasi ini memungkinkan Anda menggabungkan kekuatan analisis statistik R dengan kemampuan pengelolaan data SQL.

\begin{itemize}
\tightlist
\item
  \textbf{Skalabilitas dan Efisiensi}
\end{itemize}

Menggunakan SQL untuk mengambil dan memanipulasi data dalam basis data bisa lebih efisien daripada melakukannya dalam R, terutama untuk dataset besar.
SQL memungkinkan query yang dioptimalkan dan penggunaan indeks untuk kinerja yang lebih baik.

\begin{itemize}
\tightlist
\item
  \textbf{Data Preprocessing}
\end{itemize}

Sebelum menerapkan analisis di R, Anda mungkin perlu melakukan pra-pemrosesan pada data, seperti membersihkan data, menggabungkan tabel, dan mengisi data yang hilang. SQL dapat membantu dalam melakukan tugas-tugas ini.

Jadi, menggunakan R dan SQL bersama memungkinkan Anda menggabungkan kekuatan analisis statistik R dengan kemampuan pengelolaan data SQL. Ini bisa sangat berguna ketika Anda ingin melakukan analisis data yang luas dan kompleks dari berbagai sumber data yang berbeda.

\begin{figure}

{\centering \includegraphics[width=1\linewidth]{./images/Bab1/sql-r-friends} 

}

\caption{R dan SQL [https://irene.rbind.io](https://irene.rbind.io/post/using-sql-in-rstudio/)}\label{fig:sql-r-friends}
\end{figure}

\hypertarget{mysql-vs-postgresql}{%
\section{MySQL vs PostgreSQL}\label{mysql-vs-postgresql}}

MySQL adalah sistem manajemen basis data relasional yang memungkinkan Anda untuk menyimpan data sebagai tabel dengan baris dan kolom. Sistem ini populer sehingga digunakan di banyak aplikasi web, situs web dinamis, dan sistem tertanam. PostgreSQL adalah sistem manajemen basis data relasional-objek yang menawarkan lebih banyak fitur daripada MySQL. Sistem ini memberi Anda lebih banyak fleksibilitas dalam tipe data, skalabilitas, konkurensi, dan integrasi data.

\begin{figure}

{\centering \includegraphics[width=1\linewidth]{./images/Bab1/MySQL-VS-PostgreSQL} 

}

\caption{MySQL vs PostgreSQL [https://integrio.net/](https://integrio.net/blog/postgresql-vs-mysql)}\label{fig:MySQL-VS-PostgreSQL}
\end{figure}

MySQL dan PostgreSQL, Keduanya menyimpan data di dalam tabel yang terkait satu sama lain melalui nilai kolom umum. Namun keduanya sering dibandingkan karena terdapat beberapa perbedaan. Ingin mengenal lebih dalam? Simak penjelasan di bawah.

\hypertarget{kelebihan}{%
\subsection{Kelebihan}\label{kelebihan}}

\begin{longtable}[]{@{}
  >{\raggedright\arraybackslash}p{(\columnwidth - 2\tabcolsep) * \real{0.4545}}
  >{\raggedright\arraybackslash}p{(\columnwidth - 2\tabcolsep) * \real{0.5455}}@{}}
\toprule\noalign{}
\begin{minipage}[b]{\linewidth}\raggedright
MySQL
\end{minipage} & \begin{minipage}[b]{\linewidth}\raggedright
PostgreSQL
\end{minipage} \\
\midrule\noalign{}
\endhead
\bottomrule\noalign{}
\endlastfoot
Integrasi bahasa pemrograman sangat luas; & Support framework website modern seperti Node.js dan Django; Support framework website modern seperti Node.js dan Django; \\
Aplikasi ringan, tidak membutuhkan spesifikasi hardware yang tinggi; & Dirilis dengan lisensi PostgreSQL sendiri; \\
Struktur tabel dengan fleksibilitas tinggi; & Bersifat open source dan gratis; \\
Dibekali banyak administrative tools; & Skala besar, mampu memuat hingga ribuan transaksi data; \\
Bersifat open source dan gratis (versi basic); & Memiliki banyak fitur yang mumpuni; \\
Meski open source, MySQL menjamin keamanan tingkat tinggi; & Memiliki banyak fitur yang mumpuni; \\
Mendukung berbagai variasi Data Type; & Performa sangat baik meski menuntut query yang lebih kompleks; \\
Dapat digunakan banyak pengguna karena mendukung multi user. & Kecepatan analisis (read-write) sangat cepat; Keamanan yang lebih ketat. \\
\end{longtable}

\hypertarget{kekurangan}{%
\subsection{Kekurangan}\label{kekurangan}}

\begin{longtable}[]{@{}
  >{\raggedright\arraybackslash}p{(\columnwidth - 2\tabcolsep) * \real{0.4545}}
  >{\raggedright\arraybackslash}p{(\columnwidth - 2\tabcolsep) * \real{0.5455}}@{}}
\toprule\noalign{}
\begin{minipage}[b]{\linewidth}\raggedright
MySQL
\end{minipage} & \begin{minipage}[b]{\linewidth}\raggedright
PostgreSQL
\end{minipage} \\
\midrule\noalign{}
\endhead
\bottomrule\noalign{}
\endlastfoot
Sistem manajemen database kurang cocok untuk aplikasi mobile dan game; & PostgreSQL tidak mendukung semua stack development; \\
Technical support MySQL dinilai kurang baik; & Meski memiliki integrasi dan skalabilitas tinggi, kecepatan PostgreSQL kalah unggul dibandingkan RDBMS lain; \\
Sulit diaplikasikan untuk manajemen database berskala besar. & Sistem kompatibilitas PostgreSQL menuntut pengguna untuk bekerja lebih keras dalam perbaikan dan perawatan. \\
\end{longtable}

\hypertarget{instalasi-mysql-xampp}{%
\section{Instalasi MySQL (XAMPP)}\label{instalasi-mysql-xampp}}

\hypertarget{download-aplikasi-xampp}{%
\subsection{Download Aplikasi XAMPP}\label{download-aplikasi-xampp}}

Silakan \href{https://www.apachefriends.org/download.html}{klik disini} untuk mengunduh applikasi XAMPP, pilih salah satu saja sesuai Operating System pada Komputer anda.

\begin{figure}

{\centering \includegraphics[width=1\linewidth]{./images/Bab1/xampp0} 

}

\caption{Langkah 1, Download XAMPP)}\label{fig:download-xammp}
\end{figure}

\hypertarget{install-aplikasi}{%
\subsection{Install Aplikasi}\label{install-aplikasi}}

Temukan file XAMPP.exe yang telah anda download, secara default biasanya disimpan di;

\begin{figure}

{\centering \includegraphics[width=1\linewidth]{./images/Bab1/xampp1} 

}

\caption{Langkah 2, Instalasi XAMPP)}\label{fig:install-xammp}
\end{figure}

Selanjutnya, akan muncul Warning di klik \textbf{OK}

\begin{figure}

{\centering \includegraphics[width=1\linewidth]{./images/Bab1/xampp2} 

}

\caption{Langkah 3, Instalasi XAMPP)}\label{fig:install-xammp2}
\end{figure}

selajutunya klik next

\begin{figure}

{\centering \includegraphics[width=1\linewidth]{./images/Bab1/xampp3} 

}

\caption{Langkah 4: Instalasi XAMPP)}\label{fig:install-xammp3}
\end{figure}

Klik next lagi, karena sudah dipilih secara default oleh XAMPP

\begin{figure}

{\centering \includegraphics[width=1\linewidth]{./images/Bab1/xampp4} 

}

\caption{Langkah 5, Instalasi XAMPP)}\label{fig:install-xammp4}
\end{figure}

\hypertarget{pilih-folder}{%
\subsection{Pilih Folder}\label{pilih-folder}}

\begin{figure}

{\centering \includegraphics[width=1\linewidth]{./images/Bab1/xampp5} 

}

\caption{Langkah 6, Instalasi XAMPP)}\label{fig:install-xammp5}
\end{figure}

Secara default akan membuat folder baru \textbf{C:\textbackslash xampp}, lalu pilih next.

\textbf{note:} jika anda sudah pernah mendownload aplikasi xampp, perlu di hapus terlebih dahulu file xampp yang lama di file \textbf{C:\textbackslash xampp}

\hypertarget{jalankan-proses-instalasi}{%
\subsection{Jalankan proses Instalasi}\label{jalankan-proses-instalasi}}

\begin{figure}

{\centering \includegraphics[width=1\linewidth]{./images/Bab1/xampp6} 

}

\caption{Langkah 7, Instalasi XAMPP)}\label{fig:install-xammp6}
\end{figure}

Tunggu proses instalasi selesai \textbf{Biasanya 5-10 menit, tergantung kecepatan komputer anda}.

\hypertarget{xampp-sudah-terinstall}{%
\subsection{XAMPP sudah terinstall}\label{xampp-sudah-terinstall}}

\begin{figure}

{\centering \includegraphics[width=1\linewidth]{./images/Bab1/xampp7} 

}

\caption{Langkah 8, Instalasi XAMPP)}\label{fig:install-xammp7}
\end{figure}

Setelah aplikasi terinstall, sudah bisa digunakan.

\hypertarget{video-instalasi-xampp}{%
\subsection{Video Instalasi XAMPP}\label{video-instalasi-xampp}}

\hypertarget{instalasi-postgresql}{%
\section{Instalasi PostgreSQL}\label{instalasi-postgresql}}

Berikut ini adalah proses langkah demi langkah tentang Cara Menginstal PostgreSQL di Windows:

\hypertarget{buka-browser}{%
\subsection{Buka Browser}\label{buka-browser}}

Klik \url{https://www.postgresql.org/download} and pilih Windows

\begin{figure}

{\centering \includegraphics[width=1\linewidth]{./images/Bab1/Postgree0} 

}

\caption{Langkah 1, Instalasi PostgreeSQL)}\label{fig:install-posrgree1}
\end{figure}

\hypertarget{cek-option}{%
\subsection{Cek Option}\label{cek-option}}

Klik Download the installer Interactive Installer by EnterpriseDB

\begin{figure}

{\centering \includegraphics[width=1\linewidth]{./images/Bab1/Postgree1} 

}

\caption{Langkah 2, Instalasi PostgreeSQL)}\label{fig:install-posrgree2}
\end{figure}

\hypertarget{pilih-postgresql-version}{%
\subsection{Pilih PostgreSQL version}\label{pilih-postgresql-version}}

Anda akan diminta untuk memilih versi PostgreSQL dan sistem operasi yang diinginkan. Pilih versi PostgreSQL terbaru dan OS sesuai dengan environment Anda, \textbf{klik tombol download.}

\begin{figure}

{\centering \includegraphics[width=1\linewidth]{./images/Bab1/Postgree2} 

}

\caption{Langkah 3, Instalasi PostgreeSQL)}\label{fig:install-posrgree3}
\end{figure}

\hypertarget{open-exe-file}{%
\subsection{Open exe file}\label{open-exe-file}}

Setelah Anda mengunduh PostgreSQL, buka exe yang telah diunduh dan Klik berikutnya pada layar install welcome screen.

\begin{figure}

{\centering \includegraphics[width=1\linewidth]{./images/Bab1/Postgree3} 

}

\caption{Langkah 4, Instalasi PostgreeSQL)}\label{fig:install-posrgree4}
\end{figure}

\hypertarget{pilih-folder-1}{%
\subsection{Pilih folder}\label{pilih-folder-1}}

Ubah direktori Instalasi jika diperlukan, jika tidak, biarkan default, \textbf{klik Next.}

\begin{figure}

{\centering \includegraphics[width=1\linewidth]{./images/Bab1/Postgree4} 

}

\caption{Langkah 5, Instalasi PostgreeSQL)}\label{fig:install-posrgree5}
\end{figure}

\hypertarget{select-components}{%
\subsection{Select components}\label{select-components}}

Anda dapat memilih komponen yang ingin Anda instal di sistem Anda. Anda dapat menghapus centang pada Stack Builder (\emph{disarankan ikuti secara default}), \textbf{klik Next.}

\begin{figure}

{\centering \includegraphics[width=1\linewidth]{./images/Bab1/Postgree5} 

}

\caption{Langkah 6, Instalasi PostgreeSQL)}\label{fig:install-posrgree6}
\end{figure}

\hypertarget{check-data-location}{%
\subsection{Check data location}\label{check-data-location}}

Anda dapat mengubah lokasi data, \textbf{Klik Next.}

\begin{figure}

{\centering \includegraphics[width=1\linewidth]{./images/Bab1/Postgree6} 

}

\caption{Langkah 7, Instalasi PostgreeSQL)}\label{fig:install-posrgree7}
\end{figure}

\hypertarget{masukan-password}{%
\subsection{Masukan Password}\label{masukan-password}}

Masukkan kata sandi superuser. Catat kata sandi tersebut, \textbf{Klik Next.}

\begin{figure}

{\centering \includegraphics[width=1\linewidth]{./images/Bab1/Postgree7} 

}

\caption{Langkah 8, Instalasi PostgreeSQL)}\label{fig:install-posrgree8}
\end{figure}

\hypertarget{cek-opsi-port}{%
\subsection{Cek opsi port}\label{cek-opsi-port}}

Biarkan nomor port menjadi default, \textbf{Klik Next.}

\begin{figure}

{\centering \includegraphics[width=1\linewidth]{./images/Bab1/Postgree8} 

}

\caption{Langkah 9, Instalasi PostgreeSQL)}\label{fig:install-posrgree9}
\end{figure}

\hypertarget{cek-summary}{%
\subsection{Cek Summary}\label{cek-summary}}

Periksa pra-penginstalan summary, \textbf{Klik Next}

\begin{figure}

{\centering \includegraphics[width=1\linewidth]{./images/Bab1/Postgree9} 

}

\caption{Langkah 10, Instalasi PostgreeSQL)}\label{fig:install-posrgree10}
\end{figure}

\hypertarget{ready-to-install}{%
\subsection{Ready to Install}\label{ready-to-install}}

Klik tombol Next

\begin{figure}

{\centering \includegraphics[width=1\linewidth]{./images/Bab1/Postgree10} 

}

\caption{Langkah 11, Instalasi PostgreeSQL)}\label{fig:install-posrgree11}
\end{figure}

\hypertarget{check-stack-builder-prompt}{%
\subsection{Check stack builder prompt}\label{check-stack-builder-prompt}}

Setelah instalasi selesai, Anda akan melihat prompt Stack Builder. Hapus centang pada opsi tersebut. Kita akan menggunakan Stack Builder dalam tutorial selanjutnya, \textbf{Klik Finish.}

\begin{figure}

{\centering \includegraphics[width=1\linewidth]{./images/Bab1/Postgree11} 

}

\caption{Langkah 12, Instalasi PostgreeSQL)}\label{fig:install-posrgree12}
\end{figure}

\hypertarget{launch-postgresql}{%
\subsection{Launch PostgreSQL}\label{launch-postgresql}}

Untuk launch PostgreSQL, buka Start Menu dan cari pgAdmin 4

\begin{figure}

{\centering \includegraphics[width=1\linewidth]{./images/Bab1/Postgree12} 

}

\caption{Langkah 13, Instalasi PostgreeSQL)}\label{fig:install-posrgree13}
\end{figure}

\hypertarget{check-pgadmin}{%
\subsection{Check pgAdmin}\label{check-pgadmin}}

Anda akan melihat beranda pgAdmin

\begin{figure}

{\centering \includegraphics[width=1\linewidth]{./images/Bab1/Postgree13} 

}

\caption{Langkah 14, Instalasi PostgreeSQL)}\label{fig:install-posrgree14}
\end{figure}

\hypertarget{cari-postgresql-15}{%
\subsection{Cari PostgreSQL 15}\label{cari-postgresql-15}}

Klik pada Servers \textgreater{} PostgreSQL 15 di sub sebelah kiri

\begin{figure}

{\centering \includegraphics[width=1\linewidth]{./images/Bab1/Postgree14} 

}

\caption{Langkah 15, Instalasi PostgreeSQL)}\label{fig:install-posrgree15}
\end{figure}

\hypertarget{enter-password}{%
\subsection{Enter password}\label{enter-password}}

Masukkan kata sandi superuser yang ditetapkan selama instalasi, \textbf{Klik OK}

\begin{figure}

{\centering \includegraphics[width=1\linewidth]{./images/Bab1/Postgree15} 

}

\caption{Langkah 16, Instalasi PostgreeSQL)}\label{fig:install-posrgree16}
\end{figure}

\hypertarget{cek-dashboard}{%
\subsection{Cek Dashboard}\label{cek-dashboard}}

Anda akan melihat Dashboard

\begin{figure}

{\centering \includegraphics[width=1\linewidth]{./images/Bab1/Postgree16} 

}

\caption{Langkah 17, Instalasi PostgreeSQL)}\label{fig:install-posrgree17}
\end{figure}

\hypertarget{video-instalasi-postgresql}{%
\subsection{Video Instalasi PostgreSQL}\label{video-instalasi-postgresql}}

\hypertarget{praktikal-hands-on}{%
\section{Praktikal (Hands On)}\label{praktikal-hands-on}}

\hypertarget{referensi}{%
\chapter{Referensi}\label{referensi}}

  \bibliography{book.bib,packages.bib}

\end{document}
