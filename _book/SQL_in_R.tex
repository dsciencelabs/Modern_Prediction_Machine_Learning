% Options for packages loaded elsewhere
\PassOptionsToPackage{unicode}{hyperref}
\PassOptionsToPackage{hyphens}{url}
%
\documentclass[
]{book}
\usepackage{amsmath,amssymb}
\usepackage{iftex}
\ifPDFTeX
  \usepackage[T1]{fontenc}
  \usepackage[utf8]{inputenc}
  \usepackage{textcomp} % provide euro and other symbols
\else % if luatex or xetex
  \usepackage{unicode-math} % this also loads fontspec
  \defaultfontfeatures{Scale=MatchLowercase}
  \defaultfontfeatures[\rmfamily]{Ligatures=TeX,Scale=1}
\fi
\usepackage{lmodern}
\ifPDFTeX\else
  % xetex/luatex font selection
\fi
% Use upquote if available, for straight quotes in verbatim environments
\IfFileExists{upquote.sty}{\usepackage{upquote}}{}
\IfFileExists{microtype.sty}{% use microtype if available
  \usepackage[]{microtype}
  \UseMicrotypeSet[protrusion]{basicmath} % disable protrusion for tt fonts
}{}
\makeatletter
\@ifundefined{KOMAClassName}{% if non-KOMA class
  \IfFileExists{parskip.sty}{%
    \usepackage{parskip}
  }{% else
    \setlength{\parindent}{0pt}
    \setlength{\parskip}{6pt plus 2pt minus 1pt}}
}{% if KOMA class
  \KOMAoptions{parskip=half}}
\makeatother
\usepackage{xcolor}
\usepackage{color}
\usepackage{fancyvrb}
\newcommand{\VerbBar}{|}
\newcommand{\VERB}{\Verb[commandchars=\\\{\}]}
\DefineVerbatimEnvironment{Highlighting}{Verbatim}{commandchars=\\\{\}}
% Add ',fontsize=\small' for more characters per line
\usepackage{framed}
\definecolor{shadecolor}{RGB}{248,248,248}
\newenvironment{Shaded}{\begin{snugshade}}{\end{snugshade}}
\newcommand{\AlertTok}[1]{\textcolor[rgb]{0.94,0.16,0.16}{#1}}
\newcommand{\AnnotationTok}[1]{\textcolor[rgb]{0.56,0.35,0.01}{\textbf{\textit{#1}}}}
\newcommand{\AttributeTok}[1]{\textcolor[rgb]{0.13,0.29,0.53}{#1}}
\newcommand{\BaseNTok}[1]{\textcolor[rgb]{0.00,0.00,0.81}{#1}}
\newcommand{\BuiltInTok}[1]{#1}
\newcommand{\CharTok}[1]{\textcolor[rgb]{0.31,0.60,0.02}{#1}}
\newcommand{\CommentTok}[1]{\textcolor[rgb]{0.56,0.35,0.01}{\textit{#1}}}
\newcommand{\CommentVarTok}[1]{\textcolor[rgb]{0.56,0.35,0.01}{\textbf{\textit{#1}}}}
\newcommand{\ConstantTok}[1]{\textcolor[rgb]{0.56,0.35,0.01}{#1}}
\newcommand{\ControlFlowTok}[1]{\textcolor[rgb]{0.13,0.29,0.53}{\textbf{#1}}}
\newcommand{\DataTypeTok}[1]{\textcolor[rgb]{0.13,0.29,0.53}{#1}}
\newcommand{\DecValTok}[1]{\textcolor[rgb]{0.00,0.00,0.81}{#1}}
\newcommand{\DocumentationTok}[1]{\textcolor[rgb]{0.56,0.35,0.01}{\textbf{\textit{#1}}}}
\newcommand{\ErrorTok}[1]{\textcolor[rgb]{0.64,0.00,0.00}{\textbf{#1}}}
\newcommand{\ExtensionTok}[1]{#1}
\newcommand{\FloatTok}[1]{\textcolor[rgb]{0.00,0.00,0.81}{#1}}
\newcommand{\FunctionTok}[1]{\textcolor[rgb]{0.13,0.29,0.53}{\textbf{#1}}}
\newcommand{\ImportTok}[1]{#1}
\newcommand{\InformationTok}[1]{\textcolor[rgb]{0.56,0.35,0.01}{\textbf{\textit{#1}}}}
\newcommand{\KeywordTok}[1]{\textcolor[rgb]{0.13,0.29,0.53}{\textbf{#1}}}
\newcommand{\NormalTok}[1]{#1}
\newcommand{\OperatorTok}[1]{\textcolor[rgb]{0.81,0.36,0.00}{\textbf{#1}}}
\newcommand{\OtherTok}[1]{\textcolor[rgb]{0.56,0.35,0.01}{#1}}
\newcommand{\PreprocessorTok}[1]{\textcolor[rgb]{0.56,0.35,0.01}{\textit{#1}}}
\newcommand{\RegionMarkerTok}[1]{#1}
\newcommand{\SpecialCharTok}[1]{\textcolor[rgb]{0.81,0.36,0.00}{\textbf{#1}}}
\newcommand{\SpecialStringTok}[1]{\textcolor[rgb]{0.31,0.60,0.02}{#1}}
\newcommand{\StringTok}[1]{\textcolor[rgb]{0.31,0.60,0.02}{#1}}
\newcommand{\VariableTok}[1]{\textcolor[rgb]{0.00,0.00,0.00}{#1}}
\newcommand{\VerbatimStringTok}[1]{\textcolor[rgb]{0.31,0.60,0.02}{#1}}
\newcommand{\WarningTok}[1]{\textcolor[rgb]{0.56,0.35,0.01}{\textbf{\textit{#1}}}}
\usepackage{longtable,booktabs,array}
\usepackage{calc} % for calculating minipage widths
% Correct order of tables after \paragraph or \subparagraph
\usepackage{etoolbox}
\makeatletter
\patchcmd\longtable{\par}{\if@noskipsec\mbox{}\fi\par}{}{}
\makeatother
% Allow footnotes in longtable head/foot
\IfFileExists{footnotehyper.sty}{\usepackage{footnotehyper}}{\usepackage{footnote}}
\makesavenoteenv{longtable}
\usepackage{graphicx}
\makeatletter
\def\maxwidth{\ifdim\Gin@nat@width>\linewidth\linewidth\else\Gin@nat@width\fi}
\def\maxheight{\ifdim\Gin@nat@height>\textheight\textheight\else\Gin@nat@height\fi}
\makeatother
% Scale images if necessary, so that they will not overflow the page
% margins by default, and it is still possible to overwrite the defaults
% using explicit options in \includegraphics[width, height, ...]{}
\setkeys{Gin}{width=\maxwidth,height=\maxheight,keepaspectratio}
% Set default figure placement to htbp
\makeatletter
\def\fps@figure{htbp}
\makeatother
\setlength{\emergencystretch}{3em} % prevent overfull lines
\providecommand{\tightlist}{%
  \setlength{\itemsep}{0pt}\setlength{\parskip}{0pt}}
\setcounter{secnumdepth}{5}
\usepackage{booktabs}
\usepackage{lscape}

\ifLuaTeX
  \usepackage{selnolig}  % disable illegal ligatures
\fi
\usepackage[]{natbib}
\bibliographystyle{apalike}
\IfFileExists{bookmark.sty}{\usepackage{bookmark}}{\usepackage{hyperref}}
\IfFileExists{xurl.sty}{\usepackage{xurl}}{} % add URL line breaks if available
\urlstyle{same}
\hypersetup{
  pdftitle={Basis Data dan Penelusuran Data},
  pdfauthor={Bakti Siregar, M.Sc},
  hidelinks,
  pdfcreator={LaTeX via pandoc}}

\title{Basis Data dan Penelusuran Data}
\author{Bakti Siregar, M.Sc}
\date{2023-09-04}

\begin{document}
\maketitle

{
\setcounter{tocdepth}{1}
\tableofcontents
}
\hypertarget{kata-pengantar}{%
\chapter*{Kata Pengantar}\label{kata-pengantar}}
\addcontentsline{toc}{chapter}{Kata Pengantar}

Selamat datang dalam modul praktikum mengenai basis data dan penelusuran data. Dalam era digital yang semakin maju, pengelolaan informasi dan akses terhadap data sangatlah penting. Basis data merupakan fondasi utama dalam pengelolaan data yang efisien dan terstruktur, sedangkan penelusuran data memungkinkan kita untuk menggali wawasan berharga dari kumpulan informasi yang tersedia. Dalam modul ini, kita akan menjelajahi konsep-konsep dasar dalam basis data, termasuk jenis-jenis basis data, model data, bahasa kueri, dan praktik terbaik dalam merancang basis data yang optimal. Secara khusus, mudul ini

Selain itu, penelusuran basis data yang menjadi fokus penting adalah menggunakan R Programing dan SQL dalam membuat data analytics system. Penelusuran data melibatkan teknik-teknik dan alat-alat untuk menggali informasi yang berharga dari kumpulan data yang besar dan kompleks. Dengan adanya kemajuan dalam analisis data dan kecerdasan buatan, penelusuran data telah menjadi aspek penting dalam pengambilan keputusan dan inovasi. Penulis berharap bimbingan ini akan memberikan pemahaman yang kokoh tentang basis data dan penelusuran data, serta memberi Anda wawasan yang berguna dalam mengelola data dan mengambil informasi berharga dari sumber daya yang ada. Selamat belajar!

\hypertarget{ringkasan-materi}{%
\section*{Ringkasan Materi}\label{ringkasan-materi}}
\addcontentsline{toc}{section}{Ringkasan Materi}

Adapun isi pembelajaran dalam modul ini adalah sebagai berikut:

\begin{itemize}
\tightlist
\item
  Bab 1
\item
  Bab 2
\item
  Bab 3
\item
  Dst
\end{itemize}

\hypertarget{penulis}{%
\section*{Penulis}\label{penulis}}
\addcontentsline{toc}{section}{Penulis}

\begin{itemize}
\tightlist
\item
  \textbf{Bakti Siregar, M.Sc} adalah Ketua Program Studi di Jurusan Statistika Universitas Matana. Lulusan Magister Matematika Terapan dari National Sun Yat Sen University, Taiwan. Beliau juga merupakan dosen dan konsultan Data Scientist di perusahaan-perusahaan ternama seperti \href{https://www.jne.co.id/id/beranda}{JNE}, \href{https://www.samoragroup.co.id/home/en}{Samora Group}, \href{https://www.pertamina.com/}{Pertamina}, dan lainnya. Beliau memiliki antusiasme khusus dalam mengajar Big Data Analytics, Machine Learning, Optimisasi, dan Analisis Time Series di bidang keuangan dan investasi. Keahliannya juga terlihat dalam penggunaan bahasa pemrograman Statistik seperti R Studio dan Python. Beliau mengaplikasikan sistem basis data MySQL/NoSQL dalam pembelajaran manajemen data, serta mahir dalam menggunakan tools Big Data seperti Spark dan Hadoop. Beberapa project beliau dapat dilihat di link berikut: \href{https://rpubs.com/dsciencelabs}{Rpubs}, \href{https://github.com/dsciencelabs}{Github}, \href{https://dsciencelabs.github.io/web/index.html}{Website}, dan \href{https://www.kaggle.com/baktisiregar/code}{Kaggle}.
\end{itemize}

\hypertarget{asisten-lab}{%
\section*{Asisten Lab}\label{asisten-lab}}
\addcontentsline{toc}{section}{Asisten Lab}

\begin{itemize}
\tightlist
\item
  \textbf{Yonathan Anggraiwan, S.Stat} adalah seorang alumni Statistika yang bersemangat dalam dunia pemrograman dan analisis data. Lahir di Tangerang, minatnya terhadap teknologi dan komputer muncul sejak usia dini. Ia tumbuh dengan rasa ingin tahu yang kuat terhadap bahasa pemrograman, dan ini membawanya menuju dunia analisis data menggunakan bahasa pemrograman R dan Python. Selama menjalankan tugas sebagai asisten lab, Yonathan Anggraiwan berperan dalam membantu mahasiswa dalam memahami konsep-konsep dasar dan kompleks dalam pemrograman R dan Python. Ia memberikan penjelasan yang jelas dan dukungan kepada mahasiswa yang mengalami kesulitan. Selain itu, ia juga terlibat dalam merancang tugas dan ujian praktikum, serta memberikan umpan balik konstruktif kepada para mahasiswa. Dalam perjalanan waktu, Yonathan Anggraiwan mulai mengambil tanggung jawab lebih besar dalam laboratorium. Ia membantu mengembangkan materi pembelajaran tambahan, seperti tutorial online tentang analisis data menggunakan R dan Python. Ia juga aktif dalam berbagai proyek penelitian di bawah bimbingan dosen, yang melibatkan pengolahan data besar untuk analisis statistik dan visualisasi. Dengan semangat yang tinggi, dedikasi, dan keterampilan yang dimilikinya, Yonathan Anggraiwan adalah contoh nyata dari seorang mahasiswa yang berhasil menggabungkan minatnya dalam pemrograman R dan Python dengan peran yang produktif sebagai asisten laboratorium dan kontributor dalam dunia analisis data.
\end{itemize}

\hypertarget{ucapan-terima-kasih}{%
\section*{Ucapan Terima Kasih}\label{ucapan-terima-kasih}}
\addcontentsline{toc}{section}{Ucapan Terima Kasih}

Saya ingin mengucapkan terima kasih yang tulus kepada semua yang telah mendukung dan berkontribusi dalam perjalanan pembuatan modul ``Basis Data dan Penelusuran Data''. Modul ini tidak akan mungkin menjadi kenyataan tanpa kerja keras, semangat, dan dukungan yang luar biasa dari berbagai pihak. Terima kasih juga kepada rekan-rekan dan kolega yang telah memberikan masukan, saran, dan diskusi berharga sepanjang perjalanan penulisan modul ini. Kontribusi kalian telah membantu memperkaya isi modul dan menghadirkan sudut pandang yang beragam. Tentu saja,modul ini tidak akan lengkap tanpa rasa terima kasih kepada para peneliti dan praktisi di bidang basis data dan penelusuran data yang telah menciptakan landasan pengetahuan yang menjadi dasar dari modul ini. Pengalaman dan pengetahuan yang kalian bagikan sangat berharga. Saya juga ingin mengucapkan terima kasih kepada keluarga dan teman-teman saya atas dukungan, pengertian, dan dorongan yang tak henti-hentinya. Tanpa dukungan kalian, perjalanan menulis modul ini pastinya tidak akan semudah ini.

Akhir kata, semoga modul ini dapat memberikan manfaat dan wawasan baru kepada para pembaca yang ingin mendalami dunia basis data dan penelusuran data. Ucapan terima kasih terakhir saya tujukan untuk semua yang telah berkontribusi, baik secara langsung maupun tidak langsung, dalam menghadirkan modul ini kepada para pembaca.

\hypertarget{masukan-saran}{%
\section*{Masukan \& Saran}\label{masukan-saran}}
\addcontentsline{toc}{section}{Masukan \& Saran}

Semua masukan dan tanggapan Anda sangat berarti bagi kami untuk memperbaiki template ini kedepannya. Bagi para pembaca/pengguna yang ingin menyampaikan masukan dan tanggapan, dipersilahkan melalui kontak dibawak ini!

\textbf{Email:} \href{mailto:dsciencelabs@outlook.com}{\nolinkurl{dsciencelabs@outlook.com}}

\hypertarget{pendahuluan}{%
\chapter{Pendahuluan}\label{pendahuluan}}

Sejak tahun 1970, \textbf{Structured Query Language (SQL)} telah digunakan oleh para programmer untuk membangun dan mengakses \textbf{Sistem Basis Data (SBD)}. Banyak sekali perdebatan mengenai cara penyebutan SQL ini, namun pada kenyataannya, kita dapat melafalkannya sebagai ``sequel'' ataupun ``S.Q.L''. Mempelajari bahasa pemrograman umum seperti R adalah penting dan akan lebih baik jika memiliki kemampuan SQL dalam bidang pengolahan data.

\begin{figure}

{\centering \includegraphics[width=1\linewidth]{./images/Bab1/SQL} 

}

\caption{Beberapa Perusahaan Besar Pengguna SQL}\label{fig:user-SQL}
\end{figure}

Banyak perusahaan besar di bidang teknologi menggunakan SQL seperti Uber, Netflix, dan Airbnb. Bahkan dalam perusahaan seperti Facebook, Google dan Amazon, yang telah membuat sendiri \textbf{SBD} berkemampuan tinggi, tetap menggunakan SQL untuk melakukan query dan analisis data.

\hypertarget{apa-itu-sbd}{%
\section{Apa itu SBD?}\label{apa-itu-sbd}}

Secara umum \textbf{SBD} dapat didefinisikan sebagai berikut:

\begin{figure}

{\centering \includegraphics[width=1\linewidth]{./images/Bab1/definisi_DB} 

}

\caption{Definisi Sistem Basis Data}\label{fig:SBD}
\end{figure}

\hypertarget{komponen-sbd}{%
\subsection{Komponen SBD}\label{komponen-sbd}}

Adapun beberapa komponen dasar yang diperlukan dalam SBD adalah:

\begin{figure}

{\centering \includegraphics[width=1\linewidth]{./images/Bab1/komponen_DB} 

}

\caption{Komponen SBD}\label{fig:komponen}
\end{figure}

\begin{center}\rule{0.5\linewidth}{0.5pt}\end{center}

\hypertarget{manfaat-sbd}{%
\subsection{Manfaat SBD}\label{manfaat-sbd}}

Manfaat atau kegunaan penerapan SBD cukup banyak dan cakupannya pun luas dalam mendukung keberadaan lembaga atau organisasi maupun perusahaan, diantaranya:

\begin{figure}

{\centering \includegraphics[width=1\linewidth]{./images/Bab1/manfaat_DB} 

}

\caption{Manfaat SBD}\label{fig:manfaat}
\end{figure}

\begin{center}\rule{0.5\linewidth}{0.5pt}\end{center}

\hypertarget{definisi-sql-vs-nosql}{%
\subsection{Definisi SQL vs NoSQL}\label{definisi-sql-vs-nosql}}

Sebenarnya perbedaan antara \href{https://www.dewaweb.com/blog/sql-pengertian-fungsi-beserta-perintah-dasarnya/}{SQL} dan \href{https://aws.amazon.com/id/nosql/}{NoSQL} secara mendasar sudah dapat dijelaskan dari akronimnya.

\begin{figure}

{\centering \includegraphics[width=1\linewidth]{./images/Bab1/SQLvsNoSQL_DB} 

}

\caption{SQL vs NoSQL}\label{fig:SQLvsNoSQL}
\end{figure}

\emph{SQL} basis data relasional yang menggunakan `relasi' (yang biasanya disebut tabel) untuk menyimpan data dan mencocokkan data tersebut dengan memakai karakteristik umum di setiap dataset. Sedangkan, \emph{NoSQL} adalah database yang menggunakan format JSON untuk setiap dokumennya sehingga mudah dibaca dan dimengerti. NoSQL banyak diminati karena memiliki performa yang tinggi dan bersifat non-relasional sehingga dapat memakai berbagai model data.

\begin{center}\rule{0.5\linewidth}{0.5pt}\end{center}

\hypertarget{perbedaan-sql-vs-nosql}{%
\subsection{Perbedaan SQL vs NoSQL}\label{perbedaan-sql-vs-nosql}}

Sebenarnya banyak perbedaan yang dimiliki di antara dua database tersebut tapi inilah perbedaan yang paling mencolok antara SQL dan NoSQL:

\begin{figure}

{\centering \includegraphics[width=1\linewidth]{./images/Bab1/Perbedaan_DB} 

}

\caption{Perbedaan SQL vs NoSQL}\label{fig:Perbedaan}
\end{figure}

\begin{center}\rule{0.5\linewidth}{0.5pt}\end{center}

\hypertarget{top-7-sql}{%
\subsection{Top 7 SQL}\label{top-7-sql}}

Tercatat sampai bulan Februari 2020 ada 334 jenis database menurut db-engines.com. Berikut ini saya merangkum daftar 7 database terpopuler yang menggunakan \href{https://qwords.com/blog/database-terpopuler/}{SQL} (Relasional):

\begin{figure}

{\centering \includegraphics[width=1\linewidth]{./images/Bab1/7SQL_DB} 

}

\caption{Top 7 Perangkat Lunak SQL}\label{fig:top7SQL}
\end{figure}

\begin{center}\rule{0.5\linewidth}{0.5pt}\end{center}

\hypertarget{top-8-nosql}{%
\subsection{Top 8 NoSQL}\label{top-8-nosql}}

Kebanyakan basis data NoSQL digunakan dalam dunia aplikasi web waktu nyata (real-time web app). Berikut ini adalah ulasan 8 jenis basis data \href{https://www.codepolitan.com/7-basis-data-nosql-populer}{NoSQL} yang paling populer digunakan diseluruh dunia:

\begin{figure}

{\centering \includegraphics[width=1\linewidth]{./images/Bab1/8NoSQL_DB} 

}

\caption{Top 8 Perangkat Lunak NoSQL}\label{fig:top8noSQL}
\end{figure}

\begin{center}\rule{0.5\linewidth}{0.5pt}\end{center}

\hypertarget{mengapa-r-sql}{%
\section{Mengapa R \& SQL?}\label{mengapa-r-sql}}

Menggunakan R dan SQL merupakan kombinasi yang kuat untuk analisis data dan pengelolaan basis data. Keduanya memiliki peran yang berbeda dalam proses analisis dan pengelolaan data. Berikut adalah beberapa alasan mengapa menggunakan R dan SQL bersama:

\begin{itemize}
\tightlist
\item
  \textbf{Kekuatan Analisis R}
\end{itemize}

R adalah bahasa pemrograman yang khusus dirancang untuk analisis statistik dan visualisasi data.
R memiliki berbagai paket (packages) yang menawarkan fungsi statistik dan analisis yang kuat, termasuk regresi, pengelompokan, analisis deret waktu, dan banyak lagi.
Visualisasi yang dapat dihasilkan dengan R sangat bervariasi, dari grafik sederhana hingga visualisasi interaktif yang kompleks.

\begin{itemize}
\tightlist
\item
  \textbf{Manipulasi dan Pengelolaan Data dengan SQL}
\end{itemize}

SQL digunakan untuk mengelola dan mengambil data dari basis data terstruktur.
SQL menyediakan cara efisien untuk membuat, mengubah, menghapus, dan memanipulasi data dalam basis data.
SQL memiliki fitur untuk menggabungkan data dari berbagai tabel, melakukan agregasi, dan menyaring data.

\begin{itemize}
\tightlist
\item
  \textbf{Integrasi Antara R dan SQL}
\end{itemize}

Banyak perpustakaan R yang mendukung koneksi ke basis data menggunakan SQL.
Anda dapat menggunakan perintah SQL dalam skrip R untuk mengambil data dari basis data, memanipulasi data di dalam R, dan kemudian menerapkan analisis statistik menggunakan paket R. Integrasi ini memungkinkan Anda menggabungkan kekuatan analisis statistik R dengan kemampuan pengelolaan data SQL.

\begin{itemize}
\tightlist
\item
  \textbf{Skalabilitas dan Efisiensi}
\end{itemize}

Menggunakan SQL untuk mengambil dan memanipulasi data dalam basis data bisa lebih efisien daripada melakukannya dalam R, terutama untuk dataset besar.
SQL memungkinkan query yang dioptimalkan dan penggunaan indeks untuk kinerja yang lebih baik.

\begin{itemize}
\tightlist
\item
  \textbf{Data Preprocessing}
\end{itemize}

Sebelum menerapkan analisis di R, Anda mungkin perlu melakukan pra-pemrosesan pada data, seperti membersihkan data, menggabungkan tabel, dan mengisi data yang hilang. SQL dapat membantu dalam melakukan tugas-tugas ini.

Jadi, menggunakan R dan SQL bersama memungkinkan Anda menggabungkan kekuatan analisis statistik R dengan kemampuan pengelolaan data SQL. Ini bisa sangat berguna ketika Anda ingin melakukan analisis data yang luas dan kompleks dari berbagai sumber data yang berbeda.

\begin{figure}

{\centering \includegraphics[width=1\linewidth]{./images/Bab1/sql-r-friends} 

}

\caption{R dan SQL [https://irene.rbind.io](https://irene.rbind.io/post/using-sql-in-rstudio/)}\label{fig:sql-r-friends}
\end{figure}

\hypertarget{mysql-vs-postgresql}{%
\section{MySQL vs PostgreSQL}\label{mysql-vs-postgresql}}

MySQL adalah sistem manajemen basis data relasional yang memungkinkan Anda untuk menyimpan data sebagai tabel dengan baris dan kolom. Sistem ini populer sehingga digunakan di banyak aplikasi web, situs web dinamis, dan sistem tertanam. PostgreSQL adalah sistem manajemen basis data relasional-objek yang menawarkan lebih banyak fitur daripada MySQL. Sistem ini memberi Anda lebih banyak fleksibilitas dalam tipe data, skalabilitas, konkurensi, dan integrasi data.

\begin{figure}

{\centering \includegraphics[width=1\linewidth]{./images/Bab1/MySQL-VS-PostgreSQL} 

}

\caption{MySQL vs PostgreSQL [https://integrio.net/](https://integrio.net/blog/postgresql-vs-mysql)}\label{fig:MySQL-VS-PostgreSQL}
\end{figure}

MySQL dan PostgreSQL, Keduanya menyimpan data di dalam tabel yang terkait satu sama lain melalui nilai kolom umum. Namun keduanya sering dibandingkan karena terdapat beberapa perbedaan. Ingin mengenal lebih dalam? Simak penjelasan di bawah.

\hypertarget{kelebihan}{%
\subsection{Kelebihan}\label{kelebihan}}

\begin{longtable}[]{@{}
  >{\raggedright\arraybackslash}p{(\columnwidth - 2\tabcolsep) * \real{0.4545}}
  >{\raggedright\arraybackslash}p{(\columnwidth - 2\tabcolsep) * \real{0.5455}}@{}}
\toprule\noalign{}
\begin{minipage}[b]{\linewidth}\raggedright
MySQL
\end{minipage} & \begin{minipage}[b]{\linewidth}\raggedright
PostgreSQL
\end{minipage} \\
\midrule\noalign{}
\endhead
\bottomrule\noalign{}
\endlastfoot
Integrasi bahasa pemrograman sangat luas; & Support framework website modern seperti Node.js dan Django; Support framework website modern seperti Node.js dan Django; \\
Aplikasi ringan, tidak membutuhkan spesifikasi hardware yang tinggi; & Dirilis dengan lisensi PostgreSQL sendiri; \\
Struktur tabel dengan fleksibilitas tinggi; & Bersifat open source dan gratis; \\
Dibekali banyak administrative tools; & Skala besar, mampu memuat hingga ribuan transaksi data; \\
Bersifat open source dan gratis (versi basic); & Memiliki banyak fitur yang mumpuni; \\
Meski open source, MySQL menjamin keamanan tingkat tinggi; & Memiliki banyak fitur yang mumpuni; \\
Mendukung berbagai variasi Data Type; & Performa sangat baik meski menuntut query yang lebih kompleks; \\
Dapat digunakan banyak pengguna karena mendukung multi user. & Kecepatan analisis (read-write) sangat cepat; Keamanan yang lebih ketat. \\
\end{longtable}

\hypertarget{kekurangan}{%
\subsection{Kekurangan}\label{kekurangan}}

\begin{longtable}[]{@{}
  >{\raggedright\arraybackslash}p{(\columnwidth - 2\tabcolsep) * \real{0.4545}}
  >{\raggedright\arraybackslash}p{(\columnwidth - 2\tabcolsep) * \real{0.5455}}@{}}
\toprule\noalign{}
\begin{minipage}[b]{\linewidth}\raggedright
MySQL
\end{minipage} & \begin{minipage}[b]{\linewidth}\raggedright
PostgreSQL
\end{minipage} \\
\midrule\noalign{}
\endhead
\bottomrule\noalign{}
\endlastfoot
Sistem manajemen database kurang cocok untuk aplikasi mobile dan game; & PostgreSQL tidak mendukung semua stack development; \\
Technical support MySQL dinilai kurang baik; & Meski memiliki integrasi dan skalabilitas tinggi, kecepatan PostgreSQL kalah unggul dibandingkan RDBMS lain; \\
Sulit diaplikasikan untuk manajemen database berskala besar. & Sistem kompatibilitas PostgreSQL menuntut pengguna untuk bekerja lebih keras dalam perbaikan dan perawatan. \\
\end{longtable}

\hypertarget{instalasi-mysql-xampp}{%
\section{Instalasi MySQL (XAMPP)}\label{instalasi-mysql-xampp}}

\hypertarget{download-aplikasi-xampp}{%
\subsection{Download Aplikasi XAMPP}\label{download-aplikasi-xampp}}

Silakan \href{https://www.apachefriends.org/download.html}{klik disini} untuk mengunduh applikasi XAMPP, pilih salah satu saja sesuai Operating System pada Komputer anda.

\begin{figure}

{\centering \includegraphics[width=1\linewidth]{./images/Bab1/xampp0} 

}

\caption{Langkah 1, Download XAMPP)}\label{fig:download-xammp}
\end{figure}

\hypertarget{install-aplikasi}{%
\subsection{Install Aplikasi}\label{install-aplikasi}}

Temukan file XAMPP.exe yang telah anda download, secara default biasanya disimpan di;

\begin{figure}

{\centering \includegraphics[width=1\linewidth]{./images/Bab1/xampp1} 

}

\caption{Langkah 2, Instalasi XAMPP)}\label{fig:install-xammp}
\end{figure}

Selanjutnya, akan muncul Warning di klik \textbf{OK}

\begin{figure}

{\centering \includegraphics[width=1\linewidth]{./images/Bab1/xampp2} 

}

\caption{Langkah 3, Instalasi XAMPP)}\label{fig:install-xammp2}
\end{figure}

selajutunya klik next

\begin{figure}

{\centering \includegraphics[width=1\linewidth]{./images/Bab1/xampp3} 

}

\caption{Langkah 4: Instalasi XAMPP)}\label{fig:install-xammp3}
\end{figure}

Klik next lagi, karena sudah dipilih secara default oleh XAMPP

\begin{figure}

{\centering \includegraphics[width=1\linewidth]{./images/Bab1/xampp4} 

}

\caption{Langkah 5, Instalasi XAMPP)}\label{fig:install-xammp4}
\end{figure}

\hypertarget{pilih-folder}{%
\subsection{Pilih Folder}\label{pilih-folder}}

\begin{figure}

{\centering \includegraphics[width=1\linewidth]{./images/Bab1/xampp5} 

}

\caption{Langkah 6, Instalasi XAMPP)}\label{fig:install-xammp5}
\end{figure}

Secara default akan membuat folder baru \textbf{C:\textbackslash xampp}, lalu pilih next.

\textbf{note:} jika anda sudah pernah mendownload aplikasi xampp, perlu di hapus terlebih dahulu file xampp yang lama di file \textbf{C:\textbackslash xampp}

\hypertarget{jalankan-proses-instalasi}{%
\subsection{Jalankan proses Instalasi}\label{jalankan-proses-instalasi}}

\begin{figure}

{\centering \includegraphics[width=1\linewidth]{./images/Bab1/xampp6} 

}

\caption{Langkah 7, Instalasi XAMPP)}\label{fig:install-xammp6}
\end{figure}

Tunggu proses instalasi selesai \textbf{Biasanya 5-10 menit, tergantung kecepatan komputer anda}.

\hypertarget{xampp-sudah-terinstall}{%
\subsection{XAMPP sudah terinstall}\label{xampp-sudah-terinstall}}

\begin{figure}

{\centering \includegraphics[width=1\linewidth]{./images/Bab1/xampp7} 

}

\caption{Langkah 8, Instalasi XAMPP)}\label{fig:install-xammp7}
\end{figure}

Setelah aplikasi terinstall, sudah bisa digunakan.

\hypertarget{video-instalasi-xampp}{%
\subsection{Video Instalasi XAMPP}\label{video-instalasi-xampp}}

\hypertarget{instalasi-postgresql}{%
\section{Instalasi PostgreSQL}\label{instalasi-postgresql}}

Berikut ini adalah proses langkah demi langkah tentang Cara Menginstal PostgreSQL di Windows:

\hypertarget{buka-browser}{%
\subsection{Buka Browser}\label{buka-browser}}

Klik \url{https://www.postgresql.org/download} and pilih Windows

\begin{figure}

{\centering \includegraphics[width=1\linewidth]{./images/Bab1/Postgree0} 

}

\caption{Langkah 1, Instalasi PostgreeSQL)}\label{fig:install-posrgree1}
\end{figure}

\hypertarget{cek-option}{%
\subsection{Cek Option}\label{cek-option}}

Klik Download the installer Interactive Installer by EnterpriseDB

\begin{figure}

{\centering \includegraphics[width=1\linewidth]{./images/Bab1/Postgree1} 

}

\caption{Langkah 2, Instalasi PostgreeSQL)}\label{fig:install-posrgree2}
\end{figure}

\hypertarget{pilih-postgresql-version}{%
\subsection{Pilih PostgreSQL version}\label{pilih-postgresql-version}}

Anda akan diminta untuk memilih versi PostgreSQL dan sistem operasi yang diinginkan. Pilih versi PostgreSQL terbaru dan OS sesuai dengan environment Anda, \textbf{klik tombol download.}

\begin{figure}

{\centering \includegraphics[width=1\linewidth]{./images/Bab1/Postgree2} 

}

\caption{Langkah 3, Instalasi PostgreeSQL)}\label{fig:install-posrgree3}
\end{figure}

\hypertarget{open-exe-file}{%
\subsection{Open exe file}\label{open-exe-file}}

Setelah Anda mengunduh PostgreSQL, buka exe yang telah diunduh dan Klik berikutnya pada layar install welcome screen.

\begin{figure}

{\centering \includegraphics[width=1\linewidth]{./images/Bab1/Postgree3} 

}

\caption{Langkah 4, Instalasi PostgreeSQL)}\label{fig:install-posrgree4}
\end{figure}

\hypertarget{pilih-folder-1}{%
\subsection{Pilih folder}\label{pilih-folder-1}}

Ubah direktori Instalasi jika diperlukan, jika tidak, biarkan default, \textbf{klik Next.}

\begin{figure}

{\centering \includegraphics[width=1\linewidth]{./images/Bab1/Postgree4} 

}

\caption{Langkah 5, Instalasi PostgreeSQL)}\label{fig:install-posrgree5}
\end{figure}

\hypertarget{select-components}{%
\subsection{Select components}\label{select-components}}

Anda dapat memilih komponen yang ingin Anda instal di sistem Anda. Anda dapat menghapus centang pada Stack Builder (\emph{disarankan ikuti secara default}), \textbf{klik Next.}

\begin{figure}

{\centering \includegraphics[width=1\linewidth]{./images/Bab1/Postgree5} 

}

\caption{Langkah 6, Instalasi PostgreeSQL)}\label{fig:install-posrgree6}
\end{figure}

\hypertarget{check-data-location}{%
\subsection{Check data location}\label{check-data-location}}

Anda dapat mengubah lokasi data, \textbf{Klik Next.}

\begin{figure}

{\centering \includegraphics[width=1\linewidth]{./images/Bab1/Postgree6} 

}

\caption{Langkah 7, Instalasi PostgreeSQL)}\label{fig:install-posrgree7}
\end{figure}

\hypertarget{masukan-password}{%
\subsection{Masukan Password}\label{masukan-password}}

Masukkan kata sandi superuser. Catat kata sandi tersebut, \textbf{Klik Next.}

\begin{figure}

{\centering \includegraphics[width=1\linewidth]{./images/Bab1/Postgree7} 

}

\caption{Langkah 8, Instalasi PostgreeSQL)}\label{fig:install-posrgree8}
\end{figure}

\hypertarget{cek-opsi-port}{%
\subsection{Cek opsi port}\label{cek-opsi-port}}

Biarkan nomor port menjadi default, \textbf{Klik Next.}

\begin{figure}

{\centering \includegraphics[width=1\linewidth]{./images/Bab1/Postgree8} 

}

\caption{Langkah 9, Instalasi PostgreeSQL)}\label{fig:install-posrgree9}
\end{figure}

\hypertarget{cek-summary}{%
\subsection{Cek Summary}\label{cek-summary}}

Periksa pra-penginstalan summary, \textbf{Klik Next}

\begin{figure}

{\centering \includegraphics[width=1\linewidth]{./images/Bab1/Postgree9} 

}

\caption{Langkah 10, Instalasi PostgreeSQL)}\label{fig:install-posrgree10}
\end{figure}

\hypertarget{ready-to-install}{%
\subsection{Ready to Install}\label{ready-to-install}}

Klik tombol Next

\begin{figure}

{\centering \includegraphics[width=1\linewidth]{./images/Bab1/Postgree10} 

}

\caption{Langkah 11, Instalasi PostgreeSQL)}\label{fig:install-posrgree11}
\end{figure}

\hypertarget{check-stack-builder-prompt}{%
\subsection{Check stack builder prompt}\label{check-stack-builder-prompt}}

Setelah instalasi selesai, Anda akan melihat prompt Stack Builder. Hapus centang pada opsi tersebut. Kita akan menggunakan Stack Builder dalam tutorial selanjutnya, \textbf{Klik Finish.}

\begin{figure}

{\centering \includegraphics[width=1\linewidth]{./images/Bab1/Postgree11} 

}

\caption{Langkah 12, Instalasi PostgreeSQL)}\label{fig:install-posrgree12}
\end{figure}

\hypertarget{launch-postgresql}{%
\subsection{Launch PostgreSQL}\label{launch-postgresql}}

Untuk launch PostgreSQL, buka Start Menu dan cari pgAdmin 4

\begin{figure}

{\centering \includegraphics[width=1\linewidth]{./images/Bab1/Postgree12} 

}

\caption{Langkah 13, Instalasi PostgreeSQL)}\label{fig:install-posrgree13}
\end{figure}

\hypertarget{check-pgadmin}{%
\subsection{Check pgAdmin}\label{check-pgadmin}}

Anda akan melihat beranda pgAdmin

\begin{figure}

{\centering \includegraphics[width=1\linewidth]{./images/Bab1/Postgree13} 

}

\caption{Langkah 14, Instalasi PostgreeSQL)}\label{fig:install-posrgree14}
\end{figure}

\hypertarget{cari-postgresql-15}{%
\subsection{Cari PostgreSQL 15}\label{cari-postgresql-15}}

Klik pada Servers \textgreater{} PostgreSQL 15 di sub sebelah kiri

\begin{figure}

{\centering \includegraphics[width=1\linewidth]{./images/Bab1/Postgree14} 

}

\caption{Langkah 15, Instalasi PostgreeSQL)}\label{fig:install-posrgree15}
\end{figure}

\hypertarget{enter-password}{%
\subsection{Enter password}\label{enter-password}}

Masukkan kata sandi superuser yang ditetapkan selama instalasi, \textbf{Klik OK}

\begin{figure}

{\centering \includegraphics[width=1\linewidth]{./images/Bab1/Postgree15} 

}

\caption{Langkah 16, Instalasi PostgreeSQL)}\label{fig:install-posrgree16}
\end{figure}

\hypertarget{cek-dashboard}{%
\subsection{Cek Dashboard}\label{cek-dashboard}}

Anda akan melihat Dashboard

\begin{figure}

{\centering \includegraphics[width=1\linewidth]{./images/Bab1/Postgree16} 

}

\caption{Langkah 17, Instalasi PostgreeSQL)}\label{fig:install-posrgree17}
\end{figure}

\hypertarget{video-instalasi-postgresql}{%
\subsection{Video Instalasi PostgreSQL}\label{video-instalasi-postgresql}}

\hypertarget{praktikum}{%
\section{Praktikum}\label{praktikum}}

\begin{itemize}
\tightlist
\item
  Tutorial di MySQL (CREATE \& Drop Database, Create \& Drop Tabel)
\item
  Tutorial di PostgreeeSQL (CREATE \& Drop Database, Create \& Drop Tabel)
\end{itemize}

\hypertarget{connecting-r-to-sql}{%
\chapter{Connecting R to SQL}\label{connecting-r-to-sql}}

\hypertarget{introduction}{%
\section{Introduction}\label{introduction}}

A database is a structured set of data. Terminology is a little bit different when working with a database management system compared to working with data in R.

\begin{itemize}
\tightlist
\item
  \textbf{field:} variable or quantity
\item
  \textbf{record:} collection of fields
\item
  \textbf{table:} collection of records with all the same fields
\item
  \textbf{database:} collection of tables
\end{itemize}

The relationship between R terminology and database terminology is explained below.

\begin{longtable}[]{@{}ll@{}}
\toprule\noalign{}
R terminology & Database terminology \\
\midrule\noalign{}
\endhead
\bottomrule\noalign{}
\endlastfoot
column & field \\
row & record \\
data frame & table \\
types of columns & table schema \\
collection of data & frames database \\
\end{longtable}

\begin{center}\rule{0.5\linewidth}{0.5pt}\end{center}

\hypertarget{connecting-r-to-sql-1}{%
\section{Connecting R to SQL}\label{connecting-r-to-sql-1}}

Connecting R to SQL databases allows you to leverage the power of R for data analysis while directly interacting with and querying data stored in relational databases. This connection enables you to retrieve, manipulate, and analyze data using SQL queries within your R environment.

\begin{figure}

{\centering \includegraphics[width=1\linewidth]{./images/Bab2/coneccting_db_in_R} 

}

\caption{Connecting R to SQL [https://rviews.rstudio.com](https://rviews.rstudio.com/2017/05/17/databases-using-r/)}\label{fig:coneccting}
\end{figure}

Here's a step-by-step introduction to connecting R to an SQL database:

\hypertarget{install-required-packages}{%
\subsection{Install Required Packages}\label{install-required-packages}}

First, you need to install R packages that facilitate database connections and SQL interactions. The DBI package provides a common interface for database connections, and you'll also need a database-specific package like RMySQL for MySQL, RPostgreSQL for PostgreSQL, or RODBC for ODBC connections.

You can install these packages using the following commands:

\begin{Shaded}
\begin{Highlighting}[]
\FunctionTok{install.packages}\NormalTok{(}\FunctionTok{c}\NormalTok{(}
                   \StringTok{"RMariaDB"}\NormalTok{,}
                   \StringTok{"RMySQL"}\NormalTok{,}
                   \StringTok{"RPostgres"}\NormalTok{,}
                   \StringTok{"RSQLite"}\NormalTok{,}
\NormalTok{                   )}
\NormalTok{                 )}
\end{Highlighting}
\end{Shaded}

\hypertarget{load-packages}{%
\subsection{Load Packages}\label{load-packages}}

Then, load all these requirement packages:

\begin{Shaded}
\begin{Highlighting}[]
\FunctionTok{library}\NormalTok{(}\StringTok{"RMariaDB"}\NormalTok{)                     }\CommentTok{\# Database Interface and \textquotesingle{}MariaDB\textquotesingle{} Driver}
\FunctionTok{library}\NormalTok{(}\StringTok{"RMySQL"}\NormalTok{)                       }\CommentTok{\# Database Interface and \textquotesingle{}RMySQL\textquotesingle{} Driver}
\FunctionTok{library}\NormalTok{(}\StringTok{"RPostgres"}\NormalTok{)                    }\CommentTok{\# Database Interface and \textquotesingle{}RPostgres\textquotesingle{} Driver}
\FunctionTok{library}\NormalTok{(}\StringTok{"RSQLite"}\NormalTok{)                      }\CommentTok{\# Database Interface and \textquotesingle{}RSQLite\textquotesingle{} Driver}
\end{Highlighting}
\end{Shaded}

\hypertarget{establish-a-connection}{%
\subsection{Establish a Connection}\label{establish-a-connection}}

There are many ways to connect your database with R. This article shows you three of the most common ways:

\hypertarget{mariadb}{%
\subsubsection*{MariaDB}\label{mariadb}}
\addcontentsline{toc}{subsubsection}{MariaDB}

You'll need to establish a connection to the MySQL server first. Use the \texttt{dbConnect()} function to create a connection. Provide the necessary connection information, such as the host, user, and password. For example:

\begin{Shaded}
\begin{Highlighting}[]
\NormalTok{MariaDB}\OtherTok{\textless{}{-}} \FunctionTok{dbConnect}\NormalTok{(RMariaDB}\SpecialCharTok{::}\FunctionTok{MariaDB}\NormalTok{(), }
                  \AttributeTok{user=}\StringTok{\textquotesingle{}root\textquotesingle{}}\NormalTok{,}
                  \AttributeTok{password=}\StringTok{\textquotesingle{}\textquotesingle{}}\NormalTok{, }
                  \AttributeTok{host=}\StringTok{\textquotesingle{}localhost\textquotesingle{}}\NormalTok{)}
\NormalTok{MariaDB}
\end{Highlighting}
\end{Shaded}

To get a list of all databases on the MySQL server, you can use the dbListTables() function. It provides a list of tables in the currently selected database. To see all databases, you can run an SQL query using the dbGetQuery() function. Here's an example:

\begin{Shaded}
\begin{Highlighting}[]
\FunctionTok{dbGetQuery}\NormalTok{(MariaDB, }\StringTok{"SHOW DATABASES"}\NormalTok{)          }\CommentTok{\# daftar semua database}
\end{Highlighting}
\end{Shaded}

To create a database, you can use the dbExecute() function to run a SQL query that creates the database. To drop (delete) a database, you can use the dbExecute() function with an SQL query that deletes the database. Be careful when dropping a database, as all data in the database will be permanently deleted. Here's an example:

\begin{Shaded}
\begin{Highlighting}[]
\FunctionTok{dbExecute}\NormalTok{(MariaDB,}\StringTok{"CREATE DATABASE new\_MariaDB"}\NormalTok{) }\CommentTok{\# create a database}
\FunctionTok{dbExecute}\NormalTok{(MariaDB,}\StringTok{"DROP DATABASE new\_MariaDB"}\NormalTok{)    }\CommentTok{\# Drop a Database}
\end{Highlighting}
\end{Shaded}

Or, you can create/drop a database by using IF Exist Statement,

\begin{Shaded}
\begin{Highlighting}[]
\FunctionTok{dbExecute}\NormalTok{(MySQL,}\StringTok{"CREATE DATABASE IF NOT EXISTS new\_MariaDB"}\NormalTok{) }\CommentTok{\# create a database}
\FunctionTok{dbExecute}\NormalTok{(MariaDB1,}\StringTok{"DROP DATABASE IF EXISTS new\_MariaDB"}\NormalTok{)    }\CommentTok{\# Drop a Database}
\end{Highlighting}
\end{Shaded}

To list the tables in the selected database, you can use the dbListTables() function:

\begin{Shaded}
\begin{Highlighting}[]
\FunctionTok{dbListTables}\NormalTok{(MariaDB)                          }\CommentTok{\# table list on your database}
\end{Highlighting}
\end{Shaded}

\hypertarget{mysql}{%
\subsubsection*{MySQL}\label{mysql}}
\addcontentsline{toc}{subsubsection}{MySQL}

We can perform similar database operations using the MySQL package in R as you would with the MariaDB package. Both packages provide database connectivity and SQL query execution capabilities, and their usage is quite similar.

Here's how you can establish a database connection and list tables in a specific database using the MySQL package:

\begin{Shaded}
\begin{Highlighting}[]
\NormalTok{MySQL }\OtherTok{\textless{}{-}} \FunctionTok{dbConnect}\NormalTok{(}\FunctionTok{MySQL}\NormalTok{(), }
                  \AttributeTok{user=}\StringTok{\textquotesingle{}root\textquotesingle{}}\NormalTok{,}
                  \AttributeTok{password=}\StringTok{\textquotesingle{}\textquotesingle{}}\NormalTok{, }
                  \AttributeTok{dbname=}\StringTok{\textquotesingle{}coba\_coba\textquotesingle{}}\NormalTok{, }
                  \AttributeTok{host=}\StringTok{\textquotesingle{}localhost\textquotesingle{}}\NormalTok{)}
\FunctionTok{dbListTables}\NormalTok{(MySQL)                               }\CommentTok{\# table list on your database}
\FunctionTok{dbExecute}\NormalTok{(MySQL,}\StringTok{"CREATE DATABASE new\_MySQL"}\NormalTok{)      }\CommentTok{\# Create a new Database}
\FunctionTok{dbExecute}\NormalTok{(MySQL,}\StringTok{"DROP DATABASE new\_MySQL"}\NormalTok{)        }\CommentTok{\# Drop a Database}
\end{Highlighting}
\end{Shaded}

\hypertarget{postgres}{%
\subsubsection*{Postgres}\label{postgres}}
\addcontentsline{toc}{subsubsection}{Postgres}

Certainly! You can use similar approaches to connect to and interact with a MySQL database using the RMariaDB package in R as you would with packages like RPostgreSQL for PostgreSQL. Both packages provide similar functionalities for connecting to and querying databases, but the specific functions and connection parameters may differ.

Here's a comparison between connecting to a MySQL database using RMariaDB and connecting to a PostgreSQL database using RPostgreSQL:

\begin{Shaded}
\begin{Highlighting}[]
\NormalTok{postgres }\OtherTok{\textless{}{-}} \FunctionTok{dbConnect}\NormalTok{(RPostgres}\SpecialCharTok{::}\FunctionTok{Postgres}\NormalTok{(), }
                  \AttributeTok{user=}\StringTok{\textquotesingle{}postgres\textquotesingle{}}\NormalTok{,}
                  \AttributeTok{password=}\StringTok{\textquotesingle{}123\textquotesingle{}}\NormalTok{, }
                  \AttributeTok{dbname=}\StringTok{\textquotesingle{}postgres\textquotesingle{}}\NormalTok{, }
                  \AttributeTok{host=}\StringTok{\textquotesingle{}localhost\textquotesingle{}}\NormalTok{)}
\FunctionTok{dbListTables}\NormalTok{(postgres)                            }\CommentTok{\# table list on your database}
\FunctionTok{dbExecute}\NormalTok{(postgres,}\StringTok{"CREATE DATABASE new\_PG"}\NormalTok{)      }\CommentTok{\# Create a new Database}
\FunctionTok{dbExecute}\NormalTok{(postgres,}\StringTok{"DROP DATABASE coba\_coba"}\NormalTok{)     }\CommentTok{\# Drop a Database}
\end{Highlighting}
\end{Shaded}

\hypertarget{sqlite}{%
\subsubsection*{SQLite}\label{sqlite}}
\addcontentsline{toc}{subsubsection}{SQLite}

Last but not least, you can perform similar database operations in RMariaDB as you would in packages like SQLite or any other database package in R. The general concepts and functions for database operations are quite similar across different database management systems. Here is the RSQLite example;

\begin{Shaded}
\begin{Highlighting}[]
\NormalTok{RSQLite }\OtherTok{\textless{}{-}} \FunctionTok{dbConnect}\NormalTok{(RSQLite}\SpecialCharTok{::}\FunctionTok{SQLite}\NormalTok{(), }\StringTok{"MySQLite.sqlite"}\NormalTok{)}
\FunctionTok{dbListTables}\NormalTok{(RSQLite)                             }\CommentTok{\# table list on your database}
\end{Highlighting}
\end{Shaded}

\begin{itemize}
\tightlist
\item
  \emph{Notes:} RSQLite will store the database you created in your current working directory.
\end{itemize}

\hypertarget{import-data}{%
\section{Import Data}\label{import-data}}

This section can be ignored if the data (table) that you need is already registered in your database. If not, then it is necessary to import data set according to your available files, download it below:

\begin{itemize}
\tightlist
\item
  \href{https://raw.githubusercontent.com/dsciencelabs/dataset/master/Customers.csv}{Customers.csv}
\item
  \href{https://raw.githubusercontent.com/dsciencelabs/dataset/master/Categories.csv}{Categories.csv}
\item
  \href{https://raw.githubusercontent.com/dsciencelabs/dataset/master/Employees.csv}{Employees.csv}
\item
  \href{https://raw.githubusercontent.com/dsciencelabs/dataset/master/OrderDetails.csv}{OrderDetails.csv}\\
\item
  \href{https://raw.githubusercontent.com/dsciencelabs/dataset/master/Orders.csv}{Orders.csv}
\item
  \href{https://raw.githubusercontent.com/dsciencelabs/dataset/master/Products.csv}{Products.csv}
\item
  \href{https://raw.githubusercontent.com/dsciencelabs/dataset/master/Shippers.csv}{Shippers.csv}
\item
  \href{https://raw.githubusercontent.com/dsciencelabs/dataset/master/Suppliers.csv}{Suppliers.csv}
\item
  \href{https://view.officeapps.live.com/op/view.aspx?src=https\%3A\%2F\%2Fraw.githubusercontent.com\%2Fdsciencelabs\%2Fdataset\%2Fmaster\%2FRawDatabase.xlsx\&wdOrigin=BROWSELINK}{RawDatabase.xlsx}
\end{itemize}

\hypertarget{csv-files}{%
\subsection{CSV Files}\label{csv-files}}

When you're working with files in R, such as reading data from a CSV file or saving plots as image files, R needs to know the location of these files. By setting the working directory, you provide a starting point for R to look for and save files.

In R, the setwd() function is used to set the working directory for your R session. The working directory is the folder on your computer where R will look for files and where it will save files unless you specify a different location. Here's why the setwd() function is important and when you might use it:

\begin{Shaded}
\begin{Highlighting}[]
\CommentTok{\# Set the working directory}
\FunctionTok{setwd}\NormalTok{(}\StringTok{"/path/to/your/folder"}\NormalTok{)}
\end{Highlighting}
\end{Shaded}

\begin{Shaded}
\begin{Highlighting}[]
\CommentTok{\# Now you can read CSV files without specifying the full path}
\NormalTok{data }\OtherTok{\textless{}{-}} \FunctionTok{read.csv}\NormalTok{(}\StringTok{"file.csv"}\NormalTok{)}
\end{Highlighting}
\end{Shaded}

Replace ``/path/to/your/folder'' with the actual path to the folder containing your CSV files. Then, you can run the following code!.

\begin{Shaded}
\begin{Highlighting}[]
\NormalTok{Customers   }\OtherTok{\textless{}{-}}\FunctionTok{read.csv}\NormalTok{(}\StringTok{"data/Customers.csv"}\NormalTok{)}
\NormalTok{Categories  }\OtherTok{\textless{}{-}}\FunctionTok{read.csv}\NormalTok{(}\StringTok{"data/Categories.csv"}\NormalTok{)  }
\NormalTok{Employees   }\OtherTok{\textless{}{-}}\FunctionTok{read.csv}\NormalTok{(}\StringTok{"data/Employees.csv"}\NormalTok{)  }
\NormalTok{OrderDetails}\OtherTok{\textless{}{-}}\FunctionTok{read.csv}\NormalTok{(}\StringTok{"data/OrderDetails.csv"}\NormalTok{)  }
\NormalTok{Orders      }\OtherTok{\textless{}{-}}\FunctionTok{read.csv}\NormalTok{(}\StringTok{"data/Orders.csv"}\NormalTok{)  }
\NormalTok{Products    }\OtherTok{\textless{}{-}}\FunctionTok{read.csv}\NormalTok{(}\StringTok{"data/Products.csv"}\NormalTok{)  }
\NormalTok{Shippers    }\OtherTok{\textless{}{-}}\FunctionTok{read.csv}\NormalTok{(}\StringTok{"data/Shippers.csv"}\NormalTok{)}
\NormalTok{Suppliers   }\OtherTok{\textless{}{-}}\FunctionTok{read.csv}\NormalTok{(}\StringTok{"data/Suppliers.csv"}\NormalTok{)  }
\end{Highlighting}
\end{Shaded}

\hypertarget{xlsx-files}{%
\subsection{XLSX Files}\label{xlsx-files}}

\begin{Shaded}
\begin{Highlighting}[]
\FunctionTok{library}\NormalTok{(}\StringTok{"readxl"}\NormalTok{)                                  }
\NormalTok{Customers   }\OtherTok{\textless{}{-}}\FunctionTok{read\_excel}\NormalTok{(}\StringTok{"data/RawDatabase.xlsx"}\NormalTok{,}\AttributeTok{sheet=}\DecValTok{1}\NormalTok{)      }
\NormalTok{Categories  }\OtherTok{\textless{}{-}}\FunctionTok{read\_excel}\NormalTok{(}\StringTok{"data/RawDatabase.xlsx"}\NormalTok{,}\AttributeTok{sheet=}\DecValTok{2}\NormalTok{) }
\NormalTok{Employees   }\OtherTok{\textless{}{-}}\FunctionTok{read\_excel}\NormalTok{(}\StringTok{"data/RawDatabase.xlsx"}\NormalTok{,}\AttributeTok{sheet=}\DecValTok{3}\NormalTok{)}
\NormalTok{OrderDetails}\OtherTok{\textless{}{-}}\FunctionTok{read\_excel}\NormalTok{(}\StringTok{"data/RawDatabase.xlsx"}\NormalTok{,}\AttributeTok{sheet=}\DecValTok{4}\NormalTok{) }
\NormalTok{Orders      }\OtherTok{\textless{}{-}}\FunctionTok{read\_excel}\NormalTok{(}\StringTok{"data/RawDatabase.xlsx"}\NormalTok{,}\AttributeTok{sheet=}\DecValTok{5}\NormalTok{)}
\NormalTok{Products    }\OtherTok{\textless{}{-}}\FunctionTok{read\_excel}\NormalTok{(}\StringTok{"data/RawDatabase.xlsx"}\NormalTok{,}\AttributeTok{sheet=}\DecValTok{6}\NormalTok{) }
\NormalTok{Shippers    }\OtherTok{\textless{}{-}}\FunctionTok{read\_excel}\NormalTok{(}\StringTok{"data/RawDatabase.xlsx"}\NormalTok{,}\AttributeTok{sheet=}\DecValTok{7}\NormalTok{)  }
\NormalTok{Suppliers   }\OtherTok{\textless{}{-}}\FunctionTok{read\_excel}\NormalTok{(}\StringTok{"data/RawDatabase.xlsx"}\NormalTok{,}\AttributeTok{sheet=}\DecValTok{8}\NormalTok{)  }
\end{Highlighting}
\end{Shaded}

\hypertarget{write-dataframe-to-database}{%
\section{Write Dataframe to Database}\label{write-dataframe-to-database}}

The key here is the \texttt{dbWriteTable} function which allows us to write an R data frame directly to a database table. The data frame's column names will be used as the database table's fields. In the following example I use \texttt{RMariaDB} connection, you can apply another driver as you like.

\begin{Shaded}
\begin{Highlighting}[]
\NormalTok{new\_con }\OtherTok{\textless{}{-}} \FunctionTok{dbConnect}\NormalTok{(}\FunctionTok{MariaDB}\NormalTok{(), }
                  \AttributeTok{user=}\StringTok{\textquotesingle{}root\textquotesingle{}}\NormalTok{,}
                  \AttributeTok{password=}\StringTok{\textquotesingle{}\textquotesingle{}}\NormalTok{, }
                  \AttributeTok{dbname=}\StringTok{\textquotesingle{}new\_MariaDB\textquotesingle{}}\NormalTok{, }
                  \AttributeTok{host=}\StringTok{\textquotesingle{}localhost\textquotesingle{}}\NormalTok{)}

\FunctionTok{dbWriteTable}\NormalTok{(new\_con, }\StringTok{"Customers"}\NormalTok{, Customers, }\AttributeTok{append=}\NormalTok{T) }
\FunctionTok{dbWriteTable}\NormalTok{(new\_con, }\StringTok{"Categories"}\NormalTok{, Categories, }\AttributeTok{append=}\NormalTok{T) }
\FunctionTok{dbWriteTable}\NormalTok{(new\_con, }\StringTok{"Employees"}\NormalTok{, Employees, }\AttributeTok{append=}\NormalTok{T) }
\FunctionTok{dbWriteTable}\NormalTok{(new\_con, }\StringTok{"OrderDetails"}\NormalTok{, OrderDetails, }\AttributeTok{append=}\NormalTok{T) }
\FunctionTok{dbWriteTable}\NormalTok{(new\_con, }\StringTok{"Orders"}\NormalTok{, Orders, }\AttributeTok{append=}\NormalTok{T) }
\FunctionTok{dbWriteTable}\NormalTok{(new\_con, }\StringTok{"Products"}\NormalTok{, Products, }\AttributeTok{append=}\NormalTok{T) }
\FunctionTok{dbWriteTable}\NormalTok{(new\_con, }\StringTok{"Shippers"}\NormalTok{, Shippers, }\AttributeTok{append=}\NormalTok{T) }
\FunctionTok{dbWriteTable}\NormalTok{(new\_con, }\StringTok{"Suppliers"}\NormalTok{, Suppliers, }\AttributeTok{append=}\NormalTok{T) }
\end{Highlighting}
\end{Shaded}

\emph{Note:} Some important things that must be considered when storing table data are as follows:

\begin{itemize}
\tightlist
\item
  Data Structure adjustments
\item
  Changes Data type (especially, Date and Time)
\end{itemize}

In this case, we have a problem with the data table \texttt{Employees} and \texttt{Orders}. When you consider these Table (Employees and Orders), you will find there is no date are written correctly in the database. In order to handle this problem, just type the following code in your R console:

\begin{Shaded}
\begin{Highlighting}[]
\FunctionTok{dbRemoveTable}\NormalTok{(new\_con, }\StringTok{"Orders"}\NormalTok{)}
\end{Highlighting}
\end{Shaded}

\begin{Shaded}
\begin{Highlighting}[]
\NormalTok{Orders[}\StringTok{"OrderDate"}\NormalTok{] }\OtherTok{\textless{}{-}}\FunctionTok{as.Date}\NormalTok{(Orders}\SpecialCharTok{$}\NormalTok{OrderDate, }\AttributeTok{format =} \StringTok{"\%Y{-}\%m{-}\%d"}\NormalTok{) }
\end{Highlighting}
\end{Shaded}

\begin{Shaded}
\begin{Highlighting}[]
\FunctionTok{dbWriteTable}\NormalTok{(new\_con, }\StringTok{"Orders"}\NormalTok{, Orders, }\AttributeTok{append=}\NormalTok{T) }
\end{Highlighting}
\end{Shaded}

\emph{Your Exercise:} Do the same thing to update data table \texttt{Employees}

\hypertarget{basic-sql-in-r}{%
\section{Basic SQL in R}\label{basic-sql-in-r}}

\hypertarget{select}{%
\subsection{SELECT}\label{select}}

The SELECT statement is used to select data from a database.

\begin{Shaded}
\begin{Highlighting}[]
\FunctionTok{library}\NormalTok{(DT)}
\NormalTok{df1}\OtherTok{\textless{}{-}}\FunctionTok{dbGetQuery}\NormalTok{(new\_con,}\StringTok{\textquotesingle{}SELECT city}
\StringTok{                         FROM Customers\textquotesingle{}}\NormalTok{)}
\FunctionTok{datatable}\NormalTok{(df1)}
\end{Highlighting}
\end{Shaded}

\hypertarget{where}{%
\subsection{WHERE}\label{where}}

The WHERE clause is used to filter records, extract only those records that fulfill a specified condition.

\begin{Shaded}
\begin{Highlighting}[]
\NormalTok{df2}\OtherTok{\textless{}{-}}\FunctionTok{dbGetQuery}\NormalTok{(new\_con,}\StringTok{"SELECT * }
\StringTok{                         FROM Customers }
\StringTok{                         WHERE Country=\textquotesingle{}Germany\textquotesingle{}"}\NormalTok{)}
\FunctionTok{datatable}\NormalTok{(df2)}
\end{Highlighting}
\end{Shaded}

\hypertarget{insert-into}{%
\subsection{INSERT INTO}\label{insert-into}}

If you are adding values for all the columns of the table, you do not need to specify the column names in the SQL query. However, make sure the order of the values is in the same order as the columns in the table. The INSERT INTO syntax would be as follows:

\begin{Shaded}
\begin{Highlighting}[]
\FunctionTok{dbExecute}\NormalTok{(new\_con,}\StringTok{"INSERT INTO Customers(CustomerName,ContactName,Address,City,PostalCode, Country)}
\StringTok{                   VALUES(\textquotesingle{}Bakti\textquotesingle{},\textquotesingle{}Siregar\textquotesingle{},\textquotesingle{}Jl.Bahagia Selalu\textquotesingle{},\textquotesingle{}Tangerang\textquotesingle{},\textquotesingle{}081369\textquotesingle{},\textquotesingle{}Indonesia\textquotesingle{})"}\NormalTok{)}
\end{Highlighting}
\end{Shaded}

\hypertarget{delete}{%
\subsection{DELETE}\label{delete}}

The DELETE statement is used to delete existing records in a table.

\begin{Shaded}
\begin{Highlighting}[]
\FunctionTok{dbExecute}\NormalTok{(new\_con,}\StringTok{"DELETE FROM Customers}
\StringTok{                   WHERE CustomerName =\textquotesingle{}Bakti\textquotesingle{} "}\NormalTok{)  }
\end{Highlighting}
\end{Shaded}

\hypertarget{update}{%
\subsection{UPDATE}\label{update}}

The UPDATE statement is used to modify the existing records in a table.

\begin{Shaded}
\begin{Highlighting}[]
\FunctionTok{dbExecute}\NormalTok{(new\_con,}\StringTok{"UPDATE Customers}
\StringTok{                   SET ContactName = \textquotesingle{}Alfred Schmidt\textquotesingle{}, City= \textquotesingle{}Frankfurt\textquotesingle{}}
\StringTok{                   WHERE CustomerID = 1"}\NormalTok{)}
\end{Highlighting}
\end{Shaded}

\hypertarget{disconnect-database}{%
\subsection{Disconnect Database}\label{disconnect-database}}

If you are done with the query process and you don't want to use it anymore, you should disconnect the connection from your database.

\begin{Shaded}
\begin{Highlighting}[]
\FunctionTok{dbDisconnect}\NormalTok{(new\_con)                                        }\CommentTok{\# disconnect from your database}
\end{Highlighting}
\end{Shaded}

\hypertarget{your-job}{%
\section{Your Job}\label{your-job}}

\begin{enumerate}
\def\labelenumi{\arabic{enumi}.}
\tightlist
\item
  Write Dataframe to your Database directory, using the following Engine:
\end{enumerate}

\begin{itemize}
\tightlist
\item
  RMySQL
\item
  RPostgres\\
\item
  RSQLite
\end{itemize}

\begin{enumerate}
\def\labelenumi{\arabic{enumi}.}
\setcounter{enumi}{1}
\tightlist
\item
  Create a `Basic Queries' tutorial using all engine that you already use at task no 1, (Such as: \texttt{SELECT}, \texttt{WHERE}, \texttt{INSERT\ INTO}, \texttt{DELETE}, and \texttt{UPDATE})!
\end{enumerate}

\hypertarget{fundamental-sql-in-r}{%
\chapter{Fundamental SQL in R}\label{fundamental-sql-in-r}}

In the previous section, we learned how to connect R to a Database System (SQL) Such as RMariaDB, RMySQL, and RSQLite. In this section, we continue to cover all that you have to know about fundamental operations in SQL (Here, focus on RMySQL).

\hypertarget{connecting-r-to-mysql}{%
\section{Connecting R to MySQL}\label{connecting-r-to-mysql}}

Connecting R to MySQL is made very easy with the \texttt{RMySQL} package. To connect to a MySQL database simply install the package and load the library.

\begin{Shaded}
\begin{Highlighting}[]
\FunctionTok{library}\NormalTok{(RMySQL)}
\NormalTok{MySQL }\OtherTok{\textless{}{-}} \FunctionTok{dbConnect}\NormalTok{(}\FunctionTok{MySQL}\NormalTok{(), }
                  \AttributeTok{user=}\StringTok{\textquotesingle{}root\textquotesingle{}}\NormalTok{,}
                  \AttributeTok{password=}\StringTok{\textquotesingle{}\textquotesingle{}}\NormalTok{, }
                  \AttributeTok{dbname=}\StringTok{\textquotesingle{}mysql\textquotesingle{}}\NormalTok{, }
                  \AttributeTok{host=}\StringTok{\textquotesingle{}localhost\textquotesingle{}}\NormalTok{)}
\FunctionTok{dbListTables}\NormalTok{(MySQL)                  }\CommentTok{\# a list of the tables in our connection}
\end{Highlighting}
\end{Shaded}

\emph{Note:} Open and your XAMPP, click start on Apache and MySQL. Then, make sure you have the admin privilege before creating any database.

\hypertarget{create-db}{%
\section{Create DB}\label{create-db}}

If you want to create a new database, then the CREATE DATABASE statement would be as shown below:

\begin{Shaded}
\begin{Highlighting}[]
\FunctionTok{dbExecute}\NormalTok{(MySQL, }\StringTok{"CREATE DATABASE factory\_db"}\NormalTok{)}
\end{Highlighting}
\end{Shaded}

The result show us \texttt{1}, means that you have succeeded to create a database.

\hypertarget{drop-db}{%
\section{Drop DB}\label{drop-db}}

If you want to delete an existing database, then the DROP DATABASE statement would be as shown below:

\begin{Shaded}
\begin{Highlighting}[]
\FunctionTok{dbExecute}\NormalTok{(MySQL, }\StringTok{"DROP DATABASE factory\_db"}\NormalTok{)}
\end{Highlighting}
\end{Shaded}

The result show us \texttt{0}, means that you have succeeded to remove (Drop) a database.

\hypertarget{create-table}{%
\section{Create Table}\label{create-table}}

Once you have a database, you can continue to create table as shown below:

\begin{Shaded}
\begin{Highlighting}[]
\FunctionTok{dbExecute}\NormalTok{(MySQL, }\StringTok{"CREATE TABLE Persons(}
\StringTok{                 PersonID int,}
\StringTok{                 LastName varchar(255),}
\StringTok{                 FirstName varchar(255),}
\StringTok{                 Address varchar(255),}
\StringTok{                 City varchar(255))"}\NormalTok{)}
\end{Highlighting}
\end{Shaded}

\hypertarget{insert-value}{%
\subsection{Insert Value}\label{insert-value}}

If you are adding values for all the columns of the table, you do not need to specify the column names in the SQL query. However, make sure the order of the values is in the same order as the columns in the table. The INSERT INTO syntax would be as follows:

\begin{Shaded}
\begin{Highlighting}[]
\FunctionTok{dbExecute}\NormalTok{(MySQL,}\StringTok{"INSERT INTO Persons(PersonID,LastName,FirstName, Address,City)}
\StringTok{                 VALUES(1,\textquotesingle{}Siregar\textquotesingle{},\textquotesingle{}Bakti\textquotesingle{}, \textquotesingle{}Jl.Bahagia\textquotesingle{},\textquotesingle{}Tangerang\textquotesingle{})"}\NormalTok{)}
\end{Highlighting}
\end{Shaded}

\hypertarget{truncate-table}{%
\subsection{Truncate Table}\label{truncate-table}}

The TRUNCATE TABLE statement is used to delete the data inside a table, but not the table itself.

\begin{Shaded}
\begin{Highlighting}[]
\FunctionTok{dbExecute}\NormalTok{(MySQL, }\StringTok{"TRUNCATE TABLE Persons"}\NormalTok{)}
\end{Highlighting}
\end{Shaded}

\hypertarget{drop-table}{%
\subsection{Drop Table}\label{drop-table}}

The DROP TABLE statement is used to drop an existing table in a database.

\begin{Shaded}
\begin{Highlighting}[]
\FunctionTok{dbExecute}\NormalTok{(MySQL, }\StringTok{"DROP TABLE Persons"}\NormalTok{)}
\end{Highlighting}
\end{Shaded}

\begin{center}\rule{0.5\linewidth}{0.5pt}\end{center}

\hypertarget{write-table}{%
\subsection{Write Table}\label{write-table}}

The key here is the \texttt{dbWriteTable} function which allows us to write an R data frame directly to a database table. The data frame's column names will be used as the database table's fields.

\begin{Shaded}
\begin{Highlighting}[]
\NormalTok{Orders      }\OtherTok{\textless{}{-}}\FunctionTok{read.csv}\NormalTok{(}\StringTok{"data/Orders.csv"}\NormalTok{)  }
\FunctionTok{dbWriteTable}\NormalTok{(MySQL, }\StringTok{"Orders"}\NormalTok{, Orders, }\AttributeTok{append=}\NormalTok{T) }
\end{Highlighting}
\end{Shaded}

\hypertarget{alter-table}{%
\subsection{Alter Table}\label{alter-table}}

The ALTER TABLE statement is used to add, delete, or modify columns in an existing table. The ALTER TABLE statement is also used to add and drop various constraints on an existing table.

\hypertarget{add-column}{%
\subsection{Add Column}\label{add-column}}

To add a column in a table, use the following syntax:

\begin{Shaded}
\begin{Highlighting}[]
\FunctionTok{dbExecute}\NormalTok{(MySQL, }\StringTok{"ALTER TABLE Orders}
\StringTok{                 ADD Email varchar(255)"}\NormalTok{)}
\end{Highlighting}
\end{Shaded}

\hypertarget{drop-column}{%
\subsection{Drop Column}\label{drop-column}}

To delete a column in a table, use the following syntax (notice that some database systems don't allow deleting a column):

\begin{Shaded}
\begin{Highlighting}[]
\FunctionTok{dbSendQuery}\NormalTok{(MySQL, }\StringTok{"ALTER TABLE Orders}
\StringTok{                   DROP COLUMN Email"}\NormalTok{)}
\end{Highlighting}
\end{Shaded}

\hypertarget{modify-column}{%
\subsection{Modify Column}\label{modify-column}}

\begin{Shaded}
\begin{Highlighting}[]
\FunctionTok{dbSendQuery}\NormalTok{(MySQL,}\StringTok{" ALTER TABLE Orders}
\StringTok{                  MODIFY COLUMN OrderDate date"}\NormalTok{)}
\end{Highlighting}
\end{Shaded}

\hypertarget{constraints}{%
\section{Constraints}\label{constraints}}

SQL constraints are used to specify rules for the data in a table. Constraints are used to limit the type of data that can go into a table. This ensures the accuracy and reliability of the data in the table. If there is any violation between the constraint and the data action, the action is aborted.

Constraints can be column level or table level. Column level constraints apply to a column, and table level constraints apply to the whole table. The following constraints are commonly used in SQL:

\begin{itemize}
\tightlist
\item
  \emph{NOT NULL:} Ensures that a column cannot have a NULL value
\item
  \emph{UNIQUE:} Ensures that all values in a column are different
\item
  \emph{PRIMARY KEY:} A combination of a NOT NULL and UNIQUE. Uniquely identifies each row in a table
\item
  \emph{FOREIGN KEY:} Uniquely identifies a row/record in another table
\item
  \emph{CHECK:} Ensures that all values in a column satisfies a specific condition
\item
  \emph{DEFAULT:} Sets a default value for a column when no value is specified
\item
  \emph{INDEX:} Used to create and retrieve data from the database very quickly
\end{itemize}

\hypertarget{not-null}{%
\subsection{Not Null}\label{not-null}}

The following SQL ensures that the ``ID'', ``LastName'', and ``FirstName'' columns will NOT accept NULL values when the ``Persons\_NotNull'' table is created:

\begin{Shaded}
\begin{Highlighting}[]
\FunctionTok{dbSendQuery}\NormalTok{(MySQL,}\StringTok{"CREATE TABLE Person\_NotNull (}
\StringTok{                   ID int NOT NULL,}
\StringTok{                   LastName varchar(255) NOT NULL,}
\StringTok{                   FirstName varchar(255) NOT NULL,}
\StringTok{                   Age int)"}\NormalTok{)}
\end{Highlighting}
\end{Shaded}

\hypertarget{unique}{%
\subsection{Unique}\label{unique}}

The following SQL creates a UNIQUE constraint on the ``ID'' column when the ``Persons'' table is created:

\begin{Shaded}
\begin{Highlighting}[]
\FunctionTok{dbSendQuery}\NormalTok{(MySQL,}\StringTok{"CREATE TABLE Persons\_Unique (ID int NOT NULL UNIQUE,}
\StringTok{                   LastName varchar(255) NOT NULL,}
\StringTok{                   FirstName varchar(255) NOT NULL,}
\StringTok{                   Age int)"}\NormalTok{)}
\end{Highlighting}
\end{Shaded}

To create a UNIQUE constraint on the ``ID'' column when the table is already created, use the following SQL:

\begin{Shaded}
\begin{Highlighting}[]
\FunctionTok{dbSendQuery}\NormalTok{(MySQL,}\StringTok{"ALTER TABLE Persons\_Unique}
\StringTok{                   ADD UNIQUE (ID)"}\NormalTok{)}
\end{Highlighting}
\end{Shaded}

To define a UNIQUE constraint on multiple columns, use the following SQL syntax:

\begin{Shaded}
\begin{Highlighting}[]
\FunctionTok{dbSendQuery}\NormalTok{(MySQL,}\StringTok{"ALTER TABLE Persons\_Unique}
\StringTok{                  ADD CONSTRAINT UNIQUE (ID,LastName)"}
\end{Highlighting}
\end{Shaded}

To drop a UNIQUE constraint, use the following SQL:

\begin{Shaded}
\begin{Highlighting}[]
\FunctionTok{dbSendQuery}\NormalTok{(MySQL,}\StringTok{"ALTER TABLE Persons\_Unique}
\StringTok{                  DROP INDEX ID"}\NormalTok{)}
\end{Highlighting}
\end{Shaded}

\hypertarget{primary-key}{%
\subsection{Primary Key}\label{primary-key}}

The PRIMARY KEY constraint uniquely identifies each record in a table. Primary keys must contain UNIQUE values, and cannot contain NULL values. A table can have only ONE primary key; and in the table, this primary key can consist of single or multiple columns (fields).

\begin{Shaded}
\begin{Highlighting}[]
\FunctionTok{dbSendQuery}\NormalTok{(MySQL,}\StringTok{"CREATE TABLE Persons\_PK (ID int NOT NULL PRIMARY KEY,}
\StringTok{                   LastName varchar(255) NOT NULL,}
\StringTok{                   FirstName varchar(255),}
\StringTok{                   Age int)"}\NormalTok{)}
\end{Highlighting}
\end{Shaded}

To allow naming of a PRIMARY KEY constraint, and for defining a PRIMARY KEY constraint on multiple columns, use the following SQL syntax:

\begin{Shaded}
\begin{Highlighting}[]
\FunctionTok{dbSendQuery}\NormalTok{(MySQL,}\StringTok{"CREATE TABLE Persons\_PK (ID int NOT NULL,}
\StringTok{                  LastName varchar(255) NOT NULL,}
\StringTok{                  FirstName varchar(255),}
\StringTok{                  Age int,}
\StringTok{                  CONSTRAINT Persons\_PK PRIMARY KEY (ID,LastName))"}\NormalTok{)}
\end{Highlighting}
\end{Shaded}

To create a PRIMARY KEY constraint on the ``ID'' column when the table is already created, use the following SQL:

\begin{Shaded}
\begin{Highlighting}[]
\FunctionTok{dbSendQuery}\NormalTok{(MySQL, }\StringTok{"ALTER TABLE Persons\_PK}
\StringTok{                   ADD PRIMARY KEY (ID)"}\NormalTok{)}
\end{Highlighting}
\end{Shaded}

\hypertarget{foreign-key}{%
\subsection{Foreign Key}\label{foreign-key}}

A FOREIGN KEY is a key used to link two tables together. A FOREIGN KEY is a field (or collection of fields) in one table that refers to the PRIMARY KEY in another table. The table containing the foreign key is called the child table, and the table containing the candidate key is called the referenced or parent table.

Look at the following two tables:

\begin{itemize}
\tightlist
\item
  ``Persons'' table:
\end{itemize}

\begin{longtable}[]{@{}cccc@{}}
\toprule\noalign{}
PersonID & LastName & FirstName & Age \\
\midrule\noalign{}
\endhead
\bottomrule\noalign{}
\endlastfoot
1 & Xi & Bakti & 28 \\
2 & Li & Chong & 23 \\
3 & Gou & Mei & 20 \\
\end{longtable}

\begin{itemize}
\tightlist
\item
  ``Orders'' table:
\end{itemize}

\begin{longtable}[]{@{}ccc@{}}
\toprule\noalign{}
OrderID & OrderNumber & PersonID \\
\midrule\noalign{}
\endhead
\bottomrule\noalign{}
\endlastfoot
1 & 77895 & 3 \\
2 & 44678 & 3 \\
\end{longtable}

Notice that the ``PersonID'' column in the ``Orders'' table points to the ``PersonID'' column in the ``Persons'' table.

\begin{itemize}
\tightlist
\item
  The ``PersonID'' column in the ``Persons'' table is the PRIMARY KEY in the ``Persons'' table.
\item
  The ``PersonID'' column in the ``Orders'' table is a FOREIGN KEY in the ``Orders'' table.
\item
  The FOREIGN KEY constraint is used to prevent actions that would destroy links between tables.
\item
  The FOREIGN KEY constraint also prevents invalid data from being inserted into the foreign key column, because it has to be one of the values contained in the table it points to.
\end{itemize}

To allow naming of a FOREIGN KEY constraint, and for defining a FOREIGN KEY constraint on multiple columns, use the following SQL syntax:

\begin{Shaded}
\begin{Highlighting}[]
\FunctionTok{dbSendQuery}\NormalTok{(MySQL,}
\StringTok{"CREATE TABLE Orders (OrderID int NOT NULL,}
\StringTok{                     OrderNumber int NOT NULL,}
\StringTok{                     PersonID int,}
\StringTok{                     CONSTRAINT FOREIGN KEY (PersonID))"}\NormalTok{)}
\end{Highlighting}
\end{Shaded}

To allow naming of a FOREIGN KEY constraint, and for defining a FOREIGN KEY constraint on multiple columns, use the following SQL syntax:

\begin{Shaded}
\begin{Highlighting}[]
\FunctionTok{dbSendQuery}\NormalTok{(MySQL,}
\StringTok{"CREATE TABLE Orders (}
\StringTok{    OrderID int NOT NULL,}
\StringTok{    OrderNumber int NOT NULL,}
\StringTok{    PersonID int,}
\StringTok{    PRIMARY KEY (OrderID),}
\StringTok{    FOREIGN KEY (PersonID) REFERENCES Persons\_pk (PersonID))"}\NormalTok{)}
\end{Highlighting}
\end{Shaded}

To allow naming of a FOREIGN KEY constraint, and for defining a FOREIGN KEY constraint on multiple columns, use the following SQL syntax:

\begin{Shaded}
\begin{Highlighting}[]
\FunctionTok{dbSendQuery}\NormalTok{(MySQL,}
\StringTok{"ALTER TABLE Orders}
\StringTok{ADD CONSTRAINT FK\_Person Order}
\StringTok{FOREIGN KEY (PersonID) REFERENCES Persons(PersonID)"}\NormalTok{)}
\end{Highlighting}
\end{Shaded}

\hypertarget{check}{%
\subsection{Check}\label{check}}

The CHECK constraint is used to limit the value range that can be placed in a column. If you define a CHECK constraint on a single column it allows only certain values for this column. If you define a CHECK constraint on a table it can limit the values in certain columns based on values in other columns in the row. The following SQL creates a CHECK constraint on the ``Age'' column when the ``Persons'' table is created. The CHECK constraint ensures that the age of a person must be 18, or older:

\begin{Shaded}
\begin{Highlighting}[]
\FunctionTok{dbSendQuery}\NormalTok{(MySQL,}
\StringTok{"CREATE TABLE Persons (ID int NOT NULL,}
\StringTok{                      LastName varchar(255) NOT NULL,}
\StringTok{                      FirstName varchar(255),}
\StringTok{                      Age int,}
\StringTok{                      CHECK (Age\textgreater{}=18))"}\NormalTok{)}
\end{Highlighting}
\end{Shaded}

To allow naming of a CHECK constraint, and for defining a CHECK constraint on multiple columns, use the following SQL syntax:

\begin{Shaded}
\begin{Highlighting}[]
\FunctionTok{dbSendQuery}\NormalTok{(MySQL,}
\StringTok{"CREATE TABLE Persons (}
\StringTok{    ID int NOT NULL,}
\StringTok{    LastName varchar(255) NOT NULL,}
\StringTok{    FirstName varchar(255),}
\StringTok{    Age int,}
\StringTok{    City varchar(255),}
\StringTok{    CONSTRAINT CHK\_Person CHECK (Age\textgreater{}=18 AND City=\textquotesingle{}Sandnes\textquotesingle{}))"}\NormalTok{)}
\end{Highlighting}
\end{Shaded}

To create a CHECK constraint on the ``Age'' column when the table is already created, use the following SQL:

\begin{Shaded}
\begin{Highlighting}[]
\FunctionTok{dbSendQuery}\NormalTok{(MySQL, }\StringTok{"ALTER TABLE Persons}
\StringTok{                         ADD CHECK (Age\textgreater{}=18)"}\NormalTok{)}
\end{Highlighting}
\end{Shaded}

To allow naming of a CHECK constraint, and for defining a CHECK constraint on multiple columns, use the following SQL syntax:

\begin{Shaded}
\begin{Highlighting}[]
\FunctionTok{dbSendQuery}\NormalTok{(MySQL, }\StringTok{"ALTER TABLE Persons}
\StringTok{                    ADD CONSTRAINT CHK\_PersonAge }
\StringTok{                    CHECK (Age\textgreater{}=18 AND City=\textquotesingle{}Sandnes\textquotesingle{})"}\NormalTok{)}
\end{Highlighting}
\end{Shaded}

\hypertarget{default}{%
\subsection{Default}\label{default}}

The DEFAULT constraint is used to provide a default value for a column. The default value will be added to all new records IF no other value is specified. The following SQL sets a DEFAULT value for the ``City'' column when the ``Persons'' table is created:

\begin{Shaded}
\begin{Highlighting}[]
\FunctionTok{dbSendQuery}\NormalTok{(MySQL,}
\StringTok{"CREATE TABLE Persons\_default (ID int NOT NULL,}
\StringTok{                              LastName varchar(255) NOT NULL,}
\StringTok{                              FirstName varchar(255),}
\StringTok{                              Age int,}
\StringTok{                              City varchar(255) DEFAULT \textquotesingle{}Sandnes\textquotesingle{})"}\NormalTok{)}
\end{Highlighting}
\end{Shaded}

To create a DEFAULT constraint on the ``City'' column when the table is already created, use the following SQL:

\begin{Shaded}
\begin{Highlighting}[]
\FunctionTok{dbSendQuery}\NormalTok{(MySQL,}\StringTok{"ALTER TABLE Persons}
\StringTok{                        ALTER City SET DEFAULT \textquotesingle{}Sandnes\textquotesingle{}"}\NormalTok{)}
\end{Highlighting}
\end{Shaded}

\hypertarget{index}{%
\subsection{Index}\label{index}}

The CREATE INDEX statement is used to create indexes in tables. Indexes are used to retrieve data from the database more quickly than otherwise. The users cannot see the indexes, they are just used to speed up searches/queries. Creates an index on a table. Duplicate values are allowed:

\begin{Shaded}
\begin{Highlighting}[]
\FunctionTok{dbSendQuery}\NormalTok{(MySQL, }\StringTok{"CREATE INDEX idx\_pname}
\StringTok{                   ON Persons (LastName, FirstName)"}\NormalTok{)}
\end{Highlighting}
\end{Shaded}

\emph{Note:} Updating a table with indexes takes more time than updating a table without (because the indexes also need an update). So, only create indexes on columns that will be frequently searched against.

\hypertarget{auto-increment}{%
\subsection{Auto Increment}\label{auto-increment}}

Auto-increment allows a unique number to be generated automatically when a new record is inserted into a table. Often this is the primary key field that we would like to be created automatically every time a new record is inserted. The following SQL statement defines the ``Personid'' column to be an auto-increment primary key field in the ``Persons'' table:

\begin{Shaded}
\begin{Highlighting}[]
\FunctionTok{dbSendQuery}\NormalTok{(MySQL,}
\StringTok{"CREATE TABLE Persons\_ai (}
\StringTok{    Personid int NOT NULL AUTO\_INCREMENT,}
\StringTok{    LastName varchar(255) NOT NULL,}
\StringTok{    FirstName varchar(255),}
\StringTok{    Age int,}
\StringTok{    PRIMARY KEY (Personid))"}\NormalTok{)}
\end{Highlighting}
\end{Shaded}

\hypertarget{previewing-.sql-in-r}{%
\section{\texorpdfstring{Previewing \texttt{.sql} in R}{Previewing .sql in R}}\label{previewing-.sql-in-r}}

When you open a new \texttt{.sql} file in RStudio, it automatically populates the file with the following code:

\begin{Shaded}
\begin{Highlighting}[]
\FunctionTok{library}\NormalTok{(RSQLite)}
\FunctionTok{library}\NormalTok{(dplyr)}
\FunctionTok{library}\NormalTok{(dbplyr)}

\NormalTok{conn }\OtherTok{\textless{}{-}} \FunctionTok{src\_memdb}\NormalTok{() }\CommentTok{\# create a SQLite database in memory}
\FunctionTok{copy\_to}\NormalTok{(conn, }
\NormalTok{        storms,     }\CommentTok{\# this is a dataset built into dplyr}
        \AttributeTok{overwrite =} \ConstantTok{TRUE}\NormalTok{)}
\FunctionTok{tbl}\NormalTok{(conn, }\FunctionTok{sql}\NormalTok{(}\StringTok{"SELECT * FROM storms LIMIT 5"}\NormalTok{))}
\end{Highlighting}
\end{Shaded}

You need to create a \texttt{.sql} file with the following code:

\begin{verbatim}
-- !preview conn=src_memdb()$con

SELECT * FROM storms LIMIT 5
\end{verbatim}

Then, you will the resul like this:

\begin{Shaded}
\begin{Highlighting}[]
\FunctionTok{library}\NormalTok{(knitr)}
\FunctionTok{include\_graphics}\NormalTok{(}\StringTok{"./images/Bab3/sql{-}file{-}preview.png"}\NormalTok{)}
\end{Highlighting}
\end{Shaded}

\begin{figure}

{\centering \includegraphics[width=1\linewidth]{./images/Bab3/sql-file-preview} 

}

\caption{Previewing `.sql` in R}\label{fig:unnamed-chunk-59}
\end{figure}

\hypertarget{sql-chunks-in-rmarkdown}{%
\section{SQL chunks in RMarkdown}\label{sql-chunks-in-rmarkdown}}

I generally prefer to show RMarkdown output in the console 1 (and it looks like I'm not the only one). This means that when I run code in an .Rmd file, it feels more or less the same as when I run an .R file: the plots show up in the plots pane, code is run in the console, and so on. While you can use SQL chunks with this setting, there is NO chunk preview option. You must trust your queries and knit the file to make sure everything runs. You get the syntax highlighting razzle-dazzle but alas-- no preview.

It is in this very specific case where inline mode wins big time. SQL previews magically become an option, allowing you to interact with your beautifully colored SQL code.

\hypertarget{database-normalization-in-sql}{%
\chapter{Database Normalization in SQL}\label{database-normalization-in-sql}}

Normalization is a database design technique that reduces data redundancy and eliminates undesirable characteristics like Insertion, Update and Deletion Anomalies. Normalization rules divides larger tables into smaller tables and links them using relationships. The purpose of Normalisation in SQL is to eliminate redundant (repetitive) data and ensure data is stored logically.

The inventor of the relational model Edgar Codd proposed the theory of normalization of data with the introduction of the First Normal Form, and he continued to extend theory with Second and Third Normal Form. Later he joined Raymond F. Boyce to develop the theory of Boyce-Codd Normal Form.

Here is a list of Normal Forms in SQL:

\begin{itemize}
\tightlist
\item
  1NF (First Normal Form)
\item
  2NF (Second Normal Form)
\item
  3NF (Third Normal Form)
\item
  BCNF (Boyce-Codd Normal Form)
\item
  4NF (Fourth Normal Form)
\item
  5NF (Fifth Normal Form)
\item
  6NF (Sixth Normal Form)
\end{itemize}

The Theory of Data Normalization in MySQL server is still being developed further. For example, there are discussions even on 6th Normal Form. However, in most practical applications, normalization achieves its best in 3rd Normal Form.

\hypertarget{the-process-of-normalization}{%
\section{The Process of Normalization}\label{the-process-of-normalization}}

The process of normalization involves breaking down a large table into smaller, related tables and defining relationships between them. This helps in achieving the following benefits:

\begin{itemize}
\tightlist
\item
  \textbf{Elimination of Data Redundancy:} Redundant data can lead to inconsistencies and increased storage requirements. Normalization ensures that each piece of data is stored in only one place, reducing redundancy and promoting consistency.
\item
  \textbf{Data Integrity:} Normalization minimizes the chances of inconsistencies and anomalies that may occur when data is duplicated or updated in one place but not in another.
\item
  \textbf{Efficient Data Updates:} Since data is stored in smaller, more specific tables, updates are more efficient and require fewer changes.
\item
  \textbf{Simpler Queries:} Normalized data allows for more straightforward and efficient querying due to the structured relationships between tables.
\end{itemize}

The process of database normalization is typically divided into several ``normal forms'' (often referred to as 1NF, 2NF, 3NF, BCNF, etc.), each with its own set of rules and requirements. These normal forms build on each other, with higher normal forms addressing more complex issues of redundancy and dependency. Here's a brief overview of some common normal forms:

\begin{enumerate}
\def\labelenumi{\arabic{enumi}.}
\tightlist
\item
  \textbf{First Normal Form (1NF):}
\end{enumerate}

\begin{itemize}
\tightlist
\item
  Eliminate duplicate columns.
\item
  Create separate tables for related data.
\item
  Define a primary key for each table.
\end{itemize}

\begin{enumerate}
\def\labelenumi{\arabic{enumi}.}
\setcounter{enumi}{1}
\tightlist
\item
  \textbf{Second Normal Form (2NF):}
\end{enumerate}

\begin{itemize}
\tightlist
\item
  Meet 1NF requirements.
\item
  Remove partial dependencies (attributes dependent on only part of the primary key) by creating
\item
  separate tables.
\end{itemize}

\begin{enumerate}
\def\labelenumi{\arabic{enumi}.}
\setcounter{enumi}{2}
\tightlist
\item
  \textbf{Third Normal Form (3NF):}
\end{enumerate}

\begin{itemize}
\tightlist
\item
  Meet 2NF requirements.
\item
  Remove transitive dependencies (attributes dependent on non-key attributes) by creating separate tables.
\end{itemize}

\begin{enumerate}
\def\labelenumi{\arabic{enumi}.}
\setcounter{enumi}{3}
\tightlist
\item
  \textbf{Boyce-Codd Normal Form (BCNF):}
\end{enumerate}

\begin{itemize}
\tightlist
\item
  Meet 3NF requirements.
\item
  Remove overlapping candidate keys by creating separate tables.
\end{itemize}

Higher normal forms exist beyond these, such as \textbf{Fourth Normal Form (4NF)} and \textbf{Fifth Normal Form (5NF)}, but they are less commonly encountered and may be more relevant in specific cases of complex data modeling. While normalization offers significant benefits, it's important to strike a balance between normalization and performance. Over-normalization can lead to complex query logic and decreased query performance. Therefore, designing a database often involves considering the nature of the data and the queries that will be performed on it.

\hypertarget{simple-database-normalization}{%
\section{Simple Database Normalization}\label{simple-database-normalization}}

Let's go through a simple example of database normalization using a hypothetical scenario of an online bookstore. We'll start with an unnormalized table and then progressively normalize it through different normal forms.

Scenario: Consider an unnormalized table that stores information about books, authors, and their publishers.

\textbf{Unnormalized Table (1NF):}

\begin{longtable}[]{@{}
  >{\raggedright\arraybackslash}p{(\columnwidth - 10\tabcolsep) * \real{0.1233}}
  >{\raggedright\arraybackslash}p{(\columnwidth - 10\tabcolsep) * \real{0.2329}}
  >{\raggedright\arraybackslash}p{(\columnwidth - 10\tabcolsep) * \real{0.1918}}
  >{\raggedright\arraybackslash}p{(\columnwidth - 10\tabcolsep) * \real{0.1918}}
  >{\raggedright\arraybackslash}p{(\columnwidth - 10\tabcolsep) * \real{0.1781}}
  >{\raggedright\arraybackslash}p{(\columnwidth - 10\tabcolsep) * \real{0.0822}}@{}}
\toprule\noalign{}
\begin{minipage}[b]{\linewidth}\raggedright
Book ID
\end{minipage} & \begin{minipage}[b]{\linewidth}\raggedright
Title
\end{minipage} & \begin{minipage}[b]{\linewidth}\raggedright
Author
\end{minipage} & \begin{minipage}[b]{\linewidth}\raggedright
Author Birth
\end{minipage} & \begin{minipage}[b]{\linewidth}\raggedright
Publisher
\end{minipage} & \begin{minipage}[b]{\linewidth}\raggedright
Year
\end{minipage} \\
\midrule\noalign{}
\endhead
\bottomrule\noalign{}
\endlastfoot
1 & Algorithm & John Smith & 1980-05-15 & ABC Pub & 2000 \\
2 & Data Science & Jane Doe & 1975-10-20 & XYZ Books & 2015 \\
3 & Database System & John Smith & 1980-05-15 & ABC Pub & 2012 \\
\end{longtable}

In this unnormalized table, we have duplicate author and publisher information, leading to redundancy. John Smith's information is repeated, and if any of his details change, we need to update multiple rows.

\textbf{First Normal Form (1NF):}

To achieve 1NF, we break the table into smaller tables and remove duplicate data. We create separate tables for authors and publishers.

Authors Table:

\begin{longtable}[]{@{}lll@{}}
\toprule\noalign{}
Author ID & Author & Author Birth \\
\midrule\noalign{}
\endhead
\bottomrule\noalign{}
\endlastfoot
1 & John Smith & 1980-05-15 \\
2 & Jane Doe & 1975-10-20 \\
\end{longtable}

Publishers Table:

\begin{longtable}[]{@{}ll@{}}
\toprule\noalign{}
Publisher ID & Publisher \\
\midrule\noalign{}
\endhead
\bottomrule\noalign{}
\endlastfoot
1 & ABC Pub \\
2 & XYZ Books \\
\end{longtable}

Books Table (1NF):

\begin{longtable}[]{@{}lllll@{}}
\toprule\noalign{}
Book ID & Title & Author ID & Publisher ID & Year \\
\midrule\noalign{}
\endhead
\bottomrule\noalign{}
\endlastfoot
1 & Algorithm & 1 & 1 & 2000 \\
2 & Data Science & 2 & 2 & 2015 \\
3 & Database System & 1 & 1 & 2012 \\
\end{longtable}

We've eliminated redundancy by referencing author and publisher IDs in the books table.

\textbf{Second Normal Form (2NF):}

To achieve 2NF, we identify partial dependencies and create a separate table for author information.

Authors Table (2NF):

\begin{longtable}[]{@{}lll@{}}
\toprule\noalign{}
Author ID & Author & Author Birth \\
\midrule\noalign{}
\endhead
\bottomrule\noalign{}
\endlastfoot
1 & John Smith & 1980-05-15 \\
2 & Jane Doe & 1975-10-20 \\
\end{longtable}

Books Table (2NF):

\begin{longtable}[]{@{}lllll@{}}
\toprule\noalign{}
Book ID & Title & Author ID & Publisher ID & Year \\
\midrule\noalign{}
\endhead
\bottomrule\noalign{}
\endlastfoot
1 & Algorithm & 1 & 1 & 2000 \\
2 & Data Science & 2 & 2 & 2015 \\
3 & Database System & 1 & 1 & 2012 \\
\end{longtable}

No changes are required in the Books table for 2NF since there were no partial dependencies.

\textbf{Third Normal Form (3NF):}

To achieve 3NF, we identify transitive dependencies and create a separate table for publisher information.

Publishers Table (3NF):

\begin{longtable}[]{@{}ll@{}}
\toprule\noalign{}
Publisher ID & Publisher \\
\midrule\noalign{}
\endhead
\bottomrule\noalign{}
\endlastfoot
1 & ABC Pub \\
2 & XYZ Books \\
\end{longtable}

Books Table (3NF):

\begin{longtable}[]{@{}lllll@{}}
\toprule\noalign{}
Book ID & Title & Author ID & Publisher ID & Year \\
\midrule\noalign{}
\endhead
\bottomrule\noalign{}
\endlastfoot
1 & Algorithm & 1 & 1 & 2000 \\
2 & Data Science & 2 & 2 & 2015 \\
3 & Database System & 1 & 1 & 2012 \\
\end{longtable}

No changes are required in the Books table for 3NF since there were no transitive dependencies. The result is a normalized database structure that eliminates redundancy and ensures data integrity.

Please note that the above example is simplified for demonstration purposes. In real-world scenarios, databases can have more complex structures and relationships, which may require deeper levels of normalization to achieve higher normal forms like BCNF or 4NF.

\hypertarget{your-job-1}{%
\section{Your Job}\label{your-job-1}}

Consider a hypothetical database for an online bookstore. We'll start with a denormalized table and then go through the normalization process step by step.

Suppose we have a single table called Books with the following columns:

\begin{table}

\caption{\label{tab:unnamed-chunk-61}First Normal Form (1NF)}
\centering
\begin{tabular}[t]{l|l|l|l|l|l}
\hline
BookID & Title & Author & Genre & Publisher & PublicationYear\\
\hline
1 & Book A & Author X & Fiction & dsciencelabs & 2021\\
\hline
2 & Book B & Author Y & Mystery & Matana & 2022\\
\hline
3 & Book B & Author X & Romance & dsciencelabs & 2023\\
\hline
\end{tabular}
\end{table}

Your job is the following statements:

\begin{enumerate}
\def\labelenumi{\arabic{enumi}.}
\tightlist
\item
  Display Database Normalization Process
\item
  Create Database to your PC After Normalization Process using R and SQL
\end{enumerate}

\hypertarget{join-table-in-sql}{%
\chapter{Join Table in SQL}\label{join-table-in-sql}}

A SQL join is a Structured Query Language (SQL) instruction to combine data from two sets of data (i.e.~two tables). Before we dive into the details of a SQL join, let's briefly discuss what SQL is, and why someone would want to perform a SQL join.

SQL is a special-purpose programming language designed for managing information in a relational database management system (RDBMS). The word relational here is key; it specifies that the database management system is organized in such a way that there are clear relations defined between different sets of data. Typically, you need to extract, transform, and load data into your RDBMS before you're able to manage it using SQL, which you can accomplish by using a tool like Stitch.

\begin{center}\rule{0.5\linewidth}{0.5pt}\end{center}

\hypertarget{relational-database}{%
\section{Relational Database}\label{relational-database}}

Imagine you're running a store and would like to record information about your customers and their orders. By using a relational database, you can save this information as two tables that represent two distinct entities: customers and orders .

\hypertarget{table-customers}{%
\subsection{Table Customers}\label{table-customers}}

\begin{Shaded}
\begin{Highlighting}[]
\FunctionTok{library}\NormalTok{(DT)}
\NormalTok{costomers}\OtherTok{\textless{}{-}}\FunctionTok{read.csv}\NormalTok{(}\StringTok{"data/customers.csv"}\NormalTok{)}
\FunctionTok{datatable}\NormalTok{(}\FunctionTok{head}\NormalTok{(costomers, }\DecValTok{5}\NormalTok{), }
          \AttributeTok{caption =}\NormalTok{ htmltools}\SpecialCharTok{::}\NormalTok{tags}\SpecialCharTok{$}\FunctionTok{caption}\NormalTok{(}
            \AttributeTok{style =} \StringTok{\textquotesingle{}caption{-}side: bottom; text{-}align: center;\textquotesingle{}}\NormalTok{, }
\NormalTok{            htmltools}\SpecialCharTok{::}\FunctionTok{em}\NormalTok{(}\StringTok{\textquotesingle{}Table 1: customers.\textquotesingle{}}\NormalTok{)),}
          \AttributeTok{options =} \FunctionTok{list}\NormalTok{(}\AttributeTok{dom =} \StringTok{\textquotesingle{}t\textquotesingle{}}\NormalTok{))}
\end{Highlighting}
\end{Shaded}

Table 1, informs about each customer is stored in its own row, with columns specifying different bits of information, including their first name, last name, and email address. Additionally, we associate a unique customer number, or primary key, with each customer record.

\hypertarget{table-orders}{%
\subsection{Table Orders}\label{table-orders}}

\begin{Shaded}
\begin{Highlighting}[]
\NormalTok{orders}\OtherTok{\textless{}{-}}\FunctionTok{read.csv}\NormalTok{(}\StringTok{"data/orders.csv"}\NormalTok{)}
\FunctionTok{datatable}\NormalTok{(}\FunctionTok{head}\NormalTok{(orders,}\DecValTok{5}\NormalTok{), }
          \AttributeTok{caption =}\NormalTok{ htmltools}\SpecialCharTok{::}\NormalTok{tags}\SpecialCharTok{$}\FunctionTok{caption}\NormalTok{(}
            \AttributeTok{style =} \StringTok{\textquotesingle{}caption{-}side: bottom; text{-}align: center;\textquotesingle{}}\NormalTok{, }
\NormalTok{            htmltools}\SpecialCharTok{::}\FunctionTok{em}\NormalTok{(}\StringTok{\textquotesingle{}Table 2: orders.\textquotesingle{}}\NormalTok{)),}
          \AttributeTok{options =} \FunctionTok{list}\NormalTok{(}\AttributeTok{dom =} \StringTok{\textquotesingle{}t\textquotesingle{}}\NormalTok{))}
\end{Highlighting}
\end{Shaded}

Again, Table 2 are contains information about a specific order. Each order has its own unique identification key \texttt{order\_id} for this table -- assigned to it as well.

\hypertarget{relational-model}{%
\section{Relational Model}\label{relational-model}}

You've probably noticed that these two examples share similar information. You can see these simple relations diagrammed below:

\begin{figure}

{\centering \includegraphics[width=1\linewidth]{./images/Bab5/relational} 

}

\caption{Relational Table}\label{fig:relational}
\end{figure}

Note that the orders table contains two keys: one for the order and one for the customer who placed that order. In scenarios when there are multiple keys in a table, the key that refers to the entity being described in that table is called the primary key \emph{(PK)} and other key is called a foreign key \emph{(FK)}.

In our example, \texttt{order\_id} is a primary key in the orders table, while \texttt{customer\_id} is both a primary key in the customers table and a foreign key in the orders table. Primary and foreign keys are essential to describing relations between the tables, and in performing SQL joins.

\begin{center}\rule{0.5\linewidth}{0.5pt}\end{center}

\hypertarget{factory-database}{%
\section{Factory Database}\label{factory-database}}

To make you more convenient about all the data tables that we will use in this section. Here, I summarize the following SQL relational for database \texttt{factory\_db}:

\begin{figure}

{\centering \includegraphics[width=1\linewidth]{./images/Bab5/fullrelational} 

}

\caption{Relational Table of Factory Database}\label{fig:fullrelational}
\end{figure}

\emph{Note:} Don't forget to consider the data structure of your database (all table)

\begin{center}\rule{0.5\linewidth}{0.5pt}\end{center}

\hypertarget{basic-sql-join-types}{%
\section{Basic SQL Join Types}\label{basic-sql-join-types}}

There are four basic types of SQL joins: inner, left, right, and full. The easiest and most intuitive way to explain the difference between these four types is by using a Venn diagram, which shows all possible logical relations between data sets.

Again, it's important to stress that before you can begin using any join type, you'll need to extract the data and load it into an RDBMS like Amazon Redshift, where you can query tables from multiple sources. You build that process manually, or you can use an ETL service like Stitch, which automates that process for you.

\begin{figure}

{\centering \includegraphics[width=1\linewidth]{./images/Bab5/jointable} 

}

\caption{Basic Join Table}\label{fig:jointable}
\end{figure}

\hypertarget{connect-to-mysql}{%
\section{Connect to MySQL}\label{connect-to-mysql}}

Reading data from \texttt{MySQL} into R workspace, it requires two R libraries, \texttt{RMySQL} and \texttt{DBI}. The connection data should not be embedded in analysis code. Separate the connection code in another script. The script should set up the connection and save it into the workspace.

The saved connection is accessible by its name in the analysis code. In the dbConnect function, you need to replace dbname, username, pwd, dbserver and port with the actual values of your remote database.

\begin{Shaded}
\begin{Highlighting}[]
\CommentTok{\# set up the connection and save it into the workspace}
\CommentTok{\#\textasciitilde{}\textasciitilde{}\textasciitilde{}\textasciitilde{}\textasciitilde{}\textasciitilde{}\textasciitilde{}\textasciitilde{}\textasciitilde{}\textasciitilde{}\textasciitilde{}\textasciitilde{}\textasciitilde{}\textasciitilde{}\textasciitilde{}\textasciitilde{}\textasciitilde{}\textasciitilde{}\textasciitilde{}\textasciitilde{}\textasciitilde{}\textasciitilde{}\textasciitilde{}\textasciitilde{}\textasciitilde{}\textasciitilde{}\textasciitilde{}\textasciitilde{}\textasciitilde{}\textasciitilde{}\textasciitilde{}\textasciitilde{}\textasciitilde{}\textasciitilde{}\textasciitilde{}\textasciitilde{}\textasciitilde{}\textasciitilde{}\textasciitilde{}\textasciitilde{}\textasciitilde{}\textasciitilde{}\textasciitilde{}\textasciitilde{}\textasciitilde{}\textasciitilde{}\textasciitilde{}\textasciitilde{}\textasciitilde{}\textasciitilde{}\textasciitilde{}\textasciitilde{}\textasciitilde{}\textasciitilde{}\textasciitilde{}\textasciitilde{}\textasciitilde{}\textasciitilde{}\textasciitilde{}\textasciitilde{}\textasciitilde{}\textasciitilde{}\textasciitilde{}\textasciitilde{}\textasciitilde{}\textasciitilde{}\textasciitilde{}\textasciitilde{}\textasciitilde{}\textasciitilde{}\textasciitilde{}\textasciitilde{}\textasciitilde{}\textasciitilde{}\textasciitilde{}\textasciitilde{}\textasciitilde{}\textasciitilde{}\textasciitilde{}}
\FunctionTok{library}\NormalTok{(RMySQL)}
\FunctionTok{library}\NormalTok{(DBI)}
\NormalTok{bakti }\OtherTok{\textless{}{-}} \FunctionTok{dbConnect}\NormalTok{(RMySQL}\SpecialCharTok{::}\FunctionTok{MySQL}\NormalTok{(),}
                  \AttributeTok{dbname=}\StringTok{\textquotesingle{}factory\_db\textquotesingle{}}\NormalTok{,}
                  \AttributeTok{username=}\StringTok{\textquotesingle{}root\textquotesingle{}}\NormalTok{,}
                  \AttributeTok{password=}\StringTok{\textquotesingle{}\textquotesingle{}}\NormalTok{, }
                  \AttributeTok{host=}\StringTok{\textquotesingle{}localhost\textquotesingle{}}\NormalTok{,}
                  \AttributeTok{port=}\DecValTok{3306}\NormalTok{)}
\NormalTok{knitr}\SpecialCharTok{::}\NormalTok{opts\_chunk}\SpecialCharTok{$}\FunctionTok{set}\NormalTok{(}\AttributeTok{connection =} \StringTok{"bakti"}\NormalTok{)   }\CommentTok{\# set up the connection }
\end{Highlighting}
\end{Shaded}

After set up the connection and save it into the workspace. Next, we can run SQL in a code chunk of type sql. By setting the connection in the code chuck and adding the option output.var, the resulting table from the SQL is written into a variable in R.

\begin{verbatim}
'''{sql connection=bakti, output.var="report_model_by_make"} 
      Your SQL code Here
'''
\end{verbatim}

\hypertarget{inner-join}{%
\section{Inner Join}\label{inner-join}}

Let's say we wanted to get a list of those customers who placed an order and the details of the order they placed. This would be a perfect fit for an inner join, since an inner join returns records at the intersection of the two tables.

\begin{Shaded}
\begin{Highlighting}[]
\NormalTok{SELECT OrderID, CustomerName}
\NormalTok{  FROM Orders O}
\NormalTok{    INNER JOIN Customers C}
\NormalTok{      ON O.CustomerID }\OtherTok{=}\NormalTok{ C.CustomerID}
\end{Highlighting}
\end{Shaded}

\begin{Shaded}
\begin{Highlighting}[]
\FunctionTok{library}\NormalTok{(DT)}
\FunctionTok{datatable}\NormalTok{(Inner1, }
          \AttributeTok{caption =}\NormalTok{ htmltools}\SpecialCharTok{::}\NormalTok{tags}\SpecialCharTok{$}\FunctionTok{caption}\NormalTok{(}
            \AttributeTok{style =} \StringTok{\textquotesingle{}caption{-}side: bottom; text{-}align: center;\textquotesingle{}}\NormalTok{, }
\NormalTok{            htmltools}\SpecialCharTok{::}\FunctionTok{em}\NormalTok{(}\StringTok{\textquotesingle{}Table 3: SQL Inner Join Two Tables.\textquotesingle{}}\NormalTok{)))}
\end{Highlighting}
\end{Shaded}

The following SQL statement selects all orders with customer and shipper information:

\begin{Shaded}
\begin{Highlighting}[]
\NormalTok{SELECT }\SpecialCharTok{*} 
  \FunctionTok{FROM}\NormalTok{ ((Orders O}
\NormalTok{    INNER JOIN Customers C}
\NormalTok{      ON }\AttributeTok{O.CustomerID =}\NormalTok{ C.CustomerID)}
\NormalTok{    INNER JOIN Shippers S }
\NormalTok{      ON }\AttributeTok{O.ShipperID =}\NormalTok{ S.ShipperID)}
\end{Highlighting}
\end{Shaded}

\begin{Shaded}
\begin{Highlighting}[]
\FunctionTok{datatable}\NormalTok{(Inner2, }
          \AttributeTok{caption =}\NormalTok{ htmltools}\SpecialCharTok{::}\NormalTok{tags}\SpecialCharTok{$}\FunctionTok{caption}\NormalTok{(}
            \AttributeTok{style =} \StringTok{\textquotesingle{}caption{-}side: bottom; text{-}align: center;\textquotesingle{}}\NormalTok{, }
\NormalTok{            htmltools}\SpecialCharTok{::}\FunctionTok{em}\NormalTok{(}\StringTok{\textquotesingle{}Table 4: SQL Inner Join Three Tables.\textquotesingle{}}\NormalTok{)),}
          \AttributeTok{extensions =} \StringTok{\textquotesingle{}FixedColumns\textquotesingle{}}\NormalTok{,}
          \AttributeTok{options =} \FunctionTok{list}\NormalTok{(}\AttributeTok{scrollX =} \ConstantTok{TRUE}\NormalTok{, }\AttributeTok{fixedColumns =} \ConstantTok{TRUE}\NormalTok{)}
\NormalTok{          )}
\end{Highlighting}
\end{Shaded}

\hypertarget{left-join}{%
\section{Left Join}\label{left-join}}

If we wanted to simply append information about orders to our customers table, regardless of whether a customer placed an order or not, we would use a left join. A left join returns all records from table A and any matching records from table B. The result is NULL from the right side, if there is no match.

\begin{Shaded}
\begin{Highlighting}[]
\NormalTok{SELECT CustomerName, OrderID}
\NormalTok{  FROM Customers C}
\NormalTok{    LEFT JOIN Orders O}
\NormalTok{      ON C.CustomerID }\OtherTok{=}\NormalTok{ O.CustomerID}
\NormalTok{        ORDER BY C.CustomerName}
\end{Highlighting}
\end{Shaded}

\begin{Shaded}
\begin{Highlighting}[]
\FunctionTok{datatable}\NormalTok{(Left, }
          \AttributeTok{caption =}\NormalTok{ htmltools}\SpecialCharTok{::}\NormalTok{tags}\SpecialCharTok{$}\FunctionTok{caption}\NormalTok{(}
            \AttributeTok{style =} \StringTok{\textquotesingle{}caption{-}side: bottom; text{-}align: center;\textquotesingle{}}\NormalTok{, }
\NormalTok{            htmltools}\SpecialCharTok{::}\FunctionTok{em}\NormalTok{(}\StringTok{\textquotesingle{}Table 5: SQL Left Join Two Tables.\textquotesingle{}}\NormalTok{)))}
\end{Highlighting}
\end{Shaded}

\hypertarget{right-join}{%
\section{Right Join}\label{right-join}}

The following SQL statement will return all employees, and any orders they might have placed. The result is NULL from the left side, when there is no match.

\begin{Shaded}
\begin{Highlighting}[]
\NormalTok{SELECT OrderID, LastName, FirstName}
\NormalTok{  FROM Orders O}
\NormalTok{    RIGHT JOIN Employees E}
\NormalTok{      ON O.EmployeeID }\OtherTok{=}\NormalTok{ E.EmployeeID}
\NormalTok{        ORDER BY O.OrderID}
\end{Highlighting}
\end{Shaded}

\begin{Shaded}
\begin{Highlighting}[]
\FunctionTok{datatable}\NormalTok{(Right, }
          \AttributeTok{caption =}\NormalTok{ htmltools}\SpecialCharTok{::}\NormalTok{tags}\SpecialCharTok{$}\FunctionTok{caption}\NormalTok{(}
            \AttributeTok{style =} \StringTok{\textquotesingle{}caption{-}side: bottom; text{-}align: center;\textquotesingle{}}\NormalTok{, }
\NormalTok{            htmltools}\SpecialCharTok{::}\FunctionTok{em}\NormalTok{(}\StringTok{\textquotesingle{}Table 6: SQL Right Join Two Tables.\textquotesingle{}}\NormalTok{)))}
\end{Highlighting}
\end{Shaded}

\hypertarget{full-join}{%
\section{Full Join}\label{full-join}}

The FULL OUTER JOIN keyword returns all records when there is a match in left (table1) or right (table2) table records. FULL OUTER JOIN, FULL JOIN, and JOIN (MariaDB) are the same. The following SQL statement selects all customers, and all orders:

\begin{Shaded}
\begin{Highlighting}[]
\NormalTok{SELECT CustomerName, OrderID}
\NormalTok{  FROM Customers C}
\NormalTok{    JOIN Orders O}
\NormalTok{      ON C.CustomerID}\OtherTok{=}\NormalTok{O.CustomerID}
\NormalTok{        ORDER BY C.CustomerName}
\end{Highlighting}
\end{Shaded}

\begin{Shaded}
\begin{Highlighting}[]
\FunctionTok{datatable}\NormalTok{(Full, }
          \AttributeTok{caption =}\NormalTok{ htmltools}\SpecialCharTok{::}\NormalTok{tags}\SpecialCharTok{$}\FunctionTok{caption}\NormalTok{(}
            \AttributeTok{style =} \StringTok{\textquotesingle{}caption{-}side: bottom; text{-}align: center;\textquotesingle{}}\NormalTok{, }
\NormalTok{            htmltools}\SpecialCharTok{::}\FunctionTok{em}\NormalTok{(}\StringTok{\textquotesingle{}Table 7: SQL Full Join Two Tables.\textquotesingle{}}\NormalTok{)))}
\end{Highlighting}
\end{Shaded}

\hypertarget{self-join}{%
\section{Self JOIN}\label{self-join}}

A self JOIN is a regular join, but the table is joined with itself. The following SQL statement matches customers that are from the same city:

\begin{Shaded}
\begin{Highlighting}[]
\NormalTok{SELECT A.CustomerName AS CustomerName1, }
\NormalTok{       B.CustomerName AS CustomerName2, }
\NormalTok{       A.City}
\NormalTok{  FROM Customers A, }
\NormalTok{       Customers B}
\NormalTok{    WHERE A.CustomerID }\SpecialCharTok{\textless{}}\ErrorTok{\textgreater{}}\NormalTok{ B.CustomerID}
\NormalTok{      AND A.City }\OtherTok{=}\NormalTok{ B.City}
\NormalTok{        ORDER BY A.City}
\end{Highlighting}
\end{Shaded}

\begin{Shaded}
\begin{Highlighting}[]
\FunctionTok{datatable}\NormalTok{(Self, }
          \AttributeTok{caption =}\NormalTok{ htmltools}\SpecialCharTok{::}\NormalTok{tags}\SpecialCharTok{$}\FunctionTok{caption}\NormalTok{(}
            \AttributeTok{style =} \StringTok{\textquotesingle{}caption{-}side: bottom; text{-}align: center;\textquotesingle{}}\NormalTok{, }
\NormalTok{            htmltools}\SpecialCharTok{::}\FunctionTok{em}\NormalTok{(}\StringTok{\textquotesingle{}Table 8: SQL Self Join Two Tables.\textquotesingle{}}\NormalTok{)))}
\end{Highlighting}
\end{Shaded}

After finishing the work with the database, close the connection.

\begin{Shaded}
\begin{Highlighting}[]
\NormalTok{DBI}\SpecialCharTok{::}\FunctionTok{dbDisconnect}\NormalTok{(bakti)}
\end{Highlighting}
\end{Shaded}

\hypertarget{your-job-2}{%
\section{Your Job}\label{your-job-2}}

\begin{enumerate}
\def\labelenumi{\arabic{enumi}.}
\item
  Apply Left join and Right join to returns all records from table Orders and any matching records from table Suppliers.
\item
  Choose the correct JOIN clause to select all records from the two tables (Orders and Suppliers) where there is a match in both tables.
\item
  Choose the correct JOIN clause to select all the records from the Suppliers table plus all the matches in the Orders table.
\end{enumerate}

\hypertarget{simple-query}{%
\chapter{Simple Query}\label{simple-query}}

The real power of a relational database lies in its ability to quickly retrieve and analyze your data by running a query. Queries allow you to pull information from one or more tables based on a set of search conditions you define. In this section, you will learn how to create a simple one-table query.

First, we need to connect to our database. Please type the following code in your R console:

\begin{Shaded}
\begin{Highlighting}[]
\CommentTok{\# set up the connection and save it into the workspace}
\CommentTok{\#\textasciitilde{}\textasciitilde{}\textasciitilde{}\textasciitilde{}\textasciitilde{}\textasciitilde{}\textasciitilde{}\textasciitilde{}\textasciitilde{}\textasciitilde{}\textasciitilde{}\textasciitilde{}\textasciitilde{}\textasciitilde{}\textasciitilde{}\textasciitilde{}\textasciitilde{}\textasciitilde{}\textasciitilde{}\textasciitilde{}\textasciitilde{}\textasciitilde{}\textasciitilde{}\textasciitilde{}\textasciitilde{}\textasciitilde{}\textasciitilde{}\textasciitilde{}\textasciitilde{}\textasciitilde{}\textasciitilde{}\textasciitilde{}\textasciitilde{}\textasciitilde{}\textasciitilde{}\textasciitilde{}\textasciitilde{}\textasciitilde{}\textasciitilde{}\textasciitilde{}\textasciitilde{}\textasciitilde{}\textasciitilde{}\textasciitilde{}\textasciitilde{}\textasciitilde{}\textasciitilde{}\textasciitilde{}\textasciitilde{}\textasciitilde{}\textasciitilde{}\textasciitilde{}\textasciitilde{}\textasciitilde{}\textasciitilde{}\textasciitilde{}\textasciitilde{}\textasciitilde{}\textasciitilde{}\textasciitilde{}\textasciitilde{}\textasciitilde{}\textasciitilde{}\textasciitilde{}\textasciitilde{}\textasciitilde{}\textasciitilde{}\textasciitilde{}\textasciitilde{}\textasciitilde{}\textasciitilde{}\textasciitilde{}\textasciitilde{}\textasciitilde{}\textasciitilde{}\textasciitilde{}\textasciitilde{}\textasciitilde{}\textasciitilde{}}
\FunctionTok{library}\NormalTok{(RMySQL)}
\FunctionTok{library}\NormalTok{(DBI)}
\NormalTok{bakti }\OtherTok{\textless{}{-}} \FunctionTok{dbConnect}\NormalTok{(RMySQL}\SpecialCharTok{::}\FunctionTok{MySQL}\NormalTok{(),}
                  \AttributeTok{dbname=}\StringTok{\textquotesingle{}factory\_db\textquotesingle{}}\NormalTok{,}
                  \AttributeTok{username=}\StringTok{\textquotesingle{}root\textquotesingle{}}\NormalTok{,}
                  \AttributeTok{password=}\StringTok{\textquotesingle{}\textquotesingle{}}\NormalTok{, }
                  \AttributeTok{host=}\StringTok{\textquotesingle{}localhost\textquotesingle{}}\NormalTok{,}
                  \AttributeTok{port=}\DecValTok{3306}\NormalTok{)}
\NormalTok{knitr}\SpecialCharTok{::}\NormalTok{opts\_chunk}\SpecialCharTok{$}\FunctionTok{set}\NormalTok{(}\AttributeTok{connection =} \StringTok{"bakti"}\NormalTok{) }\CommentTok{\# to set up the connection in your Rmarkdown chunk}
\end{Highlighting}
\end{Shaded}

\hypertarget{select-1}{%
\section{SELECT}\label{select-1}}

The SQL SELECT statement is used to fetch the data from a database table which returns this data in the form of a result table. These result tables are called result-sets. The basic syntax of the SELECT statement is as follows:

\begin{verbatim}
SELECT column1, column2, columnN 
  FROM table_name;
\end{verbatim}

Assume, column1, column2\ldots{} are the fields of a table whose values you want to fetch. If you want to fetch some of the fields available in the field, then you can use the following syntax.

\begin{verbatim}
SELECT CustomerName, Address, City, Country 
  FROM CUSTOMERS;
\end{verbatim}

If you want to fetch all the fields of the CUSTOMERS table, then you should use the following query.

\begin{verbatim}
SELECT *
  FROM CUSTOMERS;
\end{verbatim}

\hypertarget{distinct}{%
\section{DISTINCT}\label{distinct}}

The SQL DISTINCT keyword is used in conjunction with the SELECT statement to eliminate all the duplicate records and fetching only unique records. There may be a situation when you have multiple duplicate records in a table. While fetching such records, it makes more sense to fetch only those unique records instead of fetching duplicate records.

The basic syntax of DISTINCT keyword to eliminate the duplicate records is as follows:

\begin{verbatim}
SELECT DISTINCT column_name
  FROM table_name
\end{verbatim}

Now, let us use the DISTINCT keyword with the above SELECT query and then see the result.

\begin{verbatim}
SELECT DISTINCT Country   
  FROM customers;
\end{verbatim}

\hypertarget{where-1}{%
\section{WHERE}\label{where-1}}

The WHERE clause is used to filter records. The WHERE clause is used to extract only those records that fulfill a specified condition.

\begin{verbatim}
SELECT column1, column2, ...
  FROM table_name
    WHERE [condition];
\end{verbatim}

\emph{Note:} The WHERE clause is not only used in SELECT statement, it is also used in UPDATE, DELETE statement, etc.!

The following SQL statement selects all the customers from the country ``Mexico'', in the ``Customers'' table:

\begin{verbatim}
SELECT * 
  FROM Customers
    WHERE Country='Mexico';
\end{verbatim}

The following operators can be used in the WHERE clause, please try it by your self!

\begin{longtable}[]{@{}
  >{\raggedright\arraybackslash}p{(\columnwidth - 2\tabcolsep) * \real{0.3636}}
  >{\raggedright\arraybackslash}p{(\columnwidth - 2\tabcolsep) * \real{0.6364}}@{}}
\toprule\noalign{}
\begin{minipage}[b]{\linewidth}\raggedright
Operator
\end{minipage} & \begin{minipage}[b]{\linewidth}\raggedright
Description
\end{minipage} \\
\midrule\noalign{}
\endhead
\bottomrule\noalign{}
\endlastfoot
= & Equal \\
\textgreater{} & Greater than \\
\textless{} & Less than \\
\textgreater= & Greater than or equal \\
\textless= & Less than or equal \\
\textless\textgreater{} & Not equal. Note: In some versions of SQL this operator may be written as != \\
IS NULL or IS NOT NUL & A field with a NULL value is a field with no value. \\
BETWEEN & Between a certain range \\
LIKE & Search for a pattern \\
IN & To specify multiple possible values for a column \\
\end{longtable}

\hypertarget{between}{%
\section{BETWEEN}\label{between}}

The BETWEEN operator selects values within a given range. The values can be numbers, text, or dates. The BETWEEN operator is inclusive: begin and end values are included.

\begin{verbatim}
SELECT column_name(s)
  FROM table_name
    WHERE column_name 
      BETWEEN value1 AND value2;
\end{verbatim}

The following SQL statement selects all products with a price BETWEEN 10 and 20:

\begin{verbatim}
SELECT * 
  FROM Products
    WHERE Price 
      BETWEEN 10 AND 20;
\end{verbatim}

The following SQL statement selects all orders with an OrderDate BETWEEN `01-July-1996' and `31-July-1996':

\begin{verbatim}
SELECT * 
  FROM Orders
    WHERE OrderDate 
      BETWEEN '1996-07-01' AND '1996-07-31';
\end{verbatim}

\hypertarget{in}{%
\section{IN}\label{in}}

The IN operator allows you to specify multiple values in a WHERE clause. The IN operator is a shorthand for multiple OR conditions.

\begin{verbatim}
SELECT column_name(s)
  FROM table_name
    WHERE column_name 
      IN (SELECT STATEMENT);
\end{verbatim}

The following SQL statement selects all customers that are located in ``Germany'', ``France'' or ``UK'':

\begin{verbatim}
SELECT * 
  FROM Customers
    WHERE Country 
      IN ('Germany', 'France', 'UK');
\end{verbatim}

The following SQL statement selects all customers that are from the same countries as the suppliers:

\begin{verbatim}
SELECT * 
  FROM Customers
    WHERE Country 
      IN (SELECT Country FROM Suppliers);
\end{verbatim}

\hypertarget{like}{%
\section{LIKE}\label{like}}

The LIKE operator is used in a WHERE clause to search for a specified pattern in a column. There are two wildcards often used in conjunction with the LIKE operator:

\begin{verbatim}
SELECT column1, column2, ...
  FROM table_name
    WHERE columnN LIKE pattern;
\end{verbatim}

\begin{itemize}
\tightlist
\item
  \% : The percent sign represents zero, one, or multiple characters
\item
  \_ : The underscore represents a single character
\end{itemize}

The following SQL statement selects all customers with a CustomerName starting with ``a'':

\begin{verbatim}
SELECT * 
  FROM Customers
    WHERE CustomerName 
      LIKE 'a%';
\end{verbatim}

Here are some examples showing different LIKE operators with `\%' and '\_' wildcards:

\begin{longtable}[]{@{}
  >{\raggedright\arraybackslash}p{(\columnwidth - 2\tabcolsep) * \real{0.5200}}
  >{\raggedright\arraybackslash}p{(\columnwidth - 2\tabcolsep) * \real{0.4800}}@{}}
\toprule\noalign{}
\begin{minipage}[b]{\linewidth}\raggedright
LIKE Operator
\end{minipage} & \begin{minipage}[b]{\linewidth}\raggedright
Description
\end{minipage} \\
\midrule\noalign{}
\endhead
\bottomrule\noalign{}
\endlastfoot
WHERE CustomerName LIKE `a\%' & Finds any values that start with ``a'' \\
WHERE CustomerName LIKE `\%a' & Finds any values that end with ``a'' \\
WHERE CustomerName LIKE `\%or\%' & Finds any values that have ``or'' in any position \\
WHERE CustomerName LIKE '\_r\%' & Finds any values that have ``r'' in the second position \\
WHERE CustomerName LIKE `a\_\%' & Finds any values that start with ``a'' and are at least 2 characters in length \\
WHERE CustomerName LIKE 'a\_\_\%' & Finds any values that start with ``a'' and are at least 3 characters in length \\
WHERE ContactName LIKE `a\%o' & \\
\end{longtable}

\hypertarget{and-or-and-not}{%
\section{AND, OR and NOT}\label{and-or-and-not}}

The WHERE clause can be combined with AND, OR, and NOT operators. The AND and OR operators are used to filter records based on more than one condition:

\begin{itemize}
\tightlist
\item
  The AND operator displays a record if all the conditions separated by AND are TRUE.
\item
  The OR operator displays a record if any of the conditions separated by OR is TRUE.
\item
  The NOT operator displays a record if the condition(s) is NOT TRUE.
\end{itemize}

\begin{verbatim}
SELECT column1, column2, ...
  FROM table_name
    WHERE condition1 AND condition2 OR condition3 NOT condition4;
\end{verbatim}

The following SQL statement selects all fields from ``Customers'' where country is ``Germany'' AND city must be ``Berlin'' OR ``München'' (use parenthesis to form complex expressions):

\begin{verbatim}
SELECT * FROM Customers
WHERE Country='Germany' AND (City='Berlin' OR City='München');
\end{verbatim}

Let see one more example, the following SQL statement selects all fields from ``Customers'' where country is NOT ``Germany'' and NOT ``USA'':

\begin{verbatim}
SELECT * FROM Customers
WHERE Country='Germany' AND (City='Berlin' OR City='München');
\end{verbatim}

\hypertarget{order-by}{%
\section{ORDER BY}\label{order-by}}

The SQL ORDER BY clause is used to sort the data in ascending or descending order, based on one or more columns. Some databases sort the query results in an ascending order by default. The basic syntax of the ORDER BY clause is as follows:

\begin{verbatim}
SELECT column-list 
  FROM table_name 
    [WHERE condition] 
      [ORDER BY column1, column2, .. columnN] [ASC | DESC];
\end{verbatim}

\begin{itemize}
\tightlist
\item
  By default ORDER BY sorts the data in ascending order.
\item
  We can use the keyword DESC to sort the data in descending order and the keyword ASC to sort in ascending order.
\end{itemize}

You can use more than one column in the ORDER BY clause. Make sure whatever column you are using to sort that column should be in the column-list.The following code block has an example, which would sort the result in an ascending order by the City and the Country:

\begin{verbatim}
SELECT * FROM Customers
WHERE Country='Germany' AND (City='Berlin' OR City='München')
  ORDER BY Country, City;
\end{verbatim}

\hypertarget{limit}{%
\section{LIMIT}\label{limit}}

If there are a large number of tuples satisfying the query conditions, it might be resourceful to view only a handful of them at a time.

\begin{itemize}
\tightlist
\item
  The LIMIT clause is used to set an upper limit on the number of tuples returned by SQL.
\item
  It is important to note that this clause is not supported by all SQL versions.
\item
  The LIMIT clause can also be specified using the SQL 2008 \href{https://www.geeksforgeeks.org/sql-offset-fetch-clause/}{OFFSET/FETCH FIRST} clauses.
\item
  The limit/offset expressions must be a non-negative integer.
\end{itemize}

\begin{verbatim}
SELECT column-list 
  FROM table_name 
    [WHERE condition] 
      [ORDER BY column1, column2, .. columnN] [ASC | DESC]
        LIMIT rows_to_skip, next_rows_to_skip;
          
\end{verbatim}

The following illustrates the LIMIT clauses to collect TOP 3 rows:

\begin{verbatim}
SELECT * FROM Customers
WHERE Country='Germany' AND (City='Berlin' OR City='München')
  ORDER BY Country, City
      LIMIT 3;
\end{verbatim}

Next, the following illustrates the LIMIT clauses to collect TOP 5 rows after TOP 3 rows:

\begin{verbatim}
SELECT CustomerName, Address, City, Country 
  FROM customers
    ORDER BY City, Country DESC
      LIMIT 3, 5;
\end{verbatim}

\hypertarget{min-and-max}{%
\section{MIN and MAX}\label{min-and-max}}

The MIN() function returns the smallest value of the selected column. The MAX() function returns the largest value of the selected column.

\begin{verbatim}
SELECT MIN/MAX(column_name)
  FROM table_name
    WHERE condition;
\end{verbatim}

The following SQL statement finds the price of the cheapest product:

\begin{verbatim}
SELECT MIN(Price) AS SmallestPrice
  FROM Products;
\end{verbatim}

The following SQL statement finds the price of the most expensive product:

\begin{verbatim}
SELECT MAX(Price) AS LargestPrice
  FROM Products;
\end{verbatim}

\hypertarget{count-sum-and-avg}{%
\section{COUNT, SUM, and AVG}\label{count-sum-and-avg}}

The COUNT() function returns the number of rows that matches a specified criterion. The AVG() function returns the average value of a numeric column. The SUM() function returns the total sum of a numeric column.

\begin{verbatim}
SELECT COUNT/SUM/AVG(column_name)
FROM table_name
WHERE condition;
\end{verbatim}

The following SQL statement finds the average price of all products:

\begin{verbatim}
SELECT AVG(Price)
  FROM Products;
\end{verbatim}

Note: Please try other functions, to get more convenient with SQL!

\hypertarget{having}{%
\section{HAVING}\label{having}}

The HAVING clause was added to SQL because the WHERE keyword could not be used with aggregate functions.

\begin{verbatim}
SELECT column_name(s)
  FROM table_name
    WHERE condition
      GROUP BY column_name(s)
        HAVING condition
          ORDER BY column_name(s);
\end{verbatim}

The following SQL statement lists the number of customers in each country, sorted high to low (Only include countries with more than 5 customers):

\begin{verbatim}
SELECT COUNT(CustomerID), Country
  FROM Customers
    GROUP BY Country
      HAVING COUNT(CustomerID) > 5
        ORDER BY COUNT(CustomerID) DESC;
\end{verbatim}

\hypertarget{case}{%
\section{CASE}\label{case}}

The CASE statement goes through conditions and returns a value when the first condition is met (like an IF-THEN-ELSE statement). So, once a condition is true, it will stop reading and return the result. If no conditions are true, it returns the value in the ELSE clause.

If there is no ELSE part and no conditions are true, it returns NULL.

\begin{verbatim}
CASE
    WHEN condition1 THEN result1
    WHEN condition2 THEN result2
    WHEN conditionN THEN resultN
    ELSE result
END;
\end{verbatim}

The following SQL goes through conditions and returns a value when the first condition is met:

\begin{verbatim}
SELECT OrderID, Quantity,
CASE
    WHEN Quantity > 30 THEN 'The quantity is greater than 30'
    WHEN Quantity = 30 THEN 'The quantity is 30'
    ELSE 'The quantity is under 30'
END AS QuantityText
FROM OrderDetails;
\end{verbatim}

The following SQL will order the customers by City. However, if City is NULL, then order by Country:

\begin{verbatim}
SELECT CustomerName, City, Country
FROM Customers
ORDER BY
(CASE
    WHEN City IS NULL THEN Country
    ELSE City
END);
\end{verbatim}

\hypertarget{your-job-3}{%
\section{Your Job}\label{your-job-3}}

\begin{itemize}
\tightlist
\item
  Select Some attributes of suppliers in alphabetical order!
\item
  Some attributes of suppliers in reverse alphabetical order!
\item
  Some attributes of suppliers ordered by country, then by city!
\item
  All atributes of suppliers and reverse alphabetical ordered by country, then by city!
\item
  All orders, sorted by total amount, the largest first!
\item
  Get all but the 10 most expensive products sorted by price!
\item
  Get the 10th to 15th most expensive products sorted by price!
\item
  List all supplier countries in alphabetical order!
\item
  Find the cheapest product and Expensive Orders!
\item
  Find the number of Supplier USA!
\item
  Compute the total Quantity of orderitem!
\item
  Compute the average UnitPrice of all product!
\item
  Get all information about customer named Thomas Hardy!
\item
  List all customers from Spain or France!
\item
  List all customers that are not from the USA!
\item
  List all orders that not between \$50 and \$15000!
\item
  List all products between \$10 and \$20
\item
  List all products not between \$10 and \$100 sorted by price!
\item
  Get the list of orders and amount sold between 1996 Jan 01 and 1996 Des 31!
\item
  List all suppliers from the USA, UK, OR Japan!
\item
  List all products that are not exactly \$10, \$20, \$30, \$40, or \$50!
\item
  List all customers that are from the same countries as the suppliers!
\item
  List all products that start with `Cha' or `Chan' and have one more character!
\item
  List all suppliers that do have a fax number!
\item
  List all customer with average orders between \$1000 and \$1200 !
\item
  List total customers in each country.
\item
  Display results with easy to understand column headers.
\item
  Measure the average order of product names from each country and order it from max to min.
\item
  Compare the average order of product names from each country in the year 1996 vs 1997 order it from max to min.
\end{itemize}

\hypertarget{join-table-queries}{%
\chapter{Join Table Queries}\label{join-table-queries}}

In the previous section, you learned how to create a simple query with one table. Most queries you design in Access will likely use multiple tables, allowing you to answer more complex questions. In this lesson, you'll learn how to design and create a multi-table query.

Queries can be difficult to understand and build if you don't have a good idea of what you're trying to find and how to find it. A one-table query can be simple enough to make up as you go along, but to build anything more powerful you'll need to plan the query in advance.

\hypertarget{planning-a-query-in-sql}{%
\section{Planning a Query in SQL}\label{planning-a-query-in-sql}}

Planning a query in SQL involves several important steps to ensure that you retrieve the desired data efficiently and accurately. Here's a step-by-step guide to planning and executing a successful SQL query:

\begin{itemize}
\tightlist
\item
  \textbf{Understand Requirements}
\end{itemize}

\begin{quote}
Clearly define the purpose of your query. Understand what specific data you need and what conditions or criteria need to be met. If you're unsure, discuss the requirements with stakeholders or team members.
\end{quote}

\begin{itemize}
\tightlist
\item
  \textbf{Select the Right Table(s)}
\end{itemize}

\begin{quote}
Identify the table(s) that contain the relevant data you need. Make sure you understand the table structure, column names, and relationships between tables (if applicable).
\end{quote}

\begin{itemize}
\tightlist
\item
  \textbf{Choose Columns}
\end{itemize}

\begin{quote}
Determine the columns you need in the query result. Select only the columns that are necessary to fulfill the query requirements. This reduces the amount of data retrieved and improves performance.
\end{quote}

\begin{itemize}
\tightlist
\item
  \textbf{Apply Filters (WHERE Clause)}
\end{itemize}

\begin{quote}
Use the WHERE clause to filter the rows that meet specific criteria. This helps narrow down the dataset and retrieves only the relevant records. Be cautious not to use overly complex conditions that might slow down the query.
\end{quote}

\begin{itemize}
\tightlist
\item
  \textbf{Sort Results (ORDER BY Clause, if needed)}
\end{itemize}

\begin{quote}
If you need the results in a specific order, use the ORDER BY clause to sort the output based on one or more columns. Sorting can impact performance, so use it judiciously.
\end{quote}

\begin{itemize}
\tightlist
\item
  \textbf{Aggregate Data (GROUP BY and HAVING Clauses, if needed)}
\end{itemize}

\begin{quote}
If you need to perform aggregate calculations (e.g., SUM, AVG, COUNT), use the GROUP BY clause to group data based on certain columns. You can also use the HAVING clause to filter groups based on aggregate conditions.
\end{quote}

\begin{itemize}
\tightlist
\item
  \textbf{Join Tables (if needed)}
\end{itemize}

\begin{quote}
If your query requires data from multiple tables, use the appropriate join operations (INNER JOIN, LEFT JOIN, etc.) to combine data based on related columns. Make sure you understand the relationships and select the appropriate join type.
\end{quote}

\begin{itemize}
\tightlist
\item
  \textbf{Optimize Performance}
\end{itemize}

\begin{quote}
Consider the performance implications of your query. Avoid using unnecessary subqueries or functions that could slow down execution. Use indexes on columns that are frequently used for filtering or joining.
\end{quote}

\begin{itemize}
\tightlist
\item
  \textbf{Test the Query}
\end{itemize}

\begin{quote}
Before executing the query in a production environment, test it in a safe environment (e.g., a development or testing database). Verify that the query returns the expected results and that the performance is acceptable.
\end{quote}

\begin{itemize}
\tightlist
\item
  \textbf{Backup Data (if applicable)}
\end{itemize}

\begin{quote}
If your query involves updating or deleting data, create a backup of the relevant tables before making any changes. This helps prevent accidental data loss.
\end{quote}

\begin{itemize}
\tightlist
\item
  \textbf{Execute and Review}
\end{itemize}

\begin{quote}
Once you're confident in your query, execute it in the production environment if necessary. Review the results to ensure they match your expectations.
\end{quote}

\begin{itemize}
\tightlist
\item
  \textbf{Monitor and Optimize}
\end{itemize}

\begin{quote}
After executing the query, monitor its performance in the production environment. Use tools like query execution plans to identify bottlenecks and optimize as needed.
\end{quote}

\begin{itemize}
\tightlist
\item
  \textbf{Document the Query}
\end{itemize}

\begin{quote}
Document the query, including its purpose, the tables involved, the filters applied, and any other relevant details. This documentation can be helpful for future reference and troubleshooting.
\end{quote}

By following these steps, you can plan and execute SQL queries effectively, ensuring that you retrieve accurate results in an efficient manner.

\hypertarget{union}{%
\section{UNION}\label{union}}

The UNION operator is used to combine the result-set of two or more SELECT statements.

\begin{itemize}
\tightlist
\item
  Each SELECT statement within UNION must have the same number of columns
\item
  The columns must also have similar data types
\item
  The columns in each SELECT statement must also be in the same order
\end{itemize}

\begin{verbatim}
SELECT column_name(s) 
  FROM table1
    UNION
      SELECT column_name(s) 
        FROM table2;
\end{verbatim}

Before we begin, first we need to connect to our database. Please type the following code in your R console:

\begin{Shaded}
\begin{Highlighting}[]
\CommentTok{\# set up the connection and save it into the workspace}
\CommentTok{\#\textasciitilde{}\textasciitilde{}\textasciitilde{}\textasciitilde{}\textasciitilde{}\textasciitilde{}\textasciitilde{}\textasciitilde{}\textasciitilde{}\textasciitilde{}\textasciitilde{}\textasciitilde{}\textasciitilde{}\textasciitilde{}\textasciitilde{}\textasciitilde{}\textasciitilde{}\textasciitilde{}\textasciitilde{}\textasciitilde{}\textasciitilde{}\textasciitilde{}\textasciitilde{}\textasciitilde{}\textasciitilde{}\textasciitilde{}\textasciitilde{}\textasciitilde{}\textasciitilde{}\textasciitilde{}\textasciitilde{}\textasciitilde{}\textasciitilde{}\textasciitilde{}\textasciitilde{}\textasciitilde{}\textasciitilde{}\textasciitilde{}\textasciitilde{}\textasciitilde{}\textasciitilde{}\textasciitilde{}\textasciitilde{}\textasciitilde{}\textasciitilde{}\textasciitilde{}\textasciitilde{}\textasciitilde{}\textasciitilde{}\textasciitilde{}\textasciitilde{}\textasciitilde{}\textasciitilde{}\textasciitilde{}\textasciitilde{}\textasciitilde{}\textasciitilde{}\textasciitilde{}\textasciitilde{}\textasciitilde{}\textasciitilde{}\textasciitilde{}\textasciitilde{}\textasciitilde{}\textasciitilde{}\textasciitilde{}\textasciitilde{}\textasciitilde{}\textasciitilde{}\textasciitilde{}\textasciitilde{}\textasciitilde{}\textasciitilde{}\textasciitilde{}\textasciitilde{}\textasciitilde{}\textasciitilde{}\textasciitilde{}\textasciitilde{}}
\FunctionTok{library}\NormalTok{(RMySQL)}
\FunctionTok{library}\NormalTok{(DBI)}
\NormalTok{bakti }\OtherTok{\textless{}{-}} \FunctionTok{dbConnect}\NormalTok{(RMySQL}\SpecialCharTok{::}\FunctionTok{MySQL}\NormalTok{(),}
                  \AttributeTok{dbname=}\StringTok{\textquotesingle{}factory\_db\textquotesingle{}}\NormalTok{,}
                  \AttributeTok{username=}\StringTok{\textquotesingle{}root\textquotesingle{}}\NormalTok{,}
                  \AttributeTok{password=}\StringTok{\textquotesingle{}\textquotesingle{}}\NormalTok{, }
                  \AttributeTok{host=}\StringTok{\textquotesingle{}localhost\textquotesingle{}}\NormalTok{,}
                  \AttributeTok{port=}\DecValTok{3306}\NormalTok{)}
\NormalTok{knitr}\SpecialCharTok{::}\NormalTok{opts\_chunk}\SpecialCharTok{$}\FunctionTok{set}\NormalTok{(}\AttributeTok{connection =} \StringTok{"bakti"}\NormalTok{) }\CommentTok{\# to set up the connection in your Rmarkdown chunk}
\end{Highlighting}
\end{Shaded}

The following SQL statement returns the cities (only distinct values) from both the ``Customers'' and the ``Suppliers'' table:

\begin{verbatim}
SELECT City 
  FROM Customers
    UNION
      SELECT City 
        FROM Suppliers
          ORDER BY City;
\end{verbatim}

\emph{Note:} If some customers or suppliers have the same city, each city will only be listed once, because UNION selects only distinct values. Use UNION ALL to also select duplicate values!

\hypertarget{union-all}{%
\subsection{UNION ALL}\label{union-all}}

The following SQL statement returns the cities (duplicate values also) from both the ``Customers'' and the ``Suppliers'' table:

\begin{verbatim}
SELECT City 
  FROM Customers
    UNION ALL
      SELECT City 
        FROM Suppliers
          ORDER BY City;
\end{verbatim}

\hypertarget{union-with-where}{%
\subsection{UNION With WHERE}\label{union-with-where}}

The following SQL statement returns the German cities (only distinct values) from both the ``Customers'' and the ``Suppliers'' table:

\begin{verbatim}
SELECT City, Country 
  FROM Customers
    WHERE Country='Germany'
      UNION
        SELECT City, Country 
          FROM Suppliers
            WHERE Country='Germany'
              ORDER BY City;
\end{verbatim}

\hypertarget{union-all-with-where}{%
\subsection{UNION ALL With WHERE}\label{union-all-with-where}}

The following SQL statement returns the German cities (duplicate values also) from both the ``Customers'' and the ``Suppliers'' table:

\begin{verbatim}
SELECT City, Country 
  FROM Customers
    WHERE Country='Germany'
      UNION ALL
        SELECT City, Country 
          FROM Suppliers
            WHERE Country='Germany'
              ORDER BY City;
\end{verbatim}

\hypertarget{exists}{%
\section{EXISTS}\label{exists}}

The EXISTS operator is used to test for the existence of any record in a subquery. The EXISTS operator returns true if the subquery returns one or more records.

\begin{verbatim}
SELECT column_name(s)
  FROM table_name
    WHERE EXISTS
      (SELECT column_name FROM table_name WHERE condition);
\end{verbatim}

The following SQL statement returns TRUE and lists the suppliers with a product price over \$50

\begin{verbatim}
SELECT SupplierName
  FROM Suppliers
    WHERE EXISTS (SELECT ProductName 
                    FROM Products 
                      WHERE Products.SupplierID = Suppliers.supplierID 
                        AND Price > 50);
\end{verbatim}

\hypertarget{any-and-all}{%
\section{ANY and ALL}\label{any-and-all}}

The ANY and ALL operators are used with a WHERE or HAVING clause. The ANY operator returns TRUE if any of the subquery values meet the condition.

\begin{verbatim}
SELECT column_name(s)
FROM table_name
WHERE column_name operator ANY
(SELECT column_name FROM table_name WHERE condition);
\end{verbatim}

The following SQL statement returns TRUE and lists the product names if it finds ANY records in the OrderDetails table that quantity = 10:

\begin{verbatim}
SELECT *
  FROM Products
    WHERE ProductID = ANY (SELECT ProductID
                            FROM OrderDetails 
                              WHERE Quantity = 10);
\end{verbatim}

The ALL operator returns TRUE if all of the subquery values meet the condition. The following SQL statement returns TRUE and lists the product names if ALL the records in the OrderDetails table has quantity = 11.

\begin{verbatim}
SELECT ProductName
  FROM Products
    WHERE ProductID = ALL (SELECT ProductID 
                            FROM OrderDetails 
                              WHERE Quantity = 11);
\end{verbatim}

\hypertarget{group-by}{%
\section{GROUP BY}\label{group-by}}

The GROUP BY statement groups rows that have the same values into summary rows, like ``find the number of customers in each country''. The GROUP BY statement is often used with aggregate functions (COUNT, MAX, MIN, SUM, AVG) to group the result-set by one or more columns.

\begin{verbatim}
SELECT column_name(s)
  FROM table_name
    WHERE condition
      GROUP BY column_name(s)
        ORDER BY column_name(s);
\end{verbatim}

The following SQL statement lists the number of orders sent by each shipper:

\begin{verbatim}
SELECT Shippers.ShipperName, COUNT(Orders.OrderID) AS NumberOfOrders 
  FROM Orders
    LEFT JOIN Shippers ON Orders.ShipperID = Shippers.ShipperID
      GROUP BY ShipperName;
\end{verbatim}

\hypertarget{having-1}{%
\section{HAVING}\label{having-1}}

The following SQL statement lists the employees that have registered more than 10 orders:

\begin{verbatim}
SELECT Employees.LastName, COUNT(Orders.OrderID) AS NumberOfOrders
  FROM (Orders
    INNER JOIN Employees ON Orders.EmployeeID = Employees.EmployeeID)
      GROUP BY LastName
        HAVING COUNT(Orders.OrderID) > 10;
\end{verbatim}

The following SQL statement lists if the employees ``Davolio'' or ``Fuller'' have registered more than 25 orders:

\begin{verbatim}
SELECT Employees.LastName, COUNT(Orders.OrderID) AS NumberOfOrders
  FROM Orders
    INNER JOIN Employees ON Orders.EmployeeID = Employees.EmployeeID
      WHERE LastName = 'Davolio' OR LastName = 'Fuller'
        GROUP BY LastName
          HAVING COUNT(Orders.OrderID) > 25;
\end{verbatim}

\emph{Note:} All the functions that we can use in simple queries, it can definitely use them in Multi-table query.

\hypertarget{your-job-4}{%
\section{Your Job}\label{your-job-4}}

\begin{itemize}
\tightlist
\item
  List all orders with customer information!
\item
  List all orders with product names, quantities, and prices!
\item
  This will list all customers, whether they placed any order or not!
\item
  List customers that have not placed orders!
\item
  List all contacts, i.e., suppliers and customers!
\item
  List products with order quantities greater than 80!
\item
  Which products were sold by the unit (i.e.~quantity =1)?
\item
  List customers who placed orders that are larger than the average of each customer order!
\item
  Find best selling products based on quantity!
\item
  Find best selling products based on revenue!
\item
  Find best selling products based on revenue for each country!
\item
  Find suppliers with a product price less than \$50!
\item
  Find top 10 best employees based on their sales quantity!
\item
  Find top 10 best supplier countries based on quantity!
\item
  Find top 10 best customer countries based on quantity!
\item
  Find top 10 best selling products based on quantity in every year!
\end{itemize}

\hypertarget{introduction-to-flexdashboard}{%
\chapter{Introduction to Flexdashboard}\label{introduction-to-flexdashboard}}

Work-in-Progress {[}\url{https://pkgs.rstudio.com/flexdashboard/articles/flexdashboard.html}{]}

\hypertarget{flexdasboard-with-sqlite}{%
\chapter{Flexdasboard with SQLite}\label{flexdasboard-with-sqlite}}

Work-in-Progress {[}\url{https://pkgs.rstudio.com/flexdashboard/articles/flexdashboard.html}{]}

\hypertarget{shiny-dashboard}{%
\chapter{Shiny Dashboard}\label{shiny-dashboard}}

Work-in-Progress {[}\url{https://rstudio.github.io/shinydashboard/index.html}{]}

\hypertarget{basic-shiny-dashboard}{%
\section{Basic Shiny Dashboard}\label{basic-shiny-dashboard}}

\hypertarget{shiny-dashboard-plus}{%
\section{Shiny Dashboard Plus}\label{shiny-dashboard-plus}}

\hypertarget{shiny-dashboard-with-sql}{%
\chapter{Shiny Dashboard with SQL}\label{shiny-dashboard-with-sql}}

Work-in-Progress {[}\url{https://rstudio.github.io/shinydashboard/index.html}{]}

\hypertarget{basic-shiny-dashboard-1}{%
\section{Basic Shiny Dashboard}\label{basic-shiny-dashboard-1}}

\hypertarget{shiny-dashboard-plus-1}{%
\section{Shiny Dashboard Plus}\label{shiny-dashboard-plus-1}}

\hypertarget{data-analytics-dashboard}{%
\chapter{Data Analytics Dashboard}\label{data-analytics-dashboard}}

Work-in-Progress

\hypertarget{referensi}{%
\chapter{Referensi}\label{referensi}}

  \bibliography{book.bib,packages.bib}

\end{document}
