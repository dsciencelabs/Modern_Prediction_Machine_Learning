% Options for packages loaded elsewhere
\PassOptionsToPackage{unicode}{hyperref}
\PassOptionsToPackage{hyphens}{url}
%
\documentclass[
]{book}
\usepackage{amsmath,amssymb}
\usepackage{iftex}
\ifPDFTeX
  \usepackage[T1]{fontenc}
  \usepackage[utf8]{inputenc}
  \usepackage{textcomp} % provide euro and other symbols
\else % if luatex or xetex
  \usepackage{unicode-math} % this also loads fontspec
  \defaultfontfeatures{Scale=MatchLowercase}
  \defaultfontfeatures[\rmfamily]{Ligatures=TeX,Scale=1}
\fi
\usepackage{lmodern}
\ifPDFTeX\else
  % xetex/luatex font selection
\fi
% Use upquote if available, for straight quotes in verbatim environments
\IfFileExists{upquote.sty}{\usepackage{upquote}}{}
\IfFileExists{microtype.sty}{% use microtype if available
  \usepackage[]{microtype}
  \UseMicrotypeSet[protrusion]{basicmath} % disable protrusion for tt fonts
}{}
\makeatletter
\@ifundefined{KOMAClassName}{% if non-KOMA class
  \IfFileExists{parskip.sty}{%
    \usepackage{parskip}
  }{% else
    \setlength{\parindent}{0pt}
    \setlength{\parskip}{6pt plus 2pt minus 1pt}}
}{% if KOMA class
  \KOMAoptions{parskip=half}}
\makeatother
\usepackage{xcolor}
\usepackage{longtable,booktabs,array}
\usepackage{calc} % for calculating minipage widths
% Correct order of tables after \paragraph or \subparagraph
\usepackage{etoolbox}
\makeatletter
\patchcmd\longtable{\par}{\if@noskipsec\mbox{}\fi\par}{}{}
\makeatother
% Allow footnotes in longtable head/foot
\IfFileExists{footnotehyper.sty}{\usepackage{footnotehyper}}{\usepackage{footnote}}
\makesavenoteenv{longtable}
\usepackage{graphicx}
\makeatletter
\def\maxwidth{\ifdim\Gin@nat@width>\linewidth\linewidth\else\Gin@nat@width\fi}
\def\maxheight{\ifdim\Gin@nat@height>\textheight\textheight\else\Gin@nat@height\fi}
\makeatother
% Scale images if necessary, so that they will not overflow the page
% margins by default, and it is still possible to overwrite the defaults
% using explicit options in \includegraphics[width, height, ...]{}
\setkeys{Gin}{width=\maxwidth,height=\maxheight,keepaspectratio}
% Set default figure placement to htbp
\makeatletter
\def\fps@figure{htbp}
\makeatother
\setlength{\emergencystretch}{3em} % prevent overfull lines
\providecommand{\tightlist}{%
  \setlength{\itemsep}{0pt}\setlength{\parskip}{0pt}}
\setcounter{secnumdepth}{5}
\usepackage{booktabs}
\usepackage{lscape}

\ifLuaTeX
  \usepackage{selnolig}  % disable illegal ligatures
\fi
\usepackage[]{natbib}
\bibliographystyle{apalike}
\IfFileExists{bookmark.sty}{\usepackage{bookmark}}{\usepackage{hyperref}}
\IfFileExists{xurl.sty}{\usepackage{xurl}}{} % add URL line breaks if available
\urlstyle{same}
\hypersetup{
  pdftitle={Judul Buku},
  pdfauthor={Bakti Siregar, M.Sc},
  hidelinks,
  pdfcreator={LaTeX via pandoc}}

\title{Judul Buku}
\author{Bakti Siregar, M.Sc}
\date{2023-08-25}

\begin{document}
\maketitle

{
\setcounter{tocdepth}{1}
\tableofcontents
}
\hypertarget{kata-pengantar}{%
\chapter*{Kata Pengantar}\label{kata-pengantar}}
\addcontentsline{toc}{chapter}{Kata Pengantar}

Tuliskan Kata Pengantar Disini

\hypertarget{ringkasan-buku-modul}{%
\section*{Ringkasan Buku (Modul)}\label{ringkasan-buku-modul}}
\addcontentsline{toc}{section}{Ringkasan Buku (Modul)}

Adapun isi pembelajaran dalam modul ini adalah sebagai berikut:

\begin{itemize}
\tightlist
\item
  Bab 1
\item
  Bab 2
\item
  Bab 3
\item
  Dst
\end{itemize}

\hypertarget{penulis}{%
\section*{Penulis}\label{penulis}}
\addcontentsline{toc}{section}{Penulis}

\begin{itemize}
\tightlist
\item
  \textbf{Bakti Siregar, M.Sc} adalah Ketua Program Studi di Jurusan Statistika Universitas Matana. Lulusan Magister Matematika Terapan dari National Sun Yat Sen University, Taiwan. Beliau juga merupakan dosen dan konsultan Data Scientist di perusahaan-perusahaan ternama seperti \href{https://www.jne.co.id/id/beranda}{JNE}, \href{https://www.samoragroup.co.id/home/en}{Samora Group}, \href{https://www.pertamina.com/}{Pertamina}, dan lainnya. Beliau memiliki antusiasme khusus dalam mengajar Big Data Analytics, Machine Learning, Optimisasi, dan Analisis Time Series di bidang keuangan dan investasi. Keahliannya juga terlihat dalam penggunaan bahasa pemrograman Statistik seperti R Studio dan Python. Beliau mengaplikasikan sistem basis data MySQL/NoSQL dalam pembelajaran manajemen data, serta mahir dalam menggunakan tools Big Data seperti Spark dan Hadoop. Beberapa project beliau dapat dilihat di link berikut: \href{https://rpubs.com/dsciencelabs}{Rpubs}, \href{https://github.com/dsciencelabs}{Github}, \href{https://dsciencelabs.github.io/web/index.html}{Website}, dan \href{https://www.kaggle.com/baktisiregar/code}{Kaggle}.
\end{itemize}

\hypertarget{asisten-lab}{%
\section*{Asisten Lab}\label{asisten-lab}}
\addcontentsline{toc}{section}{Asisten Lab}

\begin{itemize}
\tightlist
\item
  \textbf{Yonathan Anggraiwan, S.Stat} adalah seorang alumni Statistika yang bersemangat dalam dunia pemrograman dan analisis data. Lahir di Tangerang, minatnya terhadap teknologi dan komputer muncul sejak usia dini. Ia tumbuh dengan rasa ingin tahu yang kuat terhadap bahasa pemrograman, dan ini membawanya menuju dunia analisis data menggunakan bahasa pemrograman R dan Python. Selama menjalankan tugas sebagai asisten lab, Yonathan Anggraiwan berperan dalam membantu mahasiswa dalam memahami konsep-konsep dasar dan kompleks dalam pemrograman R dan Python. Ia memberikan penjelasan yang jelas dan dukungan kepada mahasiswa yang mengalami kesulitan. Selain itu, ia juga terlibat dalam merancang tugas dan ujian praktikum, serta memberikan umpan balik konstruktif kepada para mahasiswa. Dalam perjalanan waktu, Yonathan Anggraiwan mulai mengambil tanggung jawab lebih besar dalam laboratorium. Ia membantu mengembangkan materi pembelajaran tambahan, seperti tutorial online tentang analisis data menggunakan R dan Python. Ia juga aktif dalam berbagai proyek penelitian di bawah bimbingan dosen, yang melibatkan pengolahan data besar untuk analisis statistik dan visualisasi. Dengan semangat yang tinggi, dedikasi, dan keterampilan yang dimilikinya, Yonathan Anggraiwan adalah contoh nyata dari seorang mahasiswa yang berhasil menggabungkan minatnya dalam pemrograman R dan Python dengan peran yang produktif sebagai asisten laboratorium dan kontributor dalam dunia analisis data.
\end{itemize}

\hypertarget{ucapan-terima-kasih}{%
\section*{Ucapan Terima Kasih}\label{ucapan-terima-kasih}}
\addcontentsline{toc}{section}{Ucapan Terima Kasih}

Kami berharap Template ini akan menjadi panduan yang bermanfaat bagi Anda dalam menuliskan E-book.

Terima kasih kepada semua yang telah berkontribusi dalam pembuatan template ini, serta kepada Anda, pembaca, yang telah memilih template ini sebagai sumber pengetahuan Anda. Kami berharap Anda menikmati perjalanan Anda dalam memahami bahasa pemrograman R.

\hypertarget{masukan-saran}{%
\section*{Masukan \& Saran}\label{masukan-saran}}
\addcontentsline{toc}{section}{Masukan \& Saran}

Semua masukan dan tanggapan Anda sangat berarti bagi kami untuk memperbaiki template ini kedepannya. Bagi para pembaca/pengguna yang ingin menyampaikan masukan dan tanggapan, dipersilahkan melalui kontak dibawak ini!

\textbf{Email:} \href{mailto:dsciencelabs@outlook.com}{\nolinkurl{dsciencelabs@outlook.com}}

\hypertarget{judul-topik-1}{%
\chapter{Judul Topik 1}\label{judul-topik-1}}

Isikan paragraf disini (opsional)

\hypertarget{sub-topik-1.1}{%
\section{Sub Topik 1.1}\label{sub-topik-1.1}}

Isikan paragraf disini (opsional)

\hypertarget{sub-topik-1.2}{%
\section{Sub Topik 1.2}\label{sub-topik-1.2}}

Isikan paragraf disini (opsional)

\hypertarget{judul-topik-2}{%
\chapter{Judul Topik 2}\label{judul-topik-2}}

Isikan paragraf disini (opsional)

\hypertarget{sub-topik-2.1}{%
\section{Sub Topik 2.1}\label{sub-topik-2.1}}

Isikan paragraf disini (opsional)

\hypertarget{sub-topik-2.1-1}{%
\section{Sub Topik 2.1}\label{sub-topik-2.1-1}}

Isikan paragraf disini (opsional)

\hypertarget{judul-topik-3}{%
\chapter{Judul Topik 3}\label{judul-topik-3}}

Isikan paragraf disini (opsional)

\hypertarget{sub-topik-3.1}{%
\section{Sub Topik 3.1}\label{sub-topik-3.1}}

Isikan paragraf disini (opsional)

\hypertarget{sub-topik-3.2}{%
\section{Sub Topik 3.2}\label{sub-topik-3.2}}

Isikan paragraf disini (opsional)

\hypertarget{referensi}{%
\chapter{Referensi}\label{referensi}}

Berikut adalah beberapa referensi yang dapat Anda gunakan untuk mempelajari dasar-dasar pemrograman dalam bahasa R:

\begin{enumerate}
\def\labelenumi{\arabic{enumi}.}
\tightlist
\item
  Venables, W.N. Smith D.M. and R Core Team. 2018. \textbf{An Introduction to R}: \url{https://cran.r-project.org/manuals.html}
\item
  R for Data Science: \url{https://r4ds.had.co.nz/}
\item
  Codecademy - Learn R : \url{https://www.codecademy.com/learn/learn-r}
\item
  DataCamp: \url{https://www.datacamp.com/courses/tech:r}
\item
  Primartha, R. 2018. \textbf{Belajar Machine Learning Teori dan Praktik}. Penerbit Informatika : Bandung
\item
  Rosadi,D. 2016. \textbf{Analisis Statistika dengan R}. Gadjah Mada University Press: Yogyakarta
\item
  STHDA. Running RStudio and Setting Up Your Working Directory - Easy R Programming .\url{http://www.sthda.com/english/wiki/running-rstudio-and-setting-up-your-working-directory-easy-r-programming\#set-your-working-directory}
\item
  STDHA. \textbf{Getting Help With Functions In R Programming}. \url{http://www.sthda.com/english/wiki/getting-help-with-functions-in-r-programming} .
\end{enumerate}

  \bibliography{book.bib,packages.bib}

\end{document}
